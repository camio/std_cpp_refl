\section{Technical Specifications}

We propose that the basic metadata describing a program written
in C++ should be made available through a set of {\em anonymous} classes
defined by the compiler. These classes should describe various program
constructs like, namespaces, types, typedefs, classes, their member variables
(member data), member functions, inheritance, templates, template parameters,
enumerated values, etc.

The compiler should generate metadata for the program constructs defined
in the currently processed translation unit. Indexed sets of metaobjects,
like scope members, parameters of a function, etc. should be listed
in the order of appearance in the processed source code.

Since we want the metadata to be available at compile-time,
different base-level constructs should be reflected by
{\em "statically" different} metaobjects and thus by {\em different} types.
For example a metaobject reflecting the global scope namespace should
be a different {\em type} than a metaobject reflecting the \verb@std@
namespace, a metaobject reflecting the \verb@int@ type should
have a different type then a metaobject reflecting the \verb@double@
type, a metaobject reflecting \verb@::foo(int)@ function should
have a different type than a metaobject reflecting \verb@::foo(double)@,
function, etc.

In a manner of speaking these special types (metaobjects) should become
"instances" of the meta-level concepts (static interfaces which
should not exist as concrete types, but rather only at the
"specification-level" similar for example to the iterator concepts).
This section describes a set of metaobject concepts,
their interfaces, tag types for metaobject classification and
functions (or operators) providing access to the metaobjects.

\subsection{Metaobject Concepts}

This section describes the requirements that various metaobjects
need to satisfy in order to be considered models of the individual
concepts.

\subsubsection{Categorization and Traits}

In order to provide means for distinguishing between regular types
and metaobjects the \verb@is_metaobject@ trait should be added
and should "return" \verb@true_type@ for metaobjects (types defined
by the compiler providing metadata) and \verb@false_type@
for non-metaobjects (native or user defined types).

The \verb@metaobject_traits@ structure should be defined to provide
categorization and additional information about the interface of metaobjects.

\begin{lstlisting}
template <typename Metaobject>
struct metaobject_traits
{
	typedef typename Metaobject::category category;

	typedef Bool has_name;

	typedef Bool has_scope;

	typedef Bool is_scope;

	typedef Bool is_class_member;

	typedef Bool has_template;

	typedef Bool is_template;
};
\end{lstlisting}

The meaning of the individual trait typedefs is following:

\begin{itemize}
\item{\verb@category@} Is one of the following types and specifies the category
of the metaobject:
	\begin{itemize}
		\item{\verb@specifier_tag@} indicates a {\metaobject Specifier}.

		\item{\verb@namespace_tag@} indicates a {\metaobject Namespace}.

		\item{\verb@global_scope_tag@} indicates the {\metaobject GlobalScope}.

		\item{\verb@type_tag@} indicates a {\metaobject Type}.

		\item{\verb@typedef_tag@} indicates a {\metaobject Typedef}.

		\item{\verb@class_tag@} indicates a {\metaobject Class}
		or a {\metaobject Template} class.

		\item{\verb@function_tag@} indicates a {\metaobject Function}
		or a {\metaobject Template} function.

		\item{\verb@constructor_tag@} indicates a {\metaobject Constructor}.

		\item{\verb@operator_tag@} indicates an {\metaobject Operator}.

		\item{\verb@overloaded_function_tag@} indicates an {\metaobject OverloadedFunction}.

		\item{\verb@enum_tag@} indicates an {\metaobject Enum}.

		\item{\verb@inheritance_tag@} indicates an {\metaobject Inheritance}.

		\item{\verb@constant_tag@} indicates an {\metaobject Constant}.

		\item{\verb@variable_tag@} indicates a {\metaobject Variable}.

		\item{\verb@parameter_tag@} indicates a {\metaobject Parameter}.
	\end{itemize}

\item{\verb@has_name@} indicates that the reflected object is {\metaobject Named}.
By default it is defined as \verb@false_type@ unless specified otherwise in the
concept description below.

\item{\verb@has_scope@} indicates that the reflected object is {\metaobject Scoped}.
By default it is defined as \verb@false_type@ unless specified otherwise in the
concept description below.

\item{\verb@is_scope@} indicates that the reflected object is a {\metaobject Scope}.
By default it is defined as \verb@false_type@ unless specified otherwise in the
concept description below.

\item{\verb@is_class_member@} indicates that the reflected object is a {\metaobject ClassMember}.
By default it is defined as \verb@false_type@ unless specified otherwise in the
concept description below.

\item{\verb@has_template@} indicates that the reflected function or class is
a template {\metaobject Instantiation}.
By default it is defined as \verb@false_type@ unless specified otherwise in the
concept description below.

\item{\verb@is_template@} indicates that the reflected object is function or class {\metaobject Template}.
By default it is defined as \verb@false_type@ unless specified otherwise in the
concept description below.
\end{itemize}

\subsubsection{Metaobject}

{\metaobject Metaobject} is a stateless (or monostate) anonymous \verb@struct@ that provides
metadata reflecting certain program features and has the following properties:

\begin{itemize}
\item For every {\metaobject Metaobject} the \verb@is_metaobject@ trait returns \verb@true_type@.
\item For every {\metaobject Metaobject} the \verb@metaobject_traits@ structure is defined.
\item For every {\metaobject Metaobject} the {\verb@typedef Metaobject::category@} is defined
and has the same meaning as \verb@metaobject_category<Metaobject>::category@.
\end{itemize}

The exact type of a specific {\metaobject Metaobject} reflecting a specific
program feature is not defined by the standard, instances of metaobjects
should be always declared through the \verb@auto@ type specifier.

All instances (in the classical sense) of a concrete {\metaobject Metaobject} (i.e all instances 
of the concrete anonymous type satysfying the requirements of the {\metaobject Type} concept
reflecting for example the \verb@int@ type) should be equal to
the programmer.

Instances (in the classical sense) of two different metaobjects (like an instance
of the concrete anonymous type satysfying the requirements of the {\metaobject Type} concept
reflecting the \verb@int@ type and an instance 
of the concrete anonymous type satysfying the requirements of the {\metaobject Type} concept
reflecting the \verb@double@ type) of course can (and will) be different.

\begin{lstlisting}
// for all purposes these two instances of (Meta)Type
// should be equal and interchangable without any change
// to the behavior of the program
auto meta_int_1 = reflected<int>();
auto meta_int_2 = reflected<int>();
\end{lstlisting}

\subsubsection{Specifier}

{\metaobject Specifier} is a {\metaobject Metaobject}, which reflects specifiers like
\verb@const@, \verb@volatile@, \verb@private@,
\verb@protected@, \verb@public@, \verb@virtual@, etc. and has the following
requirements:

\begin{itemize}

\item{\verb@static const char* keyword(void);@} returns the keyword
of the reflected specifier. If \verb@category@ is \verb@spec_none_tag@
then \verb@keyword@ returns "" (an empty c-string).

\item{\verb@typedef Category category;@} is defined as one of the following 
types:
	\begin{itemize}
		\item{\verb@spec_none_tag@} a category for missing specifiers,
		for example a non-const member function would have a \verb@spec_none_tag@
		constness specifier or a variable with automatic storage class
		would have a \verb@spec_none_tag@ storage class specifier, etc.

		\item{\verb@spec_extern_tag@} indicates \verb@extern@ storage class / linkage.
		\item{\verb@spec_static_tag@} indicates \verb@static@ storage class / linkage.
		\item{\verb@spec_mutable_tag@} indicates \verb@mutable@ storage class / linkage.
		\item{\verb@spec_register_tag@} indicates \verb@register@ storage class / linkage.
		\item{\verb@spec_thread_local_tag@} indicates \verb@thread_local@ storage class / linkage.

		\item{\verb@spec_const_tag@} indicates \verb@const@ member functions.

		\item{\verb@spec_virtual_tag@} indicates \verb@virtual@ inheritance or function linkage.

		\item{\verb@spec_private_tag@} indicates \verb@private@ member access.
		\item{\verb@spec_protected_tag@} indicates \verb@protected@ member access.
		\item{\verb@spec_public_tag@} indicates \verb@public@ member access.

		\item{\verb@spec_class_tag@} indicates the \verb@class@ elaborated type specifier.
		\item{\verb@spec_struct_tag@} indicates the \verb@struct@ elaborated type specifier.
		\item{\verb@spec_union_tag@} indicates the \verb@union@ elaborated type specifier.
		\item{\verb@spec_enum_tag@} indicates the \verb@enum@ elaborated type specifier.
	\end{itemize}
\end{itemize}

\subsubsection{Named}

{\metaobject Named} is a {\metaobject Metaobject} reflecting program constructs,
which have a name, like namespaces, types, functions, variables, etc. and has
the following requirements:

\begin{itemize}

	\item{\verb@static const char* base_name(void);@} member function that returns the base name
	of the reflected construct, without the nested name specifier. For namespace
	\verb@std@ this function should return "std", for namespace \verb@foo::bar::baz@
	this function should return "baz", for the global scope this function
	should return "" (an empty c-string literal).\\For \verb@std::vector<int>::iterator@
	it should return "iterator". For derived and qualified types like \\
	\verb@volatile std::vector<const foo::bar::fubar*> * const *@ it should return
	"volatile vector$<$const fubar*$>$ * const *", etc. The string returned by this
	function is owned by the function and should not be freed by the caller.

	\item{\verb@static const char* full_name(void);@} member function that returns the full name
	of the reflected construct, with the nested name specifier. For namespace
	\verb@std@ this function should return "std", for namespace \verb@foo::bar::baz@
	this function should return "foo::bar::baz", for the global scope this function
	should return "" (an empty c-string literal).\\For \verb@std::vector<int>::iterator@
	it should return "std::vector$<$int$>$::iterator". For derived and qualified types like\\
	\verb@volatile std::vector<const foo::bar::fubar*> * const *@ it should return
	"volatile std::vector$<$const foo::bar::fubar*$>$ * const *", etc. For some
	metaobjects this function may return the same value as the \verb@base_name@ function.
	The string returned by this function is owned by the function and should not be freed by the caller.
	

	\item \verb@metaobject_traits<Named>::has_name@ is defined as \verb@true_type@.
\end{itemize}

\subsubsection{Scoped}

{\metaobject Scoped} is a {\metaobject Metaobject} reflecting program constructs,
which are defined inside a scope (global scope, namespace, class, etc.). {\metaobject Scoped}
metaobjects have the following requirements:

\begin{itemize}
	\item{\verb@static Scope scope(void);@} static member function returning
	a {\metaobject Scope} metaobject reflecting the scope of the scoped object.
	In concrete metaobjects the result can be a {\metaobject Namespace}, {\metaobject Class},
	etc.\\The \verb@metaobject_traits<decltype(Scoped::scope())>::category@ typedef can be used to
	query the kind of the scope.

	\item \verb@metaobject_traits<Scoped>::has_scope@ is defined as \verb@true_type@.
\end{itemize}

\subsubsection{Scope}

{\metaobject Scope} is a {\metaobject Named} and {\metaobject Scoped} metaobject,
which reflects scopes like namespaces, classes, enums, etc. {\metaobject Scope}
has the following requirements:

\begin{itemize}

	\item{\verb@static integral_constant<int,@ {\em number-of-scope-members}
	\verb@>@\\\verb@member_count(void);@} static member function returning the total number
	of various members like types, namespaces, functions, variables, etc. defined inside
	the scope reflected by a {\metaobject Scope}.

	\item{\verb@static @{\metaobject Scoped}\verb@ member(integral_constant<int, @{\em i}
	\verb@>);@} overloaded member function defined
	for $i \in \{0, 1, \dots, n-1\}$;{\em n = number-of-scope-members},
	each overload returns a different {\metaobject Scoped} metaobject reflecting the {\em i}-th member
	defined inside the scope reflected by a {\metaobject Scope}.
	In concrete metaobjects reflecting various kinds of scopes the \verb@member(...)@ function
	can return metaobjects like {\metaobject Namespace}, ({\metaobject ClassMember}) {\metaobject Variable},
	({\metaobject ClassMember}) {\metaobject OverloadedFunction}, {\metaobject Typedef},
	{\metaobject Enum}, etc.

	\item \verb@metaobject_traits<Scope>::is_scope@ is defined as \verb@true_type@.
\end{itemize}

\subsubsection{Namespace}

{\metaobject Namespace} is a {\metaobject Scope} with the following requirements:

\begin{itemize}
	\item \verb@metaobject_traits<Namespace>::category@ is defined as
	\verb@namespace_tag@.
\end{itemize}

\subsubsection{GlobalScope}

{\metaobject GlobalScope} is a {\metaobject Namespace} reflecting the global scope
and requires the following:

\begin{itemize}
	\item \verb@metaobject_traits<GlobalScope>::category@ is defined as
	\verb@global_scope_tag@.
\end{itemize}

\subsubsection{Type}

{\metaobject Type} is a {\metaobject Named} and {\metaobject Scoped} metaobject which
has the following requirements:

\begin{itemize}
	\item{\verb@typedef @{\em original-type}\verb@ original_type;@} member typedef
	defined as the original type reflected by the {\metaobject Type}.

	\item \verb@metaobject_traits<Type>::category@ is defined as \verb@type_tag@.
\end{itemize}

The \verb@is_template@ typedef in \verb@metaobject_traits@ changes the requirements
in the concepts derived from {\metaobject Type}.

\subsubsection{Typedef}

{\metaobject Typedef} is a {\metaobject Type} metaobject that reflects typedefs,
i.e. types that were defined as alternate names for another types.
{\metaobject Typedef} has the following requirements:

\begin{itemize}
	\item{\verb@static @{\metaobject Type}\verb@ type(void);@} static member function
	returning a {\metaobject Type} reflecting the "source" type of the typedef.
	In concrete {\metaobject Typedef} metaobjects \verb@type@ can return a 
	{\metaobject Type}, {\metaobject Class}, {\metaobject Enum} or {\metaobject Typedef}.

	\item \verb@metaobject_traits<Typedef>::category@ is defined as \verb@typedef_tag@.
\end{itemize}

\subsubsection{Class}

{\metaobject Class} is a {\metaobject Type} and a {\metaobject Scope} that reflects
an elaborated type (class, struct, union) or a class template.
{\metaobject Class} has the following requirements, but
the \verb@is_template@ typedef in the \verb@metaobject_traits@ changes the requirements
inherited from the {\metaobject Type} concept as described below.

\begin{itemize}
	\item{\verb@static @{\metaobject Specifier}\verb@ elaborated_type(void);@}
	static member function returning a {\metaobject Specifier} reflecting the elaborated
	type specifier used to define the class (\verb@class@, \verb@struct@, \verb@union@).

	\item{\verb@static integral_constant<int,@ {\em number-of-base-classes}
	\verb@>@\\\verb@base_class_count(void);@} static member function returning the total number
	of base classes that the class reflected by {\metaobject Class} inherits from.

	\item{\verb@static @{\metaobject Inheritance}\verb@ base_class(integral_constant<int, @{\em i}
	\verb@>);@} overloaded member function defined
	for $i \in \{0, 1, \dots, n-1\}$; {\em n = number-of-base-classes},
	each overload returns a different {\metaobject Inheritance} metaobject reflecting the inheritance
	of the {\em i}-th base class of the class reflected by {\metaobject Class}.

	\item \verb@metaobject_traits<Class>::category@ is defined as \verb@class_tag@.
\end{itemize}

If \verb@metaobject_traits<Class>::is_template@ is \verb@true_type@ it indicates 
that the reflected program feature is not a regular class, but a class template.
In such case the \verb@original_type@ typedef inherited from {\metaobject Type}
is not defined.

\subsubsection{Function}

{\metaobject Function} is a {\metaobject Scope} metaobject that reflects a function
or a function template and requires the following (the requirements are influenced
by the \verb@metaobject_traits<Function>::is_template@ typedef as described below):

\begin{itemize}
	\item{\verb@static @{\metaobject Specifier}\verb@ linkage(void);@} static member function returning
	a {\metaobject Specifier} reflecting the linkage specifier of the function.

	\item{\verb@static @{\metaobject Type}\verb@ result_type(void);@} static member function returning
	a {\metaobject Type} reflecting the result type of the function.

	\item{\verb@static integral_constant<int,@ {\em number-of-parameters}
	\verb@>@\\\verb@parameter_count(void);@} static member function returning the total number
	of parameters of the function reflected by {\metaobject Function}.

	\item{\verb@static @{\metaobject Parameter}\verb@ parameter(integral_constant<int, @{\em i}
	\verb@>);@} overloaded member function defined
	for $i \in \{0, 1, \dots, n-1\}$; {\em n = number-of-parameters},
	each overload returns a different {\metaobject Parameter} metaobject reflecting the {\em i}-th parameter
	of the function reflected by {\metaobject Function}.

	\item \verb@metaobject_traits<Function>::category@ is defined as \verb@function_tag@.
\end{itemize}

If \verb@metaobject_traits<Function>::is_template@ is defined as \verb@false_type@
i.e. the reflected feature is not a template but a regular function then the following is also
required:

\begin{itemize}
	\item{\verb@static inline ResultType::original_type call(@ {\em parameters\dots}\verb@);@}
	static inline member function with the same return value type and the same number
	and type of parameters as the original function reflected by {\metaobject Function}.
	Calls to this function should be replaced with the call of the reflected function
	with the arguments passed to \verb@call@. Additionaly if the reflected function is
	a member function, then the first of the {\em parameters} of \verb@call@ should be
	a reference to the class where the member function is defined and should be used
	as the \verb@this@ argument when calling the member function. If the member function
	is declared as \verb@const@ then the reference to the class should also be \verb@const@.
\end{itemize}

If \verb@metaobject_traits<Function>::is_class_member@ is defined as \verb@true_type@
i.e. the reflected is a member function and not a free function or lambda function,
then the following is also required:

\begin{itemize}
	\item{\verb@static @ {\metaobject Specifier} \verb@ constness(void);@} static member function
	returning  the constness {\metaobject Specifier} reflecting the constness
	of the member functions.
\end{itemize}

{\metaobject Function} metaobjects are {\em not} direct members of scopes. Instead,
all functions with the same name (even those that are not overloaded) in a specific scope
are grouped into a {\metaobject OverloadedFunction}. Individual overloaded {\metaobject Function}s
in the group can be obtained through the interface of {\metaobject OverloadedFunction}.
The same should also apply to {\metaobject Constructor}s and {\metaobject Operator}s.

The idea is that (direct) scope members (i.e. metaobjects accessible through \verb@Scope::member(...)@)
should have unique names.

The {\metaobject Scope} returned by the \verb@scope@ member function of every single
{\metaobject Function} in a {\metaobject OverloadedFunction}
is the same as the \verb@scope@ of that {\metaobject OverloadedFunction}, i.e.
the \verb@scope@ of a {\metaobject Function} can be a {\metaobject Namespace} or a {\metaobject Class}
but {\em not} a {\metaobject OverloadedFunction}.

\subsubsection{ClassMember}

{\metaobject ClassMember} is a {\metaobject Named} and {\metaobject Scoped} metaobject
that reflects a member of a class. It has the following requirements:

\begin{itemize}
	\item{\verb@static @{\metaobject Specifier}\verb@ access_type(void);@} static member function returning
	a {\metaobject Specifier} reflecting the access type specifier of the class member
	(\verb@private@, \verb@protected@ or \verb@public@).

	\item \verb@metaobject_traits<ClassMember::scope>::is_class_member@ is \verb@true_type@.
\end{itemize}

Concrete metaobjects that are models of this concept can also be some of the following:
\begin{itemize}
	\item{\metaobject Typedef}
	\item{\metaobject Class}
	\item{\metaobject Enum}
	\item{\metaobject OverloadedFunction}
\end{itemize}

\subsubsection{Constructor}

{\metaobject Constructor} is a {\metaobject ClassMember} and a {\metaobject Function} that
reflects a constructor and requires the following:

\begin{itemize}
	\item \verb@metaobject_traits<Constructor>::category@ is defined as  \verb@constructor_tag@.

	\item The result of \verb@Constructor::result_type()@ is the same as the result of
	\verb@Constructor::scope()@.
\end{itemize}

\subsubsection{Operator}

{\metaobject Operator} is a {\metaobject Function} and possibly a {\metaobject ClassMember}
that reflects an operator and requires the following:

\begin{itemize}
	\item \verb@metaobject_traits<Operator>::category@ is defined as  \verb@operator_tag@.
\end{itemize}

\subsubsection{OverloadedFunction}

{\metaobject OverloadedFunction} is a {\metaobject Function} and possibly a {\metaobject ClassMember}
that reflects a set of overloaded functions, i.e. functions with the same name.
{\metaobject OverloadedFunction} has the following requirements:

\begin{itemize}

	\item{\verb@static integral_constant<int,@ {\em number-of-overloads}
	\verb@>@\\\verb@overload_count(void);@} static member function returning the total number
	of parameters of the function reflected by {\metaobject Function}.

	\item{\verb@static @{\metaobject Function}\verb@ overload(integral_constant<int, @{\em i}
	\verb@>);@} overloaded member function defined
	for $i \in \{0, 1, \dots, n-1\}$; {\em n = number-of-overloads},
	each overload returns a different {\metaobject Function} metaobject reflecting the {\em i}-th overload
	in the set of functions reflected by {\metaobject OverloadedFunction}.

	\item \verb@metaobject_traits<OverloadedFunction>::category@ is defined as 
	\verb@overloaded_function_tag@.
\end{itemize}

\subsubsection{Template}

{\metaobject Template} is a {\metaobject Function} or a {\metaobject Class} metaobject
that reflects a function or class template. It has the following requirements:

\begin{itemize}

	\item{\verb@static integral_constant<int,@ {\em number-of-template-parameters}
	\verb@>@\\\verb@template_parameter_count(void);@} static member function returning the total number
	of parameters of the template reflected by {\metaobject Template}.

	\item{\verb@static @{\metaobject TemplateParameter}\\\verb@template_parameter(integral_constant<int, @{\em i}
	\verb@>);@} overloaded member function defined
	for $i \in \{0, 1, \dots, n-1\}$; {\em n = number-of-parameters},
	each overload returns a different {\metaobject Parameter} metaobject reflecting the {\em i}-th parameter
	of the template reflected by {\metaobject Template}.

	\item{\verb@template <@ {\em template-parameters...} \verb@>@\\
	\verb@static Instantiation instantiation(void);@} static member template function returning an {\metaobject Instantiation}
	reflecting the instantiation of the template with the specified parameters. The {\em template-parameters} passed
	to this function must be valid template parameters for the reflected template.

	\item \verb@metaobject_traits<Template>::is_template@ is defined as \verb@true_type@.
\end{itemize}

\subsubsection{TemplateParameter}

{\metaobject TemplateParameter} is a {\metaobject Typedef} or a {\metaobject Constant} that
reflects a template parameter. In class templates the types of member variables, typedefs and
the return value type and parameters of member functions may be {\metaobject TemplateParameter}
metaobjects.

\begin{itemize}
	\item{\verb@static integral_constant<int, @{\metaobject position-of-parameter}\verb@> postion(void);@}
	static member function returning the postion of the template parameter.

	\item The \verb@scope@ member function inherited from {\metaobject Scoped} returns a {\metaobject Template}
	reflecting the template which defined this template parameter.

	\item \verb@metaobject_traits<TemplateParameter>::is_template@ is defined as \verb@true_type@.
\end{itemize}

The \verb@metaobject_traits<TemplateParameter>::category@ typedef should be used to distinguish between
type and non-type template parameters.

\subsubsection{Instantiation}

{\metaobject Instantiation} is a {\metaobject Function} or {\metaobject Class} metaobject
that reflects a templated function or class for which the following is required:

\begin{itemize}

	\item{\verb@static @ {\metaobject Template} \verb@model(void);@} static member function returning
	a {\metaobject Template} reflecting the template that the class or function 
	(reflected by an {\metaobject Instantiation}) is an instantiation of.

	\item \verb@metaobject_traits<Instantiation>::has_template@ is defined as \verb@true_type@.
	This trait is used to distinguish classes and functions which are instantiations
	of a template from non-templated classes and functions.
\end{itemize}

\subsubsection{Enum}

{\metaobject Enum} is a {\metaobject Type} and a {\metaobject Scope} that reflects an enumerated
type with the following requirements:

\begin{itemize}
	\item \verb@metaobject_traits<Enum>::category@ is defined as \verb@enum_tag@.

	\item The \verb@members@ of {\metaobject Enum} are only {\metaobject Constant} metaobjects.
\end{itemize}

\subsubsection{Inheritance}

{\metaobject Inheritance} is a {\metaobject Metaobject} that is reflecting class inheritance and has
the following requirements:

\begin{itemize}
	\item{\verb@static @{\metaobject Specifier}\verb@access_type(void);@} static member function
	returning a access-type {\metaobject Specifier} that reflects the inheritance access type
	(private, protected or public).

	\item{\verb@static @{\metaobject Specifier}\verb@inheritance_type(void);@} static member function
	returning an inheritance-type {\metaobject Specifier} that reflects the inheritance access type
	(virtual or non-virtual).

	\item{\verb@static @{\metaobject Class} \verb@ base_class(void);@} static member function
	returning a {\metaobject Class} reflecting the base class in the inheritance.

	\item{\verb@static @{\metaobject Class} \verb@ derived_class(void);@} static member function
	returning a {\metaobject Class} reflecting the derived class in the inheritance.

	\item \verb@metaobject_traits<Inheritance>::category@ is defined as \verb@inheritance_tag@.
\end{itemize}

\subsubsection{Variable}

{\metaobject Variable} is a {\metaobject Named} and {\metaobject Scoped} metaobject and possibly a
{\metaobject ClassMember}, that reflects some kind of variable defined in a namespace, class, function, etc.
and has the following requirements:

\begin{itemize}
	\item{\verb@static @{\metaobject Specifier}\verb@ storage_class();@} static member function returning
	a storage-class {\metaobject Specifier} reflecting the storage class of the variable.

	\item{\verb@static @{\metaobject Type}\verb@ type();@} static member function returning
	a {\metaobject Type} reflecting the type of the variable.

	\item \verb@metaobject_traits<Variable>::category@ is defined as \verb@variable_tag@.
\end{itemize}

\subsubsection{Parameter}

{\metaobject Parameter} is a {\metaobject Variable} that reflects a parameter of a function.
The following is required for metaobjects reflecting parameters:

\begin{itemize}
	\item{\verb@static integral_constant<int, @{\em position-of-parameter}\verb@> position(void);@}
	static member function returning the position of the parameter in the function parameter list
	declaration.

	\item \verb@metaobject_traits<Parameter>::category@ is defined as \verb@parameter_tag@.

	\item The \verb@scope@ member function inherited from {\metaobject Scoped} returns
	the {\metaobject Function} that the parameter belongs to.
\end{itemize}

\subsubsection{Constant}

{\metaobject Constant} is a {\metaobject Named} and possibly {\metaobject Scoped} metaobject reflecting
named compile-time constant values like the non-type template parameters and enumeration values.

\begin{itemize}
	\item{\verb@static integral_type<@{\em value-type}\verb@, @{\em constant-value}\verb@> value(void);@}
	static member function returning the reflected value wrapped in \verb@integral_constant@.

	\item \verb@metaobject_traits<Constant>::category@ is defined as \verb@constant_tag@.
\end{itemize}


\subsection{Reflection}

The metaobjects can be provided either via a set of overloaded
functions defined in the \verb@std@ namespace or by a new operator.
Both of these approaches have advantages and disadvantages.

\subsubsection{Reflection functions}

In this approach at least two functions should be defined
in the \verb@std@ namespace:

\begin{itemize}
	\item{{\em unspecified-type} \verb@ reflected_global_scope(void);@}
	This function should return a type conforming to the {\metaobject GlobalScope}
	concept, reflecting the global scope.
	The real type or the result is not defined by the standard, i.e. it is an implementation detail. 
	If the caller needs to store the result of this function the \verb@auto@ type
	specifier should always be used.

	\item{\verb@template <typename Type>@\\
	{\em unspecified-type} \verb@ reflected(void);@}
	This function should return a type conforming to the {\metaobject Type}
	concept, reflecting the \verb@Type@ passed as template argument to this function.
	The real type or the result is not defined by the standard, i.e. it is an implementation detail. 
	If the caller needs to store the result of this function the \verb@auto@ type
	specifier should always be used.
\end{itemize}

Several other similar functions could be added to the list above
for reflection of templates, enumerated values, etc. Without defining
new rules for what regular function and template parameters can be.
The advantages of using reflection functions are following:

\begin{itemize}
	\item No need to add a new keyword to the language.

	\item Reduced chance of breaking existing code. The \verb@reflected_global_scope()@
	and \verb@reflected<Type>()@ functions are currently not defined in the \verb@std@
	namespace and therefore should not clash with existing user code.
\end{itemize}

This approach has the following disadvantages:

\begin{itemize}
	\item Less direct reflection. Using this approach it is not possible
	(at least without adding new rules for possible values of template
	and function parameters) to reflect constructors, overloaded functions
	and some other things.
\end{itemize}

\subsubsection{Reflection operator}

In this approach a new operator (we suggest the name) \verb@reflect(@{\em param}\verb@)@
should be added (for some alternatives see below).
Depending on {\em param}, which could be a type name, namespace name,
template name, overloaded function name, enumerated value name, etc. the operator should
return a {\metaobject Named} metaobject reflecting the specified feature.
If the parameter is ommited a type conforming to the {\metaobject GlobalScope} metaobject
concept should be returned.
The exact types returned by the operators should be implementation details and if the
result needs to be stored in a variable the \verb@auto@ type specifier should always
be used.
For example:

\begin{lstlisting}
// reflect the global scope
// meta_gs conforms to the GlobalScope concept
auto meta_gs = reflect();

// reflect the std namespace
// meta_std conforms to the Namespace concept
auto meta_std = reflect(std);

// reflect the errno variable
// meta_errno conforms to the Variable concept
auto meta_errno = reflect(errno);

// reflect the int type
// meta_int conforms to the Type concept
auto meta_int = reflect(int);

// reflect the std::string typedef
// meta_std_string conforms to the Typedef concept
auto meta_std_string = reflect(std::string);

// reflect the std::map template
// meta_std_map conforms to the Template
// and Class concepts
auto meta_std_map = reflect(std::map);

// reflect the std::map<int, std::string> type
// meta_std_map_int_std_string conforms to Class
// and Instantiation concepts
auto meta_std_map_int_std_string = 
	reflect(std::map<int, std::string>);

// reflect the std::string's (overloaded) constructors
// meta_std_string_string conforms to
// the OverloadedFunction concept and the individual
// overloads that it allows to traverse conform
// to the Constructor concept
auto meta_std_string_string = 
	reflect(std::string::string);

// reflect ths std::string's copy constructor
// meta_std_string_string_copy conforms to
// the Constructor concept
auto meta_std_string_string_copy =
	reflect(std::string::string(const std::string&));

// reflect the std::swap overloaded free function
// meta_std_swap conforms to OverloadedFunction
auto meta_std_swap = reflect(std::swap);

\end{lstlisting}

Using a new operator has the following advantages:

\begin{itemize}
	\item More direct reflection. Even features that
	could not be reflected by using a (templated) function
	could be reflected with an operator.

	\item More consistent reflection. Everything is reflected
	with a single operator.
\end{itemize}

and these disadvantages:

\begin{itemize}
	\item Requires a new keyword or the usage of an existing
	keyword in a new context or the usage of a character
	sequence that is currently invalid.

	\item Increased risk of breaking existing code. Could
	be resolved by using an existing operator like \verb@%@,
	\verb@|@, etc.,
	or the use of a currently invalid character or character
	sequence like \verb'@', \verb@$@ or the usage of a new set
	of quotations like \verb@`@ (backtick character).
	For example:
	\begin{lstlisting}
	// instead of:
	auto meta_std_string = reflect(std::string);
	// use
	auto meta_std_string = %std::string;
	// or 
	auto meta_std_string = |std::string;
	// or
	auto meta_std_string = @std::string;
	// or
	auto meta_std_string = `std::string`;
	\end{lstlisting}
	The problem with these may be the reflection
	of the global scope, which when using some of
	the above would result in awkward expressions
	like:
	\begin{lstlisting}
	// instead of
	auto meta_gs = reflect();
	// use
	auto meta_gs = %;
	auto meta_gs = |;
	auto meta_gs = @;
	// or 
	auto meta_gs = ``; 
	\end{lstlisting}
\end{itemize}

