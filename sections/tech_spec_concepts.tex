\subsection{Metaobject Concepts}

This section describes the requirements that various metaobjects
need to satisfy in order to be considered models of the individual
concepts.

\subsubsection{Categorization and Traits}

In order to provide means for distinguishing between regular types
and metaobjects the \verb@is_metaobject@ trait should be added
and should "return" \verb@true_type@ for metaobjects (types defined
by the compiler providing metadata) and \verb@false_type@
for non-metaobjects (native or user defined types).

The \verb@metaobject_traits@ structure should be defined to provide
categorization and additional information about the interface of metaobjects.

\begin{lstlisting}
template <typename Metaobject>
struct metaobject_traits
{
	typedef typename Metaobject::category category;

	typedef Bool has_name;

	typedef Bool has_scope;

	typedef Bool is_scope;

	typedef Bool is_class_member;

	typedef Bool has_template;

	typedef Bool is_template;
};
\end{lstlisting}

The meaning of the individual trait typedefs is following:

\begin{itemize}
\item{\verb@category@} Is one of the following types and specifies the category
of the metaobject:
	\begin{itemize}
		\item{\verb@specifier_tag@} indicates a {\metaobject Specifier}.

		\item{\verb@namespace_tag@} indicates a {\metaobject Namespace}.

		\item{\verb@global_scope_tag@} indicates the {\metaobject GlobalScope}.

		\item{\verb@type_tag@} indicates a {\metaobject Type}.

		\item{\verb@typedef_tag@} indicates a {\metaobject Typedef}.

		\item{\verb@class_tag@} indicates a {\metaobject Class}
		or a {\metaobject Template} class.

		\item{\verb@function_tag@} indicates a {\metaobject Function}
		or a {\metaobject Template} function.

		\item{\verb@constructor_tag@} indicates a {\metaobject Constructor}.

		\item{\verb@operator_tag@} indicates an {\metaobject Operator}.

		\item{\verb@overloaded_function_tag@} indicates an {\metaobject OverloadedFunction}.

		\item{\verb@enum_tag@} indicates an {\metaobject Enum}.

		\item{\verb@inheritance_tag@} indicates an {\metaobject Inheritance}.

		\item{\verb@constant_tag@} indicates an {\metaobject Constant}.

		\item{\verb@variable_tag@} indicates a {\metaobject Variable}.

		\item{\verb@parameter_tag@} indicates a {\metaobject Parameter}.
	\end{itemize}

\item{\verb@has_name@} indicates that the reflected object is {\metaobject Named}.
By default it is defined as \verb@false_type@ unless specified otherwise in the
concept description below.

\item{\verb@has_scope@} indicates that the reflected object is {\metaobject Scoped}.
By default it is defined as \verb@false_type@ unless specified otherwise in the
concept description below.

\item{\verb@is_scope@} indicates that the reflected object is a {\metaobject Scope}.
By default it is defined as \verb@false_type@ unless specified otherwise in the
concept description below.

\item{\verb@is_class_member@} indicates that the reflected object is a {\metaobject ClassMember}.
By default it is defined as \verb@false_type@ unless specified otherwise in the
concept description below.

\item{\verb@has_template@} indicates that the reflected function or class is
a template {\metaobject Instantiation}.
By default it is defined as \verb@false_type@ unless specified otherwise in the
concept description below.

\item{\verb@is_template@} indicates that the reflected object is function or class {\metaobject Template}.
By default it is defined as \verb@false_type@ unless specified otherwise in the
concept description below.
\end{itemize}

\subsubsection{Metaobject}

{\metaobject Metaobject} is a stateless (or monostate) anonymous \verb@struct@ that provides
metadata reflecting certain program features and has the following properties:

\begin{itemize}
\item For every {\metaobject Metaobject} the \verb@is_metaobject@ trait returns \verb@true_type@.
\item For every {\metaobject Metaobject} the \verb@metaobject_traits@ structure is defined.
\item For every {\metaobject Metaobject} the {\verb@typedef Metaobject::category@} is defined
and has the same meaning as \verb@metaobject_category<Metaobject>::category@.
\end{itemize}

The exact type of a specific {\metaobject Metaobject} reflecting a specific
program feature is not defined by the standard, instances of metaobjects
should be always declared through the \verb@auto@ type specifier.

All instances (in the classical sense) of a concrete {\metaobject Metaobject} (i.e all instances 
of the concrete anonymous type satysfying the requirements of the {\metaobject Type} concept
reflecting for example the \verb@int@ type) should be equal to
the programmer.

Instances (in the classical sense) of two different metaobjects (like an instance
of the concrete anonymous type satysfying the requirements of the {\metaobject Type} concept
reflecting the \verb@int@ type and an instance 
of the concrete anonymous type satysfying the requirements of the {\metaobject Type} concept
reflecting the \verb@double@ type) of course can (and will) be different.

\begin{lstlisting}
// for all purposes these two instances of (Meta)Type
// should be equal and interchangable without any change
// to the behavior of the program
auto meta_int_1 = reflected<int>();
auto meta_int_2 = reflected<int>();
\end{lstlisting}

\subsubsection{Specifier}

{\metaobject Specifier} is a {\metaobject Metaobject}, which reflects specifiers like
\verb@const@, \verb@volatile@, \verb@private@,
\verb@protected@, \verb@public@, \verb@virtual@, etc. and has the following
requirements:

\begin{itemize}

\item{\verb@static const char* keyword(void);@} returns the keyword
of the reflected specifier. If \verb@category@ is \verb@spec_none_tag@
then \verb@keyword@ returns "" (an empty c-string).

\item{\verb@typedef Category category;@} is defined as one of the following 
types:
	\begin{itemize}
		\item{\verb@spec_none_tag@} a category for missing specifiers,
		for example a non-const member function would have a \verb@spec_none_tag@
		constness specifier or a variable with automatic storage class
		would have a \verb@spec_none_tag@ storage class specifier, etc.

		\item{\verb@spec_extern_tag@} indicates \verb@extern@ storage class / linkage.
		\item{\verb@spec_static_tag@} indicates \verb@static@ storage class / linkage.
		\item{\verb@spec_mutable_tag@} indicates \verb@mutable@ storage class / linkage.
		\item{\verb@spec_register_tag@} indicates \verb@register@ storage class / linkage.
		\item{\verb@spec_thread_local_tag@} indicates \verb@thread_local@ storage class / linkage.

		\item{\verb@spec_const_tag@} indicates \verb@const@ member functions.

		\item{\verb@spec_virtual_tag@} indicates \verb@virtual@ inheritance or function linkage.

		\item{\verb@spec_private_tag@} indicates \verb@private@ member access.
		\item{\verb@spec_protected_tag@} indicates \verb@protected@ member access.
		\item{\verb@spec_public_tag@} indicates \verb@public@ member access.

		\item{\verb@spec_class_tag@} indicates the \verb@class@ elaborated type specifier.
		\item{\verb@spec_struct_tag@} indicates the \verb@struct@ elaborated type specifier.
		\item{\verb@spec_union_tag@} indicates the \verb@union@ elaborated type specifier.
		\item{\verb@spec_enum_tag@} indicates the \verb@enum@ elaborated type specifier.
	\end{itemize}
\end{itemize}

\subsubsection{Named}

{\metaobject Named} is a {\metaobject Metaobject} reflecting program constructs,
which have a name, like namespaces, types, functions, variables, etc. and has
the following requirements:

\begin{itemize}

	\item{\verb@static const char* base_name(void);@} member function that returns the base name
	of the reflected construct, without the nested name specifier. For namespace
	\verb@std@ this function should return "std", for namespace \verb@foo::bar::baz@
	this function should return "baz", for the global scope this function
	should return "" (an empty c-string literal).\\For \verb@std::vector<int>::iterator@
	it should return "iterator". For derived and qualified types like \\
	\verb@volatile std::vector<const foo::bar::fubar*> * const *@ it should return
	"volatile vector$<$const fubar*$>$ * const *", etc. The string returned by this
	function is owned by the function and should not be freed by the caller.

	\item{\verb@static const char* full_name(void);@} member function that returns the full name
	of the reflected construct, with the nested name specifier. For namespace
	\verb@std@ this function should return "std", for namespace \verb@foo::bar::baz@
	this function should return "foo::bar::baz", for the global scope this function
	should return "" (an empty c-string literal).\\For \verb@std::vector<int>::iterator@
	it should return "std::vector$<$int$>$::iterator". For derived and qualified types like\\
	\verb@volatile std::vector<const foo::bar::fubar*> * const *@ it should return
	"volatile std::vector$<$const foo::bar::fubar*$>$ * const *", etc. For some
	metaobjects this function may return the same value as the \verb@base_name@ function.
	The string returned by this function is owned by the function and should not be freed by the caller.
	

	\item \verb@metaobject_traits<Named>::has_name@ is defined as \verb@true_type@.
\end{itemize}

\subsubsection{Scoped}

{\metaobject Scoped} is a {\metaobject Metaobject} reflecting program constructs,
which are defined inside a scope (global scope, namespace, class, etc.). {\metaobject Scoped}
metaobjects have the following requirements:

\begin{itemize}
	\item{\verb@typedef Scope scope;@} member typedef defined as
	a {\metaobject Scope} metaobject reflecting the scope of the scoped object.
	In concrete metaobjects this typedef can be a {\metaobject Namespace}, {\metaobject Class},
	etc. The \verb@metaobject_traits<Scoped::scope>::category@ typedef can be used to
	query the kind of the scope.

	\item \verb@metaobject_traits<Scoped>::has_scope@ is defined as \verb@true_type@.
\end{itemize}

\subsubsection{Scope}

{\metaobject Scope} is a {\metaobject Named} and {\metaobject Scoped} metaobject,
which reflects scopes like namespaces, classes, enums, etc. {\metaobject Scope}
has the following requirements:

\begin{itemize}

	\item{\verb@typedef integral_constant<int,@ {\em number-of-scope-members}
	\verb@>@\\\verb@member_count;@} member typedef specifying the total number
	of various members like types, namespace, functions, variables, etc. defined inside
	the scope reflected by a {\metaobject Scope}.

	\item{\verb@static @{\metaobject Scoped}\verb@ member(integral_constant<int, @{\em i}
	\verb@>);@} overloaded member function defined
	for $i \in \{0, 1, \dots, n-1\}$;{\em n = number-of-scope-members},
	each overload returns a different {\metaobject Scoped} metaobject reflecting the {\em i}-th member
	defined inside the scope reflected by a {\metaobject Scope}.
	In concrete metaobjects reflecting various kinds of scopes the \verb@member(...)@ function
	can return metaobjects like {\metaobject Namespace}, ({\metaobject ClassMember}) {\metaobject Variable},
	({\metaobject ClassMember}) {\metaobject OverloadedFunction}, {\metaobject Typedef},
	{\metaobject Enum}, etc.

	\item \verb@metaobject_traits<Scope>::is_scope@ is defined as \verb@true_type@.
\end{itemize}

\subsubsection{Namespace}

{\metaobject Namespace} is a {\metaobject Scope} with the following requirements:

\begin{itemize}
	\item \verb@metaobject_traits<Namespace>::category@ is defined as
	\verb@namespace_tag@.
\end{itemize}

\subsubsection{GlobalScope}

{\metaobject GlobalScope} is a {\metaobject Namespace} reflecting the global scope
and requires the following:

\begin{itemize}
	\item \verb@metaobject_traits<GlobalScope>::category@ is defined as
	\verb@global_scope_tag@.
\end{itemize}

\subsubsection{Type}

{\metaobject Type} is a {\metaobject Named} and {\metaobject Scoped} metaobject which
has the following requirements:

\begin{itemize}
	\item{\verb@typedef @{\em original-type}\verb@ original_type;@} member typedef
	defined as the original type reflected by the {\metaobject Type}.

	\item \verb@metaobject_traits<Type>::category@ is defined as \verb@type_tag@.
\end{itemize}

The \verb@is_template@ typedef in \verb@metaobject_traits@ changes the requirements
in the concepts derived from {\metaobject Type}.

\subsubsection{Typedef}

{\metaobject Typedef} is a {\metaobject Type} metaobject that reflects typedefs,
i.e. types that were defined as alternate names for another types.
{\metaobject Typedef} has the following requirements:

\begin{itemize}
	\item{\verb@typedef @{\metaobject Type}\verb@ type;@} member typedef
	defined as the {\metaobject Type} reflecting the "source" type of the typedef.
	In concrete {\metaobject Typedef} metaobjects \verb@type@ can be a 
	{\metaobject Type}, {\metaobject Class}, {\metaobject Enum} or {\metaobject Typedef}.

	\item \verb@metaobject_traits<Typedef>::category@ is defined as \verb@typedef_tag@.
\end{itemize}

\subsubsection{Class}

{\metaobject Class} is a {\metaobject Type} and a {\metaobject Scope} that reflects
an elaborated type (class, struct, union) or a class template.
{\metaobject Class} has the following requirements, but
the \verb@is_template@ typedef in the \verb@metaobject_traits@ changes the requirements
inherited from the {\metaobject Type} concept as described below.

\begin{itemize}
	\item{\verb@typedef @{\metaobject Specifier}\verb@ elaborated_type;@}
	member typedef defined as a {\metaobject Specifier} reflecting the elaborated
	type specifier used to define the class (\verb@class@, \verb@struct@, \verb@union@).

	\item \verb@metaobject_traits<Class>::category@ is defined as \verb@class_tag@.
\end{itemize}

If \verb@metaobject_traits<Class>::is_template@ is \verb@true_type@ it indicates 
that the reflected program feature is not a regular class, but a class template.
In such case the \verb@original_type@ typedef inherited from {\metaobject Type}
is not defined.

\subsubsection{Function}

{\metaobject Function} is a {\metaobject Scope} metaobject that reflects a function
or a function template and requires the following (the requirements are influenced
by the \verb@metaobject_traits<Function>::is_template@ typedef as described below):

\begin{itemize}
	\item{\verb@typedef @{\metaobject Specifier}\verb@ linkage;@} member typedef defined as 
	a {\metaobject Specifier} reflecting the linkage specifier of the function.

	\item{\verb@typedef @{\metaobject Type}\verb@ result_type;@} member typedef defined as
	a {\metaobject Type} reflecting the result type of the function.

	\item{\verb@typedef integral_constant<int,@ {\em number-of-parameters}
	\verb@>@\\\verb@parameter_count;@} member typedef specifying the total number
	of parameters of the function reflected by {\metaobject Function}.

	\item{\verb@static @{\metaobject Parameter}\verb@ parameter(integral_constant<int, @{\em i}
	\verb@>);@} overloaded member function defined
	for $i \in \{0, 1, \dots, n-1\}$; {\em n = number-of-parameters},
	each overload returns a different {\metaobject Parameter} metaobject reflecting the {\em i}-th parameter
	of the function reflected by {\metaobject Function}.

	\item \verb@metaobject_traits<Function>::category@ is defined as \verb@function_tag@.
\end{itemize}

If \verb@metaobject_traits<Function>::is_template@ is defined as \verb@false_type@
i.e. the reflected feature is not a template but a regular function then the following is also
required:

\begin{itemize}
	\item{\verb@static inline ResultType::original_type call(@ {\em parameters\dots}\verb@);@}
	static inline member function with the same return value type and the same number
	and type of parameters as the original function reflected by {\metaobject Function}.
	Calls to this function should be replaced with the call of the reflected function
	with the parameters specified to \verb@call@.
\end{itemize}

If \verb@metaobject_traits<Function>::is_class_member@ is defined as \verb@true_type@
i.e. the reflected is a member function and not a free function or lambda function,
then the following is also required:

\begin{itemize}
	\item{\verb@typedef @ {\metaobject Specifier} \verb@ constness;@} member typedef
	defined as the constness {\metaobject Specifier} reflecting the constness
	of the member functions.
\end{itemize}

{\metaobject Function} metaobjects are {\em not} direct members of scopes. Instead,
all functions with the same name (even those that are not overloaded) in a specific scope
are grouped into a {\metaobject OverloadedFunction}. Individual overloaded functions
in the group can be obtained through the interface of {\metaobject OverloadedFunction}.
The same should also apply to {\metaobject Constructor}s and {\metaobject Operator}s.

The idea is that (direct) scope members (i.e. metaobjects accessible through \verb@Scope::member(...)@)
should have unique names.

The \verb@scope@ typedef of every single {\metaobject Function} in a {\metaobject OverloadedFunction}
is the same as the \verb@scope@ of that {\metaobject OverloadedFunction}, i.e.
the \verb@scope@ of a {\metaobject Function} can be a {\metaobject Namespace} or a {\metaobject Class}
but {\em not} a {\metaobject OverloadedFunction}.

\subsubsection{ClassMember}

{\metaobject ClassMember} is a {\metaobject Named} and {\metaobject Scoped} metaobject
that reflects a member of a class. It has the following requirements:

\begin{itemize}
	\item{\verb@typedef @{\metaobject Specifier}\verb@ access_type;@} member typedef defined as
	a {\metaobject Specifier} reflecting the access type specifier of ther class member
	(\verb@private@, \verb@protected@ or \verb@public@).

	\item \verb@metaobject_traits<ClassMember::scope>::is_class_member@ is \verb@true_type@.
\end{itemize}

Concrete metaobjects that are models of this concept can also be some of the following:
\begin{itemize}
	\item{\metaobject Typedef}
	\item{\metaobject Class}
	\item{\metaobject Enum}
	\item{\metaobject OverloadedFunction}
\end{itemize}

\subsubsection{Constructor}

{\metaobject Constructor} is a {\metaobject ClassMember} and a {\metaobject Function} that
reflects a constructor and requires the following:

\begin{itemize}
	\item \verb@metaobject_traits<Constructor>::category@ is defined as  \verb@constructor_tag@.
\end{itemize}

\subsubsection{Operator}

{\metaobject Operator} is a {\metaobject Function} and possibly a {\metaobject ClassMember}
that reflects an operator and requires the following:

\begin{itemize}
	\item \verb@metaobject_traits<Operator>::category@ is defined as  \verb@operator_tag@.
\end{itemize}

\subsubsection{OverloadedFunction}

{\metaobject OverloadedFunction} is a {\metaobject Function} and possibly a {\metaobject ClassMember}
that reflects a set of overloaded functions, i.e. functions with the same name.
{\metaobject OverloadedFunction} has the following requirements:

\begin{itemize}

	\item{\verb@typedef integral_constant<int,@ {\em number-of-overloads}
	\verb@>@\\\verb@overload_count;@} member typedef specifying the total number
	of parameters of the function reflected by {\metaobject Function}.

	\item{\verb@static @{\metaobject Function}\verb@ overload(integral_constant<int, @{\em i}
	\verb@>);@} overloaded member function defined
	for $i \in \{0, 1, \dots, n-1\}$; {\em n = number-of-overloads},
	each overload returns a different {\metaobject Function} metaobject reflecting the {\em i}-th overload
	in the set of functions reflected by {\metaobject OverloadedFunction}.

	\item \verb@metaobject_traits<OverloadedFunction>::category@ is defined as 
	\verb@overloaded_function_tag@.
\end{itemize}

\subsubsection{Template}

{\metaobject Template} is a {\metaobject Function} or a {\metaobject Class} metaobject
that reflects a function or class template. It has the following requirements:

\begin{itemize}

	\item{\verb@typedef integral_constant<int,@ {\em number-of-template-parameters}
	\verb@>@\\\verb@template_parameter_count;@} member typedef specifying the total number
	of parameters of the template reflected by {\metaobject Template}.

	\item{\verb@static @{\metaobject TemplateParameter}\\\verb@template_parameter(integral_constant<int, @{\em i}
	\verb@>);@} overloaded member function defined
	for $i \in \{0, 1, \dots, n-1\}$; {\em n = number-of-parameters},
	each overload returns a different {\metaobject Parameter} metaobject reflecting the {\em i}-th parameter
	of the template reflected by {\metaobject Template}.

%TODO
% template <parameters...>
% struct instantiation
% {
%     typedef Instantiation type;
% };

	\item \verb@metaobject_traits<Template>::is_template@ is defined as \verb@true_type@.
\end{itemize}

\subsubsection{TemplateParameter}

{\metaobject TemplateParameter} is a {\metaobject Typedef} or a {\metaobject Constant} that
reflects a template parameter. 

\begin{itemize}
	\item \verb@metaobject_traits<TemplateParameter>::is_template@ is defined as \verb@true_type@.
\end{itemize}

\subsubsection{Instantiation}

{\metaobject Instantiation} is a {\metaobject Function} or {\metaobject Class} metaobject
that reflects a templated function or class for which the following is required:

\begin{itemize}

	\item{\verb@typedef @ {\metaobject Template} \verb@model;@} member typedef defined
	as the {\metaobject Template} reflecting the template that the class or function 
	(reflected by an {\metaobject Instantiation}) is an instantiation of.

	\item \verb@metaobject_traits<Instantiation>::has_template@ is defined as \verb@true_type@.
	This trait is used to distinguish classes and functions which are instantiations
	of a template from non-templated classes and functions.
\end{itemize}

\subsubsection{Enum}

{\metaobject Enum} is a {\metaobject Type} and a {\metaobject Scope} that reflects an enumerated
type with the following requirements:

\begin{itemize}
	\item \verb@metaobject_traits<Enum>::category@ is defined as \verb@enum_tag@.

	\item The \verb@members@ of {\metaobject Enum} are only {\metaobject Constant} metaobjects.
\end{itemize}

\subsubsection{Inheritance}

\subsubsection{Variable}

\subsubsection{Constant}

\subsubsection{Parameter}
