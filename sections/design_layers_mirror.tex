\subsubsection{Mirror}

Mirror is a compile-time functional-style reflective programming library,
which is based directly on the basic metadata and is suitable for generic programming,
similar to the standard \verb@type_traits@ library.

Mirror is the original library from which the Mirror reflection utilities started.

It provides a more user-friendly and rich interface than the basic-metaobjects.
and a set of metaprogramming utilities which allow
to write compile-time meta-programs, which can generate efficient
and optimized program code using only those metadata that are required.

The following text contains several (rather simple) examples of usage
and the functional style of the algorithms based on metadata provided by Mirror.

The first example prints some information about the members of selected
namespaces to \verb@std::cout@.

\begin{lstlisting}
struct info_printer
{
    template <typename MetaObject>
    void operator()(MetaObject mo) const
    {
        MIRRORED_META_OBJECT(MetaObject) mmo;
        std::cout
            << mmo.construct_name()
            << ": "
            << mo.full_name()
            << std::endl;
    }
};

int main(void)
{
    using namespace mirror;

    // print the info about each of the members of the global scope
    mirror::mp::for_each<
        members<

            // this should be in standard C++ be replaced by 
            // a specialstandard library function or operator
            MIRRORED_GLOBAL_SCOPE()
        >
    >(info_printer());

    // print the info about each of the members of the std namespace
    mp::for_each<
        members<

            // this should be in standard C++ be replaced by
            // a special standard library function or operator
            MIRRORED_NAMESPACE(std)
        >
    >(info_printer());
    //
    return 0;
}
\end{lstlisting}

This program produces the following output:

\begin{verbatim}
   namespace: std
   namespace: boost
   type: void
   type: bool
   type: char
   type: unsigned char
   type: wchar_t
   type: short int
   type: int
   type: long int
   type: unsigned short int
   type: unsigned int
   type: unsigned long int
   type: float
   type: double
   type: long double
   class: std::string
   class: std::wstring
   class: std::tm
   template: std::pair
   template: std::tuple
   template: std::allocator
   template: std::equal_to
   template: std::not_equal_to
   template: std::less
   template: std::greater
   template: std::less_equal
   template: std::greater_equal
   template: std::vector
   template: std::list
   template: std::deque
   template: std::map
   template: std::set
\end{verbatim}

The next example gets all types in the global scope,
applies some \verb@type_traits@ modifiers like \verb@std::add_pointer@
\verb@std::add_const@ and for each of such modified types calls a functor
that prints the names of the individual types to the standard output:

\begin{lstlisting}
struct name_printer
{
    template <typename MetaNamedObject>
    void operator()(MetaNamedObject mo) const
    {
        std::cout << mo.base_name() << std::endl;
    }
};

int main(void)
{
  using namespace mirror;

  // this function calls the name_printer functor passed
  // as the function argument on each element in the 
  // range that is passed as the template argument
  mp::for_each<

    // this template transforms the elements in the range
    // passed as the first argument by the unary template
    // passed as the second argument
    mp::transform<

      // this template filters out only those metaobjects
      // that satisfy the predicate passed as the second
      // argument from the range of metaobjects passed
      // as the first argument
      mp::only_if<

        // this template "returns" a range of metaobjects
        // reflecting the members of the namespace
        // (or other scope) that is passed as argument
        members<

          // this macro expands into a class
          // conforming to the Mirror's MetaNamespace
          // concept and provides metadata describing
          // the global scope namespace.
          // in the proposed solution for standard C++
          // this should be relaced by a special stdlib
          // function or by an operator.
          MIRRORED_GLOBAL_SCOPE()
        >,

        // this is a lambda function testing if its first
        // argument falls to the MetaType category
        mp::is_a<
          mp::arg<1>,
          meta_type_tag
        >
      >,

      // this is a unary lambda function that modifies
      // the type passed as its argument by
      // the add_pointer and add_const type traits
      apply_modifier<
        mp::arg<1>,
        mp::protect<
          std::add_pointer<
            std::add_const<
              mp::arg<1>
            >
          >
        >
      >
    >
  >(name_printer());
  std::cout << std::endl;
  return 0;
}

\end{lstlisting}

This short program produces the following output:

\begin{verbatim}
   void const *
   bool const *
   char const *
   unsigned char const *
   wchar_t const *
   short int const *
   int const *
   long int const *
   unsigned short int const *
   unsigned int const *
   unsigned long int const *
   float const *
   double const *
   long double const *
\end{verbatim}

The printing of names is definitely not the only usage
of reflection. The scope of this proposal does not allow
to include and fully explain the more elaborated applications.
For some other examples of usage see ~\cite{mirror-doc-mirror-examples}.

