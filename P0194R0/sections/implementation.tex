\section{Experimental implementation}

A fork of the clang compiler with a partial experimental implementation
of this proposal can be found on github \cite{clang-reflexpr-impl}.
The modified compiler can be built by following the instructions listed
in \cite{clang-getting-started}, but instead of checking out the official clang
repository, the sources on the \texttt{mirror-reflection} branch of the modified
repository should be used.

The implementation required changes (mostly additions) to roughly 3500 lines
of code in the compiler plus circa 540 lines in the \texttt{<reflexpr>} header.

All examples listed in Appendix \ref{section-reflexpr-examples} are working with
the modified compiler.

This particular implementation works by generating and adding new \texttt{CXXRecordDecl}s,
with member typedefs and member constants providing basic meta-data, to the AST.
The templates implementing the metaobject operations are referencing and
occasionally transforming the members of the metaobject types.

The \texttt{is\_metaobject} trait is implemented by extending the existing
clang's type-trait framework and adding the \texttt{\_\_is\_metaobject} builtin
operator.

The \meta{ObjectSequence} operations, namely \texttt{get\_size} and
\texttt{get\_element} are implemented lazily by using the new
\texttt{\_\_reflexpr\_size} and \texttt{\_\_reflexpr\_element} operators.

No rigorous measurements were made, but the implementation is not expected to
incur any noticeable overhead in terms of memory footprint or processing
times\footnote{Besides the fact that several new tokens were added.} when
compiling programs, which do not use reflection.

Also note that this specification underwent some last-minute changes,
which were not incorporated into the implementation yet.
