\subsubsection{\texttt{is\_metaobject}}

In order to distinguish between regular types and metaobjects generated
by the compiler, the \texttt{is\_metaobject} trait should be added
directly to the \texttt{std} namespace as one of the type traits. 

\begin{minted}{cpp}
template <typename T>
struct is_metaobject : integral_constant<bool, ...> { };

template <typename T>
constexpr bool is_metaobject_v = is_metaobject<T>::value;
\end{minted}

The expression \texttt{is\_metaobject<X>::value} should be \texttt{true}
if \texttt{X} is a metaobject generated by the compiler, otherwise it should
be \texttt{false}.

The \texttt{is\_metaobject} trait should be defined in the standard
\texttt{<type\_traits>} header.

Several additional trait templates should be defined in the nested namespace
\texttt{std::meta} to provide further information about metaobjects.
These traits are listed below together with their related metaobject concepts
and should be defined in a new
\hyperref[section-reflexpr-header]{\texttt{<reflexpr>} header} file.
