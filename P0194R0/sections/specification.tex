\section{Concept specification}
\label{section-current-Concepts}

We propose that the basic metadata describing a program written
in C++ should be made available through a set of {\em anonymous} types -- {\em metaobjects},
defined by the compiler and through related template classes implementing
the interface of the metaobjects.
At the moment these types should describe only the following program
declarations: namespaces\footnote{in a limited form}, types, typedefs,
classes and their data members and enum types.

In the future, the set of metaobjects should be extended to reflect
class inheritance, free functions, class member functions, templates,
template parameters, enumerated values, possibly the C++ specifiers, etc.

The compiler should generate metadata for the program declarations
in the currently processed translation unit, when requested by the invocation
of the reflection operator. Members of ordered sets (sequences) of metaobjects,
like scope members, parameters of a function, base classes, and so on, should be listed
in the order of appearance in the processed translation unit.

Since we want the metadata to be available at compile-time,
different base-level declarations should be reflected by
{\em statically different} metaobjects and thus by {\em different} types.
For example a metaobject reflecting the global scope namespace should
be a different {\em type} than a metaobject reflecting the \verb@std@
namespace\footnote{this means that they should be distinguishable for
example by the \texttt{std::is\_same} type trait},
a metaobject reflecting the \verb@int@ type should
have a different type then a metaobject reflecting the \verb@double@
type, etc.

This section describes a set of metaobject concepts
and their requirements, traits for metaobject classification and operations
providing the individual bits of meta-data.

Unless stated otherwise, all proposed named templates described below should
go into the \verb@std::meta@ nested namespace in order to contain reflection-related
definitions and to help avoiding potential name conflicts.

\subsection{StringConstant}
\label{concept-StringConstant}

\texttt{StringConstant} represents a compile-time character string type. 

\begin{minted}[tabsize=4]{cpp}
struct StringConstant {
	typedef StringConstant type;
	typedef const char value_type[N+1];
	static constexpr const char value[N+1];

	operator const char* (void) const noexcept;
	const char* operator (void) const noexcept;
};
\end{minted}

This concept could be replaced by the \texttt{basic\_string\_constant}
from N4236.


\subsection{Meta-Object}
\label{concept-Meta-Object}

A \meta{Object} is a stateless anonymous type generated by the compiler
(when requested by the programmer through the 
\hyperref[section-reflection-operator]{\texttt{reflexpr} operator}),
providing metadata reflecting a specific program feature.

\subsubsection{\texttt{is\_metaobject}}

In order to distinguish between regular types and metaobjects generated
by the compiler, the \texttt{is\_metaobject} trait should be added
to the \texttt{std} namespace as one of the type traits. 

\begin{minted}{cpp}
template <typename T>
struct is_metaobject
{
	typedef bool value_type;
	static constexpr const bool value = /*
		true: if T is a metaobject
		false: otherwise
	*/

	typedef integral_constant<bool, value> type;

	operator value_type (void) const noexcept;
	value_type operator(void) const noexcept;
};

template <typename T>
using is_metaobject_t = typename is_metaobject<T>::type;

template <typename T>
constexpr bool is_metaobject_v = is_metaobject<T>::value;
\end{minted}

The expression \texttt{is\_metaobject<X>::value} should evaluate to \texttt{true}
if \texttt{X} is a metaobject generated by the compiler, otherwise it should
be \texttt{false}.

There are also several other trait templates nested in the namespace
\texttt{std::meta} which provide further information about metaobjects.
These traits are listed below together with their related metaobject concepts.



\subsubsection{Definition}

\begin{minted}{cpp}
namespace meta {
\end{minted}
\begin{minted}[xleftmargin=1em,tabsize=4,breakafter=&]{cpp}
template <typename T>
concept bool Object = is_metaobject_v<T>;

\end{minted}
\begin{minted}{cpp}
} // namespace meta
\end{minted}


The following operations are defined for each type satisfying the \meta{Object}
concept.



\subsubsection{\texttt{get\_source\_location}}

returns the source location info of the declaration of a base-level program feature reflected by a \meta{Object}.

\begin{minted}{cpp}
namespace meta {
\end{minted}
\begin{minted}[xleftmargin=1em,tabsize=4]{cpp}
template <Object T>
struct get_source_location : source_location { };

\end{minted}
\begin{minted}{cpp}
} // namespace meta
\end{minted}



The returned instance of \texttt{std::source\_location} should be \texttt{constexpr},
and the source file name and function name strings should be compile-time constants.

Also the source information for built-in types and other
such implicit declarations which are declared internally by the compiler
should return an empty string as a source file path and zero as source file
line and column.

\subsection{Meta-ObjectSequence}
\label{concept-Meta-ObjectSequence}

A \meta{ObjectSequence} is representing an ordered sequence of metaobjects.


\subsubsection{\texttt{is\_sequence}}

The \texttt{meta::is\_sequence}
trait indicates if the \meta{Object} passed as argument is a \meta{ObjectSequence}.


\begin{minted}[tabsize=4]{cpp}
namespace meta {

template <typename T> requires Declaration<T>
struct is_sequence : integral_constant<bool, ... > { };

template <typename T>
constexpr bool is_sequence_v = is_sequence<T>::value;

} // namespace meta
\end{minted}



\subsubsection{Definition}

\begin{minted}{cpp}
namespace meta {
\end{minted}
\begin{minted}[xleftmargin=1em,tabsize=4,breakafter=&]{cpp}

template <typename T>
concept bool ObjectSequence = Object<T> && is_sequence_v<T>;

\end{minted}
\begin{minted}{cpp}
} // namespace meta
\end{minted}




\subsubsection{\texttt{get\_size}}

returns a number of elements in the sequence.

\begin{minted}{cpp}
namespace meta {
\end{minted}
\begin{minted}[xleftmargin=1em,tabsize=4]{cpp}
template <ObjectSequence T>
struct get_size : integral_constant<size_t, ...> { };
template <ObjectSequence T>
constexpr auto get_size_v = get_size<T>::value;
\end{minted}
\begin{minted}{cpp}
} // namespace meta
\end{minted}



\subsubsection{\texttt{get\_element}}

returns the i-th element in a ObjectSequence.

\begin{minted}{cpp}
namespace meta {
\end{minted}
\begin{minted}[xleftmargin=1em,tabsize=4]{cpp}
template <ObjectSequence T1, size_t Index>
struct get_element
{
	typedef /* generated by the compiler */ type;
};
	
template <ObjectSequence T1, size_t Index>
using get_element_t = typename get_element<T1, Index>::type;

\end{minted}
\begin{minted}{cpp}
} // namespace meta
\end{minted}


Note that \texttt{get\_element<...>::type}
must conform to the \meta{Object} concept.



\subsection{Meta-Named}
\label{concept-Meta-Named}

A \meta{Named} is a \meta{Declaration} reflecting a named base-level declaration
(namespace, type, variable, function, etc.).


\subsubsection{\texttt{has\_name}}

The \texttt{meta::has\_name}
trait indicates if the \meta{Declaration} passed as argument is a \meta{Named}.


\begin{minted}[tabsize=4]{cpp}
namespace meta {

template <typename T>
requires Metaobject<T>
struct has_name
{
	typedef bool value_type;
	static constexpr const bool value = /*
		true: if T is a MetaNamed
		false: otherwise
	*/

	typedef integral_constant<bool, value> type;

	operator value_type (void) const noexcept;
	value_type operator(void) const noexcept;
};

template <typename T>
using has_name_t = typename has_name<T>::type;

template <typename T>
constexpr bool has_name_v = has_name<T>::value;

} // namespace meta
\end{minted}



\subsubsection{Definition}

\begin{minted}{cpp}
namespace meta {
\end{minted}
\begin{minted}[xleftmargin=1em,tabsize=4,breakafter=&]{cpp}

template <typename T>
concept bool Named = Object<T> && has_name_v<T>;

\end{minted}
\begin{minted}{cpp}
} // namespace meta
\end{minted}


\subsubsection{\texttt{get\_name}}

returns the basic name of the a named declaration reflected by a \meta{Named}.

\begin{minted}{cpp}
namespace meta {
\end{minted}
\begin{minted}[xleftmargin=1em,tabsize=4]{cpp}
template <Named T>
struct get_name : StringConstant { };
template <Named T>
constexpr auto get_name_v = get_name<T>::value;
\end{minted}
\begin{minted}{cpp}
} // namespace meta
\end{minted}


In case of types, the \texttt{get\_name} operation returns the type (or type alias)
name without any qualifiers, asterisks\footnote{in case of pointers},
ampersands\footnote{in case of references}, angle or square brackets\footnote{
in case of templates or arrays}, etc.

\subsection{Meta-Typed}
\label{concept-Meta-Typed}

A \meta{Typed} is a \meta{Declaration} reflecting base-level declaration with a type
(like a variable).


\subsubsection{\texttt{has\_type}}

The \texttt{meta::has\_type}
trait indicates if the \meta{Object} passed as argument is a \meta{Typed}.


\begin{minted}[tabsize=4]{cpp}
namespace meta {

template <typename T> requires Metaobject<T>
struct has_type {
	typedef integral_constant<bool, value> type;
	typedef bool value_type;
	static constexpr const bool value;

	operator bool(void) const noexcept;
	bool operator(void) const noexcept;
};

template <typename T>
using has_type_t = typename has_type<T>::type;
template <typename T>
constexpr bool has_type_v = has_type<T>::value;

} // namespace meta
\end{minted}



\subsubsection{Definition}

\begin{minted}{cpp}
namespace meta {
\end{minted}
\begin{minted}[xleftmargin=1em,tabsize=4,breakafter=&]{cpp}

template <typename T>
concept bool Typed = Object<T> && has_type_v<T>;

\end{minted}
\begin{minted}{cpp}
} // namespace meta
\end{minted}



\subsubsection{\texttt{get\_type}}

returns the Type reflecting the type of base-level declaration with a type reflected by a \meta{Typed}.

\begin{minted}{cpp}
namespace meta {
\end{minted}
\begin{minted}[xleftmargin=1em,tabsize=4]{cpp}
template <Typed T>
struct get_type
{
	typedef /* generated by the compiler */ type;
};
	
template <Typed T>
using get_type_t = typename get_type<T>::type;

\end{minted}
\begin{minted}{cpp}
} // namespace meta
\end{minted}


Note that \texttt{get\_type<...>::type}
must conform to the \meta{Type} concept.


\subsection{Meta-Scoped}
\label{concept-Meta-Scoped}

A \meta{Scoped} is a \meta{Declaration} reflecting base-level declaration nested
inside of a scope.


\subsubsection{\texttt{has\_scope}}

The \texttt{meta::has\_scope}
trait indicates if the \meta{Declaration} passed as argument is a \meta{Scoped}.


\begin{minted}[tabsize=4]{cpp}
namespace meta {

template <typename T> requires Metaobject<T>
struct has_scope {
	typedef integral_constant<bool, value> type;
	typedef bool value_type;
	static constexpr const bool value;

	operator bool(void) const noexcept;
	bool operator(void) const noexcept;
};

template <typename T>
using has_scope_t = typename has_scope<T>::type;
template <typename T>
constexpr bool has_scope_v = has_scope<T>::value;

} // namespace meta
\end{minted}



\subsubsection{Definition}

\begin{minted}{cpp}
namespace meta {
\end{minted}
\begin{minted}[xleftmargin=1em,tabsize=4,breakafter=&]{cpp}
template <Object T>
concept bool Scoped = has_scope_v<T>;

\end{minted}
\begin{minted}{cpp}
} // namespace meta
\end{minted}


\subsubsection{\texttt{get\_scope}}

returns the Scope reflecting the scope of a scoped declaration reflected by a \meta{Scoped}.

\begin{minted}[tabsize=4]{cpp}

template <typename T>
requires Scoped<T>
struct get_scope
{
	typedef Scope type;
};
	
template <typename T>
using get_scope_t = typename get_scope<T>::type;

\end{minted}


\subsection{Meta-Scope}
\label{concept-Meta-Scope}

A \meta{Scope} is a \meta{Object} which is usually also a \meta{Named}
and possibly a \meta{Scoped} reflecting a scope.


\subsubsection{\texttt{is\_scope}}

The \texttt{meta::is\_scope}
trait indicates if the \meta{Declaration} passed as argument is a \meta{Scope}.


\begin{minted}[tabsize=4]{cpp}
namespace meta {

template <typename T> requires Metaobject<T>
struct is_scope {
	typedef bool value_type;
	static constexpr const bool value;
	typedef integral_constant<bool, value> type;

	operator bool(void) const noexcept;
	bool operator(void) const noexcept;
};

template <typename T>
using is_scope_t = typename is_scope<T>::type;
template <typename T>
constexpr bool is_scope_v = is_scope<T>::value;

} // namespace meta
\end{minted}



\subsubsection{Definition}

\begin{minted}{cpp}
namespace meta {
\end{minted}
\begin{minted}[xleftmargin=1em,tabsize=4,breakafter=&]{cpp}

template <typename T>
concept bool Scope = Object<T> && is_scope_v<T>;

\end{minted}
\begin{minted}{cpp}
} // namespace meta
\end{minted}


\subsection{Meta-Alias}
\label{concept-Meta-Alias}

A \meta{Alias} is a \meta{Named} reflecting a type or namespace alias.  


\subsubsection{\texttt{is\_alias}}

The \texttt{meta::is\_alias}
trait indicates if the \meta{Object} passed as argument is a \meta{Alias}.


\begin{minted}[tabsize=4]{cpp}
namespace meta {

template <typename T>
requires Metaobject<T>
struct is_alias
{
	typedef bool value_type;
	static constexpr const bool value;
	typedef integral_constant<bool, value> type;

	operator value_type (void) const noexcept;
	value_type operator(void) const noexcept;
};

template <typename T>
using is_alias_t = typename is_alias<T>::type;

template <typename T>
constexpr bool is_alias_v = is_alias<T>::value;

} // namespace meta
\end{minted}



\subsubsection{Definition}

\begin{minted}{cpp}
namespace meta {
\end{minted}
\begin{minted}[xleftmargin=1em,tabsize=4,breakafter=&]{cpp}
template <Object T>
concept bool Alias = Named<T> && is_alias_v<T>;

\end{minted}
\begin{minted}{cpp}
} // namespace meta
\end{minted}


\subsubsection{\texttt{get\_aliased}}

returns the Named reflecting the original declaration of a type or namespace alias reflected by a \meta{Alias}.

\begin{minted}{cpp}
namespace meta {
\end{minted}
\begin{minted}[xleftmargin=1em,tabsize=4]{cpp}
template <Alias T>
struct get_aliased
{
	typedef /* generated by the compiler */ type;
	static_assert(Named<type>, "");
};
	
template <Alias T>
constexpr auto get_aliased_v = get_aliased<T>::value;
\end{minted}
\begin{minted}{cpp}
} // namespace meta
\end{minted}


\subsection{Meta-Linkable}
\label{concept-Meta-Linkable}

A \meta{Linkable} is a \meta{Named} and a \meta{Scoped} reflecting declaration with storage duration and/or linkage. 


\subsubsection{\texttt{is\_linkable}}

The \texttt{meta::is\_linkable}
trait indicates if the \meta{Declaration} passed as argument is a \meta{Linkable}.


\begin{minted}[tabsize=4]{cpp}
namespace meta {

template <typename T> requires Metaobject<T>
struct is_linkable {
	typedef bool value_type;
	static constexpr const bool value;
	typedef integral_constant<bool, value> type;

	operator bool(void) const noexcept;
	bool operator(void) const noexcept;
};

template <typename T>
using is_linkable_t = typename is_linkable<T>::type;
template <typename T>
constexpr bool is_linkable_v = is_linkable<T>::value;

} // namespace meta
\end{minted}



\subsubsection{Definition}

\begin{minted}{cpp}
namespace meta {
\end{minted}
\begin{minted}[xleftmargin=1em,tabsize=4,breakafter=&]{cpp}

template <typename T>
concept bool Linkable = Named<T> && Scoped<T> && is_linkable_v<T>;

\end{minted}
\begin{minted}{cpp}
} // namespace meta
\end{minted}



\subsubsection{\texttt{is\_static}}

returns whether the declaration with storage duration and/or linkage reflected by a \meta{Linkable} was declared with the static specifier.

\begin{minted}{cpp}
namespace meta {
\end{minted}
\begin{minted}[xleftmargin=1em,tabsize=4]{cpp}
template <Linkable T>
struct is_static : integral_constant<bool, ...> { };
template <Linkable T>
constexpr auto is_static_v = is_static<T>::value;
\end{minted}
\begin{minted}{cpp}
} // namespace meta
\end{minted}



\subsection{Meta-ClassMember}
\label{concept-Meta-ClassMember}

A \meta{ClassMember} is a \meta{Scoped} reflecting a class member.


\subsubsection{\texttt{is\_class\_member}}

The \texttt{meta::is\_class\_member}
trait indicates if the \meta{Declaration} passed as argument is a \meta{ClassMember}.


\begin{minted}[tabsize=4]{cpp}
namespace meta {

template <typename T> requires Declaration<T>
struct is_class_member : integral_constant<bool, ... > { };

template <typename T>
constexpr bool is_class_member_v = is_class_member<T>::value;

} // namespace meta
\end{minted}



\subsubsection{Definition}

\begin{minted}{cpp}
namespace meta {
\end{minted}
\begin{minted}[xleftmargin=1em,tabsize=4,breakafter=&]{cpp}

template <typename T>
concept bool ClassMember = Scoped<T> && is_class_member_v<T> && is_class_v<get_scope_t<T>>;

\end{minted}
\begin{minted}{cpp}
} // namespace meta
\end{minted}


\subsubsection{\texttt{is\_public}}

returns whether the class member reflected by a \meta{ClassMember} was declared with public access.

\begin{minted}[tabsize=4]{cpp}

template <typename T>
requires ClassMember<T>
struct is_public : integral_constant<bool, ...> { };
template <typename T>
using is_public_t = typename is_public<T>::type;

\end{minted}



\subsection{Meta-GlobalScope}
\label{concept-Meta-GlobalScope}

A \meta{GlobalScope} is a \meta{Scope} reflecting the global scope.


\subsubsection{\texttt{is\_global\_scope}}

The \texttt{meta::is\_global\_scope}
trait indicates if the \meta{Declaration} passed as argument is a \meta{GlobalScope}.


\begin{minted}[tabsize=4]{cpp}
namespace meta {

template <typename T> requires Metaobject<T>
struct is_global_scope {
	typedef bool value_type;
	static constexpr const bool value;
	typedef integral_constant<bool, value> type;

	operator bool(void) const noexcept;
	bool operator(void) const noexcept;
};

template <typename T>
using is_global_scope_t = typename is_global_scope<T>::type;
template <typename T>
constexpr bool is_global_scope_v = is_global_scope<T>::value;

} // namespace meta
\end{minted}



\subsubsection{Definition}

\begin{minted}[tabsize=8]{cpp}
namespace meta {

template <typename T>
concept bool GlobalScope =
	Scope<T> &&
	meta::is_global_scope_v<T>;

} // namespace meta
\end{minted}


\subsection{Meta-Namespace}
\label{concept-Meta-Namespace}

A \meta{Namespace} is a \meta{Named}, a \meta{Scoped} and a \meta{Scope}
reflecting a namespace.


\subsubsection{\texttt{is\_namespace}}

The \texttt{meta::is\_namespace}
trait indicates if the \meta{Declaration} passed as argument is a \meta{Namespace}.


\begin{minted}[tabsize=4]{cpp}
namespace meta {

template <typename T> requires Metaobject<T>
struct is_namespace {
	typedef integral_constant<bool, value> type;
	typedef bool value_type;
	static constexpr const bool value;

	operator bool(void) const noexcept;
	bool operator(void) const noexcept;
};

template <typename T>
using is_namespace_t = typename is_namespace<T>::type;
template <typename T>
constexpr bool is_namespace_v = is_namespace<T>::value;

} // namespace meta
\end{minted}



\subsubsection{Definition}

\begin{minted}{cpp}
namespace meta {
\end{minted}
\begin{minted}[xleftmargin=1em,tabsize=4,breakafter=&]{cpp}
template <Object T>
concept bool Namespace = Named<T> && Scope<T> && Scoped<T> && is_namespace_v<T>;

\end{minted}
\begin{minted}{cpp}
} // namespace meta
\end{minted}


\subsection{Meta-NamespaceAlias}
\label{concept-Meta-NamespaceAlias}

A \meta{NamespaceAlias} is a \meta{Namespace} and a \meta{Alias}
reflecting a namespace alias.


\subsubsection{Definition}

\begin{minted}[tabsize=8]{cpp}
namespace meta {

template <typename T>
concept bool NamespaceAlias =
	Namespace<T> &&
	Alias<T> &&
	meta::is_namespace_v<get_aliased_t<T>>;

} // namespace meta
\end{minted}



\subsection{Meta-Type}
\label{concept-Meta-Type}

A \meta{Type} is a \meta{Named} and a \meta{Scoped}
reflecting a type.


\subsubsection{\texttt{is\_type}}

The \texttt{meta::is\_type}
trait indicates if the \meta{Object} passed as argument is a \meta{Type}.


\begin{minted}[tabsize=4]{cpp}
namespace meta {

template <typename T> requires Metaobject<T>
struct is_type {
	typedef bool value_type;
	static constexpr const bool value;
	typedef integral_constant<bool, value> type;

	operator bool(void) const noexcept;
	bool operator(void) const noexcept;
};

template <typename T>
using is_type_t = typename is_type<T>::type;
template <typename T>
constexpr bool is_type_v = is_type<T>::value;

} // namespace meta
\end{minted}



\subsubsection{Definition}

\begin{minted}[tabsize=8]{cpp}
namespace meta {

template <typename T>
concept bool Type =
	Named<T> &&
	Scoped<T> &&
	meta::is_type_v<T>;

} // namespace meta
\end{minted}



\subsubsection{\texttt{get\_reflected\_type}}

returns the the base-level type reflected by a \meta{Type}.

\begin{minted}{cpp}
namespace meta {
\end{minted}
\begin{minted}[xleftmargin=1em,tabsize=4]{cpp}
template <Type T>
struct get_reflected_type
{
	typedef /* generated by the compiler */ type;
};
	
template <Type T>
constexpr auto get_reflected_type_v = get_reflected_type<T>::value;
\end{minted}
\begin{minted}{cpp}
} // namespace meta
\end{minted}



\subsection{Meta-TypeAlias}
\label{concept-Meta-TypeAlias}

A \meta{TypeAlias} is a \meta{Type} and a \meta{Alias}
reflecting a typedef, a type alias or a substituted template type parameter.


\subsubsection{Definition}

\begin{minted}{cpp}
namespace meta {
\end{minted}
\begin{minted}[xleftmargin=1em,tabsize=4,breakafter=&]{cpp}

template <typename T>
concept bool TypeAlias = Type<T> && Alias<T> && is_type_v<get_aliased_t<T>>;

\end{minted}
\begin{minted}{cpp}
} // namespace meta
\end{minted}


If the reflected type alias or typedef refers to a class then the reflecting
\meta{TypeAlias} is also a \meta{Class}, if the reflected alias refers
to an enum then the \meta{TypeAlias} is also a \meta{Enum}.

\subsection{Meta-Class}
\label{concept-Meta-Class}

A \meta{Class} is a \meta{Type} and a \meta{Scope}
reflecting a class, struct or union.


\subsubsection{\texttt{is\_class}}

The \texttt{meta::is\_class}
trait indicates if the \meta{Object} passed as argument is a \meta{Class}.


\begin{minted}[tabsize=4]{cpp}
namespace meta {

template <typename T>
requires Metaobject<T>
struct is_class
{
	typedef bool value_type;
	static constexpr const bool value;
	typedef integral_constant<bool, value> type;

	operator value_type (void) const noexcept;
	value_type operator(void) const noexcept;
};

template <typename T>
using is_class_t = typename is_class<T>::type;

template <typename T>
constexpr bool is_class_v = is_class<T>::value;

} // namespace meta
\end{minted}



\subsubsection{Definition}

\begin{minted}{cpp}
namespace meta {
\end{minted}
\begin{minted}[xleftmargin=1em,tabsize=4,breakafter=&]{cpp}

template <typename T>
concept bool Class = Type<T> && Scope<T> && is_class_v<T>;

\end{minted}
\begin{minted}{cpp}
} // namespace meta
\end{minted}


\subsubsection{\texttt{get\_data\_members}}

returns a sequence of objects reflecting the public data members of a class reflected by a \meta{Class}.

\begin{minted}[tabsize=4]{cpp}

template <typename T>
requires Class<T>
struct get_data_members
{
	typedef DeclarationSequence type;
};
	
template <typename T>
using get_data_members_t = typename get_data_members<T>::type;

\end{minted}

\subsubsection{\texttt{get\_all\_data\_members}}

returns a sequence of objects reflecting all    (including the private and protected)   data members of a class reflected by a \meta{Class}.

\begin{minted}[tabsize=4]{cpp}

template <typename T>
requires Class<T>
struct get_all_data_members
{
	typedef DeclarationSequence type;
};
	
template <typename T>
using get_all_data_members_t = typename get_all_data_members<T>::type;

\end{minted}


The \meta{DeclarationSequence} returned by \texttt{get\_data\_members} and
\texttt{get\_all\_data\_members} should \emph{not} include metaobjects reflecting
inherited data members.

\subsection{Meta-Enum}
\label{concept-Meta-Enum}

A \meta{Enum} is a \meta{Type} and possibly also a \meta{Scope}
reflecting an enum or a scoped enum.


\subsubsection{\texttt{is\_enum}}

The \texttt{meta::is\_enum}
trait indicates if the \meta{Object} passed as argument is a \meta{Enum}.


\begin{minted}[tabsize=4]{cpp}
namespace meta {

template <typename T> requires Metaobject<T>
struct is_enum {
	typedef integral_constant<bool, value> type;
	typedef bool value_type;
	static constexpr const bool value;

	operator bool(void) const noexcept;
	bool operator(void) const noexcept;
};

template <typename T>
using is_enum_t = typename is_enum<T>::type;
template <typename T>
constexpr bool is_enum_v = is_enum<T>::value;

} // namespace meta
\end{minted}



\subsubsection{Definition}

\begin{minted}{cpp}
namespace meta {
\end{minted}
\begin{minted}[xleftmargin=1em,tabsize=4,breakafter=&]{cpp}

template <typename T>
concept bool Enum = Type<T> && is_enum_v<T>;

\end{minted}
\begin{minted}{cpp}
} // namespace meta
\end{minted}


\subsection{Meta-EnumClass}
\label{concept-Meta-EnumClass}

A \meta{EnumClass} is a \meta{Enum} and a \meta{Scope}
reflecting a scoped, strongly-typed enumeration.


\subsubsection{Definition}

\begin{minted}[tabsize=8]{cpp}
namespace meta {

template <typename T>
concept bool EnumClass =
	Enum<T> &&
	Scope<T>;

} // namespace meta
\end{minted}


\subsection{Meta-Variable}
\label{concept-Meta-Variable}

A \meta{Variable} is a \meta{Named}, a \meta{Typed} and a \meta{Linkable}
reflecting a variable\footnote{At the moment only class data members fall
into this category, but variable reflection should be introduced in a future
proposal}.


\subsubsection{\texttt{is\_variable}}

The \texttt{meta::is\_variable}
trait indicates if the \meta{Declaration} passed as argument is a \meta{Variable}.


\begin{minted}{cpp}
namespace meta {
\end{minted}
\begin{minted}[xleftmargin=1em,tabsize=4]{cpp}

template <Object T>
struct is_variable : integral_constant<bool, ... > { };
template <Object T>
constexpr bool is_variable_v = is_variable<T>::value;

\end{minted}
\begin{minted}{cpp}
} // namespace meta
\end{minted}



\subsubsection{Definition}

\begin{minted}{cpp}
namespace meta {
\end{minted}
\begin{minted}[xleftmargin=1em,tabsize=4,breakafter=&]{cpp}

template <typename T>
concept bool Variable = Named<T> && Typed<T> && Linkable<T> && is_variable_v<T>;

\end{minted}
\begin{minted}{cpp}
} // namespace meta
\end{minted}


\subsubsection{\texttt{get\_pointer}}

returns a pointer to the a variable reflected by a \meta{Variable}.   If the variable is a class member then the pointer is a class data member pointer,   otherwise it is a plain pointer.

\begin{minted}[tabsize=4]{cpp}

template <typename T>
requires Variable<T>
struct get_pointer
{
	typedef conditional_t<
		is_class_member_v<T> && !is_static_v<T>,
		get_reflected_type_t<get_type_t<T>>
		get_reflected_type_t<get_scope_t<T>>::*,
		get_reflected_type_t<get_type_t<T>>*
	> value_type;

	static const value_type value;
};
	
template <typename T>
const auto get_pointer_v = get_pointer<T>::value;

\end{minted}


\subsection{Meta-DataMember}
\label{concept-Meta-DataMember}

A \meta{DataMember} is a \meta{Variable} and a \meta{ClassMember}
reflecting a class data member.


\subsubsection{Definition}

\begin{minted}[tabsize=8]{cpp}
namespace meta {

template <typename T>
concept bool DataMember =
	Variable<T> &&
	ClassMember<T>;

} // namespace meta
\end{minted}


