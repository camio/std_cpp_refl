\section{Issues}

\subsection{Alternatives for breaking access restrictions}
\label{issue-breaking-access}

\textbf{Q:} {\em Some people were objecting to reflection being able to break
access restrictions. Would this capability be more acceptable if it had to be
explicitly allowed or if it could be denied?}

Breaking class member access restrictions is important for many use cases,
like non-intrusive serialization of third-party classes, which may include
the serialization of non-public data-members.
But there may be situations where the programmer would like to prevent
reflection from having access to non-public members.

This may be achieved either by annotating such members with an attribute;

\begin{minted}[tabsize=4]{cpp}
class very_private
{
private:
	[[nonreflectable]] int _really_secret;
public:
	/* ... */
};
\end{minted}

or by explicitly allowing access to reflection, for example by extending
the \verb@friend@ mechanism:

\begin{minted}[tabsize=4]{cpp}
class not_so_private
{
private:
	friend reflexpr;
	int _public_secret;
public:
	/* ... */
};
\end{minted}

If this is deemed necessary, then we favor the first option.

Another option is to explicitly ask permission when accessing class members,
which are considered private or protected in a particular context.
This can be done either by using a new variant of the access operation
like \verb@get_private_pointer@ and \verb@get_protected_pointer@
vs. \verb@get_pointer@:

\begin{minted}[tabsize=4]{cpp}
class mycls
{
private:
	typedef int _my_t;
	_my_t _secret;
public:
	typedef value_type;
	value_type value ;
	/* ... */
};

using meta_mycls = reflexpr(meta_mycls);
using meta_mycls_dm = meta::get_all_data_members_m<meta_mycls>;
using meta_mycls_mt = meta::get_all_member_types_m<meta_mycls>;

using meta::get_element_m;
using meta::get_pointer_t;
using meta::get_private_pointer_t;

get_private_pointer_v<get_element_m<meta_mycls_dm, 0>>; // OK, asking nicely
get_pointer_v<get_element_m<meta_mycls_dm, 0>>; // error, member is private
get_pointer_v<get_element_m<meta_mycls_dm, 1>>; // OK, member is public

using meta::get_reflected_type_t;
using meta::get_private_reflected_type_t;

get_private_reflected_type_t<get_element_m<meta_mycls_mt, 0>>; // OK, asking nicely
get_reflected_type_t<get_element_m<meta_mycls_mt, 0>>; // error, member is private
get_reflected_type_t<get_element_m<meta_mycls_mt, 1>>; // OK, member is public
\end{minted}

However this approach would not scale very well because it requires three\footnote
{or at least two} variants of all such operations.

A better alternative to the above might be to use a special variant of the
\hyperref[fut-reverse-reflection]{reverse reflection operator} like
\verb@reflexpr(private, Metaobject)@ and \verb@reflexpr(protected, Metaobject)@\\
vs. \verb@reflexpr(Metaobject)@:

\begin{minted}[tabsize=4]{cpp}
class mycls
{
private:
	typedef int _my_t;
	_my_t _secret;
public:
	typedef value_type;
	value_type value ;
	/* ... */
};

using meta_mycls = reflexpr(meta_mycls);
using meta_mycls_dm = meta::get_all_data_members_m<meta_mycls>;
using meta_mycls_mt = meta::get_all_member_types_m<meta_mycls>;

// un-reflect the data members
reflexpr(private, get_element_m<meta_mycls_dm, 0>); // OK, asking nicely
reflexpr(get_element_m<meta_mycls_dm, 0>); // error, member is private
reflexpr(get_element_m<meta_mycls_dm, 1>); // OK, member is public

// un-reflect the member types
reflexpr(private, get_element_m<meta_mycls_mt, 0>); // OK, asking nicely
reflexpr(get_element_m<meta_mycls_mt, 0>); // error, member is private
reflexpr(get_element_m<meta_mycls_mt, 1>); // OK, member is public
\end{minted}

\subsection{Interaction with attributes}

\textbf{Q:} {\em Should we even reflect attributes? If so, how will the reflection
of generalized attributes look like?}

Currently generalized attributes are used {\em mostly} as hints to the compiler.
They can help the compiler to do some optimizations\footnote{for example the
\texttt{probably(true)} or \texttt{carries\_dependency} attributes} or to
indicate that the user really means to do something what in other circumstances
could be a bug\footnote{like the \texttt{fallthrough} or \texttt{maybe\_unused}
attributes} that the compiler warns about, etc.

Some people argue that generalized attributes are not supposed to have any
semantic effects -- a compiler should be able to completely strip 
a program of all attributes and it should not have any effects on the behavior
of the program.

Attributes could however be viewed more broadly as a generic mechanism for
annotating declarations\footnote{Associating one or more constant values with
the declarations.} in the code, without explicitly saying that the recipient
can only be the compiler.
There are valid use cases for user annotations and since the annotations
usually provide additional conceptual (meta-)information about a declaration,
reflection is a natural mechanism for accessing the annotations.

We favor the view that on the base-level the compiler should be able to
strip attributes\footnote{or to ignore attributes which it does not understand},
but it should reflect them on the meta-level on demand.

Having covered that, the reflection of the \say{simple} attributes like
\verb@[[attr1]]@ or \verb@[[namespace::attr2]]@ could be pretty straightforward:

We add a \meta{Attribute} concept and its trait, say \verb@meta::is_attribute@,
make it conform to \meta{Named} and \meta{Scoped} concepts and then add an
operation to the interface of \meta{Object} like \verb@get_attributes@, returning
a sequence of metaobjects reflecting the attributes on a base-level declaration.

The complication is that attributes can have nearly arbitrary parameters;
\verb@[[probably(true)]]@, \verb@[[deprecated("reason")]]@,
\verb@[[visibility(hidden)]]@, \verb@[[gnu::aligned(64)]]@. The reflection of
such attributes and their arguments is currently unresolved.

\subsection{Interaction with concepts}

\textbf{Q:} {\em Are there any special interactions between compile-time
reflection and concepts?}

Unresolved.

\subsection{Interaction with modules}

\textbf{Q:} {\em Are there any special interactions between compile-time
reflection and modules?}

Unresolved.

