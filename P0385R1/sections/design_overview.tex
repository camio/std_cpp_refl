\subsection{Basic overview}

As the introduction briefly mentions, the metadata reflecting base-level
program declarations has its own structure. One way to maintain this structure
and to organize the individual, but related pieces of metadata reflecting
for example the structure of a class is to compose them into {\em metaobjects}.

In P0194R2 we propose to add support for
compile-time reflection to C++ by the means of lightweight, compiler-generated
types -- {\em metaobjects}, providing \hyperref[term-metadata]{metadata}
describing various \hyperref[term-base-meta-level]{base-level} program declarations.

The Metaobjects themselves are opaque, without any visible internal structure.
Values of such a Metaobject type are default- and copy-constructible.

Their primary purpose is to give a \hyperref[term-first-class]{first-class identity}
to the reflected entity\footnote{Namespace, type alias, function, parameter,
specifier, etc.}, so that we can pass it as an argument or a return value
in metaprograms and to separate the reflection of a declaration from the querying
of metadata\footnote {Which will happen very often in the more complex use cases}.

We introduce a new reflection operator -- \verb@reflexpr@ which returns a
metaobject type reflecting its operand.

For example:

\begin{minted}{cpp}
typedef reflexpr() meta_global_scope;
typedef reflexpr(int) meta_int;
typedef reflexpr(std) meta_std;
typedef reflexpr(std::size_t) meta_std_size_t;
typedef reflexpr(std::thread) meta_std_thread;
typedef reflexpr(std::pair) meta_std_pair;
typedef reflexpr(std::launch) meta_std_launch;
typedef reflexpr(std::launch::async) meta_std_launch_async;
typedef reflexpr(static) meta_static;
typedef reflexpr(public) meta_public;
\end{minted}

Since there are many different kinds of
base-level reflectable declarations, the metaobjects reflecting them are
modeling various {\em metaobject concepts}, which also serve to classify
metaobjects and to indicate whether a metaobjects has or has not a particular
property;

\begin{minted}{cpp}
static_assert(meta::Named<reflexpr(std)>, "");
static_assert(meta::ScopeMember<reflexpr(int)>, "");
static_assert(meta::Scope<reflexpr(std::string)>, "");
static_assert(meta::Alias<reflexpr(std::string::size_type)>, "");
\end{minted}

or if is falls into a particular category:

\begin{minted}{cpp}
static_assert(meta::GlobalScope<reflexpr()>, "");
static_assert(meta::Namespace<reflexpr(std)>, "");
static_assert(meta::Type<reflexpr(int)>, "");
static_assert(meta::Class<reflexpr(std::string)>, "");
static_assert(meta::Specifier<reflexpr(virtual)>, "");
\end{minted}

The individual pieces of metadata can be obtained from a metaobject by using one
of the class templates which comprise its interface.

Some of this metadata like the class name or number of base classes is provided
as compile-time constant values, some as base-level types, like the original
type of the class and some in the form of other metaobjects, like the metaobject
reflecting the scope, the sequence of metaobjects reflecting the members, etc.:

\begin{minted}{cpp}
using meta_str = reflexpr(std::string);

get_base_name_v<meta_str>; // a compile-time constant string
get_reflected_type_t<meta_str>; // a base-level type: std::string
get_scope_m<meta_str>; // another metaobject reflecting the scope
\end{minted}

P0194R2 also defines the initial subset
of metaobject concepts which we assume to be essential
and which will provide a good starting point for future extensions.

