\subsection{Concept transformation}
\label{section-concept-rendering}

The concepts described above need to be transformed into concrete
C++ code in order to be usable to the programmer and there are several
possible sets of transformation rules, with various advantages and
disadvanateges.

Please see the appendix~\ref{appendix-concept-renderings} for concrete
examples of several possible transformations.

Consider a \emph{Concept} with the following requirements:

\begin{itemize}

\item {\em instance:}  \verb@some_instance@ -- Instance of a tag concept.

\item {\em attribute:}  \verb@Constant@ \verb@constant_attrib@-- Attribute specifying a compile-time constant value.

\item {\em attribute:}  \verb@Type@ \verb@type_attrib@ -- Attribute specifying a base-level type.

\item {\em attribute:}  \verb@Concept@ \verb@concept_attrib@ -- Attributes specifying a metaobject conforming to a concept.

\item {\em trait:}  \verb@bool@ \verb@some_trait@ -- Boolean trait.

\item {\em element:}  \verb@Element@ \verb@element(position)@ -- Unary random access getter for indexed attributes.

\item {\em template:}  \verb@some_template(X)@ -- Class template which when instantiated results in a concrete type.

\item {\em function:}  \verb@result_type@ \verb@some_function(@$\dots$\verb@)@ -- N-ary function.

\end{itemize}

This logical concept could be transformed into usable C++ as:

\begin{itemize}
\item an anonymous structure with member typedefs, functions, constant values, etc.:
\begin{minted}{cpp}
// definition
struct Concept
{
	static constexpr size_t constant_attrib = /*...*/;

	typedef /*...*/ type_attrib;

	static constexpr AnotherConcept concept_attrib(void);

	static constexpr bool some_trait = /*...*/;

	static Element element(integral_constant<size_t, /*...*/>);

	template <typename X>
	struct some_template { /*...*/ };

	static result_type some_function(/*...*/);
};

// usage
auto metaobject = /* use reflection to get an instance of Concept */
size_t x = metaobject.constant_attrib;
auto y = metaobject.concept_attrib();
// etc.
\end{minted}

\item an anonymous type (with its own identity so that it could be used
in function overloads or template specializations) without any internal structure,
where the related attributes, etc. would be accessed by free functions or
by the means of templates, for example:
\begin{minted}{cpp}
// by function
// declared as
constexpr size_t constant_attrib(Concept);
constexpr AnotherConcept concept_attrib(Concept);
//
// used as
auto metaobject = /* use reflection to get an instance of Concept */
size_t x = constant_attrib(metaobject);
AnotherConcept y = concept_attrib(metaobject);
result_type some_function(Concept, /*...*/);

// or by template
// defined as
template <typename Concept>
struct constant_attr
{
	static constexpr size_t value = /*...*/;
};
template <typename Concept>
struct concept_attrib
{
	typedef AnotherConcept type;
};
template <typename Concept, typename X>
struct some_template
{
	/*...*/
};
template <typename Concept>
result_type some_function(/*...*/);

// used as
typedef /* get metaobject */ metaobject;
size_t x = constant_attrib<metaobject>::value;
typedef concept_attrib<metaobject>::type y;
\end{minted}

\end{itemize}

The advantage of the first approach is that it does not "pollute"
the \verb@std@ (nor any other namespace) with the \emph{getter} functions or templates.
Everything necessary to get the metadata from the metaobject is contained
in the metaobject itself.

The advantage of the second approach is that the getters for a particular
metaobject do not need to be defined all-in-one, which may simplify
resolution of circular-dependencies.

\subsubsection{Resolving conflicts with C++ keywords}
\label{section-resolving-keyword-conflicts}

In the concepts defined above, several requirements like instances, attributes, etc.
have names which if transformed directly to C++ would conflict with C++ keywords.

There are several ways to resolve these conflicts:
\begin{itemize}
\item Changing the letter case or naming convention for example to \verb@ALL_CAPS@, \verb@CamelCase@, etc. --
this approach is however inconsistent with the naming convention adopted by the standard
library.

\item Applying a prefix and/or postfix -- the prefixes/suffixes can be meaningful or just underscores,
for which there is a precedent in the standard - the placeholders.

\item Changing the name to a synonym -- this can however lead to confusion.

\end{itemize}

\subsection{Suggested transformation}

We suggest the following set of rules to transform the individual requirement
kinds to C++ (see appendix~\ref{appendix-preferred-concept-rendering} for
details). 

Concrete metaobjects should be anonymous structures with the attributes,
instances, elements, templates, etc. defined as members.
If the name is in conflict with a C++ keyword it should be postfixed
with an underscore, unless stated otherwise.

\subsubsection{Instances of tag concepts}

Tag concepts define a set of "instances" tagging other metaobjects.
The instances are intended to be used for metaobject classification and
tag dispatching of function overloads and template specializations --
i.e. they must have their own identity.

Tags can be either types or compile-time constant values, for example
the values of a strongly typed enumeration.

We suggest that the instances of the tag concepts are transformed
into tag types and in order to avoid conflicts with keywords
they should be prefixed with \verb@meta_@ (in case of {\metaobject{MetaobjectCategory}})
and \verb@spec_@ (in case of {\metaobject{SpecifierCategory}}) and suffixed with
\verb@_tag@.

For example the {\em namespace} instance of {\metaobject{MetaobjectCategory}}
should be transformed into:
\begin{minted}{cpp}
namespace std {
struct meta_namespace_tag { };
} // namespace std
\end{minted}
and the {\em static} instance of {\metaobject{SpecifierCategory}} should be
transformed into:
\begin{minted}{cpp}
namespace std {
struct spec_static_tag { };
} // namespace std
\end{minted}

\subsubsection{Traits}

Concept traits should be transformed to templates similar to \verb@type_traits@.
For example the {\em is\_template} trait of {\metaobject{Metaobject}}
should be transformed:
\begin{minted}{cpp}
struct _unspecified_instance_of_Metaobject { };

template <typename Metaobject>
struct is_template
 : integral_constant<bool, true-or-false>
{ };
\end{minted}

\subsubsection{Constant value attributes}

Attributes specifying a constant value should be transformed to
static constexpr data members.
For example the {\em size} attribute of {\metaobject{String}}
should be transformed into:
\begin{minted}{cpp}
struct _unspecified_instance_of_String
{
	static constexpr size_t size = /*...*/;
	/*...*/
};
\end{minted}

\subsubsection{Type attributes}

Attributes specifying a (base-level not meta-level) type should be
transformed to member typedefs.
For example the {\em original\_type} attribute of {\metaobject{Type}}
should be transformed to:
\begin{minted}{cpp}
struct _unspecified_instance_of_Type
{
	typedef unspecified-type original_type;
	/*...*/
};
\end{minted}

\subsubsection{Concept attributes}

Attributes specifying a (meta-level) type conforming to a Concept should be
transformed to static constexpr member functions.
For example the {\em scope} attribute of {\metaobject{Scoped}}
should be transformed to:
\begin{minted}{cpp}
struct _unspecified_instance_of_Scoped
{
	static constexpr Scope scope(void);
	/*...*/
};
\end{minted}
The {\em template} attribute of {\metaobject{Instantiation}} that is in conflict
with a reserved identifier should be transformed to:
\begin{minted}{cpp}
struct _unspecified_instance_of_Instantiation
{
	static constexpr Template template_(void);
	/*...*/
};
\end{minted}

There are cicrular dependencies in the metaobjects and using functions
can help to avoid problems when defining the metaobjects.

\subsubsection{Range elements}

Indexed attributes specifying the i-th element in a {\metaobject{Range}}
should be transformed to a set of static constexpr functions overloaded
for every element in the range. The \verb@integral_constant@ template
is used at the parameter distinguishing the overloads.
For example the {\em at} element of a {\metaobject{Range}} should
be transformed to:
\begin{minted}{cpp}
struct _unspecified_instance_of_Range<Element>
{
	// overloaded for i in (0, 1, ..., N-1) where
	// N is the value of Range::size
	static constexpr Element at(integral_constant<size_t, i>);
	/*...*/
};
\end{minted}

\subsubsection{Templates}

Class templates should be transformed to nested member templates.
For example the {\em named\_typedef} template in {\metaobject{NamedScoped}}
should be transformed to:
\begin{minted}{cpp}
struct _unspecified_instance_of_NamedScoped
{
	template <typename X>
	struct named_typedef
	{
		/*...*/
	};
	/*...*/
};
\end{minted}

\subsubsection{Functions}

Functions should be transformed to static member functions.
For example the {\em call} function in {\metaobject{Function}}
should be transformed into:
\begin{minted}{cpp}
struct _unspecified_instance_of_Function
{
	static result_type call(/*same as the reflected function*/);
	/*...*/
};
\end{minted}

