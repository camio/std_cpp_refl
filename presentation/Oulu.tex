\documentclass[compress,table,xcolor=table]{beamer}
\mode<presentation>
\usetheme{Warsaw}
\hypersetup{pdfstartview={Fit}}

\usecolortheme{crane}
\useoutertheme{smoothbars}

\setbeamertemplate{bibliography item}[text]
\logo{%
  \usebeamerfont{footline}%
  \usebeamercolor[fg]{footline}%
  \hspace{1em}%
  \small{\insertframenumber/}\scriptsize{\inserttotalframenumber}
}

\definecolor{alert-color}{rgb}{0.40, 0.15, 0.50}
\setbeamercolor{alerted text}{fg=alert-color}

\usepackage[utf8]{inputenc}
\usepackage{caption}
\usepackage{enumitem}
\usepackage{dirtytalk}
\usepackage{multicol}

\definecolor{listing-bgcolor}{rgb}{0.85, 0.85, 1.00}

\usepackage{listings}
\lstset{
  backgroundcolor = \color{listing-bgcolor},
  basicstyle=\scriptsize\ttfamily,
  language=C++,
  breaklines=true,
  tabsize=2,
  escapeinside={(*@}{@*)}
}

\usepackage{url}

\setlist[itemize,1]{label=$\circ$,leftmargin=1em}
\setlist[itemize,2]{label=$\bullet$,leftmargin=1em}
\setlist[itemize,3]{label=$\diamond$,leftmargin=1em}

\author[Chochl\'{i}k M. -- Naumann A.]{Mat\'{u}\v{s} Chochl\'{i}k \and Axel Naumann}

\begin{document}

\title{P0194R1 -- Static reflection}

\section{Intro}

\begin{frame}
\begin{center}
  \Huge
  \textbf{P0194} -- {\em Static reflection}\\
  \normalsize
  \vspace{3em}
  Mat\'{u}\v{s} Chochl\'{i}k\\
  Axel Naumann
\end{center}
\end{frame}

\subsection{The CFP}

\begin{frame}[fragile]
\frametitle{{\em Call for Compile-Time Reflection Proposals} -- \textbf{N3814}}
  \large
  \begin{itemize}
    \item Targeted use cases:
    \begin{itemize}
      \item \textbf{Generation of common functions}
      \begin{itemize}
        \scriptsize
        \item comparison operators,
        \item serialization functions,
        \item hash functions,
        \item \ldots
      \end{itemize}
      \item \textbf{Type transformations}
      \begin{itemize}
        \scriptsize
        \item {\em struct-of-arrays},
        \item object-relational mapping,
        \item \ldots
      \end{itemize}
      \item \textbf{Compile-time context information}
      \begin{itemize}
        \scriptsize
        \item replacement for \verb@__FILE__@, \verb@__LINE__@, \ldots
      \end{itemize}
      \item \textbf{Enumeration of other entities}
      \begin{itemize}
        \scriptsize
        \item types in a namespace
        \item parameters of a function
        \item global variables in a namespace
        \item enumerators in an enumeration
        \item \ldots
      \end{itemize}
    \end{itemize}
  \end{itemize}
\end{frame}

\subsection{The proposal}

\begin{frame}
\frametitle{Static reflection proposal}
  \begin{itemize}
    \large
    \item \textbf{Current iteration}
    \begin{itemize}
      \item \textbf{P0194R1} -- {\em Static reflection (rev.5)} {\small -- wording}
      \item \textbf{P0385R0} -- {\em Static reflection: {\small Rationale, design and evolution}}
    \end{itemize}
    \normalsize
    \item Previous revisions\footnote{see the revision history in P0194R1}
    \begin{itemize}
      \small
      \item \textbf{P0194R0} -- {\em Static reflection (rev.4)}
      \item \textbf{N4451} -- {\em Static reflection (rev.3)}
      \item \textbf{N4452} -- {\em A case for strong static reflection}
      \item \textbf{N4111} -- {\em Static reflection (rev.2)}
      \item \textbf{P3996} -- {\em Static reflection (rev.1)}
    \end{itemize}
  \end{itemize}
\end{frame}

\section{Overview}

\subsection{Metaobjects}
\begin{frame}
\frametitle{{\em Metaobject} -- abstract definition}
  \large
  \begin{itemize}
    \item \textbf{{\em A representation of a base-level program declaration.}}
    \item {\Large \textbf{Gives a first-class identity to the reflected entity.}}
    \begin{itemize}
      \small
      \item Can be stored in a \say{variable}.
      \item Can be passed as an argument or a return value in a metaprogram.
      \item Separates the reflection of a declaration from
      the querying of the metadata.
    \end{itemize}
    \item \textbf{Can be reasoned-about at compile-time.}
    \item In some cases allows to \say{go back} to the reflected declaration.
    \item Conforms to one of the {\em metaobject concepts}.
  \end{itemize}
\end{frame}

\begin{frame}[fragile]
\frametitle{{\em Metaobjects} -- implementation}
  \large
  \begin{itemize}
    \item {\Large \textbf{Anonymous, implementation-defined types.}}
    \begin{itemize}
      \small
      \item have first-class identity at compile-time
      \item lightweight
      \item no visible internal structure
      \item creation of run-time objects of these types is not required
      \item can be implemented as template-wrapped constants 
    \end{itemize}
    \item Returned by the \textbf{reflexpr} operator.
    \begin{lstlisting}[basicstyle=\normalsize\ttfamily]
	using meta_global_scope = (*@\alert{reflexpr}@*)(::);
	using meta_std = (*@\alert{reflexpr}@*)(std);
	using meta_int = (*@\alert{reflexpr}@*)(int);
    \end{lstlisting}
  \end{itemize}
\end{frame}

\begin{frame}
\frametitle{Metaobject kinds}
    \large
    \begin{columns}[T]
      \begin{column}{0.50\textwidth}
      \begin{itemize}
        \item \textbf{In the current proposal}
        \begin{itemize}
          \normalsize
          \item the global scope,
          \item a namespace,
          \item a type,
          \item a class,
          \item a class data member,
          \item an enumeration,
          \item an enumeration value,
          \item a namespace or type alias,
          \item a specifier.
        \end{itemize}
      \end{itemize}
      \end{column}
      \begin{column}{0.45\textwidth}
      \begin{itemize}
        \item Future extensions\footnotemark
        \begin{itemize}
          \normalsize
          \item a template,
          \item a variable
          \item a function,
          \item a constructor,
          \item an operator,
          \item a template parameter,
          \item a function parameter,
          \item \ldots
        \end{itemize}
      \end{itemize}
      \end{column}
    \end{columns}
    \footnotetext{Using the same principles. See also N4111}
\end{frame}

\subsection{Metaobject operations}
\begin{frame}[fragile]
\frametitle{{\em Metaobject operations}}
  \large
  \begin{itemize}
    \item \textbf{Compile-time operations adding functionality to the {\em metaobjects}.}
    \item One or several of their arguments are {\em metaobjects}.
    \item Return compile-time metadata or other metaobjects.
    \item {\Large \textbf{Implemented as class templates:}}
    \begin{lstlisting}[basicstyle=\normalsize\ttfamily]
	using meta_str = reflexpr(std::string);
	(*@\alert{get\_name\_v}@*)<meta_str>;          // "string"
	using meta_std = (*@\alert{get\_scope\_m}@*)<meta_str>;
	(*@\alert{get\_name\_v}@*)<meta_std>;          // "std"
	(*@\alert{reflected\_type\_t}@*)<meta_str> str("blah");
    \end{lstlisting}
  \end{itemize}
\end{frame}

\begin{frame}
\frametitle{{\em Metaobject operations} -- \say{return values}}
  \large
  \begin{itemize}
    \item \textbf{Compile-time metadata} in the form of:
    \begin{itemize}
      \normalsize
      \item boolean constants,
      \item integral constants,
      \item enumerator constants,
      \item string constants,
      \item other (data member pointers, \ldots)
    \end{itemize}
    \item \textbf{Metaobjects} reflecting:
    \begin{itemize}
      \normalsize
      \item scope,
      \item class data members, class member typedefs, class inheritance,
      \item parameters, the return value,
      \item aliased declarations,
      \item various specifiers,
      \item \ldots
    \end{itemize}
  \end{itemize}
\end{frame}

\subsection{Metaobject concepts}
\begin{frame}
\frametitle{{\em Metaobject concepts}}
  \begin{itemize}
    \item Determine the \textbf{category} of a metaobject -- \textbf{what does it reflect}.
    \item \large \textbf{Determine which operations can be invoked on a metaobject.}
  \end{itemize}
  \begin{columns}
    \scriptsize
    \begin{column}{0.35\textwidth}
      \begin{itemize}
      \setlength\itemsep{0.1em}
      \item \texttt{Object}
      \item \texttt{ObjectSequence}
      \item \texttt{Reversible}
      \item \texttt{Named}
      \item \texttt{Typed}
      \item \texttt{ScopeMember}
      \item \texttt{Scope}
      \item \texttt{Alias}
      \item \texttt{ClassMember}
      \item \texttt{Linkable}
      \item \texttt{Constant}
      \item \texttt{Specifier}
      \end{itemize}
    \end{column}
    \begin{column}{0.35\textwidth}
      \begin{itemize}
      \setlength\itemsep{0.1em}
      \item \texttt{Namespace}
      \item \texttt{GlobalScope}
      \item \texttt{NamespaceAlias}
      \item \texttt{Type}
      \item \texttt{TypeAlias}
      \item \texttt{Class}
      \item \texttt{Enum}
      \item \texttt{EnumClass}
      \item \texttt{Variable}
      \item \texttt{DataMember}
      \item \texttt{MemberType}
      \item \texttt{EnumValue}
      \end{itemize}
    \end{column}
  \end{columns}
\end{frame}

\begin{frame}[fragile]
\frametitle{{\em Metaobject concepts}}
  \begin{itemize}
    \item \textbf{Distinguishing metaobjects} from other types -- \textbf{the
    \texttt{is\_metaobject} type trait}:
    \begin{lstlisting}
	template <typename T> struct is_metaobject 
	 : integral_constant<bool, (*@{\em implementation-defined}@*)> { };

	template <typename T> constexpr bool is_metaobject_v
	 = is_metaobject<T>::value;
    \end{lstlisting}
    \item \textbf{The \texttt{meta::Object} concept:}
    \begin{lstlisting}
	template <typename T> concept bool Object
	 = is_metaobject_v<T>;
    \end{lstlisting}
    \item All other \textbf{metaobjects form a} generalization-specialization
      \textbf{hierarchy} and have additional requirements:
    \begin{lstlisting}
	template <typename T> concept bool Named
	 = Object<T> && (*@{\em implementation-defined}@*);
    \end{lstlisting}
  \end{itemize}
\end{frame}

\begin{frame}[fragile]
\frametitle{{\em Metaobject sequences}}
  \begin{itemize}
    \item Lightweight \say{handle} to an ordered sequence of metaobjects.
    \begin{itemize}
      \footnotesize
      \item class or namespace members,
      \item list of storage class specifiers,
      \item list of base class inheritance specifiers,
      \item \ldots
    \end{itemize}
    \item \textbf{Does not instantiate \say{contained} metaobjects eagerly.}
    \item Definition:
    \begin{lstlisting}
	template <typename T> concept bool ObjectSequence
	 = Object<T> && (*@{\em implementation-defined}@*);
    \end{lstlisting}
    \item \textbf{Operations:}
    \begin{lstlisting}
	template <ObjectSequence S>           struct get_size;
	template <ObjectSequence S, size_t I> struct get_element;

	template <ObjectSequence S, template <class...> class Tpl>
	struct unpack_sequence;
    \end{lstlisting}
  \end{itemize}
\end{frame}

\subsection{The header}
\begin{frame}[fragile]
\frametitle{\em The \texttt{<reflexpr>} header file}
  \begin{itemize}
    \item \textbf{Must be included} prior to the use of \texttt{reflexpr}.
    \item Defines the metaobject concepts.
    \item Implements the metaobject operations.
    \item All \textbf{reflection-related declarations go to} the \textbf{namespace
    \texttt{std::meta}}, \textbf{except for} the \textbf{\texttt{is\_metaobject} trait}:
    \begin{lstlisting}
	namespace std {
	  template <typename T> is_metaobject;

	  namespace meta {
	    template <typename T> concept bool Object;
	    template <Object T> concept bool Named;
	    ...
	    template <Named MO> struct get_name;
	    ...
	  } // namespace meta
	} // namespace std
    \end{lstlisting}
  \end{itemize}
\end{frame}

\subsection{Why the TMP interface?}
\begin{frame}[fragile]
\frametitle{Why a template metaprogramming interface?}
  \large
  \begin{itemize}
    \item \textbf{Tried paradigm} with approximately 15 years of experience.
    \item The pros and cons are generally known.
    \item Lots of \textbf{existing TMP libraries} and utilities.
    \item \textbf{Consistent with} the existing \textbf{type-traits} and other
      standard and third-party metaprogramming utilities.
    \item \textbf{Other interfaces}, both compile-time and run-time \textbf{can}
      and will \textbf{be implemented on top of\footnote{or parallel to, using
      the same machinery} the TMP interface}.
    \begin{lstlisting}
	template <Object MO> struct metaobject;

	template <Object MO>
	constexpr auto get_name(metaobject<MO>) {
	    return meta::get_name_v<MO>;
	}
    \end{lstlisting}
  \end{itemize}
\end{frame}

\subsection{Design evaluation}
\begin{frame}
\frametitle{Design -- the good}
  \large
  \begin{itemize}
    \item All the \say{magic} is contained within the \texttt{reflexpr} operator.
    \item No changes to the core language required\footnote{except for the operator}.
    \item Metaobjects are first-class entities.
    \item Allows for lightweight and efficient implementation.
    \item Fairly powerful and expressive.
    \item Covers many use cases.
    \item Non-intrusive.
    \item Fine-grained.
    \item Extensible.
    \item Lazy\footnote{can be a virtue too}.
  \end{itemize}
\end{frame}

\begin{frame}
\frametitle{Design -- the bad}
  \large
  \begin{itemize}
    \item Requires a new keyword -- \texttt{reflexpr}.
    \begin{itemize}
      \item We did some checking -- \say{reflexpr} is not a very common word.
      \item Zero occurrences in ACTCD16\footnote{\url{http://www.tomazos.com/actcd16/}}
    \end{itemize}
  \end{itemize}
\end{frame}

\begin{frame}
\frametitle{Design -- the ugly}
  \large
  \begin{itemize}
    \item It's template metaprogramming\footnote{for the moment \ldots}
    \begin{itemize}
      \item May be too verbose for some use cases -- can\footnote{and will} be
      solved by implementing a simplifying fa\c{c}ade.
    \end{itemize}
    \item \textbf{But, we don't want to trade \say{simplicity} for usefulness.}
  \end{itemize}
\end{frame}

\section{Notable features}

\subsection{Typedef reflection}
\begin{frame}[fragile]
\frametitle{Typedef reflection -- unnamed \texttt{struct}}
  \small
  \begin{itemize}
    \item Suppose that we have the following \verb@typedef@ to an unnamed
      \verb@struct@\footnote{possibly in third-party code}:
    \begin{lstlisting}{cpp}[basicstyle=\small\ttfamily]
	typedef struct { int a, b, c; } my_struct;
    \end{lstlisting}
    \item \textbf{We want to serialize instances} of this \verb@struct@
      for network transport, \textbf{including the type name} in the data format:
      \begin{lstlisting}{cpp}{
	cout << "{" << '"' << "type" << '"' << ": " << '"'
	     << get_name_v<reflexpr(my_struct)>
	     << '"' << ", " << '"' << "attr" << '"' << ": {";
	/* ... */
	cout << "}}";
      \end{lstlisting}
    \item Output \textbf{without \texttt{typedef}} reflection:
    \begin{lstlisting}[basicstyle=\small\ttfamily]
	{"type":        (*@\alert{""}@*), "attr": { ... }}
    \end{lstlisting}
    \item Output \textbf{with \texttt{typedef}} reflection:
    \begin{lstlisting}[basicstyle=\small\ttfamily]
	{"type": (*@\alert{"my\_struct"}@*), "attr": { ... }}
    \end{lstlisting}
  \end{itemize}
\end{frame}

\begin{frame}[fragile]
\frametitle{Typedef reflection -- inter-platform differences}
  \small
  \begin{itemize}
    \item Let's suppose that we have the following declarations:
    \begin{lstlisting}{cpp}
	#if PLATFORM_X
	  typedef int       int32_t;
	  typedef (*@\alert{long}@*)       int64_t;
	#elif PLATFORM_Y
	  typedef long      int32_t;
	  typedef (*@\alert{long long}@*)   int64_t;
	#endif
	struct abc { (*@\alert{int64\_t}@*) a, b, c; };
    \end{lstlisting}
    \item We serialize instances of this \verb@struct@ \textbf{on platform X} and
      send them over the network to platform Y:
    \item \textbf{Without \texttt{typedef}} reflection:
    \begin{lstlisting}[basicstyle=\footnotesize\ttfamily]
	"attr":{"a":{"t":"(*@\alert{long}@*)", "v":"(*@\alert{92233720368547758}@*)"},...}
    \end{lstlisting}
    \item \textbf{With \texttt{typedef}} reflection:
    \begin{lstlisting}[basicstyle=\footnotesize\ttfamily]
	"attr":{"a":{"t":"(*@\alert{int64\_t}@*)","v":"(*@\alert{92233720368547758}@*)"},...}
    \end{lstlisting}
  \end{itemize}
\end{frame}

\begin{frame}[fragile]
\frametitle{Typedef reflection -- inter-platform differences (cont.)}
  \small
  \begin{itemize}
  \item \ldots meanwhile at the receiving end \textbf{on platform Y}:
    \begin{itemize}
    \item \textbf{Without \texttt{typedef}} reflection
      \begin{lstlisting}{cpp}
	json_obj obj = json_obj::from_string(json_str);
	auto attr = obj["attr"]["a"];
	/* ... */
	if(attr["t"] == "long") {
	   (*@\alert{long}@*) tmp;
	   attr["v"] >> tmp; // (*@\alert{32-bit int overflow?, exception?}@*)
	} /* ... */
      \end{lstlisting}
    \item \textbf{With \texttt{typedef}} reflection
      \begin{lstlisting}{cpp}
	json_obj obj = json_obj::from_string(json_str);
	auto attr = obj["attr"]["a"];
	/* ... */
	if(attr["t"] == "int64_t") {
	   (*@\alert{int64\_t}@*) tmp;
	   attr["v"] >> tmp; // (*@\alert{A-OK}@*)
	} /* ... */
      \end{lstlisting}
    \end{itemize}
  \end{itemize}
\end{frame}

\begin{frame}[fragile]
\frametitle{Typedef reflection -- Diagnostics and logging}
  \begin{itemize}
  \item We want to log function calls -- why not \textbf{use reflection to get
    type names}, source location info, etc.
  \item Output \textbf{without \texttt{typedef}} reflection:
    \begin{itemize}
      \footnotesize
      \item Depends on the platform, compiler and the implementation of
        the standard library.
      \item May depend on build configuration (debug/release, 32-bit/64-bit)
    \end{itemize}
  \item Output \textbf{with \texttt{typedef}} reflection:
    \begin{itemize}
      \small
      \item Same on every platform
      \item Same for every build configuration
      \item Preserves semantic meaning
      \item Matches the names in output to what is used in the code.
    \end{itemize}
  \end{itemize}
  \Large Compiler diagnostics use typedefs for a good reason. Same holds for reflection.
\end{frame}

\subsection{Non-public member reflection}
\begin{frame}[fragile]
\frametitle{Non-public member reflection}
  \begin{itemize}
    \item \textbf{Necessary for several use cases} (serialization, etc.)
    \item Guidance from JAX: {\em Non-public reflection yes, but {\em greppable}.}
    \item \textbf{Variants of operations} for {\em obtaining} the reflections
      of the non-public members:
    \begin{itemize}
      \footnotesize
      \item \texttt{get\_data\_members} vs. \texttt{\alert{get\_all\_data\_members}}
      \item \texttt{get\_member\_types} vs. \texttt{\alert{get\_all\_member\_types}}
      \item \ldots
    \end{itemize}
    \item Variants of operations for {\em going back} to the base-level?\footnote
     {not in the proposal at the moment}
    \begin{itemize}
      \footnotesize
      \item \texttt{get\_pointer} vs. \texttt{\alert{get\_nonpublic\_pointer}}
      \item \texttt{get\_reflected\_type} vs. \texttt{\alert{get\_nonpublic\_reflected\_type}}
      \item {\em greppable}, but does not scale very well
    \end{itemize}
    \item Freely access metadata; name, type information, scope information, etc.
  \end{itemize}
\end{frame}

\subsection{Reverse reflection}
\begin{frame}[fragile]
\frametitle{Reverse reflection}
  \begin{itemize}
    \item {\large \textbf{What happens when a metaobject is passed as the argument
      to \texttt{reflexpr}}?}
    \begin{itemize}
      \item If the metaobject is not a model of \texttt{Reversible} -- ill-formed.
      \item If the metaobject is a \texttt{Reversible} -- {\em reverse reflection}.
    \end{itemize}
    \item \textbf{Reverse reflection} -- as if the reflected declaration was used:
    \begin{itemize}
      \item \textbf{Already in the proposal}:
      \begin{lstlisting}[basicstyle=\small\ttfamily]
	using meta_int = reflexpr(int);
	// same as: int i = 123;
	(*@\alert{reflexpr(meta\_int)}@*) i = 123;
	// equivalent to:
	get_reflected_type_t(meta_int) i = 123;
      \end{lstlisting}
      \item Extensions of this mechanism are planned for the future.
    \end{itemize}
  \end{itemize}
\end{frame}

\section{Examples}

\subsection{Reflexpr usage}
\begin{frame}[fragile]
\frametitle{Using the reflexpr operator}

\textbf{Valid operands} for the \texttt{reflexpr} operator:
\begin{lstlisting}[basicstyle=\small\ttfamily]
reflexpr() // reflects the global namespace
reflexpr(::) // reflects the global namespace
reflexpr(std) // reflects the namespace
reflexpr(unsigned) // reflects the type
reflexpr(std::size_t) // reflects the typedef
reflexpr(std::launch) // reflects the enum type
reflexpr(std::vector<int>) // reflects the class
\end{lstlisting}

\textbf{Some} of these \textbf{will become valid} in the future:
\begin{lstlisting}[basicstyle=\small\ttfamily]
reflexpr(1) // ill-formed
reflexpr(std::sin) // ill-formed
reflexpr(std::vector) // ill-formed
reflexpr(is_same_v<void,void>) // ill-formed
\end{lstlisting}
\end{frame}

\subsection{Getting class member types}
\begin{frame}[fragile]
\frametitle{Getting class member types}
\begin{lstlisting}[basicstyle=\small\ttfamily]
	struct foo {
	  int a;
	  bool b;
	  char c;
	};

	template <meta::DataMember ... MDM> using (*@\alert{helper}@*)
	 = tuple<meta::get_reflected_type_t<MDM>...>;

	using X = meta::(*@\alert{unpack\_sequence\_t}@*)<
	  meta::get_data_members<reflexpr(foo)>, (*@\alert{helper}@*)
	>;

	is_same_v<X, tuple<int,bool,char>>; // true
\end{lstlisting}
\end{frame}

\subsection{Getting class member names}
\begin{frame}[fragile]
\frametitle{Getting class member names}
\begin{lstlisting}[basicstyle=\small\ttfamily]
	struct foo { int a; bool b; char c; double d; };

	template <meta::DataMember...MDM> struct (*@\alert{helper}@*) {
	  static const char** get(void) {
	    static const char* n[] = {
	      meta::get_name_v<MDM>...
	    };
	    return n;
	  };
	};
	template <typename T>
	using name_getter = meta::(*@\alert{unpack\_sequence\_t}@*)<
	  meta::get_data_members<reflexpr(T)>, (*@\alert{helper}@*)
	>;

	name_getter<foo>::get()[0]; // "a"
\end{lstlisting}
\end{frame}

%\section{Conclusion}

%\begin{frame}
%\begin{center}
%\Huge
%Thanks for your attention\\
%\textbf{That's all folks!}\footnote{not really, but we want to finish before the sunset}
%\end{center}
%\end{frame}

\section{Issues}

\subsection{Open design decisions}
\begin{frame}[fragile]
\frametitle{Unnamed named}
  \begin{itemize}
    \item What about this:
    \begin{lstlisting}[basicstyle=\footnotesize\ttfamily]
	struct outer {
	  struct {
	    int i;
	  } inner;
	};
using meta_X = get_type_m<reflexpr(outer.inner)>;
    \end{lstlisting}
    \item {\bf A} -- \verb@meta_X@ \textbf{is not a} \verb@meta::Named@
    \begin{itemize}
      \small
      \item \verb@meta::get_name_v<meta_X>@ is ill-formed
      \item Easy to detect unnamed declarations
      \item May complicate other things
    \end{itemize}
    \item {\bf B} -- \verb@meta_X@ \textbf{is a} \verb@meta::Named@
    \begin{itemize}
      \small
      \item \verb@meta::get_name_v<meta_X>@ returns \verb@""@
      \item More clumsy to detect unnamed declarations
      \begin{itemize}
        \item Add a new operation -- \verb@bool(meta::is_unnamed<X>::value)@?
      \end{itemize}
    \end{itemize}
  \end{itemize}
\end{frame}

\section{Implementation}

\subsection{Constant value-based metaobjects}
\begin{frame}[fragile]
\frametitle{New implementation in \texttt{clang}}
  \large
  \begin{itemize}
    \item \textbf{New fundamental type \texttt{\_\_metaobject\_id}} -- basically
      a ripped-off of \texttt{long}.
    \item \textbf{New operator \texttt{\_\_reflexpr}} -- same arguments as
      \texttt{reflexpr}, returns \texttt{\_\_metaobject\_id}.
    \item A metaobject is implemented as:
    \begin{lstlisting}[basicstyle=\ttfamily\normalsize]
	template <__metaobject_id MoId>
	struct __metaobject;
    \end{lstlisting}
  \end{itemize}
\end{frame}

\begin{frame}[fragile]
  \large
  \begin{itemize}
    \item \textbf{Operations implemented by built-ins:}
    \begin{lstlisting}[basicstyle=\ttfamily\normalsize]
	constexpr __metaobject_id
	__metaobj_get_scope(__metaobject_id);
    \end{lstlisting}
    \item \textbf{Wrapped into templates:}
    \begin{lstlisting}[basicstyle=\ttfamily\normalsize]
	template <typename MO>
	struct get_scope;

	template <__metaobject_id MoId>
	struct get_scope<__metaobject<MoId>> {
	    typedef __metaobject<
	       __metaobj_get_scope(MoId)
	    > type;
	};
    \end{lstlisting}
  \end{itemize}
\end{frame}

\section{The future}

\begin{frame}
\begin{center}
\Huge \alert{DISCLAIMER:}\\
\Large The following are \textbf{future extensions}, not part of this
proposal. We want to lay the foundation before we paint the walls!\\
\normalsize
\end{center}
\end{frame}

\subsection{Reverse reflection}
\begin{frame}[fragile]
\frametitle{Reverse reflection 2.0}
    \begin{itemize}
      \item We already know that \ldots
      \begin{lstlisting}{cpp}
	using meta_int = reflexpr(int);
	// same as: int i = 123;
	(*@\alert{reflexpr(meta\_int)}@*) i = 123;
      \end{lstlisting}
      \item {\Large what if \dots}
      \begin{lstlisting}{cpp}
	struct S { int i; };
	S s{123};
	using meta_S_i = reflexpr(S::i);
	// same as:  assert(s.i == 123);
	assert(s.(*@\alert{reflexpr(meta\_S\_i)}@*) == 123);

	using meta_std = reflexpr(std);
	// same as: std::string str;
	(*@\alert{reflexpr(meta\_std)}@*)::string str;

	using meta_std_pair = reflexpr(std::pair); 
	//        std::pair<int, int> pii;
	(*@\alert{reflexpr(meta\_std\_pair)}@*)<int, int> pii;
      \end{lstlisting}
    \end{itemize}
\end{frame}

\subsection{Identifier formatting}
\begin{frame}[fragile]
\frametitle{Identifier formatting}
  \begin{itemize}
    \item {\large The \textbf{ability to generate identifiers programmatically}\footnote{
       without the preprocessor} is important for several use cases.}
    \begin{itemize}
      \footnotesize
      \item Structure-of-arrays generators
      \item Object-relational mapping
      \item \ldots
    \end{itemize}
    \item Let's use special operator, say \verb@identifier@\footnote{let the
      bike-shedding begin}, with a \textbf{format string literal
      and a set of \texttt{meta::Named} metaobjects}:
      \begin{lstlisting}{cpp}
	struct bar { int foo; };
	using m_bar = reflexpr(bar);
	using m_foo = reflexpr(bar::foo);

	int (*@\alert{identifier("baz")}@*);               //int baz;
	int (*@\alert{identifier("\%1", m\_bar)}@*);          //int bar;
	int (*@\alert{identifier("\%2\_\%1", m\_bar, m\_foo)}@*);  //int foo_bar;
	int (*@\alert{identifier("my\_\%1\_\%2", m\_bar, m\_foo)}@*);//int my_bar_foo;
      \end{lstlisting}
  \end{itemize}
\end{frame}

\subsection{Variadic composition}
\begin{frame}[fragile]
\frametitle{Variadic composition}
  \small
    \begin{itemize}
      \item suppose we have a \verb@struct@ which we want to transform;
      \begin{lstlisting}{cpp}
	struct original { T1 attr1; T2 attr2; T3 attr3; };
      \end{lstlisting}
      \item by using identifier generation and variadic composition,
      \begin{lstlisting}{cpp}
	template <DataMember ... MDM> struct soa_helper {
	    std::vector<reflexpr(meta::get_type_t<MDM>)>
	        (*@\alert{identifier("vec\_\%1", MDM)...}@*);
	};
      \end{lstlisting}
      \item instantiating the \verb@soa_helper@:
      \begin{lstlisting}{cpp}
	using mdm = get_data_members_m<reflexpr(original)>;
	using soa_orig = expand_sequence_t<mdm, soa_helper>;
      \end{lstlisting}
      \item would result in the following structure:
      \begin{lstlisting}{cpp}
	struct {
	    std::vector<T1> vec_attr1;
	    std::vector<T2> vec_attr2;
	    std::vector<T3> vec_attr3;
	};
      \end{lstlisting}
    \end{itemize}
\end{frame}

\subsection{Context-dependent reflection}
\begin{frame}[fragile]
\frametitle{Context-dependent reflection}
  \begin{itemize}
    \item {\Large Allows to \textbf{obtain metadata based on the context in
      which the reflection operator is invoked}, instead of on the declaration
      name:}
    \begin{itemize}
      \footnotesize
      \item \verb@reflexpr(this::namespace)@ -- the innermost enclosing namespace.
      \item \verb@reflexpr(this::class)@ -- the innermost enclosing class.
      \item \verb@reflexpr(this::template)@ -- the innermost enclosing template.
      \item \verb@reflexpr(this::function)@ -- the enclosing function.
    \end{itemize}
    \item {\large Helpful in several use cases}
    \begin{itemize}
      \small
        \item Implementation of logging
        \item Implementation of cross-cutting aspects
        \item \ldots
    \end{itemize}
  \end{itemize}
\end{frame}

\section{Outro}

\subsection{TOC}
\begin{frame}
\begin{multicols}{2}
  \small
  \tableofcontents
\end{multicols}
\end{frame}

\end{document}
