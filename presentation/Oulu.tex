\documentclass[compress,table,xcolor=table]{beamer}
\mode<presentation>
\usetheme{Warsaw}
\hypersetup{pdfstartview={Fit}}

\usecolortheme{crane}
\useoutertheme{smoothbars}

\setbeamertemplate{bibliography item}[text]
\usepackage[utf8]{inputenc}
\usepackage{caption}
\usepackage{enumitem}
\usepackage{dirtytalk}
\usepackage{multirow}

\definecolor{light-gray}{gray}{0.95}

\usepackage{listings}
\lstset{
backgroundcolor = \color{light-gray},
basicstyle=\footnotesize\ttfamily,
language=C++,
breaklines=true,
tabsize=2,
escapeinside={(*@}{@*)}
}

\usepackage{url}

\setlist[itemize,1]{label=$\bullet$,leftmargin=1em}
\setlist[itemize,2]{label=$\circ$,leftmargin=1em}
\setlist[itemize,3]{label=$\diamond$,leftmargin=1em}

\begin{document}

\title{Static reflection}

\section{Overview}

\subsection{Metaobject}
\begin{frame}
\frametitle{Overview -- {\em Metaobject}}
  \begin{itemize}
    \small
    \item A representation of a base-level program declaration\footnote
      {only some of these are part of the initial proposal}:
    \begin{itemize}
      \scriptsize
      \item the global scope, a namespace,
      \item a type, a class, an enumeration,
      \item a namespace or type alias,
      \item a template,
      \item a variable, a class data member,
      \item a function, a constructor, an operator,
      \item a template parameter, a function parameter,
      \item \ldots
    \end{itemize}
    \item Gives a first-class identity to the reflected entity
    \begin{itemize}
      \scriptsize
      \item Can be stored in a \say{variable}.
      \item Can be passed as an argument or a return value around a metaprogram.
      \item Allows to separate the reflection of a declaration from
      the querying of the metadata.
    \end{itemize}
    \item Can be reasoned-about at compile-time.
    \item Conform to various {\em concepts}.
  \end{itemize}
\end{frame}

\subsection{Metaobject operations}
\begin{frame}
\frametitle{Overview -- {\em Metaobject operations}}
  \begin{itemize}
    \small
    \item Compile-time operations adding functionality to the {\em metaobjects}.
    \item One or several of their arguments are {\em metaobjects}.
    \item The return value can be one of the following:
    \begin{itemize}
      \footnotesize
      \item Metadata in the form of:
      \begin{itemize}
        \item boolean constants,
        \item integral constants,
        \item enumeration constants,
        \item string constants,
        \item other (class data member pointers, \ldots),
      \end{itemize}
      \item Metaobjects reflecting:
      \begin{itemize}
        \item scope,
        \item class data members, class member typedefs, class inheritance,
        \item parameters, the return value,
        \item aliased declarations,
        \item various specifiers,
        \item \ldots
      \end{itemize}
    \end{itemize}
  \end{itemize}
\end{frame}

\subsection{Metaobject concepts}
\begin{frame}
\frametitle{Overview -- {\em Metaobject concepts}}
  \begin{itemize}
    \small
    \item Determine the category of a metaobject (what does it reflect).
    \item Determine which operations can be invoked on a metaobject.
  \end{itemize}
\end{frame}

\section{Notable features}

\subsection{Typedef reflection}
\begin{frame}
\frametitle{Typedef reflection}
\Huge TODO
\end{frame}

\section{Future extensions}

\subsection{Reverse reflection}
\begin{frame}
\frametitle{Reverse reflection}
\Huge TODO
\end{frame}

\subsection{Context-dependent reflection}
\begin{frame}
\frametitle{Context-dependent reflection}
\Huge TODO
\end{frame}

\subsection{Identifier formatting}
\begin{frame}
\frametitle{Identifier formatting}
\Huge TODO
\end{frame}


\end{document}
