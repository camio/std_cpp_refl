\section{Frequently asked questions}

\subsection{Why metaobjects, why not reflect directly with type traits?}
\label{faq-why-metaobjects}

\textbf{Q:} {\em Why should we bother with defining a set of metaobject concepts,
let the compiler generate models of these concepts and use those to obtain
the metadata? Why not just extend the existing type traits?}

\textbf{A:} The most important reasons are the \hyperref[design-completeness]{completeness} and the scope of reflection.
Type traits\footnote{As they are defined now.} work just with types.
A reflection facility should however provide much more metadata.
It should be able to reflect namespaces, functions, constructors,
class inheritance, templates, specifiers, variables, etc.

Without drastically changing the rules specifying what can or cannot be a template
parameter, we cannot properly reflect C++'s second-class declarations like
namespaces.

This is also connected with the usability of reflection in metaprograms.
Using types to reflect various base-level declarations, allows to
pass a representation of a namespace, a constructor, a specifier or a parameter,
around various \say{subroutines} of a metaprogram as a parameter or a return value.

In order to achieve this by other means it would be necessary to make namespaces,
etc. first-class objects or at least to allow them to become template arguments.

So with type-based metaobjects we can do for example the following:

\begin{minted}{cpp}
template <typename Metaobject>
void foo(void)
{
	// do something terribly useful
}

foo<reflexpr(int)>(); // works
foo<reflexpr(std)>(); // works
foo<reflexpr(std::string)>(); // works
foo<reflexpr(std::string::size_type)>(); // works
foo<reflexpr(std::size_t)>(); // works and is different from the above
foo<reflexpr(std::pair)>(); // works
foo<reflexpr(std::sqrt)>(); // works
foo<reflexpr(static)>(); // works
\end{minted}

meanwhile without metaobjects:

\begin{minted}{cpp}
template <something Param>
void foo(void)
{
	// do something terribly useful, if possible
}

foo<int>(); // works
foo<std>(); // error?
foo<std::string>(); // works
foo<std::string::size_type>(); // works
foo<std::size_t>(); // works but same as the above
foo_tpl<std::pair>(); // works (if there are two versions of foo)
foo<std::sqrt>(); // error?
foo<static>(); // error?
\end{minted}

\subsection{OK, but why not reflect directly with several different operators?}

\textbf{Q:} {\em Alright, we partially agree with the reasoning in the
\hyperref[faq-why-metaobjects]{previous question}, but why don't we use several
different operators to get the various pieces of metadata (like declaration names,
types, scopes, etc.) directly, skipping the metaobjects?
For example we already have \verb@decltype(expr)@ so why not add,
\verb@declname(expr)@ or \verb@declscope(expr)@, etc.}

\textbf{A:} The main reason is that there are many base-level declarations
with various properties which we want to reflect and having a separate
operator for every one of them\footnote{There may be even several dozen.},
will require many new reserved keywords which can break existing code.

And still, what would the operator reflecting for example a scope return?
In case of class scopes it could return a type, but what if we are asking
about the scope of a global-scope or namespace-level declaration? We would
either end up with something like the proposed metaobjects,
or we would have to make some other (not type-based) representation of any
possible scope.
In case of the latter we would loose the ability to pass the representation
of the scope as parameters to metaprograms.

The same applies to every other operation returning an expression which does
not have first-class identity in base-level C++.

It is possible that, the templates implementing the metaobject operations
will internally use compiler built-ins\footnote{Just like some of the type traits
are implemented.}, but these can use compiler-specific reserved keywords
and they will all operate on the metaobjects.

\subsection{Creating a separate type for each metaobject is so heavyweight.}

\textbf{Q:} {\em Isn't the creation of a new type for each metaobject too
heavyweight? Won't it lead to unacceptable increases in the memory footprint
and compilation times?}

\textbf{A:} This might have been a problem if the metaobjects were regular
types and if the implementation was too coarse-grained or too eager.

For example if upon the invocation of \verb@reflexpr(std::string)@, the representations
of all the metadata related to the \verb@std::string@ type\footnote{Like
the compile-time representation of its name and the metaobjects reflecting its scope
or its members.} were generated immediately.

This is however {\em not} the case.

The metaobjects need to be types just enough so that they can be used
as template arguments\footnote{Similar to the \texttt{void} type}.
They don't need to have any name, scope, implicitly generated constructors,
destructors, assignment operators, virtual method tables, run-time type information,
etc. All they need to have is a unique identity and their internal representations
need to point to the internal representations of the declarations\footnote
{Which the compiler needs to maintain anyway, regardless of reflection.}
which they reflect.

Also our proposal allows very fine granularity. The result of \verb@reflexpr@
can be a very lightweight type, as we just described and the individual
\say{attributes} like the name, scope, members, specifiers, etc. related
to the metaobject are materialized only when requested by one of the operations
defined for that particular metaobject concept, like \verb@get_name@,
\verb@get_scope@, \verb@get_data_members@, etc.

\subsection{Creating a separate type for each string is so heavyweight.}

\textbf{Q:} {\em Isn't the creation of a new type for each string returned
for example from the \verb@get_name@ operation too heavyweight?
Won't that lead to unacceptable increases in the memory footprint
and compilation times?}

\textbf{A:} The answer is that we {\em do not} insist on creating a separate
type for each string returned by reflection. What we insist on is that we
should have the ability to reason about and manipulate the strings at compile-time.
A static, \verb@constexpr@-initialized, zero-terminated array of \verb@char@s
gives us this ability.

So for example the implementation of the \verb@get_name@ operation should
{\em be equivalent} to the following:

\begin{minted}[tabsize=4]{cpp}
template <Named T>
struct get_name
{
	typedef const char value_type[N+1];
	static constexpr const char value[N+1] = {...,'\0'};
};
\end{minted}

Some members of the committee suggested that for example the
\texttt{basic\_string\_constant} from N4121~\cite{ISOCPP-N4121} or
N4236~\cite{ISOCPP-N4236} or whichever compile-time
string representation eventually makes it to the standard could
be used to implement the \verb@get_name@ operation,
but this in just an option not a requirement for our part.
We trust in the ability of the compiler vendors to pick and
implement the best option and to use all the tricks at their disposal
to make expressions like;

\begin{minted}{cpp}
std::string(std::meta::get_name_v<reflexpr(std::pair)>);
\end{minted}

as efficient as possible.

\subsection{There's already a type trait for that!}

\textbf{Q:} {\em Why do you introduce a metaobject trait like
\verb@std::meta::is_class<Metaobject>@  when there already are equivalent
type traits like \verb@std::is_class<T>@?}

\textbf{A:} The reason is consistency.
While it is true that there are a few metaobject traits, which indicate the
same thing as already existing type traits, omitting the metaobject trait
in favor of the type trait would break consistency and make the interface less generic.

For example the following is much more consistent and nicer,

\begin{minted}{cpp}
template <typename Metaobject>
void foo(void)
{
	if(std::meta::is_class_v<Metaobject>) { ... }
	else if(std::meta::is_typedef_v<Metaobject>) { ... }
	else if(std::meta::is_type_v<Metaobject>) { ... }
	else if(std::meta::is_namespace_v<Metaobject>) { ... }
	else if(std::meta::is_function_v<Metaobject>) { ... }
}
\end{minted}

than,

\begin{minted}{cpp}
template <typename T, typename Metaobject>
void foo(void)
{
	if(std::is_class_v<T>) { ... }
	else if(meta::is_typedef_v<Metaobject>) { ... }
	else if(meta::is_type_v<Metaobject>) { ... }
	else if(meta::is_namespace_v<Metaobject>) { ... }
	else if(meta::is_function_v<Metaobject>) { ... }
}
\end{minted}

or even,

\begin{minted}{cpp}
template <typename Metaobject>
void foo(void)
{
	if(std::is_class_v<get_reflected_type_t<Metaobject>>) { ... }
	else if(meta::is_typedef_v<Metaobject>) { ... }
	else if(meta::is_type_v<Metaobject>) { ... }
	else if(meta::is_namespace_v<Metaobject>) { ... }
	else if(meta::is_function_v<Metaobject>) { ... }
}
\end{minted}

Even more important is the problem with breaking genericity:

\begin{minted}{cpp}
template <typename Metaobject, template <class> class Trait>
void bar(void)
{
	if(Trait<Metaobject>::value) { ... }
	else { ... }
}

bar<reflexpr(std), std::meta::is_namespace>(); // works
bar<reflexpr(static), std::meta::is_specifier>(); // works
bar<reflexpr(std::string), std::meta::is_class>(); // works
bar<reflexpr(std::string), std::is_class>(); // error!
\end{minted}


In C++ there are many examples of various expressions doing the same thing,
this is just one of them.
Having several options is not a bad thing, pick the one most appropriate for your
particular use case.


\subsection{There's already another expression for that!}

\textbf{Q:} {\em There are situations where you can do something much more
easily without reflection than with it, like \verb@&variable@ vs.
\verb@meta::get_pointer_v<reflexpr(variable)>@ or\\\verb@decltype(var)@ vs.
\verb@meta::get_reflected_type_t<meta::get_type_m<reflexpr(var)>>@, etc.}

\textbf{A:} Of course there are! There are also situations where the decision
that you want to get a pointer to or a to get the type of a (reflected) variable
can be separated from the actual site where you access (reflect) the variable by
several layers of metaprogram templates.

Consider for example the following pseudo-code:

\begin{minted}[tabsize=4]{cpp}

template <typename T>
void handle_var(const char* name, const T* ptr)
{
	/* ... */
}

template <MetaVariable MV>
void handle_meta_var(void)
{
	// this may be true only in a fraction of cases
	if(some_logic())
	{
		// we need the pointer just here
		handle_var(
			meta::get_name_v<MV>,
			meta::get_pointer_v<MV>
		);
	}
}

template <MetaClass MC>
void handle_meta_class(void);

template <MetaType MT>
void handle_meta_type(void);

template <MetaVariable MV>
void handle_meta_ns(void);

template <Metaobject MO>
void generic_func_1(void)
{
	// this may be true only in a fraction of cases
	if(meta::is_variable_v<MO>)
	{
		handle_meta_var<MO>();
	}
	else if(meta:is_class_v<MO>)
	{
		handle_meta_class<MO>();
	}
	else if(meta:is_type_v<MO>)
	{
		handle_meta_type<MO>();
	}
	else if(meta:is_namespace_v<MO>)
	{
		handle_meta_ns<MO>();
	}
}

template <Metaobject MO>
void generic_func_2(void)
{
	if(some_trait_v<MO>)
	{
		generic_func_1<MO>();
	}
	else ...
}

// ... generic_func_3 - generic_func_19 ...
// all eventually calling generic_func_1

template <Metaobject MO>
void generic_func_20(void)
{
	if(some_logic())
	{
		generic_func_19<MO>();
	}
	else
	{
		generic_func_15<MO>();
	}
}

void main(int argc, const char** argv)
{
	std::string str;
	generic_func_20<reflexpr(str)>();
	generic_func_18<reflexpr(std)>();
	generic_func_19<reflexpr(argc)>();
	generic_func_20<reflexpr(std::pair)>();
	// etc.

	return 0;
}
\end{minted}

In this case we actually need the pointer or a reference to a variable
only in a fraction of cases, deep inside of the (meta-)program.

If we didn't have the \verb@get_pointer@ operation, we would have to
get the pointer very early and pass it through the whole algorithm.
Furthermore in many cases there would not be any meaningful pointer to
get so we would have pass \verb@nullptr@. Now imagine that there were more
metaobjects, and potentially more pointers to pass around.

Having said this, we of course do not force anybody to use \verb@get_pointer@
when simply using the ampersand operator directly would be enough.

Both cases are valid and again you should pick the most appropriate option
for your situation.

\subsection{All these additional options make it hard for the novices.}
\label{faq-hard-on-novices}

\textbf{Q:} {\em This reflection proposal is too complex to be learned by
beginning C++ programmers and the novices might start to use unnecessarily
complicated expressions to do simple things, like doing
\verb@get_pointer_v<reflexpr(x)>@ instead of \verb@&x@.}

\textbf{A:} 
A novice could equally do:

\begin{minted}{cpp}
double a[2];
*(a+int(-exp(complex<double>(0, M_PI)).real())) = 123.456;
\end{minted}

instead of

\begin{minted}{cpp}
double a[2];
a[1] = 123.456;
\end{minted}

Doing mistakes is a part of being a beginner in something.

C++ is a complex\footnote{No pun intended.} language which means that
the reflection mechanism also must be expressive enough to capture
the complexity of the base-level.
However we do expect that once static reflection is available and the most
common use cases are identified a more convenient and straightforward
fa\c{c}ade will be added on top of it for simple and common things,
while maintaining the usability of reflection in more elaborate use cases.

\subsection{Why are the metaobjects anonymous?}

\textbf{Q:} {\em Why should the metaobjects be anonymous types as opposed to
types with well defined and standardized names or concrete template classes, (possibly with some
special kind of parameter accepting different arguments than types and constants)?}

\textbf{A:} We wanted to avoid defining a specific naming convention, because it would
be difficult to do so and very probably not user friendly (see C++ name mangling). There
already are precedents for anonymous types -- for example C++ {\em lambdas}.

Another option would be to define a concrete set of template classes like:

\begin{minted}[tabsize=4]{cpp}
namespace std {

	template <typename T>
	class meta_type /* Model of MetaType */
	{ };

} // namespace std
\end{minted}

which could work with types, classes, etc., but would not work with namespaces,
constructors, etc. (see also the Q/A above):

\begin{minted}[tabsize=4]{cpp}
namespace std {

	template <something X> // problem
	class meta_constructor /* Model of Meta-Constructor */
	{ };

	template <something X> // problem
	class meta_namespace /* Model of Meta-Namespace */
	{ };

} // namespace std

typedef std::meta_namespace<std> meta_std; // problem
\end{minted}

Instead of this, the metaobjects are anonymous and their (internal) identification
is left to the compiler. From the user's point of view, the metaobject can be distinguished
by the means of the metaobject traits.

\subsection{Reflection violating access restrictions to class members?}

\textbf{Q:} {\em Why do you allow reflection to bypass the class member access
restrictions? This is like giving people nuclear weapons. I don't want people
to have nuclear weapons [sic].}

\textbf{A:}
This is a valid point, but there are several different ways to break access restrictions
if the programmer wants to and we feel that restricting reflection only to public
class members would severely limit its usefulness in implementing things like
non-intrusive serialization, object-relational mapping, etc.

Having said that we also try to design the interface to indicate to the users
that they are breaking encapsulation. For example the \say{basic} operation
for getting class members -- \verb@get_data_members@, returns only the
reflections of {\em public} class data members. To get access to all data members
including the private or protected ones, the \verb@get_all_data_members@\footnote
{We are open to suggestions for a better, more expressive or dissuasive name.}
operation has to be used.

Furthermore the \verb@is_public@ trait can be used to test if a class member
is publicly accessible or not.

See also \hyperref[issue-breaking-access]{the issues section}.

\subsection{We need to get around access restrictions, but not in reflection.}

\textbf{Q:} {\em OK, we understand that there are use cases where we need
to get around access restrictions, but why do this in reflection?
Why don't we use some other, unrelated, \say{magic} operator for that?}

\textbf{A:}
In our opinion if want\footnote{and we do!} to add a way of sneaking past
access restrictions, then reflection is the perfect place to do it.

Reflection is like a way to look at a program from a higher dimension,
not perceivable from the base-level itself. The access restrictions are in
the base-level language for a reason and having to go through the meta-level
to get around them seems appropriate.

Another reason for not using an operator -- a reserved identifier, is to avoid
causing conflicts with names in existing code.

\subsection{Why do we need typedef reflection?}

\textbf{Q:} {\em Why is it necessary to distinguish between types and typedefs
or type aliases on the meta-level when they are not distinguishable on the
base-level? Or why do we need to reflect on syntax rather than just on semantics?}

\textbf{A:} Our preferred answer is that it's better to have more information
then less. Being able to distinguish typedefs or aliases on the meta-level,
brings additional information about what the programmer meant, not just how
it is implemented;
\begin{itemize}
	\item \verb@size_type@ vs. \verb@unsigned@,
	\item \verb@rank_type@ vs. \verb@short@,
	\item \verb@const_iterator@ vs. \verb@const Element*@,
	\item \verb@height_cm@ vs. \verb@float@,
	\item \verb@pid_t@ vs. \verb@int@,
	\item \verb@sighandler_t@ vs. \verb@void (*)(int)@,
	\item etc.
\end{itemize}

If we reflect on typedefs or aliases we still know the underlying types,
the opposite is not true.  

Having this information is important\footnote{or at least very nice to have}
in several use cases.

\begin{itemize}
	\item Debugging or trace messages (containing function signatures)
		are much more readable and informative if they also contain 
		the typedef names of parameter types, return values or variables,
		not just their underlying types.
	\item Cross-platform serialization or remote procedure calls may require
		typedef names instead of native type names which vary between
		platforms.
\end{itemize}

%TODO: mention some other use cases

\subsection{Why \meta{ObjectSequence}s? Why not replace them with typelists?}

\textbf{Q:} {\em Why do you define the \meta{ObjectSequence} concept and its
operations? There are type-lists proposed for the inclusion into the standard,
why not use those?}

\textbf{A:} The reason why we use \meta{ObjectSequence}s is efficiency.
Operations like \verb@get_data_members@ are currently designed to return
metaobjects --  very lightweight types representing a whole set of
other metaobjects, without actually instantiating the elements eagerly,
unlike typelists which would require that all contained
metaobjects are generated as a part of the definition of the typelist.
There are use cases where returning a typelist directly would be much
less efficient than returning the lightweight metaobject sequence.

For example the user might want to test a big set of classes and find those,
which have a precise number of members or find those which have multiple base
classes or just get the metaobject reflecting only the first\footnote
{zero-th} data member of a class having hundreds of members:

\begin{minted}{cpp}
get_element_m<get_data_members_m<reflexpr(my_class)>, 0>;
\end{minted}

This operation would involve the creation of two metaobjects:
\begin{enumerate}
	\item the \meta{ObjectSequence} and
	\item the \meta{DataMember} reflecting the first data member.
\end{enumerate}

On the other hand if this operation returned a typelist of metaobjects,
then all metaobjects would have to be generated, even if most of them
were not used afterwards.

What we can do is to add an operation converting a \meta{ObjectSequence}
into a typelist on demand, which can be implemented trivially with the
\verb@meta::unpack_sequence@ template:

\begin{minted}{cpp}
template &lt;typename ... T&gt;
struct type_list { /* ... */ };


template &lt;ObjectSequence MOS&gt;
using convert_to_list = unpack_sequence&lt;MOS, type_list&gt;;
\end{minted}

\subsection{Why not reflect with constexpr instances of the same type?}

\textbf{Q:} {\em Why are the metaobjects implemented as distinct types and
their interface as specializations of templates? Why not provide the metadata
as \verb@constexpr@ instances of the same type? Then we could replace the metaobject
sequences with arrays of \verb@constexpr@ instances and the templates with
\verb@constexpr@ functions. Wouldn't the latter be more nice and efficient?}

\textbf{A:}
In other words why don't we instead of:

\begin{minted}{cpp}
using mo = reflexpr(my_ns::my_class);

std::cout << meta::get_name_v<mo> << std::endl;
std::cout << meta::get_name_v<meta::get_scope_m<mo>> << std::endl;
meta::for_each<meta::get_data_members_m<mo>>(my_func);
\end{minted}

do something like one of the following:

\begin{enumerate}[label=\textbf{\Alph*)}]

\item
\begin{minted}{cpp}
meta::object mo = reflexpr(my_ns::my_class);

std::cout << meta::get_name(mo) << std::endl;
std::cout << meta::get_name(meta::get_scope_t(mo)) << std::endl;
meta::for_each(meta::get_data_members_t(mo), my_func);
\end{minted}

\item
\begin{minted}{cpp}
meta::object mo = reflexpr(my_ns::my_class);

std::cout << mo.get_name() << std::endl;
std::cout << mo.get_scope().get_name() << std::endl;
mo.get_data_members().for_each(my_func)
meta::for_each(meta::get_data_members(mo, my_func);
\end{minted}

\end{enumerate}

One of the reasons for the chosen representation of metaobjects and their
interface was consistency with the already existing type traits and the
established practices in metaprogramming\footnote{Yes we are aware of the
Boost.Hana library and the metaprogramming paradigm which it brings.}.

Having said that, there is nothing preventing us to implement a fa\c{c}ade
with similar syntax as either of the above\footnote{and even others}, on top
of our chosen representation and we do plan to do that in the future.
\say{Niceness} is generally a matter of taste and we want to accommodate
various programming styles. See the various layers implemented by the Mirror
reflection utilities \cite{Chochlik-Mirror-doc} for some examples.

In regard to efficiency, the types representing the metaobjects are very
lightweight up to the point of being comparable to compile-time constants.
Also the representation which we have chosen allows for a very fine granularity
and metadata which are not queried don't have to be generated by the compiler.
This may not be true if the metadata is represented as instances of \verb@constexpr@
structures, which must be initialized or would require some \say{magic}
implementation.

Also note that the metaobjects reflecting the various members of a scope\footnote{
namespaces, native types, structured types, enumerations, functions, etc.}
can and will be heterogeneous. This means that we will either have to implement
them only as homogeneous compile-time identifiers and implement their interface
with \say{magic} \verb@constexpr@ functions very similar to our templates, or
we cannot store them in an array.

Furthermore, representing metaobject sequences as arrays of such \verb@constexpr@
objects also requires all of the metaobjects in the sequence to be instantiated
even if only a few of them will actually be used. See also the previous
question for more details on this.

\subsection{Why not return fully qualified names?}

\textbf{Q:} {\em Why doesn't the \texttt{get\_name} operation return fully
qualified names instead of base names?}

\textbf{A:} This is because it is much easier to generate full names from
base names by using metaprogramming, than to do the reverse -- to parse the
base name from a fully qualified and decorated name at compile-time.

Also there are several possibilities how the full name can be formatted;
should it contain any whitespaces, if so where and how many?, should the
\verb@const@ and \verb@volatile@ qualifiers be put before or after the type
name?, etc.

These details shouldn't be handled by the basic reflection facility, but by
a higher-level library.

\subsection{What about the use cases from the committee's CFP?}

\textbf{Q:} {\em How does this proposal handle the targeted use cases
from the committee's Call for Compile-Time Reflection
Proposals~\cite{ISOCPP-N3814}?}

\textbf{A:} Some of these use cases are discussed in the sections
\ref{use-case-common-func}, \ref{use-case-enum-to-string}
and \ref{use-case-struct-transf} of this paper.

