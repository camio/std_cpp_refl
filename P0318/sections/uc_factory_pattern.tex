\subsection{Implementing the factory pattern}

The purpose of the {\em Factory} pattern is to separate its caller,
who requires a new instance of a {\em Product} type, from the details
of this instance's construction. The caller only supplies the input data
to the factory and collects the new instance. There are several aspects
that need to be considered when designing and implementing a factory.

The input data for the construction of an instance of the {\em Product}
can be stored in an external representation (an XML fragment, a RDBS database dataset,
a JSON document, etc.) or even entered by the user through a GUI oron the
command-line and so on, and would need to be converted into a native C++ representation.
The new instance also might be constructed as a copy of another already existing
\emph{prototype} instance of the same type sitting in an object pool.

The product may be polymorphic and the exact type may not even be
known to the user. It may have one or several constructors, each of which
may require a different set of arguments. It may or may not have
constructors with a specific signature, for example a default constructor.

A default constructor does not make sense for many types and requiring it
just because the type will be used with a factory is problematic\footnote{
or even impossible with third-party code}.
Consider for example what a \say{default} instance of \verb@person@ or \verb@address@ would
look like -- it would not have any meaning at all.
Thus well-designed factories should not depend on the presence of constructors
with specific signatures.

Furthermore it might be desirable, that the constructor used to construct
a particular instance is picked based on the available input data which is known
only at run-time, but not when the factory is designed and implemented.

Let's consider the implementation of a factory for a rather simple \verb@point@ class,
representing a point in 3-dimensional space:

\begin{minted}[tabsize=4]{cpp}
struct point
{
    double _x, _y, _z;

    point(double x, double y, double z): _x(x), _y(y), _z(z) { }

    point(double w): _x(w), _y(w), _z(w) { }

    point(void): _x(0.0), _y(0.0), _z(0.0) { }

    point(const point&) = default;

    // ... other declarations
};
\end{minted}

A naive hand-coded implementation, of a factory constructing
\verb@point@s from some \verb@Data@ type (for example an XML node) might look like this:

\begin{minted}[tabsize=4]{cpp}
class point_factory
{
private:
    unsigned pick_constructor(Data data)
    {
        // somehow examine the data and pick
        // the most suitable constructor of the point class
    }

    double extract(Data data, string param)
    {
        // somehow extract and convert the value
        // of a named parameter from the data
    }
public:
    point create(Data data)
    {
        switch(pick_constructor(data))
        {
            case 0: return point();

            case 1: return point(extract(data, "w"));

            case 2: return point(
                        extract(data, "x"),
                        extract(data, "y"),
                        extract(data, "z")
                    );

            default: throw exception(...);
        }
    }
};
\end{minted}

Now suppose that there is some pool of existing \verb@point@ objects
and let's extend the factory to use this pool and return copies if applicable:

\begin{minted}[tabsize=4]{cpp}

extern pool_of<point>& point_pool;

class point_factory
{
private:
    unsigned pick_constructor(Data data)
    {
        // same as before but also allow the copy
        // constructor to be picked if the data says so
    }

    double extract(Data data, string param);
public:
    point create(Data data)
    {
        switch(pick_constructor(data))
        {
            // same as before, but add a new case
            // returning copies from the pool

            case 3: return point_pool.get(data);

            default: throw exception(...);
        }
    }
};
\end{minted}

When looking at the hand-coded factories above, it is obvious that
implementing and maintaining\footnote{as the constructed types evolve and change}
factories for several dozens of classes in a larger
application is a highly repetitive, tedious and possibly error-prone
process and at least partial automation is desirable.

Factory classes must generally handle several tasks
which fit into two distinct and nearly orthogonal categories:

\begin{itemize}
\item{\emph{Product type-related}}
        \begin{itemize}
        \item{\emph{Constructor description}} -- providing the metadata describing
        the individual constructors, their parameters, etc.
        \item{\emph{Constructor dispatching}} -- calling the selected constructor.
        with the supplied arguments which results in a new instance of the product type.
        \end{itemize}

\item{\emph{Input data representation-related}}
        \begin{itemize}
        \item{\emph{Input data validation}} -- checking if the input data match
        the available constructors.
        \item{\emph{Constructor selection}} -- examining the input data, comparing it
        to the metadata describing product's constructors and determining
        which constructor should be called.
        \item{\emph{Getting the argument values}} -- determining where the argument
        values should come from and getting them:
                \begin{itemize}
                \item{\emph{Conversion from the external representation}} -- this usually applies
                to intrinsic C++ types, but complex types could be converted directly too.
                \item{\emph{Recursive construction by using another factory}} -- this usually
                requires some form of cooperation between the parent and its child factories
                and it means that all the tasks discussed here must be repeated also for
                the recursively constructed parameter(s).
                \item{\emph{Copying an existing instance}} -- for example from an object pool.
                \end{itemize}
        \end{itemize}
\end{itemize}

Parts from each category can be combined with parts from the other to create new
factories which promotes code re-usability. Factories constructing instances of a single
product from various data representations share the product-related components
and factories constructing instances of various product types from a single input data representation
share the input-data-related parts. This approach has several advantages like better
maintainability or the ability to develop the components separately and combine them later
via metaprogramming.

If the input data for a metaprogram generating the factory class, that is
the metadata describing the Product type\footnote{specifically the constructors of Product}
can be obtained by using compile-time reflection then new factory classes can be generated
automatically for nearly arbitrary type provided that the input data type-related parts are
implemented.

The scope of this paper does not allow to fully explain the implementation
of the factory generators. Please see \cite{Chochlik-CAI} for further details.
