\section{Reflection operator}
\label{section-reflection-operator}

The metaobjects reflecting some program feature \verb@X@ as
described above should be made available to the user by
the means of a new operator \texttt{reflexpr}.
More precisely, the reflection operator should return
a type\footnote{Like the \texttt{decltype} operator} conforming to a particular
metaobject concept, depending on the reflected expression.

The reflected expression \verb@X@ in the items listed above can at the moment
be any of the following\footnote{Future proposals may add new reflectible expressions,
like variables, overloaded functions, constructors, etc.}:

\begin{itemize}
\item{No expression or \verb@::@} -- The global scope, the returned metaobject is a {\meta{GlobalScope}}.
\item{{\em Namespace name}} -- (\verb@std@) the returned metaobject is a {\meta{Namespace}}.
\item{{\em Namespace alias name}} -- the returned metaobject is a {\meta{NamespaceAlias}}.
\item{{\em Type name}} -- (\verb@long double@) the returned metaobject is a {\meta{Type}}.
\item{{\em Typedef name}} -- (\verb@std::size_t@ or \verb@std::string@)
     the returned metaobject is a {\meta{TypeAlias}}.
\item{{\em Class name}} -- (\verb@std::thread@ or \verb@std::map<int, double>@)
     the returned metaobject is a {\meta{Class}}.
\item{{\em Enum name}} -- 
     the returned metaobject is a {\meta{Enum}}.
\item{{\em Enum class name}} -- (\verb@std::launch@)
     the returned metaobject is a {\meta{EnumClass}}.
\end{itemize}

The reflection operator should have access to \verb@private@ and
\verb@protected@ members of classes.

