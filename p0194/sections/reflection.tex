\section{Reflection operator}
\label{section-reflection-operator}

The metaobjects reflecting some program declaration \verb@X@ 
should be made available to the user by
the means of a new operator \texttt{reflexpr}.
More precisely, the reflection operator should return
a type\footnote{Like the \texttt{decltype} operator} conforming to a particular
metaobject concept, depending on the argument.

The operand \verb@X@ of the \texttt{reflexpr} operator can at the moment
be any of the following\footnote{Future proposals may add new operands
like variables, overloaded functions, constructors, etc.}:

\begin{itemize}
\item{No expression or \verb@::@} -- The global scope, the returned metaobject is a {\meta{GlobalScope}}.
\item{{\em Namespace name}} -- (\verb@std@) the returned metaobject is a {\meta{Namespace}}.
\item{{\em Namespace alias name}} -- the returned metaobject is a {\meta{NamespaceAlias}}.
\item{{\em Type name}} -- (\verb@long double@) the returned metaobject is a {\meta{Type}}.
\item{{\em Typedef name}} -- (\verb@std::size_t@ or \verb@std::string@)
     the returned metaobject is a {\meta{TypeAlias}}.
\item{{\em Class name}} -- (\verb@std::thread@ or \verb@std::map<unsigned, float>@)
     the returned metaobject is a {\meta{Class}}.
\item{{\em Enum name}} -- 
     the returned metaobject is a {\meta{Enum}}.
\item{{\em Enum class name}} -- (\verb@std::launch@)
     the returned metaobject is a {\meta{EnumClass}}.
\end{itemize}

\subsection{Redeclarations}

The meta data queried by \texttt{reflexpr} depends on the point of invocation
of the operator; only declarations in front of the invocation will be visible.
Subsequent invocations are independent of prior invocations, as if all compiler
generated types were unique to each reflexpr invocation.

\begin{minted}{cpp}
struct foo;
using meta_foo_fwd1 = reflexpr(foo);
constexpr size_t n1 = get_size_v<get_data_members_t<meta_foo_fwd1>>; // 0

struct foo;
using meta_foo_fwd2 = reflexpr(foo);
constexpr size_t n2 = get_size_v<get_data_members_t<meta_foo_fwd2>>; // 0

constexpr bool b1 = is_same_v<meta_foo_fwd1, meta_foo_fwd2>; // unspecified
constexpr bool b2 = meta::reflects_same_v<meta_foo_fwd1, meta_foo_fwd2>; // true

struct foo { int a,b,c,d; };
using meta_foo = reflexpr(foo);
constexpr size_t n3 = get_size_v<get_data_members_t<meta_foo>>; // 4

constexpr bool b3 = is_same_v<meta_foo_fwd1, meta_foo>; // false
constexpr bool b4 = meta::reflects_same_v<meta_foo_fwd1, meta_foo>; // true
\end{minted}

\subsection{Required header}
\label{section-reflexpr-header}

The standard \texttt{<reflexpr>} header \emph{must} be included before
the \emph{reflexpr} operator can be used\footnote{Similar to the
\texttt{<typeinfo>} header being required for the \texttt{typeid} operator.}.
This header file defines the metaobject trait templates described above
and the templates implementing the \emph{interface} of the metaobjects.

\subsection{Future extensions}

In a future proposal an additional, new variant of the \texttt{reflexpr} operator
-- \texttt{reflexpr(X, Concept)} could be introduced.
This operator would return a pack of \meta{Object}s reflecting members of
the base-level declaration \texttt{X} which conform to the specified
\texttt{Concept}. The parameter \texttt{X} is the same as in the single argument
version, \texttt{Concept} is the name of a metaobject concept.

For example:

\begin{minted}[tabsize=4]{cpp}
typedef std::tuple<reflexpr(std, std::meta::Typedef)...> meta_std_typedefs;
\end{minted}
