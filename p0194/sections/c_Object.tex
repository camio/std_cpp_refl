\subsection{Meta-Object}
\label{concept-Meta-Object}

A \meta{Object} is a stateless anonymous type generated by the compiler
(at the request of the programmer via the invocation of the
\hyperref[section-reflection-operator]{\texttt{reflexpr} operator}),
providing metadata reflecting a specific program declaration.

Construction of metaobject instances is not required at this moment, i.e. metaobjects
do not need to have any constructors.
If the need for instantiation of metaobject types arises in practical use,
then this requirement can be added in the future.

\subsubsection{\texttt{is\_metaobject}}

In order to distinguish between regular types and metaobjects generated
by the compiler, the \texttt{is\_metaobject} trait should be added
to the \texttt{std} namespace as one of the type traits. 

\begin{minted}{cpp}
template <typename T>
struct is_metaobject
{
	typedef bool value_type;
	static constexpr const bool value = /*
		true: if T is a metaobject
		false: otherwise
	*/

	typedef integral_constant<bool, value> type;

	operator value_type (void) const noexcept;
	value_type operator(void) const noexcept;
};

template <typename T>
using is_metaobject_t = typename is_metaobject<T>::type;

template <typename T>
constexpr bool is_metaobject_v = is_metaobject<T>::value;
\end{minted}

The expression \texttt{is\_metaobject<X>::value} should evaluate to \texttt{true}
if \texttt{X} is a metaobject generated by the compiler, otherwise it should
be \texttt{false}.

There are also several other trait templates nested in the namespace
\texttt{std::meta} which provide further information about metaobjects.
These traits are listed below together with their related metaobject concepts.



\subsubsection{Definition}

\begin{minted}{cpp}
namespace meta {
\end{minted}
\begin{minted}[xleftmargin=1em,tabsize=4,breakafter=&]{cpp}
template <typename T>
concept bool Object = is_metaobject_v<T>;

\end{minted}
\begin{minted}{cpp}
} // namespace meta
\end{minted}


The following operations are defined for each type satisfying the \meta{Object}
concept.



\subsubsection{\texttt{get\_source\_location}}

returns the source location info of the declaration of a base-level program feature reflected by a \meta{Object}.

\begin{minted}{cpp}
namespace meta {
\end{minted}
\begin{minted}[xleftmargin=1em,tabsize=4]{cpp}
template <Object T>
struct get_source_location : source_location { };

\end{minted}
\begin{minted}{cpp}
} // namespace meta
\end{minted}



The returned instance of \texttt{std::source\_location} should be \texttt{constexpr},
and even the source file name and function name strings should be compile-time constants.

The source information for built-in types and other
such implicit declarations which are declared internally by the compiler
should return an empty string as the function name and source file path and
zero as the source file line and column.
