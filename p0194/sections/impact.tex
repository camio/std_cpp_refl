\section{Impact on the standard}

This proposal requires the addition of a new operator
\texttt{reflexpr} which {\em may} cause conflicts with identifiers in existing code.

For what it's worth, we have performed a quick analysis on \num{994} third-party, open-source
repositories hosted on \url{http://github.com/}\footnote{The main branches
of original repositories, not forks.}, where we counted identifiers
in the C++ source files. We have found \num{646313149} instances of
\num{7903042} {\em distinct} words matching the C++ identifier rules.
We did not find {\em any} occurence of \say{reflexpr}.

No other changes to the core language are required, especially no new rules
for template parameters. The metaobjects can be implemented by (basically) regular
types and the interface is implemented by regular templates.
An implementation with lazy evaluation may require the addition of several compiler builtin
operators similar to those already used for the type-traits, but other
implementations are possible.

\subsection{Alternative spelling of the operator}

Alternatively the \texttt{constexpr(X)} or \texttt{std::reflexpr(X)} syntax could 
be used instead of unscoped \texttt{reflexpr(X)} to further minimise possible name
conflicts with existing identifiers.
