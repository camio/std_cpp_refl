\subsection{Type reflection}

\begin{minted}[tabsize=4]{cpp}
#include <reflexpr>
#include <iostream>

int main(void)
{
	using namespace std;

	typedef reflexpr(unsigned) meta_unsigned;

	static_assert(is_metaobject_v<meta_unsigned>, "");
	static_assert(meta::is_type_v<meta_unsigned>, "");
	static_assert(!meta::is_alias_v<meta_unsigned>, "");

	static_assert(is_same<
		meta::get_reflected_type_t<meta_unsigned>,
		unsigned
	>::value, "");

	static_assert(meta::has_name_v<meta_unsigned>, "");
	cout << meta::get_name_v<meta_unsigned> << endl;

	typedef reflexpr(unsigned*) meta_ptr_unsigned;
	static_assert(meta::has_name_v<meta_ptr_unsigned>, "");
	cout << meta::get_name_v<meta_ptr_unsigned> << endl;

	return 0;
}
\end{minted}

Output:

\begin{verbatim}
unsigned int
unsigned int
\end{verbatim}

Note that the \texttt{get\_name} operation returns the base type name without
any qualifiers, asterisks\footnote{in case of pointers},
ampersands\footnote{in case of references}, angle or square brackets\footnote{
in case of templates or arrays}, etc.
