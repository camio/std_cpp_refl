\section{Introduction}

In this paper we propose to add native support for
compile-time reflection to C++ by the means of compiler generated
types providing basic metadata describing various program declarations.
This paper introduces a new reflection operator and the initial subset
of metaobject concepts which we assume to be essential
and which will provide a good starting point for future extensions.

When finalized, these metaobjects, together with some additions to the standard
library can later be used to implement other third-party libraries
providing both compile-time and run-time, high-level reflection utilities.

Please refer to the previous papers \cite{n3996,n4111,n4451,n4452}
for the motivation for this proposal, including some examples,
design considerations and rationale. In \cite{ITFPWTHOR} a complex use-case
for a static reflection is described.

\subsection{Revision history}

\begin{itemize}
\item{\textbf{Revision 4}} -- Further refines the concepts from N4111; prefixes
the names of the metaobject operations with \texttt{get\_}, adds new operations,
replaces the metaobject category tags with new metaobject traits.
Introduces a nested namespace \texttt{std::meta} which contains most
of the reflection-related additions to the standard library.
Rephrases definition of meta objects using Concepts Lite. Specifies the
reflection operator name -- \texttt{reflexpr}.
Introduces an experimental implementation of the reflection operator in clang.
Drops the context-dependent reflection from N4111 (will be re-introduced later).

\item{\textbf{Revision 3} (N4451 \cite{n4451})} -- Incorporates the feedback from the discussion
about N4111 at the Urbana meeting, most notably reduces the set of metaobject concepts and refines their
definitions, removes some of the additions to the standard library added in the previous revisions.
Adds context-dependent reflection.

\item{\textbf{Revision 2} (N4111 \cite{n4111})} -- Refines the metaobject concepts and introduces
a concrete implementation of their interface by the means of templates similar
to the standard type traits. Describes some additions to the standard library
(mostly meta-programming utilities), which simpilfy the use of the metaobjects.
Answers some questions from the discussion about N3996 and expands the design
rationale.

\item{\textbf{Revision 1} (N3996 \cite{n3996})} -- Describes the method of static reflection
by the means of compiler-generated anonymous types. Introduces the first version
of the metaobject concepts and some possiblities of their implementation.
Also includes discussion about the motivation and the design rationale for the proposal.
\end{itemize}
