\subsubsection{\texttt{is\_metaobject}}

In order to distinguish between regular types and metaobjects generated
by the compiler, the \texttt{is\_metaobject} trait should be added
to the \texttt{std} namespace as one of the type traits. 

\begin{minted}{cpp}
template <typename T>
struct is_metaobject {
	typedef integral_constant<bool, value> type;
	typedef bool value_type;
	static constexpr const bool value;

	operator bool(void) const noexcept;
	bool operator(void) const noexcept;
};

template <typename T>
using is_metaobject_t = typename is_metaobject<T>::type;
template <typename T>
constexpr bool is_metaobject_v = is_metaobject<T>::value;
\end{minted}

The expression \texttt{is\_metaobject<X>::value} should evaluate to \texttt{true}
if \texttt{X} is a metaobject generated by the compiler, otherwise it should
be \texttt{false}.

The \texttt{is\_metaobject} trait should be defined in the standard
\texttt{<type\_traits>} header.

There are also several other trait templates defined in the nested namespace
\texttt{std::meta} which provide further information about metaobjects.
These traits are listed below together with their related metaobject concepts
and should be defined in a new
\hyperref[section-reflexpr-header]{\texttt{<reflexpr>} header} file.
