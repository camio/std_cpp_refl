\subsection{MetaClassMember}
\label{concept-MetaClassMember}

\begin{tikzpicture}
\node [concept] (Metaobject) {Metaobject};
\node [concept] (MetaNamed)[above right=of Metaobject] {MetaNamed}
	edge [inheritance, bend right] (Metaobject);
\node [concept] (MetaScoped)[below right=of Metaobject] {MetaScoped}
	edge [inheritance, bend left] (Metaobject);
\node [concept] (MetaNamedScoped)[below right=of MetaNamed, above right=of MetaScoped] {MetaNamedScoped}
	edge [inheritance, bend right] (MetaNamed)
	edge [inheritance, bend left] (MetaScoped);
\node [concept] (MetaClassMember)[right=of MetaNamedScoped] {MetaClassMember}
	edge [inheritance] (MetaNamedScoped);
\end{tikzpicture}

\meta{Class} is a \meta{Type} and a \meta{Scope} if reflecting a regular class or possibly
also a \meta{Template} if it reflects a class template.

In addition to the requirements inherited from \meta{NamedScoped},
the following is required for \meta{ClassMember}s:

The \verb@is_class_member@ template class specialization for a \meta{ClassMember} should
inherit from \verb@true_type@:

\begin{minted}{cpp}
template <>
struct is_class_member<MetaClassMember>
 : true_type
{ };
\end{minted}

\subsubsection{\texttt{access\_specifier}}

A template class called \verb@access_specifier@ should be defined and should inherit from
a \meta{Specifier} reflecting the \verb@private@, \verb@protected@ or \verb@public@
access specifier:

\begin{minted}{cpp}
template <typename T>
struct access_specifier;

template <>
struct access_specifier<MetaClassMember>
 : MetaSpecifier
{ };
\end{minted}

