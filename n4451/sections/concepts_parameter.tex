\subsection{MetaParameter}
\label{concept-MetaParameter}

\begin{tikzpicture}
\node [concept] (Metaobject) {Metaobject};
\node [concept] (MetaNamed)[above right=of Metaobject] {MetaNamed}
	edge [inheritance, bend right] (Metaobject);
\node [concept] (MetaScoped)[below right=of Metaobject] {MetaScoped}
	edge [inheritance, bend left] (Metaobject);
\node [concept] (MetaPositional)[below=of MetaScoped] {MetaPositional}
	edge [inheritance, bend left] (Metaobject);
\node [concept] (MetaParameter)[below right=of MetaNamed] {MetaParameter}
	edge [inheritance, bend right] (MetaNamed)
	edge [inheritance, bend left] (MetaScoped)
	edge [inheritance, bend left] (MetaPositional);
\end{tikzpicture}

\meta{Parameter} is a \meta{Named}, \meta{Scoped} and \meta{Positional},
reflecting a function parameter or a parameter pack.

In addition to the requirements inherited from \meta{Named} and \meta{Scoped},
the following must be satisfied:

The \verb@metaobject_category@ template class specialization for a \meta{Parameter} should
inherit from \verb@parameter_tag@:

\begin{minted}{cpp}
template <>
struct metaobject_category<MetaParameter>
 : parameter_tag
{ };
\end{minted}

The \verb@scope@ of a \meta{Parameter} should be a \meta{Function} reflecting
the function to which the parameter belongs:

\begin{minted}{cpp}
template <>
struct scope<MetaParameter>
 : MetaFunction
{ };
\end{minted}

The \verb@full_name@ inherited from \meta{Named} should return the same \concept{StringConstant}
as \verb@base_name@ for models of \meta{Parameter}, i.e. the plain parameter
name without any qualifications.

\subsubsection{\texttt{is\_pack}}

A template class called \verb@is_pack@ should be defined and should
inherit from \verb@true_type@ if the parameter is a part of an expanded
parameter pack or from \verb@false_type@ otherwise.

\begin{minted}{cpp}
template <typename T>
struct is_pack;

template <>
struct is_pack<MetaParameter>
 : integral_constant<bool, B>
{ };
\end{minted}

When a concrete instantiation of a function template with a parameter pack
is reflected then the individual \meta{Parameter}s reflect the actual parameters
of that instantiation. In such case the \verb@parameters@ template "returns"
a \concept{MetaobjectSequence} of \meta{Parameter}s where some of the \meta{Parameter}
have the same name (the name of the pack template parameter)
and \verb@is_pack@ inherits from \verb@true_type@.

\subsubsection{\texttt{type}}

A template class \verb@type@ should be added and should inherit
from a \meta{Type} reflecting the type of the parameter:

\begin{minted}{cpp}
template <typename T>
struct type;

template <>
struct type<MetaParameter>
 : MetaType
{ };
\end{minted}

\subsubsection{\texttt{pointer}}

A template class \verb@pointer@ should be implemented as follows:

\begin{minted}{cpp}
template <typename T>
struct pointer;

template <>
struct pointer<MetaParameter>
{
	typedef typename original_type<type<MetaParameter>>::type* type;

	static type get(void);
};
\end{minted}

The static member function \verb@get@ should return the address of the reflected parameter
instance if it is invoked (directly or indirectly) inside of the body of the function to which
the reflected parameter belongs to. Otherwise it should return \verb@nullptr@.

In case of recursively called functions, pointers to the arguments of the innermost
invocation should be returned.
