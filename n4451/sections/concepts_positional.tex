\subsection{MetaPositional}
\label{concept-MetaPositional}

\begin{tikzpicture}
\node [concept] (Metaobject) {Metaobject};
\node [concept] (MetaPositional)[right=of Metaobject] {MetaPositional}
	edge [inheritance] (Metaobject);
\end{tikzpicture}

\meta{Positional} is a \meta{object}, which is fixed to an ordinal position in some context
usually together with other similar metaobjects, like those reflecting parameters of a function,
or the inheritance of base classes, etc.

In addition to the requirements inherited from \meta{object},
the following must be satisfied:

The \verb@has_position@ template class specialization for a \meta{Positional} should
inherit from \verb@true_type@:

\begin{minted}{cpp}
template <>
struct has_position<MetaPositional>
 : true_type
{ };
\end{minted}

\subsubsection{\texttt{position}}

A template class \verb@position@ should be defined and should
inherit from \verb@integral_constant<size_t, I>@ type where \verb@I@ is
a zero-based position (index) of the reflected base-level language construct,
for example the position of a parameter in a list of function parameters,
or the position of an inheritance clause in the list of base classes.

\begin{minted}{cpp}
template <typename T>
struct position;

template <>
struct position<MetaPositional>
 : integral_constant<size_t, I>
{ };
\end{minted}

