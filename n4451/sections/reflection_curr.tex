\section{Reflection operator}
\label{section-reflection-operator}

The metaobjects reflecting some program feature \verb@X@ as
described above should be made available to the user by
the means of a new operator or expression.
More precisely, the reflection operator should return a type conforming to a particular
metaobject concept, depending on the reflected expression.

Since adding a new keyword has the potential to break existing code,
we do not insist on any particular expression, here follows a list of suggestions
in order of preference (from the most to the least preferrable):

\begin{itemize}
\item{\verb@mirrored(X)@}
\item{\verb@reflected(X)@}
\item{\verb@reflexpr(X)@}
\item{\verb@constexpr(X)@}
\item{\verb@|X@}
\item{\verb@[[X]]@}
\item{\verb@<<X>>@}
\end{itemize}

The reflected expression \verb@X@ in the items listed above can be any of the following:

\begin{itemize}
\item{\verb@::@} -- The global scope, the returned metaobject is a {\meta{GlobalScope}}.
\item{{\em Namespace name}} -- (\verb@std@) the returned metaobject is a {\meta{Namespace}}.
\item{{\em Type name}} -- (\verb@long double@) the returned metaobject is a {\meta{Type}}.
\item{{\em \verb@typedef@ name}} -- (\verb@std::size_t@ or \verb@std::string@)
     the returned metaobject is a {\meta{Typedef}}.
\item{{\em Template name}} -- (\verb@std::tuple@ or \verb@std::map@)
     the returned metaobject is a {\meta{Template}}.
\item{{\em Class name}} -- (\verb@std::thread@ or \verb@std::map<int, double>@)
     the returned metaobject is a {\meta{Class}}.
\end{itemize}

The reflection operator or expression should have access to \verb@private@ and
\verb@protected@ members of classes. The following should be valid:

\begin{minted}{cpp}
struct A
{
	int a;
};

class B
{
protected:
	int b;
};

class C
 : protected A
 , public B
{
private:
	int c;
};

typedef mirrored(A::a) meta_A_a;
typedef mirrored(B::b) meta_B_b;
typedef mirrored(C::a) meta_C_a;
typedef mirrored(C::b) meta_C_b;
typedef mirrored(C::c) meta_C_c;

\end{minted}

\subsection{Context-dependent reflection}
\label{section-context-dependent-reflection}

We also propose to define a set of special expressions that can be used
inside of the reflection operator, to obtain metadata based on the context
where it is invoked, instead of the identifier.

\subsubsection{Namespaces}

If the \verb@this::namespace@ expression is used as the argument of the reflection
operator, then it should return a \meta{Namespace} reflecting the namespace
inside of which the reflection operator was invoked.

For example:

\begin{minted}{cpp}

typedef mirrored(this::namespace) _meta_gs;

\end{minted}

reflects the global scope namespace and is equivalent to

\begin{minted}{cpp}

typedef mirrored(::) _meta_gs;

\end{minted}

For named namespaces:

\begin{minted}{cpp}

namespace foo {

typedef mirrored(this::namespace) _meta_foo;

namespace bar {

typedef mirrored(this::namespace) _meta_foo_bar;

} // namespace bar

} // namespace foo
\end{minted}

\subsubsection{Classes}

If the \verb@this::class@ expression is used as the argument of the reflection
operator, then it should return a \meta{Class} reflecting the class
inside of which the reflection operator was invoked.

For example:

\begin{minted}{cpp}

struct foo
{
	const char* _name;

	// reflects foo
	typedef mirrored(this::class) _meta_foo1;

	foo(void)
	 : _name(base_name<mirrored(this::class)>())
	{ }

	void f(void)
	{
		// reflects foo
		typedef mirrored(this::class) _meta_foo2;
	}

	double g(double, double);

	struct bar
	{
		// reflects foo::bar
		typedef mirrored(this::class) _meta_foo_bar;
	};
};

double foo::g(double a, double b)
{
	// reflects foo
	typedef mirrored(this::class) _meta_foo3;
	return a+b;
}

class baz
{
private:
	typedef mirrored(this::class) _meta_baz;
};

typedef mirrored(this::class); // <- error: not used inside of a class.

\end{minted}

