\section{Introduction}

Reflection and reflective programming can be used
for a wide range of tasks, such as the implementation
of serialization-like operations, object-relational mapping,
remote procedure calls, scripting, automated GUI-generation,
implementation of several software design patterns, etc.
C++ as one of the most prevalent programming languages 
lacks a standardized reflection facility.

In this paper we propose to add native support for
compile-time reflection to C++ by the means of compiler generated
types providing basic metadata describing various program constructs.
These metaobjects, together with some additions to the standard
library can later be used to implement other third-party libraries
providing both compile-time and run-time high-level
reflection utilities.

The basic static metadata provided by compile-time reflection
should be as complete as possible to be applicable in a wide
range of scenarios and allow to implement custom higher-level
static and dynamic reflection libraries and reflection-based
utilities.

\subsection{Differences between N3996, N4111 and N4451}

This proposal (N4451) is a revision of N4111, with several modification
listed below:

\begin{itemize}
\item \verb@position@ has been added to the definition of the \meta{Inheritance} concept.
\item \verb@for_each@ has been added to the definition of the \meta{objectSequence} concept.
\item TODO
\end{itemize}
