\subsection{Metaobject Sequence}
\label{concept-MetaobjectSequence}

As the name implies \concept{MetaobjectSequence}s are used to store seqences
or tuples of metaobjects.
A model of \concept{MetaobjectSequence} is a class conforming to the following:

It is a nullary metafunction returning itself:

\begin{minted}{cpp}
template <typename Metaobject>
struct MetaobjectSequence
{
	typedef MetaobjectSequence type;
};
\end{minted}

\textbf{Note:} The definition above is only a psedo-code and
the template parameter \verb@Metaobject@ indicates here the minimal
metaobject concept which all elements in the sequence must satisfy.
For example a \verb@MetaobjectSequence<MetaConstructor>@ denotes a sequence
of metaobjects that all satisfy the \meta{Constructor} concept, etc.

\subsubsection{\texttt{size}}

A template class \verb@size@ is defined as follows:

\begin{minted}{cpp}
template <typename T>
struct size;

template <>
struct size<MetaobjectSequence<Metaobject>>
 : integral_type<size_t, N>
{ };
\end{minted}

Where \verb@N@ is the number of elements in the sequence.

\subsubsection{\texttt{at}}

A template class \verb@at@, providing random access to metaobjects in a sequence
is defined (for values of \verb@I@ between \verb@0@ and \verb@N-1@ where \verb@N@
is the number of elements returned by \verb@size@) as follows:

\begin{minted}{cpp}
template <typename T, size_t I>
struct at;

template <size_t I>
struct at<MetaobjectSequence<Metaobject>, I>
 : Metaobject
{ };
\end{minted}

For example if \verb@__meta_seq_ABC@ is a metaobject sequence containing three metaobjects;
\verb@__meta_A@, \verb@__meta_B@ and \verb@__meta_C@ (in that order), then:

\begin{minted}{cpp}
template <>
struct size<__meta_seq_ABC>
 : integral_constant<size_t, 3>
{ };
\end{minted}

and 

\begin{minted}{cpp}
template <>
struct at<__meta_seq_ABC, 0>
 : __meta_A
{ };

template <>
struct at<__meta_seq_ABC, 1>
 : __meta_B
{ };

template <>
struct at<__meta_seq_ABC, 2>
 : __meta_C
{ };
\end{minted}

\textbf{Note:} The order of the metaobjects in a sequence is determined by the order
of appearance in the processed translation unit.

\subsubsection{\texttt{for\_each}}

A template function \verb@for_each@ should be defined and should
execute the specified unary function on every \meta{object} in
the sequence, in the order of appearance in the processed translation unit.

\begin{minted}{cpp}
template <typename MetaobjectSequence, typename UnaryFunc>
void for_each(UnaryFunc func)
{
	func(Metaobject1());
	func(Metaobject2());
	/* ... */
	func(MetaobjectN());
}

\end{minted}

\textbf{Note:} The interface of \meta{objectSequence}, in particular \verb@size@
and \verb@at@ together with \verb@std::index_sequence@ should be enough
to implement a single generic version of \verb@for_each@.

