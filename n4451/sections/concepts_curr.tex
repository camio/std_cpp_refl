\section{Currently proposed metaobject concepts}
\label{section-current-Concepts}

We propose that the basic metadata describing a program written
in C++ should be made available through a set of {\em anonymous} types
defined by the compiler and through related functions and template classes.
At the moment these types should describe only the following program
constructs: namespaces\footnote{in a very limited form}, types, typedefs,
classes and their data members.

In the future, the set of metaobjects should be extended to reflect also
class inheritance, free functions, class member functions, templates,
template parameters, enumerated values, specifiers, etc.
See appendix~\ref{section-all-Concepts} for more details.

The compiler should generate metadata for the program constructs defined
in the currently processed translation unit. Members of ordered sets (ranges) of metaobjects,
like scope members, parameters of a function, and so on, should be listed
in the order of appearance in the processed source code.

Since we want the metadata to be available at compile-time,
different base-level constructs should be reflected by
{\em statically different} metaobjects and thus by {\em different} types.
For example a metaobject reflecting the global scope namespace should
be a different {\em type} than a metaobject reflecting the \verb@std@
namespace\footnote{this means that they should be distinguishable for
example by the \texttt{std::is\_same} type trait},
a metaobject reflecting the \verb@int@ type should
have a different type then a metaobject reflecting the \verb@double@
type, etc.

In a manner of speaking these metaobjects should become
"instances" of the meta-level concepts\footnote{conceptual interfaces which
describe the requirements of types modelling them, but
should not exist as concrete types}, but rather only at the
"specification-level" similar for example to the iterator concepts.

This section describes a set of metaobject concepts,
their interfaces\footnote{the requirements that the various metaobjects
need to satisfy in order to be considered models of the individual
concepts},
tag types for metaobject classification and
functions (or operators) providing access to the metaobjects.

Unless stated otherwise, all named templates proposed and described below should
go into the \verb@std@ namespace. Alternatively, if any of the definitions
proposed here would clash with existing members (or new members proposed elsewhere)
of the \verb@std@ namespace, then they can be nested in a namespace like
\verb@std::meta@ or \verb@std::mirror@.

Also note, that in the sections below, the examples use names for concrete
metaobjects, like \verb@__meta_std_string@, etc. This convention
is {\em NOT} part of this proposal. The actual naming of the metaobjects
should be left to the compiler implementations and for all purposes,
from the users point of view, the metaobjects should be anonymous types.

\subsection{Categorization and Traits}

In order to provide means for distinguishing between regular types
and metaobjects generated by the compiler,
the \verb@is_metaobject@ trait should be added to the standard \verb@type_traits@ library
and should inherit from \verb@true_type@ for metaobjects\footnote{types generated
by the compiler providing metadata} and from \verb@false_type@
for non-metaobjects\footnote{native or user defined types}:

\begin{minted}{cpp}
template <typename T>
struct is_metaobject
 : false_type
{ };
\end{minted}

\subsubsection{Metaobject category tags}
\label{metaobject-category-tags}

To distiguish between various metaobject kinds\footnote{metaobjects satisfying different concepts
as described below} a set of tag \verb@struct@s indicating the metaobject kind
should be added:

\begin{minted}{cpp}

struct namespace_tag { };

struct global_scope_tag { };

struct type_tag { };

struct class_tag { };

struct enum_tag { };

struct enum_class_tag { };

struct variable_tag { };

\end{minted}

These tags are referred-to as \verb@MetaobjectCategory@:


\subsection{StringConstant}
\label{concept-StringConstant}

A \concept{StringConstant} is a class conforming to the following:

\begin{minted}{cpp}
struct StringConstant
{
	typedef StringConstant type;

	// null terminated char array with size (string length+1)
	// known to sizeof at compile-time
        static constexpr const char value[Length+1] = {..., '\0'};

	// implicit compile-time conversion to null terminated
	// c-string
        constexpr operator const char* (void) const
        {
                return value;
        }
};

constexpr const char StringConstant::value[Length+1];
\end{minted}

Concrete models of \concept{StringConstant} are used to return compile-time string values.
For example the \verb@_str_void@ type defined below, conforms to the \concept{StringConstant}
concept:

\begin{minted}{cpp}
template <char ... C>
struct string_constant
{
	typedef string_constant type;

	static constexpr const char value[sizeof...(C)+1] = {C...,'\0'};

	constexpr operator const char* (void) const
	{
		return value;
	}
};

template <char ... C>
constexpr const char string_constant::value[sizeof...(C)+1];

//...

typedef string_constant<'v','o','i','d'> _str_void;

cout << _str_void::value << std::endl;
cout << _str_void() << std::endl;
static_assert(sizeof(_str_void::value) == 4+1, "");
\end{minted}

The strings stored in the \verb@value@ array should be UTF-8 encoded.

\textbf{Note:} the \verb@string_constant@ class as defined above is just one of the
possible implementations of \concept{StringConstant}, we do not however imply
that it must be implemented this way.

\subsection{Metaobject Sequence}
\label{concept-MetaobjectSequence}

As the name implies \concept{MetaobjectSequence}s are used to store seqences
or tuples of metaobjects.
A model of \concept{MetaobjectSequence} is a class conforming to the following:

It is a nullary metafunction returning itself:

\begin{minted}{cpp}
template <typename Metaobject>
struct MetaobjectSequence
{
	typedef MetaobjectSequence type;
};
\end{minted}

\textbf{Note:} The definition above is only a psedo-code and
the template parameter \verb@Metaobject@ indicates here the minimal
metaobject concept which all elements in the sequence must satisfy.
For example a \verb@MetaobjectSequence<MetaConstructor>@ denotes a sequence
of metaobjects that all satisfy the \meta{Constructor} concept, etc.

\subsubsection{\texttt{size}}

A template class \verb@size@ is defined as follows:

\begin{minted}{cpp}
template <typename T>
struct size;

template <>
struct size<MetaobjectSequence<Metaobject>>
 : integral_type<size_t, N>
{ };
\end{minted}

Where \verb@N@ is the number of elements in the sequence.

\subsubsection{\texttt{at}}

A template class \verb@at@, providing random access to metaobjects in a sequence
is defined (for values of \verb@I@ between \verb@0@ and \verb@N-1@ where \verb@N@
is the number of elements returned by \verb@size@) as follows:

\begin{minted}{cpp}
template <typename T, size_t I>
struct at;

template <size_t I>
struct at<MetaobjectSequence<Metaobject>, I>
 : Metaobject
{ };
\end{minted}

For example if \verb@__meta_seq_ABC@ is a metaobject sequence containing three metaobjects;
\verb@__meta_A@, \verb@__meta_B@ and \verb@__meta_C@ (in that order), then:

\begin{minted}{cpp}
template <>
struct size<__meta_seq_ABC>
 : integral_constant<size_t, 3>
{ };
\end{minted}

and 

\begin{minted}{cpp}
template <>
struct at<__meta_seq_ABC, 0>
 : __meta_A
{ };

template <>
struct at<__meta_seq_ABC, 1>
 : __meta_B
{ };

template <>
struct at<__meta_seq_ABC, 2>
 : __meta_C
{ };
\end{minted}

\textbf{Note:} The order of the metaobjects in a sequence is determined by the order
of appearance in the processed translation unit.

\subsubsection{\texttt{for\_each}}

A template function \verb@for_each@ should be defined and should
execute the specified unary function on every \meta{object} in
the sequence, in the order of appearance in the processed translation unit.

\begin{minted}{cpp}
template <typename MetaobjectSequence, typename UnaryFunc>
void for_each(UnaryFunc func)
{
	func(Metaobject1());
	func(Metaobject2());
	/* ... */
	func(MetaobjectN());
}

\end{minted}

This design requires the metaobjects to be default constructible.
If this should prove to be a problem then an \verb@identity@ wrapper could be used
instead:

\begin{minted}{cpp}
template <typename X>
struct identity
{
	typedef X type;
};

template <typename MetaobjectSequence, typename UnaryFunc>
void for_each(UnaryFunc func)
{
	func(identity<Metaobject1>());
	func(identity<Metaobject2>());
	/* ... */
	func(identity<MetaobjectN>());
}

\end{minted}

\textbf{Note:} The interface of \meta{objectSequence}, in particular \verb@size@
and \verb@at@ together with \verb@std::index_sequence@ should be enough
to implement a single generic version of \verb@for_each@.


\subsection{Metaobject}
\label{concept-Metaobject}

A \meta{object} is a stateless anonymous type generated by the compiler which
provides metadata reflecting a specific program feature. Each metaobject
should satisfy the following:

Every metaobject should be a nullary metafunction returning itself:

\begin{minted}{cpp}
struct Metaobject
{
	typedef Metaobject type;
};
\end{minted}

One possible way how to achieve this is to define {\em basic metaobjects}
as plain types (without any internal structure) and define a class template like:

\begin{minted}{cpp}
template <typename BasicMetaobject>
struct metaobject
{
	typedef metaobject type;
};
\end{minted}

and then, implement the actual \meta{object}s as instantiations of this template.
For example if \verb@__base_meta_int@ is a basic metaobject reflecting the \verb@int@
type then the actual metaobject \verb@__meta_int@ conforming to this concept could 
be defined as:

\begin{minted}{cpp}
typedef metaobject<__base_meta_int> __meta_int;
\end{minted}

Although, this is just one possibility not a requirement of this proposal.

\subsubsection{\texttt{is\_metaobject}}

The \verb@is_metaobject@ template should inherit from \verb@true_type@ for all \meta{object}s,
and inherit from \verb@false_type@ otherwise.

\begin{minted}{cpp}
template <typename T>
struct is_metaobject
 : false_type
{ };

template <>
struct is_metaobject<Metaobject>
 : true_type
{ };
\end{minted}

\subsubsection{\texttt{metaobject\_category}}

A template class \verb@metaobject_category@ should be defined in the \verb@std@ namespace
(even if everything else is defined inside of a nested namespace like \verb@std::meta@)
and should inherit from
one of the \hyperref[metaobject-category-tags]{metaobject category tags}, depending on
the actual kind of the metaobject.

\begin{minted}{cpp}
template <typename T>
struct metaobject_category;

template <>
struct metaobject_category<Metaobject>
 : MetaobjectCategory
{ };
\end{minted}

For example if the \verb@__meta_std@ metaobject reflects the \verb@std@ namespace,
then the specialization of \verb@metaobject_category@ should be:

\begin{minted}{cpp}
template <>
struct metaobject_category<__meta_std>
 : namespace_tag
{ };
\end{minted}

\subsubsection{Traits}

The following template classes indicating various properties of a \meta{object}
should be defined and should by default inherit from \verb@false_type@ unless stated
otherwise below:

\verb@has_name@ -- indicates that a \meta{object} is a \meta{Named}:
\begin{minted}{cpp}
template <typename T>
struct has_name
 : false_type
{ };
\end{minted}

\verb@has_scope@ -- indicates that a \meta{object} is a \meta{Scoped}:
\begin{minted}{cpp}
template <typename T>
struct has_scope
 : false_type
{ };
\end{minted}

\verb@is_scope@ -- indicates that a \meta{object} is a \meta{Scope}:
\begin{minted}{cpp}
template <typename T>
struct is_scope
 : false_type
{ };
\end{minted}

\verb@has_name@ -- indicates that a \meta{object} is a \meta{Positional}:
\begin{minted}{cpp}
template <typename T>
struct has_position
 : false_type
{ };
\end{minted}

\verb@is_class_member@ -- indicates that a \meta{object} is a \meta{ClassMember}:
\begin{minted}{cpp}
template <typename T>
struct is_class_member
 : false_type
{ };
\end{minted}

\subsubsection{\texttt{source\_file}}

A template class \verb@source_file@ should be defined and should return the
path to the source file where the base-level construct reflected by a
metaobject is defined (similar to what the preprocessor \verb@__FILE__@ macro
expands to).

\begin{minted}{cpp}
template <typename T>
struct source_file;

template <>
struct source_file<MetaObject>
 : StringConstant
{ };
\end{minted}

For base-level constructs like namespaces which don't have a single specific
declaration, an empty string should be returned.

\subsubsection{\texttt{source\_line}}

A template class \verb@source_line@ should be defined and should return the (positive)
line number in the source file where the base-level construct reflected by a
metaobject is defined (similar to what the preprocessor \verb@__LINE__@ symbol
expands to).

\begin{minted}{cpp}
template <typename T>
struct source_line;

template <>
struct source_line<MetaObject>
 : integral_constant<unsigned, Line>
{ };
\end{minted}

For base-level constructs like namespaces which don't have a single specific
declaration, line number zero should be returned.


\subsection{MetaNamed}
\label{concept-MetaNamed}

\meta{Named} is a \meta{object} reflecting program constructs, which have a name
(are identified by an identifier) like namespaces, types, functions, variables, parameters, etc.

In addition to the requirements inherited from \meta{object}, the following requirements must
be satisfied:

The \verb@has_name@ template class specialization for a \meta{Named} should
inherit from \verb@true_type@:

\begin{minted}{cpp}
template <>
struct has_name<MetaNamed>
 : true_type
{ };
\end{minted}

\subsubsection{\texttt{base\_name}}

A template class \verb@base_name@ should be defined an should return the base name
of the reflected construct, without the nested name specifier nor any qualifications
or other decorations, as a
\concept{StringConstant}:

\begin{minted}{cpp}
template <typename T>
struct base_name;

template <>
struct base_name<MetaNamed>
 : StringConstant
{ };
\end{minted}

For example, if \verb@__meta_std_size_t@ reflects the \verb@std::size_t@ type,
then the matching specialization of \verb@base_name@ could be implemented in the following
way:

\begin{minted}{cpp}
template <>
struct base_name<__meta_std_size_t>
 : string_constant<'s','i','z','e','_','t'>
{ };
\end{minted}

where the \verb@string_constant<'s','i','z','e','_','t'>@ class is a model
of \concept{StringConstant} as described above.

For namespace \verb@std@ the value should be \verb@"std"@, for namespace
\verb@foo::bar::baz@ it should be \verb@"baz"@, for the global scope the
value should be an empty string.

For \verb@unsigned long int * const *@ it should be \verb@"unsigned long int"@.

For \verb@std::vector<int>::iterator@ it should be \verb@"iterator"@. For derived,
qualified types like \verb@volatile std::vector<const foo::bar::fubar*> * const *@
it should be \verb@"vector"@, etc. For the global scope, anonymous namespaces and types
an empty string should be returned.

\subsubsection{\texttt{full\_name}}

A template class \verb@full_name@ should be defined and should return the fully
qualified name of the reflected construct, including the nested name specifier
and all qualifiers.

For namespace \verb@std@ the value 
should be \verb@"std"@, for namespace \verb@foo::bar::baz@ the value should
be \verb@"foo::bar::baz"@, for the global scope the value should be an empty
\concept{StringConstant}.
For \verb@std::vector<int>::iterator@ it should be \verb@"std::vector<int>::iterator"@.
For derived qualified types like
\verb@volatile std::vector<const foo::bar::fubar*> * const *@ it should be defined as
\verb@"volatile std::vector<const foo::bar::fubar*> * const *"@, etc.
For the global scope, anonymous namespaces and types an empty string should be returned.

\begin{minted}{cpp}
template <typename T>
struct full_name;

template <>
struct full_name<MetaNamedScoped>
 : StringConstant
{ };
\end{minted}


\subsection{MetaScoped}
\label{concept-MetaScoped}

\meta{Scoped} is a \meta{object} reflecting program constructs defined inside
of a named scope (like the global scope, a namespace, a class, etc.\footnote{
and in the future an enum, an enum class etc.})

In addition to the requirements inherited from \meta{object}, the following requirements must
be satisfied:

The \verb@has_scope@ template class specialization for a \meta{Scoped} should
inherit from \verb@true_type@:

\begin{minted}{cpp}
template <>
struct has_scope<MetaScoped>
 : true_type
{ };
\end{minted}

\subsubsection{\texttt{scope}}

A template class \verb@scope@ should be defined and should inherit from the
\meta{Scope} which reflects the parent scope of the program construct reflected
by this \meta{Scoped}.

\begin{minted}{cpp}
template <typename T>
struct scope;

template <>
struct scope<MetaScoped>
 : MetaScope
{ };
\end{minted}


\subsection{MetaNamedScoped}
\label{concept-MetaNamedScoped}

\begin{tikzpicture}
\node [concept] (Metaobject) {Metaobject};
\node [concept] (MetaNamed)[above right=of Metaobject] {MetaNamed}
	edge [inheritance, bend right] (Metaobject);
\node [concept] (MetaScoped)[below right=of Metaobject] {MetaScoped}
	edge [inheritance, bend left] (Metaobject);
\node [concept] (MetaNamedScoped)[below right=of MetaNamed, above right=of MetaScoped] {MetaNamedScoped}
	edge [inheritance, bend right] (MetaNamed)
	edge [inheritance, bend left] (MetaScoped);
\end{tikzpicture}

Models of \meta{NamedScoped} combine the requirements of \meta{Named} and \meta{Scoped}.

\subsection{MetaScope}
\label{concept-MetaScope}

\meta{Scope} is a \meta{NamedScoped} reflecting program constructs defined inside
of a named or anonymous scope (like the global scope, a namespace, a class, etc.)

In addition to the requirements inherited from \meta{NamedScoped}, the following is required:

The \verb@is_scope@ template class specialization for a \meta{Scope} should
inherit from \verb@true_type@:

\begin{minted}{cpp}
template <>
struct is_scope<MetaScope>
 : true_type
{ };
\end{minted}

\subsubsection{\texttt{members}}

A template class \verb@members@ should be defined and should inherit from a
\concept{MetaobjectSequence} containing \meta{NamedScoped} metaobjects reflecting
the members of the base-level scope reflected by this \meta{Scope}.

Furthermore the boolean \verb@All@ parameter (defaulting to \verb@false@ if not specified)
should control which scope members are listed in the resulting metaobject sequence.

If \verb@All@ is 
\begin{itemize}
\item \verb@true@, then all members should be listed,
\item \verb@false@, then the following scope member should {\em not} be included:
	\begin{itemize}
	\item private or protected class members,
	\item scope members with identifiers starting with an underscore followed by a
	(uppercase or lowercase) letter,
	\item scope members annotated by a generalized attributes as hidden (for example
	\verb@[[mirror::hidden]]@ or \verb@[[reflection::hidden]]@).
	\end{itemize}
\end{itemize}



\begin{minted}{cpp}
template <typename T, bool All=false>
struct members;

template <>
struct members<MetaScope, true>
 : MetaobjectSequence<MetaNamedScoped> // all members
{ };

template <>
struct members<MetaScope, false>
 : MetaobjectSequence<MetaNamedScoped> // only 'visible' members
{ };
\end{minted}


\subsection{MetaPositional}
\label{concept-MetaPositional}

\meta{Positional} is a \meta{object}, which is fixed to an ordinal position in some context
usually together with other similar metaobjects, like those reflecting parameters of a function,
or the inheritance of base classes, etc.

In addition to the requirements inherited from \meta{object},
the following must be satisfied:

The \verb@has_position@ template class specialization for a \meta{Positional} should
inherit from \verb@true_type@:

\begin{minted}{cpp}
template <>
struct has_position<MetaPositional>
 : true_type
{ };
\end{minted}

\subsubsection{\texttt{position}}

A template class \verb@position@ should be defined and should
inherit from \verb@integral_constant<size_t, I>@ type where \verb@I@ is
a zero-based position (index) of the reflected base-level language construct,
for example the position of a parameter in a list of function parameters,
or the position of an inheritance clause in the list of base classes.

\begin{minted}{cpp}
template <typename T>
struct position;

template <>
struct position<MetaPositional>
 : integral_constant<size_t, I>
{ };
\end{minted}


\subsection{MetaClassMember}
\label{concept-MetaClassMember}

\begin{tikzpicture}
\node [concept] (Metaobject) {Metaobject};
\node [concept] (MetaNamed)[above right=of Metaobject] {MetaNamed}
	edge [inheritance, bend right] (Metaobject);
\node [concept] (MetaScoped)[below right=of Metaobject] {MetaScoped}
	edge [inheritance, bend left] (Metaobject);
\node [concept] (MetaNamedScoped)[below right=of MetaNamed, above right=of MetaScoped] {MetaNamedScoped}
	edge [inheritance, bend right] (MetaNamed)
	edge [inheritance, bend left] (MetaScoped);
\node [concept] (MetaClassMember)[right=of MetaNamedScoped] {MetaClassMember}
	edge [inheritance] (MetaNamedScoped);
\end{tikzpicture}

\meta{Class} is a \meta{Type} and a \meta{Scope} if reflecting a regular class or possibly
also a \meta{Template} if it reflects a class template.

In addition to the requirements inherited from \meta{NamedScoped},
the following is required for \meta{ClassMember}s:

The \verb@is_class_member@ template class specialization for a \meta{ClassMember} should
inherit from \verb@true_type@:

\begin{minted}{cpp}
template <>
struct is_class_member<MetaClassMember>
 : true_type
{ };
\end{minted}

\subsubsection{\texttt{access\_specifier}}

A template class called \verb@access_specifier@ should be defined and should inherit from
a \meta{Specifier} reflecting the \verb@private@, \verb@protected@ or \verb@public@
access specifier:

\begin{minted}{cpp}
template <typename T>
struct access_specifier;

template <>
struct access_specifier<MetaClassMember>
 : MetaSpecifier
{ };
\end{minted}


\subsection{MetaGlobalScope}
\label{concept-MetaGlobalScope}

\meta{GlobalScope} is a \meta{Scope} reflecting the global scope.

In addition to the requirements inherited from \meta{Scope}, the following must
be satisfied:

The \verb@metaobject_category@ template class specialization for a \meta{GlobalScope} should
inherit from \verb@global_scope_tag@:

\begin{minted}{cpp}
template <>
struct metaobject_category<MetaNamespace>
 : global_scope_tag
{ };
\end{minted}

The \verb@scope@ template class specialization (required by \meta{Scoped}) for \meta{GlobalScope}
should inherit from the \meta{GlobalScope} itself:

\begin{minted}{cpp}
template <>
struct scope<MetaGlobalScope>
 : MetaGlobalScope
{ };
\end{minted}


\subsection{MetaNamespace}
\label{concept-MetaNamespace}

\meta{Namespace} is a \meta{Scope} reflecting a namespace.

In addition to the requirements inherited from \meta{Scope}, the following must
be satisfied:

The \verb@metaobject_category@ template class specialization for a \meta{Namespace} should
inherit from \verb@namespace_tag@:

\begin{minted}{cpp}
template <>
struct metaobject_category<MetaNamespace>
 : namespace_tag
{ };
\end{minted}


\subsection{MetaType}
\label{concept-MetaType}

\begin{tikzpicture}
\node [concept] (Metaobject) {Metaobject};
\node [concept] (MetaNamed)[above right=of Metaobject] {MetaNamed}
	edge [inheritance, bend right] (Metaobject);
\node [concept] (MetaScoped)[below right=of Metaobject] {MetaScoped}
	edge [inheritance, bend left] (Metaobject);
\node [concept] (MetaNamedScoped)[below right=of MetaNamed, above right=of MetaScoped] {MetaNamedScoped}
	edge [inheritance, bend right] (MetaNamed)
	edge [inheritance, bend left] (MetaScoped);
\node [concept] (MetaType)[right=of MetaNamedScoped] {MetaType}
	edge [inheritance] (MetaNamedScoped);
\end{tikzpicture}

\meta{Type} is a \meta{NamedScoped} reflecting types.

In addition to the requirements inherited from \meta{NamedScoped}, the following is required:

The \verb@metaobject_category@ template class specialization for a \meta{Type} should
inherit from \verb@type_tag@:

\begin{minted}{cpp}
template <>
struct metaobject_category<MetaType>
 : type_tag
{ };
\end{minted}

\subsubsection{\texttt{original\_type}}

A template class \verb@original_type@ should be defined and should "return"
the original type reflected by this \meta{Type}:

\begin{minted}{cpp}
template <typename T>
struct original_type;

template <>
struct original_type<MetaType>
{
	static_assert(not(is_template<MetaType>::value), "");
	typedef original-type type;
};
\end{minted}

For example, if \verb@__meta_int@ is a metaobject reflecting the \verb@int@ type,
then the specialization of \verb@original_type@ should be following:

\begin{minted}{cpp}
template <>
struct original_type<__meta_int>
{
	typedef int type;
};
\end{minted}

\textbf{Note:} If a concept derived from \meta{Type}, for example a \meta{Class},
is also a \meta{Template} (i.e. is reflecting a template not a concrete type),
then the \verb@original_type@ template should be left undefined.


\subsection{MetaTypedef}
\label{concept-MetaTypedef}

\meta{Typedef} is a \meta{Type} reflecting \verb@typedef@s.

In addition to the requirements inherited from \meta{Type}, the following is required:

The \verb@category@ template class specialization for a \meta{Typedef} should
inherit from \verb@typedef_tag@:

\begin{minted}{cpp}
template <>
struct category<MetaTypedef>
 : typedef_tag
{ };
\end{minted}

\subsubsection{\texttt{decl\_type}}

A template class called \verb@decl_type@ should be defined and should inherit from the \meta{Type}
reflecting the "source" type of the typedef:

\begin{minted}{cpp}
template <typename T>
struct decl_type;

template <>
struct decl_type<MetaTypedef>
 : MetaType
{ };
\end{minted}

For example if \verb@__meta_std_string@ is a \meta{Typedef} reflecting the \verb@std::string@
typedef and \verb@__meta_std_basic_string_char@ is the \meta{Type} that reflects
the \verb@std::basic_string<char>@ type, and \verb@std::string@ is defined as:

\begin{minted}{cpp}
namespace std {
typedef basic_string<char> string;
}
\end{minted}

then the specialization of \verb@decl_type@ for \verb@__meta_std_string@ should be following:

\begin{minted}{cpp}
template <>
struct decl_type<__meta_std_string>
 : __meta_std_basic_string_char
{ };
\end{minted}

\textbf{Note:} If this feature proves to be too difficult to implement 
at this point\footnote{since some compilers do not keep typedef information},
it can be added later. We, however, think that leaving it out completely would
seriously limit the utility of reflection in certain use cases.

\subsection{MetaClass}
\label{concept-MetaClass}

\begin{tikzpicture}
\node [concept] (Metaobject) {Metaobject};
\node [concept] (MetaNamed)[above right=of Metaobject] {MetaNamed}
	edge [inheritance, bend right] (Metaobject);
\node [concept] (MetaScoped)[below right=of Metaobject] {MetaScoped}
	edge [inheritance, bend left] (Metaobject);
\node [concept] (MetaNamedScoped)[below right=of MetaNamed, above right=of MetaScoped] {MetaNamedScoped}
	edge [inheritance, bend right] (MetaNamed)
	edge [inheritance, bend left] (MetaScoped);
\node [concept] (MetaType)[above right=of MetaNamedScoped] {MetaType}
	edge [inheritance, bend right] (MetaNamedScoped);
\node [concept] (MetaScope)[below right=of MetaNamedScoped] {MetaScope}
	edge [inheritance, bend left] (MetaNamedScoped);
\node [concept] (MetaClass)[below right=of MetaType] {MetaClass}
	edge [inheritance, bend right] (MetaType)
	edge [inheritance, bend left] (MetaScope);
\end{tikzpicture}

\meta{Class} is a \meta{Type} and a \meta{Scope} if reflecting a regular class or possibly
also a \meta{Template} if it reflects a class template.

In addition to the requirements inherited from \meta{Type}, \meta{Scope}
and optionally from \meta{Template},
models of \meta{Class} are subject to the following:

The \verb@metaobject_category@ template class specialization for a \meta{Class} should
inherit from \verb@class_tag@:

\begin{minted}{cpp}
template <>
struct metaobject_category<MetaClass>
 : class_tag
{ };
\end{minted}

If a \meta{Class} reflects a class template, then the \verb@is_template@
trait should inherit from \verb@true_type@

\subsubsection{\texttt{elaborated\_type\_specifier}}

A template class called \verb@elaborated_type_specifier@ should be defined and should inherit from
a \meta{Specifier} reflecting the \verb@class@, \verb@struct@ or \verb@union@
specifiers:

\begin{minted}{cpp}
template <typename T>
struct elaborated_type_specifier;

template <>
struct elaborated_type_specifier<MetaClass>
 : MetaSpecifier
{ };
\end{minted}

\subsubsection{\texttt{base\_classes}}

A template class \verb@base_classes@ should be defined and should inherit from
a \concept{MetaobjectSequence} of \meta{Inheritance}s, each one of which reflects the inheritance
of a single base class of the class reflected by the \meta{Class}:

\begin{minted}{cpp}
template <typename T>
struct base_classes;

template <>
struct base_classes<MetaClass>
 : MetaobjectSequence<MetaInheritance>
{ };
\end{minted}


\subsection{MetaVariable}
\label{concept-MetaVariable}

\begin{tikzpicture}
\node [concept] (Metaobject) {Metaobject};
\node [concept] (MetaNamed)[above right=of Metaobject] {MetaNamed}
	edge [inheritance, bend right] (Metaobject);
\node [concept] (MetaScoped)[below right=of Metaobject] {MetaScoped}
	edge [inheritance, bend left] (Metaobject);
\node [concept] (MetaNamedScoped)[below right=of MetaNamed, above right=of MetaScoped] {MetaNamedScoped}
	edge [inheritance, bend right] (MetaNamed)
	edge [inheritance, bend left] (MetaScoped);
\node [concept] (MetaVariable)[above right=of MetaNamedScoped] {MetaVariable}
	edge [inheritance, bend right] (MetaNamedScoped);
\end{tikzpicture}

\meta{Variable} is a \meta{NamedScoped} reflecting a variable.

In addition to the requirements inherited from \meta{NamedScoped}, the following must
be satisfied:

The \verb@metaobject_category@ template class specialization for a \meta{Variable} should
inherit from \verb@variable_tag@:

\begin{minted}{cpp}
template <>
struct metaobject_category<MetaVariable>
 : variable_tag
{ };
\end{minted}

\subsubsection{\texttt{storage\_specifier}}

A template class \verb@storage_specifier@ should be added and should
inherit from a \meta{Specifier} reflecting a storage class specifier:

\begin{minted}{cpp}
template <typename T>
struct storage_specifier;

template <>
struct storage_specifier<MetaVariable>
 : MetaSpecifier
{ };
\end{minted}

\subsubsection{\texttt{type}}

A template class \verb@type@ should be added and should inherit
from a \meta{Type} reflecting the type of the variable:

\begin{minted}{cpp}
template <typename T>
struct type;

template <>
struct type<MetaVariable>
 : MetaType
{ };
\end{minted}

\subsubsection{\texttt{pointer}}

If the reflected variable is a namespace-level variable, then a template
class \verb@pointer@ should be implemented as follows:

\begin{minted}{cpp}
template <typename T>
struct pointer;

template <>
struct pointer<MetaVariable>
{
	typedef typename original_type<type<MetaVariable>>::type* type;

	static type get(void);
};
\end{minted}

The static member function \verb@get@ should return the address of the reflected variable.

If the reflected variable is a class member variable (i.e. if the \meta{Variable}
is also a \meta{ClassMember}), then the \verb@pointer@ template class should be
defined as follows:

\begin{minted}{cpp}

template <>
struct pointer<MetaVariable>
{
	typedef typename original_type<type<MetaVariable>>::type
		_mv_t;
	typedef typename original_type<type<scope<MetaVariable>>>::type
		_cls_t;

	typedef _mv_t _cls_t::* type;

	static type get(void);
};

\end{minted}

The static member function \verb@get@ should return a data member pointer to
the reflected member variable. The \verb@_mv_t@ and \verb@_cls_t@ typedefs
are implementation details and are not a part of this specification.

