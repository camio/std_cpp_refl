\documentclass[11pt,a4paper,oneside]{scrartcl}

\newcommand{\mchname}{Mat\'{u}\v{s} Chochl\'{i}k}
\newcommand{\mchmail}{chochlik@gmail.com}
\newcommand{\docname}{A case for strong static reflection}
\newcommand{\docnum}{N4452}
\newcommand{\docdate}{2015-04-11}

\usepackage[utf8]{inputenc}
\usepackage{url}
\usepackage[colorlinks=true]{hyperref}
\usepackage{parskip}
\usepackage[titletoc]{appendix}

\usepackage{listings}
\usepackage{minted}
\lstset{basicstyle=\footnotesize\ttfamily,breaklines=true}

\usepackage{fancyhdr}
\setlength{\headheight}{14pt}
\pagestyle{fancyplain}
\lhead{\fancyplain{}{\docnum - \docname}}
\rhead{}
\rfoot{\fancyplain{}{\thepage}}
\cfoot{}

\usepackage[pdftex]{graphicx}
\DeclareGraphicsExtensions{.pdf,.png,.jpg,.mps,.eps}
\graphicspath{{images/}}

\usepackage{tikz}
\usetikzlibrary{arrows,positioning}
\tikzstyle{concept}=[
	rectangle,	
	very thick,
	draw=red!80!black!80,
	top color=white,
	bottom color=red!20,
	node distance=0.5em and 1.5em
]
\tikzstyle{inheritance}=[
	->,
	shorten >=1pt,
	>=open triangle 90,
	very thick
]
   

\setcounter{tocdepth}{3} 

\title{\docname}

\author{\mchname (\mchmail)}

\newcommand{\meta}[1]{{\em \textbf{Meta#1}}}

\begin{document}

\begin{tabular}{r l}
Document number: & \docnum\\
Date: & \docdate\\
Project: & Programming Language C++, SG7, Reflection\\
Reply-to: & \mchname (\href{mailto:\mchmail}{\mchmail})\\
\end{tabular}

\begin{center}
\vskip 2em
{\Huge \docname}
\vskip 1em
{\emph \mchname}
\vskip 2em
\end{center}

\paragraph{Abstract}

In N3996 \cite{n3996} and N4111 \cite{n4111} we proposed the design
and some hints on possible implementation of a compile-time reflection
facility for standard C++. N3996 contained list of possible use-cases
and a discussion about the usefulness of reflection. During the presentation
of N4111 concerns were expressed about the level-of-detail and scope
of the presented proposal and possible dangers of giving non-expert language users
a {\em too powerful} tool to use and dangers for the implementers of this proposal.
Examples and more detailed description of use cases were called for.
This paper aims to address these issues.


\tableofcontents

\section{Introduction}

Reflection and reflective programming can be used
for a wide range of tasks such as implementation of serialization-like operations,
remote procedure calls, scripting, automated GUI-generation,
implementation of several software design patterns, etc.
C++ as one of the most prevalent programming languages 
lacks a standardized reflection facility.

In this paper we propose the addition of native support for
compile-time reflection to C++ and a library built
on top of the metadata provided by the compiler.

The basic static metadata provided by compile-time reflection
should be as complete as possible to be applicable in a wide
range of scenarios and allow to implement custom higher-level
static and dynamic reflection libraries and reflection-based
utilities.

The term \emph{reflection} refers to the ability of a computer program
to observe and possibly alter its own structure and/or its behavior.
This includes building new or altering the existing data structures,
doing changes to algorithms or changing the way the program code
is interpreted. Reflective programming is a particular kind
of \emph{metaprogramming}.

The advantage of using reflection is in the fact that everything
is implemented in a single programming language, and the human-written
code can be closely tied with the customizable reflection-based
code which is automatically generated by compiler metaprograms,
based on the metadata provided by reflection.

The solution proposed in this paper is based on the
\href{http://kifri.fri.uniza.sk/~chochlik/mirror-lib/html/}{\em Mirror}
reflection utilities~\cite{mirror-doc-cpp11} and on several years
of user experience with reflection-based metaprogramming.

\section{Use cases and examples}
\label{use-cases-examples}

Note that some of the examples listed in this section use features which
are not part of the initial reflection specification, but which are planned
as future additions.

\subsection{Portable (type) names}

One of the notorious problems of \verb@std::type_info@ is that the string
returned by its \verb@name@ member function is not standardized and is
not even guaranteed to return any meaningful, unique human-readable string,
at least not without de-mangling, which is platform specific.
Furthermore the returned string is not \verb@constexpr@ and cannot be
reasoned about at compile-time and is applicable only to types.
One other problem with \verb@typeid@ that it is not always aware of \verb@typedef@s.
In some cases we would like to obtain the alias name, instead of the
\say{real} name of a type or a class member or function parameter.

The ability to uniquely map any type used in a program to a human-readable,
portable, compile-time string has several use-cases described in this paper.

The \meta{Named} concept reflects named language constructs
and provides the \verb@get_name@ operation
returning their basic name without any qualifiers or decorations.
This can be with the help of metaprogramming turned into a fully-qualified
name.


\subsection{Logging}

When logging the execution of functions (especialy templated ones) it is sometimes
desirable to also include the names of the parameter types or even the names of the parameters
and other variables.

The best we can do with just the \verb@std::type_info@ is the following:

\begin{minted}[tabsize=4]{cpp}
#if __PLATFORM_ABC__
std::string demangled_type_name(const char*) { /* implementation 1 */ }
#else if __PLATFORM_MNO__
std::string demangled_type_name(const char*) { /* implementation 2 */ }
#else if __PLATFORM_XYZ__
std::string demangled_type_name(const char*) { /* implementation N */ }
#else
std::string demangled_type_name(const char* mangled_name)
{
	// don't know how to demangle this; let's try our luck
	return mangled_name;
}
#endif

template <typename T>
T min(const T& a, const T& b)
{
	log()   << "min<"
	        << demangled_type_name(typeid(T).name())
	        << ">(" << a << ", " << b << ") = ";

	T result = a<b?a:b;

	log()   << result << std::endl;

	return result;
}

\end{minted}

Which may or may not work, depending on the platform.

With the help of reflection as proposed in N4111 we could do:

\begin{minted}[tabsize=4]{cpp}
template <typename T>
T min(const T& a, const T& b)
{
	log()   << "min<"
	        << full_name<mirrored(T)>()
	        << ">(" << a << ", " << b << ") = ";

	T result = a<b?a:b;

	log()   << result << std::endl;

	return result;
}
\end{minted}

The \verb@__PRETTY_FUNCTION__@ macro generated by the compiler could be also
used in this case, but the format of the string which this macro expands into is not customizable
(which may be necessary for logs formatted in XML, JSON, etc.

A more elaborated output containing also the parameter names could be achieved
by using reflection:

\begin{minted}[tabsize=4]{cpp}
template <typename T>
T min(const T& a, const T& b)
{
	log()   << "function: min<"
	        << full_name<mirrored(T)>()
	        << ">"
		<< std::endl
		<< base_name<mirrored(a)>() << ": "
		<< a << std::endl
		<< base_name<mirrored(b)>() << ": "
		<< b << std::endl;

	T result = a<b?a:b;

	log()   << base_name<mirrored(result)>() << ": "
		<< b << std::endl;

	return result;
}
\end{minted}

It is true that the lines:
\begin{minted}[tabsize=4]{cpp}
		<< base_name<mirrored(a)>() << ": "
		<< base_name<mirrored(b)>() << ": "
\end{minted}

could be replaced by preprocessor stringization or just hard coded
strings, like

\begin{minted}[tabsize=4]{cpp}
		<< BOOST_PP_STRINGIZE(a) << ": "
		<< BOOST_PP_STRINGIZE(b) << ": "
\end{minted}

or

\begin{minted}[tabsize=4]{cpp}
		<< "a: "
		<< "b: "
\end{minted}

but the compiler would not force the programmer to change the macro parameter
or the content of the string the if the parameters \verb@a@ and \verb@b@ were renamed
for example to \verb@first@ and \verb@second@. If would enforce the change if
reflection was used.

Furthermore, if the \meta{Function} concept was implemented
and if it was possible to reflect the {\em 'current function'} (i.e. to get a \meta{Function}
from inside of a function body via some invocation of the reflection operator),
then even more would be possible; The function name and even the parameter names could
be obtained from reflection and encapsulated into a function.

\begin{minted}[tabsize=4]{cpp}
template <typename MetaFunction, typename ... P>
void log_function_exec(MetaFunction, const std::tuple<P&...>& params)
{
	log()   << "function: "
		<< base_name<MetaFunction>()
		<< std::endl;

	// obtain the MetaParameter(s) from the MetaFunction
	// and print them pairwise with the values from params.
	for_each<parameters<MetaFunction>>(
		[&params](auto meta_param)
		{
			typedef decltype(meta_param) MP;
			log()  << base_name<MP>() << ": "
			       << std::get<position<MP>::value>(params)
			       << std::endl;
		}
	);
}

template <typename T>
T min(T a, T b)
{
	log_function_exec(mirrored(this::function), std::tie(a, b));
	/* ... */
}

template <typename T>
T max(T a, T b)
{
	log_function_exec(mirrored(this::function), std::tie(a, b));
	/* ... */
}

template <typename T>
T avg(T a, T b)
{
	log_function_exec(mirrored(this::function), std::tie(a, b));
	/* ... */
}
\end{minted}

This example used the following features:

\begin{itemize}
\item{function reflection,}
\item{function parameter reflection,}
\item{context-dependent reflection\footnote{See appendix~\ref{appendix-context-dependent-reflection}.}},
\item{use of metaobject sequences,}
\item{use of the reflection operator,}
\item{base names and the \meta{Named} concept.}
\end{itemize}

\subsection{Generation of common functions}

This use case was part of the \say{targeted use cases} in the committee's
call for compile-time reflection proposals~\cite{ISOCPP-N3814}: 

{\em There are many functions that generally consist of boilerplate code,
performing some action for each member of a class. Such functions include
equality operators, comparison operators, serialization functions,
hash functions and swap functions.
}

In other words for arbitrary structured type, for example:

\begin{minted}[tabsize=4]{cpp}
struct S
{
	int i;
	long l;
	float f;
};
\end{minted}

we want to create equality or non-equality comparison function like:

\begin{minted}[tabsize=4]{cpp}
bool S_equal(const S& a, const S& b)
{
	bool result = true;
	result &= a.i == b.i;
	result &= a.l == b.l;
	result &= a.f == b.f;
	return result;
}

bool S_not_equal(const S& a, const S& b)
{
	bool result = false;
	result |= a.i != b.i;
	result |= a.l != b.l;
	result |= a.f != b.f;
	return result;
}
\end{minted}

or a hash function:

\begin{minted}[tabsize=4]{cpp}
std::size_t S_hash(const S& a)
{
	std::size_t result = 0u;
	result ^= std::hash<int>()(a.i);
	result ^= std::hash<long>()(a.l);
	result ^= std::hash<float>()(a.f);
	return result;
}
\end{minted}

This is one of the many use cases where the \verb@for_each@ function
described in section \ref{fac-for-each} comes in handy. The above could be
implemented along the lines of:

\begin{minted}[tabsize=4]{cpp}
template <typename T>
struct compare_data_members
{
	const T& a;
	const T& b;
	bool& result;

	template <typename MetaDataMember>
	void operator()(identity<MetaDataMember>) const
	{
		auto mem_ptr = meta::get_pointer_v<MetaDataMember>;
		result &= a.*mem_ptr == b.*mem_ptr;
	}
};

template <typename T>
bool generic_equal(const T& a, const T& b)
{
	using metaT = reflexpr(T);
	bool result = true;

	meta::for_each<meta::get_all_data_members_t<metaT>>(
		compare_data_members<T>{a, b, result}
	);

	return result;
}
\end{minted}

If the reversible reflection feature described in section \ref{fut-reverse-reflection}
was implemented then the helper could take advantage of it:

\begin{minted}[tabsize=4]{cpp}
template <typename T>
struct compare_data_members
{
	const T& a;
	const T& b;
	bool& result;

	template <typename MetaDataMem>
	void operator()(identity<MetaDataMem>) const
	{
		result &= a.reflexpr(MetaDataMem) == b.reflexpr(MetaDataMem);
	}
};
\end{minted}

The helper could also be implemented by using a lambda function:

\begin{minted}[tabsize=4]{cpp}

template <typename T>
std::size_t generic_hash(const T& a)
{
	std::size_t result = 0u;

	meta::for_each<meta::get_all_data_members_t<reflexpr(T)>>(
		[&result,&a](auto meta_dm)
		{
			using MetaDataMem = decltype(meta_dm)::type;
			using MetaT = meta::get_type_t<MetaDataMem>;

			using T = meta::get_original_type_t<MetaT>;
			// or T = reflexpr(metaT);

			auto mem_ptr = meta::get_pointer_v<MetaDataMem>;

			result ^= std::hash<T>(a.*mem_ptr);
			// or  ^= std::hash<T>(a.reflexpr(MetaDataMem));
		}
	);

	return result;
}
\end{minted}


\subsection{Enumerator to string and vice versa}
\label{use-case-enum-to-string}

This is another use case from the \say{targeted use cases} in the committee's
call for compile-time reflection proposals~\cite{ISOCPP-N3814}. 

The goal is to automate the implementation of functions which for a given
enumeration value, return the name of the enumeration value:

\begin{minted}[tabsize=4]{cpp}
enum class E
{
	a, b, c, d, e, f
};

string E_to_string(E value)
{
	switch(value)
	{
		case E::a: return "a";
		case E::b: return "b";
		case E::c: return "c";
		case E::d: return "d";
		case E::e: return "e";
		case E::f: return "f";
	}
	return {};
}
\end{minted}

or the other way around:

\begin{minted}[tabsize=4]{cpp}
E string_to_E(const string& name)
{
	if(name == "a") return E::a;
	if(name == "b") return E::b;
	if(name == "c") return E::c;
	if(name == "d") return E::d;
	if(name == "e") return E::e;
	if(name == "f") return E::f;

	// or throw here
	return {};
}
\end{minted}

The Mirror reflection library shows a possible implementation of the
\verb@enum_to_string@,

\begin{minted}[tabsize=4]{cpp}
template <typename Enum>
class enum_to_string
{
private:
	template <typename ... MEC>
	struct _hlpr
	{
		static void _eat(bool ...) { }

		static std::map<Enum, std::string> _make_map(void)
		{
			using namespace std;

			map<Enum, string> res;
			_eat(res.emplace(
				meta::get_constant_v<MEC>,
				string(meta::get_base_name<MEC>())
			).second...);
			return res;
		}
	};
public:
	const std::string& operator()(Enum e) const
	{
		using namespace std;

		using ME = reflexpr(Enum);
		using hlpr = meta::unpack_sequence_t<
			meta::get_enumerators_m<ME>,
			_hlpr
		>;
		static auto m = hlpr::_make_map();
		return m[e];
	}
};
\end{minted}

and the \verb@string_to_enum@ utility:

\begin{minted}[tabsize=4]{cpp}
template <typename Enum>
class string_to_enum
{
private:
	template <typename ... MEC>
	struct _hlpr
	{
		static void _eat(bool ...) { }

		static std::map<std::string, Enum> _make_map(void)
		{
			using namespace std;

			map<string, Enum> res;
			_eat(res.emplace(
				string(meta::get_base_name<MEC>()),
				meta::get_constant_v<MEC>
			).second...);
			return res;
		}
	};
public:
	Enum operator()(const std::string& s) const
	{
		using namespace std;

		using ME = reflexpr(Enum);
		using hlpr = meta::unpack_sequence_t<
			meta::get_enumerators_m<ME>,
			_hlpr
		>;
		static auto m = hlpr::_make_map();
		auto p = m.find(s);
		if(p == m.end()) {
			throw runtime_error("Invalid enumerator name");
		}
		return p->second;
	}
};
\end{minted}


\subsection{Simple serialization}

We need to serialize the instances of selected classes into a structured external format
like XML, JSON, XDR or even into a format like Graphviz dot for the purpose of creating
a visualization of a static class or dynamic object hierarchy or graph.

Reflection makes this task trivial\footnote{
Admittedly this is not the most clever XML schema ever devised, but let's stick to the basics.}:

\begin{minted}[tabsize=4]{cpp}

template <typename T>
void to_xml(const T& instance, std::true_type atomic)
{
	typedef mirrored(T) MetaType;
	std::cout << "<" << base_name<MetaType>() << ">";
	std::cout << instance;
	std::cout << "</" << base_name<MetaType>() << ">";
}

template <typename T>
void to_xml(const T& instance)
{
	to_xml(instance, std::is_fundamental<T>());
}

template <typename T>
void to_xml(const T& instance, std::false_type atomic)
{
	typedef mirrored(T) MetaType;
	std::cout << "<" << base_name<MetaType>() << ">";

	for_each<base_classes<MetaType>>(
		[](auto meta_inheritance)
		{
			typedef decltype(meta_inhertance) MetaInh;
			typedef original_type<base_class<MetaInh>>::type BT;

			to_xml(const BT&(instance));
		}
	);

	for_each<members<MetaType>>(
		[](auto meta_cls_mem)
		{
			typedef decltype(meta_cls_mem) MetaClsMem;
			typedef original_type<type<MetaClsMem>>::type MT;

			if(std::is_base_of<
				meta_variable_tag,
				metaobject_category<MetaClsMem>
			>())
			{
				auto mvp = pointer<MetaClsMem>::get();
				std::cout << "<" << base_name<MetaClsMem> << ">";
				to_xml(instance.*mvp);
				std::cout << "</" << base_name<MetaClsMem> << ">";
			}
		}
	);

	std::cout << "</" << base_name<MetaType>() << ">";
}

\end{minted}

Where necessary explicit specializations can override the generic implementation:

\begin{minted}[tabsize=4]{cpp}

template <typename Bool>
void to_xml(const std::string& instance, Bool)
{
	std::cout << "<string>";
	std::cout << instance;
	std::cout << "</string>";
}

\end{minted}

This use-case shows the following:

\begin{itemize}
\item{class member reflection,}
\item{inheritance reflection,}
\item{class member variable reflection,}
\item{use of metaobject sequences,}
\item{use of the interface of various metaobjects,}
\item{use of the reflection operator,}
\item{metaobject categorization,}
\item{base names and the \meta{Named} concept.}
\end{itemize}


\subsection{Cross-cutting aspects}

We need to execute the same action (or a set of actions) at the entry of or at the exit from the body of
a function (from a set of multiple functions meeting some conditions) each time it is called.

The action may be related to logging, debugging, profiling, but also access control, etc.
The condition which selects the functions for which the action is invoked might be something like:
\begin{itemize}
\item each member function of a particular class,
\item each function defined in some namespace,
\item each function returning values of a particular type or having a particular set of parameters,
\item each function whose name matches a search expression,
\item each function declared in a particular source file,
\item etc. and various combinations of the above.
\end{itemize}

It may not be possible to tell in advance the relations between the aspects and the individual functions
or these relations may vary for different builds or build configurations.
Furthermore we want to be able to quickly change the assignment of actions to functions in one
place instead of going through the whole project source which may consists of dozens or even hundreds of files.

We want for example temporarily enable logging of the entry and exit of each member function of class \verb@foo@,
or we need to count the number of invocations of functions defined in the \verb@bar@ namespace with
names not starting with an underscore, or we want to throw the \verb@not_logged_in@ exception at the entry
of each member function of class \verb@secure@ if the global \verb@user_logged_in@ function returns \verb@false@.

Without reflection something like this could be implemented in the following way:

\begin{minted}[tabsize=4]{cpp}

class logging_aspect
{
public:
	template <typename ... P>
	logging_aspect(const char* func_name, P&&...)
	{
		// write to clog
	}
};

class profiling_aspect
{
	/* ... */
};

class authorization_aspect
{
public:
	template <typename ... P>
	authorization_aspect(const char* func_name, P&&...)
	{
		if(contains(func_name, "secure"))
		{
			if(!::is_user_logged_in())
			{
				throw not_authorized(func_name);
			}
		}
	}
};

template <typename RV, typename ... P>
class func_aspects
 : logging_aspect
 , profiling_aspect
 , authorization_aspect
/* ... etc. ... */
{
public:
	func_aspects(
		const char* name,
		const char* file,
		unsigned line,
		P&&... args
	): logging_aspect(name, file, line, args...)
	 , profiling_aspect(name, file, line, args...)
	 , authorization_aspect(name, file, line, argc...)
	/* ... etc. ... */
	{ }
};

template <typename RV, typename ... P>
func_aspects<RV, P...>
make_func_aspects(
	const char* name,
	const char* file,
	unsigned line,
	P&&...args
);


void func1(int a, int b)
{
	auto _fa = make_func_aspects<void>(
		__func__,
		__FILE__,
		__LINE__,
		a, b
	);
	/* function body */
}

double func2(double a, float b, long c)
{
	auto _fa = make_func_aspects<double>(
		__func__,
		__FILE__,
		__LINE__,
		a, b, c
	);
	/* function body */
}

namespace foo {

long func3(int x)
{
	auto _fa = make_func_aspects<long>(
		__func__,
		__FILE__,
		__LINE__,
		x
	);
	/* function body */
}

} // namespace foo
\end{minted}

Obviously this is very repetitive and it can get quite tedious and error-prone
to supply all this information to the aspects in each function manually.
Also if the signature or the name of the function changes the construction
of the \verb@func_aspects@ instance must be updated accordingly.
With the help of reflection things can be simplified considerably:

\begin{minted}[tabsize=4]{cpp}

template <typename MetaFunction, typename Enabled>
class logging_aspect_impl;

template <typename MetaFunction>
class logging_aspect_impl<MetaFunction, false_type>
{ };

template <typename MetaFunction>
class logging_aspect_impl<MetaFunction, true_type>
{
public:
	logging_aspect_impl(void)
	{
		clog
			<< base_name<MetaFunction>()
			<< "("
			/* ... */
			<< ")"
			<< endl;
	}
};

template <typename MetaFunction>
struct logging_enabled
 : integral_constant<
	bool,
	is_base_of<mirrored(std), scope<MetaFunction>>() &&
	is_same<std::string, original_type<result<MetaFunction>>::type> &&
	/* ... etc. ... */
>
{ };

template <typename MetaFunction>
using logging_aspect =
	logging_aspect_impl<
		MetaFunction,
		typename logging_enabled<MetaFunction>::type
	>;

template <typename MetaFunction, typename Enabled>
class authorization_aspect_impl;

template <typename MetaFunction>
class authiorization_aspect_impl<MetaFunction, false_type>
{ };

template <typename MetaFunction>
class authorization_aspect_impl<MetaFunction, true_type>
{
public:
	authorization_aspect_impl(void)
	{
		if(!::is_user_logged_in())
		{
			throw not_authorized(
				full_name<MetaFunction>()
			);
		}
	}
};

template <typename MetaFunction>
struct autorization_enabled
 : integral_constant<
	bool,
	is_base_of<mirrored(foo::bar), scope<MetaFunction>>() &&
	constexpr_starts_with(base_name<MetaFunction>(), "secure_") &&
	/* ... etc. ... */
>
{ };

template <typename MetaFunction>
using authorization_aspect =
	authorization_aspect_impl<
		MetaFunction,
		typename authorization_enabled<MetaFunction>::type
	>;

template <typename MetaFunction>
class func_aspects
 : logging_aspect<MetaFunction>
 , profiling_aspect<MetaFunction>
 , authorization_aspect<MetaFunction>
/* ... etc. ... */
{
public:
};

void func1(int a, int b)
{
	func_aspects<mirrored(this::function)> _fa;
	/* function body */
}

double func2(double a, float b, long c)
{
	func_aspects<mirrored(this::function)> _fa;
	/* function body */
}

namespace foo {

long func3(int x)
{
	func_aspects<mirrored(this::function)> _fa;
	/* function body */
}

} // namespace foo

\end{minted}

In this case the same expression is used in all functions
regardless of their name and signature and the aspects get all the information
they require from the metaobject reflecting the function. All the data
obtained from the metaobjects is available at compile-time so various
specializations of the aspect classes can be implemented as required.

This same technique could also be used with instances of classes:

\begin{minted}[tabsize=4]{cpp}

template <typename MetaClass>
class class_aspects
 : logging_aspects<MetaClass>
/* ... etc. ... */
{
public:
	class_aspects(typename original_type<MetaClass>::type* that);
};

class cls1
{
private:
	int member1;
	/* ... other members ... */
	class_aspect<mirrored(this::class)> _ca;
public:
	cls1(void)
	 : member1(...)
	 , _ca(this)
	{ }
};

\end{minted}

Class aspects like these could also be used for logging, monitoring of object instantation,
leak detection, etc.

This use-case shows the following:

\begin{itemize}
\item{current function reflection,}
\item{current class reflection.}
\end{itemize}


\subsection{Implementing the factory pattern}

The purpose of the {\em Factory} pattern is to separate its caller,
who requires a new instance of a {\em Product} type, from the details
of this instance's construction. The caller only supplies the input data
to the factory and collects the new instance. There are several aspects
that need to be considered when designing and implementing a factory.

The input data for the construction of an instance of the {\em Product}
can be stored in an external representation (an XML fragment, a RDBS database dataset,
a JSON document, etc.) or even entered by the user through a GUI oron the
command-line and so on, and would need to be converted into a native C++ representation.
The new instance also might be constructed as a copy of another already existing
\emph{prototype} instance of the same type sitting in an object pool.

The product may be polymorphic and the exact type may not even be
known to the user. It may have one or several constructors, each of which
may require a different set of arguments. It may or may not have
constructors with a specific signature, for example a default constructor.

A default constructor does not make sense for many types and requiring it
just because the type will be used with a factory is problematic\footnote{
or even impossible with third-party code}.
Consider for example what a \say{default} instance of \verb@person@ or \verb@address@ would
look like -- it would not have any meaning at all.
Thus well-designed factories should not depend on the presence of constructors
with specific signatures.

Furthermore it might be desirable, that the constructor used to construct
a particular instance is picked based on the available input data which is known
only at run-time, but not when the factory is designed and implemented.

Let's consider the implementation of a factory for a rather simple \verb@point@ class,
representing a point in 3-dimensional space:

\begin{minted}[tabsize=4]{cpp}
struct point
{
    double _x, _y, _z;

    point(double x, double y, double z): _x(x), _y(y), _z(z) { }

    point(double w): _x(w), _y(w), _z(w) { }

    point(void): _x(0.0), _y(0.0), _z(0.0) { }

    point(const point&) = default;

    // ... other declarations
};
\end{minted}

A naive hand-coded implementation, of a factory constructing
\verb@point@s from some \verb@Data@ type (for example an XML node) might look like this:

\begin{minted}[tabsize=4]{cpp}
class point_factory
{
private:
    unsigned pick_constructor(Data data)
    {
        // somehow examine the data and pick
        // the most suitable constructor of the point class
    }

    double extract(Data data, string param)
    {
        // somehow extract and convert the value
        // of a named parameter from the data
    }
public:
    point create(Data data)
    {
        switch(pick_constructor(data))
        {
            case 0: return point();

            case 1: return point(extract(data, "w"));

            case 2: return point(
                        extract(data, "x"),
                        extract(data, "y"),
                        extract(data, "z")
                    );

            default: throw exception(...);
        }
    }
};
\end{minted}

Now suppose that there is some pool of existing \verb@point@ objects
and let's extend the factory to use this pool and return copies if applicable:

\begin{minted}[tabsize=4]{cpp}

extern pool_of<point>& point_pool;

class point_factory
{
private:
    unsigned pick_constructor(Data data)
    {
        // same as before but also allow the copy
        // constructor to be picked if the data says so
    }

    double extract(Data data, string param);
public:
    point create(Data data)
    {
        switch(pick_constructor(data))
        {
            // same as before, but add a new case
            // returning copies from the pool

            case 3: return point_pool.get(data);

            default: throw exception(...);
        }
    }
};
\end{minted}

When looking at the hand-coded factories above, it is obvious that
implementing and maintaining\footnote{as the constructed types evolve and change}
factories for several dozens of classes in a larger
application is a highly repetitive, tedious and possibly error-prone
process and at least partial automation is desirable.

Factory classes must generally handle several tasks
which fit into two distinct and nearly orthogonal categories:

\begin{itemize}
\item{\emph{Product type-related}}
        \begin{itemize}
        \item{\emph{Constructor description}} -- providing the metadata describing
        the individual constructors, their parameters, etc.
        \item{\emph{Constructor dispatching}} -- calling the selected constructor.
        with the supplied arguments which results in a new instance of the product type.
        \end{itemize}

\item{\emph{Input data representation-related}}
        \begin{itemize}
        \item{\emph{Input data validation}} -- checking if the input data match
        the available constructors.
        \item{\emph{Constructor selection}} -- examining the input data, comparing it
        to the metadata describing product's constructors and determining
        which constructor should be called.
        \item{\emph{Getting the argument values}} -- determining where the argument
        values should come from and getting them:
                \begin{itemize}
                \item{\emph{Conversion from the external representation}} -- this usually applies
                to intrinsic C++ types, but complex types could be converted directly too.
                \item{\emph{Recursive construction by using another factory}} -- this usually
                requires some form of cooperation between the parent and its child factories
                and it means that all the tasks discussed here must be repeated also for
                the recursively constructed parameter(s).
                \item{\emph{Copying an existing instance}} -- for example from an object pool.
                \end{itemize}
        \end{itemize}
\end{itemize}

Parts from each category can be combined with parts from the other to create new
factories which promotes code re-usability. Factories constructing instances of a single
product from various data representations share the product-related components
and factories constructing instances of various product types from a single input data representation
share the input-data-related parts. This approach has several advantages like better
maintainability or the ability to develop the components separately and combine them later
via metaprogramming.

If the input data for a metaprogram generating the factory class, that is
the metadata describing the Product type\footnote{specifically the constructors of Product}
can be obtained by using compile-time reflection then new factory classes can be generated
automatically for nearly arbitrary type provided that the input data type-related parts are
implemented.

\subsection{SQL schema generation}

We need to create an SQL/DDL (data definition language) script for creating a schema
with tables which will be storing the values of all structures in namespace C++ \verb@foo@
having names starting with \verb@persistent_@:

\begin{minted}[tabsize=4]{cpp}

const char* translate_to_sql(const std::string& type_name)
{
	if(type_name == "int")
		return "INTEGER";
	/* .. etc. */
}

template <typename MetaMemVar>
void create_table_column_from(MetaMemVar)
{
	if(!std::is_base_of<
		variable_tag,
		metaobject_category<MetaMemVar>
	>()) return;

	std::cout << base_name<MetaMemVar>() << " ";

	std::cout << translate_to_sql(base_name<type<MetaMemVar>());

	if(starts_with(base_name<MetaMemVar>(), "id_"))
	{
		std::cout << " PRIMARY KEY";
	}
	std::cout << std::endl;
}

template <typename MetaClass>
void create_table_from(MetaClass)
{
	if(!std::is_base_of<
		class_tag,
		metaobject_category<MetaClass>
	>()) return;

	std::cout << "CREATE TABLE "
	          << strip_prefix("persistent_", base_name<MetaClass>())
	          << "(" << std::endl;

	for_each<members<MetaClass>>(create_table_column_from);

	std::cout << ");"
}

template <typename MetaNamespace>
void create_schema_from(MetaNamespace)
{
	std::cout << "CREATE SCHEMA "
	          << base_name<MetaNamespace>()
	          << ";" << std::endl;

	for_each<members<MetaNamespace>>(create_table_from);
}

int main(void)
{
	create_schema_from(reflected(foo));
	return 0;
}

\end{minted}

This example shows the following features from N4111:

\begin{itemize}
\item{namespace reflection,}
\item{namespace member reflection,}
\item{class member reflection,}
\item{use of metaobject sequence,}
\item{metaobject categorization,}
\item{base names and the \meta{Named} concept.}
\end{itemize}

Furthermore reflection could be used to implement actual object-relational mapping,
together with a library like \verb@SOCI@, \verb@ODBC@, \verb@libpq@, etc.

\subsection{Structure data member transformations}
\label{use-case-struct-transf}

We need to create a new structure, which has data members with the same names
as an original structure, but we need to change some of the properties
of the data members (for example their types).

For example we need to transform:

\begin{minted}[tabsize=4]{cpp}

struct foo
{
	bool b;
	char c;
	double d;
	float f;
	std::string s;
};

\end{minted}

into

\begin{minted}[tabsize=4]{cpp}

struct rdbs_table_placeholder_foo
{
	column_placeholder<bool>::type b;
	column_placeholder<char>::type c;
	column_placeholder<double>::type d;
	column_placeholder<float>::type f;
	column_placeholder<std::string>::type s;
};

\end{minted}

By using the proposed \verb@identifier@ operator\footnote{See appendix~\ref{appendix-operator-identifier}.}, class member reflection
and multiple inheritance we can create a new structure that is
nearly equivalent to \verb@rdbs_table_foo@ via metaprogramming:

\begin{minted}[tabsize=4]{cpp}

template <typename MetaVariable>
struct rdbs_table_placeholder_helper
{
	typename column_placeholder<
		typename original_type<type<MetaVariable>>::type
	>::type identifier(base_name<MetaVariable>::value);
};

template <typename ... MetaVariables>
struct rdbs_table_placeholder_helpers
 : rdbs_table_placeholder_helper<MetaVariables>...
{ };

template <typename MetaVariableSeq, typename IdxSeq>
class rdbs_table_impl;

template <typename MetaVariableSeq, std::size_t ... I>
class rdbs_table_impl<MetaFunctionSeq, std::index_sequence<I...>>
 : public rdbs_table_placeholder_helpers<at<MetaFunctionSeq, I>...>
{ };

typedef rdbs_table_impl<
	members<mirrored(foo)>,
	std::make_index_sequence<size<members<mirrored(foo)>>::value>
> rdbs_table_placeholder_foo;

\end{minted}

This examples uses the following features.

\begin{itemize}
\item{class member reflection,}
\item{use of the reflection operator,}
\item{use of the \verb@identifier@ operator or the \verb@named_mem_var@ templates.}
\end{itemize}


%\subsection{Replacing Qt's MOC with standard C++}
\label{use-case-replacing-moc}

Replacing the functionality currently provided by the Qt's meta-object
compiler -- an external tool implementing some basic reflection capabilities
and the signal slot mechanism, with standard C++, is another relatively important
use case to be considered when designing C++ reflection.

TODO


\subsection{Simple examples of usage}
\label{working-examples}

This section shows several simple, but complete examples of usage already working
with the experimental implementation of the initial specification.

\subsubsection{Scope reflection}

\begin{minted}[tabsize=4]{cpp}
#include <reflexpr>
#include <iostream>

namespace foo {

struct bar
{
	typedef int baz;
};

} // namespace foo

typedef long foobar;

int main(void)
{
	using namespace std;

	typedef reflexpr(int) meta_int;
	typedef reflexpr(foo::bar) meta_foo_bar;
	typedef reflexpr(foo::bar::baz) meta_foo_bar_baz;
	typedef reflexpr(foobar) meta_foobar;

	static_assert(meta::has_scope_v<meta_int>, "");
	static_assert(meta::has_scope_v<meta_foo_bar>, "");
	static_assert(meta::has_scope_v<meta_foo_bar_baz>, "");
	static_assert(meta::has_scope_v<meta_foobar>, "");

	typedef meta::get_scope_m<meta_int> meta_int_s;
	typedef meta::get_scope_m<meta_foo_bar> meta_foo_bar_s;
	typedef meta::get_scope_m<meta_foo_bar_baz> meta_foo_bar_baz_s;
	typedef meta::get_scope_m<meta_foobar> meta_foobar_s;

	static_assert(meta::is_scope_v<meta_int_s>, "");
	static_assert(meta::is_scope_v<meta_foo_bar_s>, "");
	static_assert(meta::is_scope_v<meta_foo_bar_baz_s>, "");
	static_assert(meta::is_scope_v<meta_foobar_s>, "");

	static_assert(meta::is_namespace_v<meta_int_s>, "");
	static_assert(meta::is_global_scope_v<meta_int_s>, "");
	static_assert(meta::is_namespace_v<meta_foo_bar_s>, "");
	static_assert(!meta::is_global_scope_v<meta_foo_bar_s>, "");
	static_assert(meta::is_type_v<meta_foo_bar_baz_s>, "");
	static_assert(meta::is_class_v<meta_foo_bar_baz_s>, "");
	static_assert(!meta::is_namespace_v<meta_foo_bar_baz_s>, "");
	static_assert(meta::is_namespace_v<meta_foobar_s>, "");
	static_assert(meta::is_global_scope_v<meta_foobar_s>, "");
	static_assert(!meta::is_class_v<meta_foobar_s>, "");

	cout << meta::get_name_v<meta_foo_bar_baz> << endl;
	cout << meta::get_name_v<meta_foo_bar_baz_s> << endl;
	cout << meta::get_name_v<meta_foo_bar_s> << endl;

	return 0;
}
\end{minted}

Output:

\begin{verbatim}
baz
bar
foo
\end{verbatim}


\subsubsection{Namespace reflection}

\begin{minted}[tabsize=4]{cpp}
#include <reflexpr>
#include <iostream>

namespace foo { namespace bar { } }

namespace foobar = foo::bar;

int main(void)
{
	using namespace std;

	typedef reflexpr(foo) meta_foo;

	static_assert(is_metaobject_v<meta_foo>, "");

	static_assert(!meta::is_global_scope_v<meta_foo>, "");
	static_assert(meta::is_namespace_v<meta_foo>, "");
	static_assert(!meta::is_type_v<meta_foo>, "");
	static_assert(!meta::is_alias_v<meta_foo>, "");

	static_assert(meta::has_name_v<meta_foo>, "");
	static_assert(meta::has_scope_v<meta_foo>, "");
	cout << meta::get_name_v<meta_foo> << endl;

	typedef reflexpr(foo::bar) meta_foo_bar;

	static_assert(is_metaobject_v<meta_foo_bar>, "");

	static_assert(!meta::is_global_scope_v<meta_foo_bar>, "");
	static_assert(meta::is_namespace_v<meta_foo_bar>, "");
	static_assert(!meta::is_type_v<meta_foo_bar>, "");
	static_assert(!meta::is_alias_v<meta_foo_bar>, "");

	static_assert(meta::has_name_v<meta_foo_bar>, "");
	static_assert(meta::has_scope_v<meta_foo_bar>, "");
	cout << meta::get_name_v<meta_foo_bar> << endl;

	typedef reflexpr(foobar) meta_foobar;

	static_assert(is_metaobject_v<meta_foobar>, "");

	static_assert(!meta::is_global_scope_v<meta_foobar>, "");
	static_assert(meta::is_namespace_v<meta_foobar>, "");
	static_assert(!meta::is_type_v<meta_foobar>, "");
	static_assert(meta::is_alias_v<meta_foobar>, "");

	static_assert(meta::has_name_v<meta_foobar>, "");
	static_assert(meta::has_scope_v<meta_foobar>, "");
	cout << meta::get_name_v<meta_foobar> << " a.k.a ";
	cout << meta::get_name_v<meta::get_aliased_t<meta_foobar>> << endl;

	return 0;
}
\end{minted}

Output:

\begin{verbatim}
foo
bar
foobar a.k.a bar
\end{verbatim}


\subsubsection{Type reflection}

\begin{minted}[tabsize=4]{cpp}
#include <reflexpr>
#include <iostream>

int main(void)
{
	using namespace std;

	typedef reflexpr(unsigned) meta_unsigned;

	static_assert(is_metaobject_v<meta_unsigned>, "");
	static_assert(meta::is_type_v<meta_unsigned>, "");
	static_assert(!meta::is_alias_v<meta_unsigned>, "");

	static_assert(is_same_v<
		meta::get_reflected_type_t<meta_unsigned>,
		unsigned
	>, "");

	static_assert(meta::has_name_v<meta_unsigned>, "");
	cout << meta::get_name_v<meta_unsigned> << endl;

	typedef reflexpr(unsigned*) meta_ptr_unsigned;
	static_assert(meta::has_name_v<meta_ptr_unsigned>, "");
	cout << meta::get_name_v<meta_ptr_unsigned> << endl;

	return 0;
}
\end{minted}

Output:

\begin{verbatim}
unsigned int
unsigned int
\end{verbatim}


\subsubsection{Typedef reflection}

\begin{minted}[tabsize=4]{cpp}
#include <reflexpr>
#include <iostream>

namespace foo {

typedef int bar;
using baz = bar;

} // namespace foo

int main(void)
{
	using namespace std;

	typedef reflexpr(foo::baz) meta_foo_baz;

	static_assert(is_metaobject_v<meta_foo_baz>, "");
	static_assert(meta::is_type_v<meta_foo_baz>, "");
	static_assert(meta::is_alias_v<meta_foo_baz>, "");

	static_assert(is_same_v<
		meta::get_reflected_type_t<meta_foo_baz>,
		foo::baz
	>, "");

	static_assert(meta::has_name_v<meta_foo_baz>, "");
	cout << meta::get_base_name_v<meta_foo_baz> << endl;

	typedef meta::get_aliased_m<meta_foo_baz> meta_foo_bar;

	static_assert(is_metaobject_v<meta_foo_bar>, "");
	static_assert(meta::is_type_v<meta_foo_bar>, "");
	static_assert(meta::is_alias_v<meta_foo_bar>, "");

	static_assert(is_same_v<
		meta::get_reflected_type_t<meta_foo_bar>,
		foo::bar
	>, "");

	static_assert(meta::has_name_v<meta_foo_bar>, "");
	cout << meta::get_base_name_v<meta_foo_bar> << endl;

	typedef meta::get_aliased_m<meta_foo_bar> meta_int;

	static_assert(is_metaobject_v<meta_int>, "");
	static_assert(meta::is_type_v<meta_int>, "");
	static_assert(!meta::is_alias_v<meta_int>, "");

	static_assert(is_same_v<
		meta::get_reflected_type_t<meta_int>,
		int
	>, "");

	static_assert(meta::has_name_v<meta_int>, "");
	cout << meta::get_base_name_v<meta_int> << endl;

	return 0;
}

\end{minted}

Output:

\begin{verbatim}
baz
bar
int
\end{verbatim}


\subsubsection{Class alias reflection}

\begin{minted}[tabsize=4]{cpp}
#include <reflexpr>
#include <iostream>

struct foo { };
using bar = foo;

int main(void)
{
	using namespace std;

	typedef reflexpr(bar) meta_bar;

	static_assert(is_metaobject_v<meta_bar>, "");
	static_assert(meta::is_type_v<meta_bar>, "");
	static_assert(meta::is_class_v<meta_bar>, "");
	static_assert(meta::is_alias_v<meta_bar>, "");

	static_assert(is_same_v<meta::get_reflected_type_t<meta_bar>, bar>, "");

	static_assert(meta::has_name_v<meta_bar>, "");
	cout << meta::get_base_name_v<meta_bar> << endl;

	typedef meta::get_aliased_m<meta_bar> meta_foo;

	static_assert(is_metaobject_v<meta_foo>, "");
	static_assert(meta::is_type_v<meta_foo>, "");
	static_assert(meta::is_class_v<meta_foo>, "");
	static_assert(!meta::is_alias_v<meta_foo>, "");

	static_assert(is_same_v<meta::get_reflected_type_t<meta_foo>, foo>, "");

	static_assert(meta::has_name_v<meta_foo>, "");
	cout << meta::get_base_name_v<meta_foo> << endl;

	return 0;
}
\end{minted}

Output:

\begin{verbatim}
bar
foo
\end{verbatim}

Note that \texttt{meta\_bar} is both a \meta{Alias} and \meta{Class}.


\subsubsection{Class data members (1)}

\begin{minted}[tabsize=4]{cpp}
#include <reflexpr>
#include <iostream>

struct foo
{
private:
	int _i, _j;
public:
	static constexpr const bool b = true;
	float x, y, z;
private:
	static double d;
};

int main(void)
{
	using namespace std;

	typedef reflexpr(foo) meta_foo;

	// (public) data members
	typedef meta::get_data_members_m<meta_foo> meta_data_mems;

	static_assert(is_metaobject_v<meta_data_mems>, "");
	static_assert(meta::is_sequence_v<meta_data_mems>, "");

	cout << meta::get_size_v<meta_data_mems> << endl;

	// 0-th (public) data member
	typedef meta::get_element_m<meta_data_mems, 0> meta_data_mem0;

	static_assert(is_metaobject_v<meta_data_mem0>, "");
	static_assert(meta::is_variable_v<meta_data_mem0>, "");
	static_assert(meta::has_type_v<meta_data_mem0>, "");

	cout << meta::get_base_name_v<meta_data_mem0> << endl;

	// 2-nd (public) data member
	typedef meta::get_element_m<meta_data_mems, 2> meta_data_mem2;

	static_assert(is_metaobject_v<meta_data_mem2>, "");
	static_assert(meta::is_variable_v<meta_data_mem2>, "");
	static_assert(meta::has_type_v<meta_data_mem2>, "");

	cout << meta::get_base_name_v<meta_data_mem2> << endl;

	// all data members
	typedef meta::get_all_data_members_m<meta_foo> meta_all_data_mems;

	static_assert(is_metaobject_v<meta_all_data_mems>, "");
	static_assert(meta::is_sequence_v<meta_all_data_mems>, "");

	cout << meta::get_size_v<meta_all_data_mems> << endl;

	// 0-th (overall) data member
	typedef meta::get_element_m<meta_all_data_mems, 0> meta_all_data_mem0;

	static_assert(is_metaobject_v<meta_all_data_mem0>, "");
	static_assert(meta::is_variable_v<meta_all_data_mem0>, "");
	static_assert(meta::has_type_v<meta_all_data_mem0>, "");

	cout << meta::get_base_name_v<meta_all_data_mem0> << endl;

	// 3-rd (overall) data member
	typedef meta::get_element_m<meta_all_data_mems, 3> meta_all_data_mem3;

	static_assert(is_metaobject_v<meta_all_data_mem3>, "");
	static_assert(meta::is_variable_v<meta_all_data_mem3>, "");
	static_assert(meta::has_type_v<meta_all_data_mem3>, "");

	cout << meta::get_base_name_v<meta_all_data_mem3> << endl;

	return 0;
}

\end{minted}

This produces the following output:

\begin{verbatim}
4
b
y
7
_i
x
\end{verbatim}

\subsubsection{Class data members (2)}

\begin{minted}[tabsize=4]{cpp}
#include <reflexpr>
#include <iostream>

struct foo
{
private:
	int _i, _j;
public:
	static constexpr const bool b = true;
	float x, y, z;
private:
	static double d;
};

template <typename ... T>
void eat(T ... ) { }

template <typename Metaobjects, std::size_t I>
int do_print_data_member(void)
{
	using namespace std;

	typedef meta::get_element_m<Metaobjects, I> metaobj;

	cout	<< I << ": "
		<< (meta::is_public_v<metaobj>?"public":"non-public")
		<< " "
		<< (meta::is_static_v<metaobj>?"static":"")
		<< " "
		<< meta::get_base_name_v<meta::get_type_m<metaobj>>
		<< " "
		<< meta::get_base_name_v<metaobj>
		<< endl;

	return 0;
}

template <typename Metaobjects, std::size_t ... I>
void do_print_data_members(std::index_sequence<I...>)
{
	eat(do_print_data_member<Metaobjects, I>()...);
}

template <typename Metaobjects>
void do_print_data_members(void)
{
	using namespace std;

	do_print_data_members<Metaobjects>(
		make_index_sequence<
			meta::get_size_v<Metaobjects>
		>()
	);
}

template <typename MetaClass>
void print_data_members(void)
{
	using namespace std;

	cout<< "Public data members of " << meta::get_base_name_v<MetaClass> << endl;

	do_print_data_members<meta::get_data_members_m<MetaClass>>();
}

template <typename MetaClass>
void print_all_data_members(void)
{
	using namespace std;

	cout << "All data members of " << meta::get_base_name_v<MetaClass> << endl;

	do_print_data_members<meta::get_all_data_members_m<MetaClass>>();
}

int main(void)
{
	print_data_members<reflexpr(foo)>();
	print_all_data_members<reflexpr(foo)>();

	return 0;
}
\end{minted}

This program produces the following output:

\begin{verbatim}
Public data members of foo
0: public static bool b
1: public  float x
2: public  float y
3: public  float z
All data members of foo
0: non-public  int _i
1: non-public  int _j
2: public static bool b
3: public  float x
4: public  float y
5: public  float z
6: non-public static double d
\end{verbatim}


\subsubsection{Class data members (3)}

\begin{minted}[tabsize=4]{cpp}
#include <reflexpr>
#include <iostream>

struct A
{
	int a;
};

class B : public A
{
private:
	bool b;
};

class C : public B
{
public:
	char c;
};

int main(void)
{
	using namespace std;

	typedef reflexpr(A) meta_A;
	typedef reflexpr(B) meta_B;
	typedef reflexpr(C) meta_C;

	cout << meta::get_size_v<meta::get_public_data_members_m<meta_A>> << endl;
	cout << meta::get_size_v<meta::get_public_data_members_m<meta_B>> << endl;
	cout << meta::get_size_v<meta::get_public_data_members_m<meta_C>> << endl;

	cout << meta::get_size_v<meta::get_data_members_m<meta_A>> << endl;
	cout << meta::get_size_v<meta::get_data_members_m<meta_B>> << endl;
	cout << meta::get_size_v<meta::get_data_members_m<meta_C>> << endl;

	return 0;
}
\end{minted}

This program produces the following output:

\begin{verbatim}
1
0
1
1
1
1
\end{verbatim}

Note that neither the result of \texttt{get\_data\_members} nor the result of
\texttt{get\_public\_data\_members} includes the inherited data members.


%\section{Design preferences}


\subsection{Consistency}

The reflection facility as a whole
should be consistent, instead of being composed of ad-hoc, individually
designed parts.

\subsection{Reusability}

The provided metadata should be reusable
in many situations and for many different purposes. It should not focus only
on the obvious use cases. This is closely related to {\em completeness} (below).

\subsection{Flexibility}

The basic reflection and the libraries
built on top of it should be designed
in a way that they are eventually usable during both compile-time
and run-time and under various paradigms (object-oriented, functional, etc.),
depending on the application needs.

\subsection{Encapsulation}

The metadata should not be exposed directly to the used by compiler built-ins, etc.
Instead it should be accessible through conceptually well-defined interfaces.

\subsection{Stratification}

Reflection should be non-intrusive,
and the meta-level should be separated from the base-level language
constructs it reflects. Also, reflection should not be implemented
in a all-or-nothing manner. Things that are not needed for a particular application,
should not generally be compiled-into such application.

\subsection{Ontological correspondence}

The meta-level facilities should
correspond to the ontology of the base-level C++ language constructs
which they reflect. This basically means that all existing language
features should be reflected and new ones should not be invented.

\subsection{Completeness}

The proposed reflection facility should
provide as much useful metadata as possible, including various specifiers,
(like constness, storage-class, access, etc.), namespace members,
enumerated types, iteration of namespace members and much more.

\subsection{Ease of use}

Although reflection-based metaprogramming
allows to implement very complicated things, simple things
should be kept simple.

\subsection{Cooperation with rest of the standard and other librares}

Reflection should be easily
usable with the existing introspection facilites (like \verb@type_traits@)
already provided by the standard library and with other libraries.

\subsection{Extensibility}

The programmers should be able to define their own models of metaobject concepts
and these should be usable with the rest of the metaobjects provided by the
compiler. This way if some of the metaobjects generated by the compiler are not
suitable for a particular purpose they can be individually replaced with
hand-coded variants.


\renewcommand\refname{\arabic{section}\hspace{1em}References}

\stepcounter{section}
\addcontentsline{toc}{section}{\refname}

\begin{thebibliography}{100}

\bibitem{mirror-doc-cpp11}
Mirror C++ reflection library documentation (C++11 version),
\url{http://kifri.fri.uniza.sk/~chochlik/mirror-lib/html/}.

\end{thebibliography}{100}



\begin{appendices}

\section{Additional use cases}
\label{appendix-other-use-cases}

This sections describes further, less common, use-cases for reflection or use-cases
requiring advanced reflection features.

\subsection{SQL schema generation}

We need to create an SQL/DDL (data definition language) script for creating a schema
with tables which will be storing the values of all structures in namespace C++ \verb@foo@
having names starting with \verb@persistent_@:

\begin{minted}[tabsize=4]{cpp}

const char* translate_to_sql(const std::string& type_name)
{
	if(type_name == "int")
		return "INTEGER";
	/* .. etc. */
}

template <typename MetaMemVar>
void create_table_column_from(MetaMemVar)
{
	if(!std::is_base_of<
		variable_tag,
		metaobject_category<MetaMemVar>
	>()) return;

	std::cout << base_name<MetaMemVar>() << " ";

	std::cout << translate_to_sql(base_name<type<MetaMemVar>());

	if(starts_with(base_name<MetaMemVar>(), "id_"))
	{
		std::cout << " PRIMARY KEY";
	}
	std::cout << std::endl;
}

template <typename MetaClass>
void create_table_from(MetaClass)
{
	if(!std::is_base_of<
		class_tag,
		metaobject_category<MetaClass>
	>()) return;

	std::cout << "CREATE TABLE "
	          << strip_prefix("persistent_", base_name<MetaClass>())
	          << "(" << std::endl;

	for_each<members<MetaClass>>(create_table_column_from);

	std::cout << ");"
}

template <typename MetaNamespace>
void create_schema_from(MetaNamespace)
{
	std::cout << "CREATE SCHEMA "
	          << base_name<MetaNamespace>()
	          << ";" << std::endl;

	for_each<members<MetaNamespace>>(create_table_from);
}

int main(void)
{
	create_schema_from(reflected(foo));
	return 0;
}

\end{minted}

This example shows the following features from N4111:

\begin{itemize}
\item{namespace reflection,}
\item{namespace member reflection,}
\item{class member reflection,}
\item{use of metaobject sequence,}
\item{metaobject categorization,}
\item{base names and the \meta{Named} concept.}
\end{itemize}

Furthermore reflection could be used to implement actual object-relational mapping,
together with a library like \verb@SOCI@, \verb@ODBC@, \verb@libpq@, etc.

\subsection{Implementing delegation or decorators}

We need to create a decorator class, which wraps an instance of another
class, implements similar interface as the original class, writes info about
each member function call into a log and then delegates the call to the private member object:

\begin{minted}[tabsize=4]{cpp}

class foo
{
public:
	void f1(void);

	int f2(int a, int b);

	double f3(float a, long b, double c, const std::string& d);
};

class logging_foo
{
private:
	foo _obj;
	loglib::log_sink _log;

	template <typename MetaFunction, typename ... P>
	void _do_log_call(const P&...);
public:
	void f1(void)
	{
		_do_log_call<mirrored(this::function)>();
		_obj.f1();
	}

	int f2(int a, int b);
	{
		_do_log_call<mirrored(this::function)>();
		return _obj.f2(a, b);
	}

	double f3(float a, long b, double c, const std::string& d);
	{
		_do_log_call<mirrored(this::function)>();
		return _obj.f3(a, b, c, d);
	}
};

\end{minted}

Obviously the definition of \verb@logging_foo@ is very repetitive and if this
pattern is recurring in the code it may lead to subtle, hard to track bugs,
so we may wish to automate the implementation.

Reflection to the rescue!

\begin{minted}[tabsize=4]{cpp}

template <typename Wrapped>
class logging_base
{
protected:
	Wrapped _obj;
	loglib::log_sink _log;

	template <typename MetaFunction, typename ... P>
	void _do_log_call(const P&...);
};

\end{minted}

\verb@logging_base@ is a common virtual base class holding the wrapped object
and the log sink.

\begin{minted}[tabsize=4]{cpp}

template <typename Wrapped, typename MetaFunction>
class logging_helper
 : virtual public logging_base<Wrapped>
{
public:
	template <typename ... P>
	auto identifier(base_name<MetaFunction>::value)(P&& ... p)
	{
		this->_do_log_call<MetaFunction>(std::forward<P>(p)...);

		auto mfp = pointer<MetaFunction>::get();

		return (this->_obj.*mfp)(std::forward<P>(p)...);
	}
};

\end{minted}

\verb@logging_helper@ is a unit implementing the delegation of a single function
call from the interface of the \verb@Wrapped@ class. 

The \verb@identifier@ operator is used here to define the name of the member function
to be the same as the name of the wrapped function.

If the idea of the \verb@identifier@ operator is scrapped, it would still be doable
in terms of the \verb@named_mem_var@ template as defined in N4111, or some variation
on that theme.

\begin{minted}[tabsize=4]{cpp}

template <typename Wrapped, typename ... MetaFunctions>
class logging_helpers
 : public logging_helper<Wrapped, MetaFunctions>...
{ };

\end{minted}

\verb@logging_helpers@ inherits from multiple \verb@logging_helper@ units
each having a single \meta{Function} reflecting respective member functions
of the \verb@Wrapper@ class.

\begin{minted}[tabsize=4]{cpp}

template <typename Wrapped, typename MetaFunctionSeq, typename IdxSeq>
class logging_impl;

template <typename Wrapped, typename MetaFunctionSeq, std::size_t ... I>
class logging_impl<Wrapped, MetaFunctionSeq, std::index_sequence<I...>>
 : public logging_helpers<Wrapped, at<MetaFunctionSeq, I>...>
{ };

\end{minted}

\verb@logging_impl@ uses a standard \verb@index_sequence@ to extract the
individual \meta{Functions} from the metafunction sequence and passes them
to \verb@logging_helpers@ as a parameter pack.


\begin{minted}[tabsize=4]{cpp}

template <typename Wrapped>
class logging
 : public logging_impl<
	Wrapped,
	members<mirrored(Wrapped)>,
	std::make_index_sequence<size<members<mirrored(Wrapped)>>::value>
>
{ };

typedef logging<foo> logging_foo;

\end{minted}

The \verb@logging@ template makes the use of \verb@logging_impl@ convenient.

Note that the metaobject sequence 'returned' by \verb@members<...>@ should
be filtered to contain only \meta{Function}s.

This programming pattern of creating a new class with the same or similar interface
than another class is quite frequent and includes not just typical decorators or delegation
but also adapters, type-erasures, etc.\footnote{See also the following use-case.}

The following features are shown in this use-case:

\begin{itemize}
\item{class member reflection,}
\item{class member function reflection,}
\item{use of the reflection operator,}
\item{use of the \verb@identifier@ operator or the \verb@named_mem_var@ templates.}
\end{itemize}


\subsection{Structure data member transformations}
\label{use-case-struct-transf}

We need to create a new structure, which has data members with the same names
as an original structure, but we need to change some of the properties
of the data members (for example their types).

For example we need to transform:

\begin{minted}[tabsize=4]{cpp}

struct foo
{
	bool b;
	char c;
	double d;
	float f;
	std::string s;
};

\end{minted}

into

\begin{minted}[tabsize=4]{cpp}

struct rdbs_table_placeholder_foo
{
	column_placeholder<bool>::type b;
	column_placeholder<char>::type c;
	column_placeholder<double>::type d;
	column_placeholder<float>::type f;
	column_placeholder<std::string>::type s;
};

\end{minted}

By using the proposed \verb@identifier@ operator\footnote{See appendix~\ref{appendix-operator-identifier}.}, class member reflection
and multiple inheritance we can create a new structure that is
nearly equivalent to \verb@rdbs_table_foo@ via metaprogramming:

\begin{minted}[tabsize=4]{cpp}

template <typename MetaVariable>
struct rdbs_table_placeholder_helper
{
	typename column_placeholder<
		typename original_type<type<MetaVariable>>::type
	>::type identifier(base_name<MetaVariable>::value);
};

template <typename ... MetaVariables>
struct rdbs_table_placeholder_helpers
 : rdbs_table_placeholder_helper<MetaVariables>...
{ };

template <typename MetaVariableSeq, typename IdxSeq>
class rdbs_table_impl;

template <typename MetaVariableSeq, std::size_t ... I>
class rdbs_table_impl<MetaFunctionSeq, std::index_sequence<I...>>
 : public rdbs_table_placeholder_helpers<at<MetaFunctionSeq, I>...>
{ };

typedef rdbs_table_impl<
	members<mirrored(foo)>,
	std::make_index_sequence<size<members<mirrored(foo)>>::value>
> rdbs_table_placeholder_foo;

\end{minted}

This examples uses the following features.

\begin{itemize}
\item{class member reflection,}
\item{use of the reflection operator,}
\item{use of the \verb@identifier@ operator or the \verb@named_mem_var@ templates.}
\end{itemize}




\section{Additions to N4111}

The examples described in this paper use several features not described in N4111
(these will be added to the next revision of that paper -- N4451):

\subsection{Source file and line}

Template class \verb@source_file@ should be defined for \meta{object}s
and should "return" a compile-time string containing the path to the source file
where the base-level construct reflected by the metaobject was declared.

\begin{minted}[tabsize=4]{cpp}
template <>
struct source_file<MetaObject>
 : String
{ };
\end{minted}

Template class \verb@source_line@ should be defined for \meta{object}s
and should inherit from \verb@integral_constant<unsigned, Line>@ where
\verb@Line@ is the line number in the source file
where the base-level construct reflected by the metaobject was declared.

\begin{minted}[tabsize=4]{cpp}
template <>
struct source_file<MetaObject>
 : String
{ };
\end{minted}

\subsection{For each element in Metaobject sequence}

Template function \verb@for_each@, should be defined for every metaobject sequence
and should call the specified unary functor taking values of types conforming to
the same metaobject concept as the elements of the metaobject sequence as arguments.

\begin{minted}[tabsize=4]{cpp}
template <typename MetaobjectSequence, typename UnaryFunc>
void for_each(UnaryFunc func)
{
	/* call func on each element in the sequence */
}
\end{minted}

The interface of \meta{objectSequence} as defined in N4111 should be enough
to define a single generic implementation of this function without the need
to write specialization for every type modelling this concept.

\subsection{The \meta{Positional} concept}

The \meta{Positional} concept defines the interface for metaobjects reflecting
base-level constructs having a fixed position, like function or template parameters,
class inheritance clauses, etc.

The \verb@has_position@ template class can be used to distinguish metaobjects modelling
this concept. Is should inherit from \verb@true_type@ for \meta{Positional}s and from
\verb@false_type@ otherwise.

\begin{minted}{cpp}
template <typename X>
struct has_position
 : false_type
{ };

template <>
struct has_position<MetaPositional>
 : true_type
{ };
\end{minted}

The \verb@position@ template class inheriting
from \verb@integral_constant<size_t, I>@ type (where \verb@I@ is
a zero-based position of the reflected base-level language construct)
can be used to obtain the value of the index.

\begin{minted}{cpp}
template <typename T>
struct position;

template <>
struct position<MetaPositional>
 : integral_constant<size_t, I>
{ };
\end{minted}

\subsection{Position of a base in the list of base classes}

Every model of \meta{Inheritance} should also conform to the \meta{Positional}
concept described above.

\subsection{Pointers to reflected variables and functions}

For models of \meta{Variable} and \meta{Function} the \verb@pointer@ template class
should be defined as:

\begin{minted}[tabsize=4]{cpp}
template <>
struct pointer<MetaVariable>
{
	typedef _unspecified_ type;

	static type get(void);
};

template <>
struct pointer<MetaFunction>
{
	typedef _unspecified_ type;

	static type get(void);
};
\end{minted}

For \meta{Variable}s or \meta{Function}s reflecting namespace-level variables or functions
the \verb@get@ function should return a pointer to that variable or function respectively.

If the \meta{Variable} or \meta{Function} reflect class members then the \verb@get@ function
should return a pointer to data member or pointer to member function respectively.

\subsection{Context-dependent reflection}
\label{appendix-context-dependent-reflection}

Special expressions in the form of \verb@this::{namespace,class,function}@ should be added
as valid arguments for the reflection operator and should return metaobjects depending
on the context where such invocation of the reflection operator was used.

\begin{itemize}
\item{\verb@mirrored(this::namespace)@} reflects the namespace inside of which the reflection
operator was invoked.
\item{\verb@mirrored(this::class)@} reflects the class inside of which the reflection
operator was invoked. This should also work inside of member functions, constructors and operators
of that class.
\item{\verb@mirrored(this::function)@} reflects the function inside of which the reflection
operator was invoked.
\end{itemize}

\subsection{Turning compile-time strings into identifiers}
\label{appendix-operator-identifier}

Inspired by the idea of {\em name literals} as mentioned on the WG mailing list,
we suggest to consider adding a new functionality to the core language, allowing to specify
identifiers as compile-time constant C-string literal expression, i.e. expressions
evaluating into values of \verb@constexpr const char [N]@.

This could be implemented either by using a new operator (or recycling an old one),
or maybe by using generalized attributes.
In the use-cases described in this paper the \verb@identifier@ operator is used, but we do not have
any strong preference for the name of this operator.

For example:

\begin{minted}[tabsize=4]{cpp}

identifier("int") identifier("main")(
	int idenitifier("argc"),
	const identifier("char")* identifier("argv")
)
{
	using namespace identifier(base_name<mirrored(std)>::value);
	for(int i=0; i<argc; ++i)
	{
		cout << argv[i] << endl;
	}
	return 0;
}

\end{minted}

would be equivalent to

\begin{minted}[tabsize=4]{cpp}
int main(int argc, const char* argv)
{
	using namespace std;
	/* ... */
}
\end{minted}

The content of the constexpr string passed as the argument to \verb@identifier@
should be encoded in the source character set and subject to the same restrictions
which are placed on identifiers.

The idea is to replace preprocessor token pasting with much more flexible constexpr C++ expressions.
Adding this feature would also allow to remove the \verb@named_mem_var@ and
\verb@named_typedef@ metafunctions which were in N4111 defined as part of
the interface of \meta{Named}.

This addition has the potential to complicate the processing of a translation unit
by the compiler and would logically fit somewhere between phases 6 and 8 as described
in the standard. If the use of regular templates for the purpose of creating the
constexpr identifier strings would be too complicated to implement, phase 6 could
be extended to allow simple compile-time text manipulation (comparison, concatenation,
substrings, etc.) by a set of dedicated functions.



\end{appendices}

\end{document}
