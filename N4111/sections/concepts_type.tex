\subsection{MetaType}
\label{concept-MetaType}

\begin{tikzpicture}
\node [concept] (Metaobject) {Metaobject};
\node [concept] (MetaNamed)[above right=of Metaobject] {MetaNamed}
	edge [inheritance, bend right] (Metaobject);
\node [concept] (MetaScoped)[below right=of Metaobject] {MetaScoped}
	edge [inheritance, bend left] (Metaobject);
\node [concept] (MetaNamedScoped)[below right=of MetaNamed, above right=of MetaScoped] {MetaNamedScoped}
	edge [inheritance, bend right] (MetaNamed)
	edge [inheritance, bend left] (MetaScoped);
\node [concept] (MetaType)[right=of MetaNamedScoped] {MetaType}
	edge [inheritance] (MetaNamedScoped);
\end{tikzpicture}

\meta{Type} is a \meta{NamedScoped} reflecting types.

In addition to the requirements inherited from \meta{NamedScoped}, the following is required:

The \verb@metaobject_category@ template class specialization for a \meta{Type} should
inherit from \verb@type_tag@:

\begin{minted}{cpp}
template <>
struct metaobject_category<MetaType>
 : type_tag
{ };
\end{minted}

\subsubsection{\texttt{original\_type}}

A template class \verb@original_type@ should be defined and should "return"
the original type reflected by this \meta{Type}:

\begin{minted}{cpp}
template <typename T>
struct original_type;

template <>
struct original_type<MetaType>
{
	static_assert(not(is_template<MetaType>::value), "");
	typedef original-type type;
};
\end{minted}

For example, if \verb@__meta_int@ is a metaobject reflecting the \verb@int@ type,
then the specialization of \verb@original_type@ should be following:

\begin{minted}{cpp}
template <>
struct original_type<__meta_int>
{
	typedef int type;
};
\end{minted}

\textbf{Note:} If a concept derived from \meta{Type}, for example a \meta{Class},
is also a \meta{Template} (i.e. is reflecting a template not a concrete type),
then the \verb@original_type@ template should be left undefined.

