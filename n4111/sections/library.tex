\section{Additions to the library}
\label{section-Library}

In order to simplify composition of the metaobjects and metafunctions defined
\hyperref[section-Concepts]{above}, several further additions to the standard
library should be made.

\subsection{Metaobject expressions}

A {\em metaobject expression} is a class, which has a member typedef called \verb@type@
which is a model of \meta{object}:

\begin{minted}{cpp}
struct MetaobjectExpression
{
	typedef Metaobject type;
};

static_assert(is_metaobject<MetaobjectExpression::type>(), "");
\end{minted}

Any \meta{object} is a {\em metaobject expression}.

\subsubsection{\texttt{evaluate}}

A template class called \verb@evaluate@ should be defined and "return" a \meta{object}
resulting from a {\em metaobject expression}:

\begin{minted}{cpp}
template <class MetaobjectExpression>
struct evaluate
 : Metaobject
{ };
\end{minted}

that could be implemented for example as follows:

\begin{minted}{cpp}
template <class X, bool IsMetaobject>
struct do_evaluate;

template <class X>
struct do_evaluate<X, true>
 : X
{ };

template <class X>
struct do_evaluate<X, false>
 : do_evaluate<
	typename X::type,
	is_metaobject<typename X::type>::value
> { };

template <class X>
struct evaluate
 : do_evaluate<X, is_metaobject<X>::value>
{ };

\end{minted}

\subsection{Default implementation of metafunctions}

The default implementation of the metafunction template classes defined above,
should follow this pattern:

\begin{minted}{cpp}
template <typename T>
struct Template;

template <typename T>
struct Template
 : Template<typename evaluate<T>::type>
{ };
\end{minted}

Where \verb@Template@ is each of the following:

\begin{minipage}[t]{0.5\textwidth}
\begin{itemize}
\item \verb@metaobject_category@
\item \verb@specifier_category@
\item \verb@keyword@
\item \verb@base_name@
\item \verb@full_name@
\item \verb@named_typedef@
\item \verb@named_mem_var@
\item \verb@scope@
\item \verb@members@
\item \verb@overloads@
\item \verb@type@
\item \verb@base_classes@
\item \verb@base_class@
\item \verb@base_type@
\item \verb@result_type@
\item \verb@parameters@
\item \verb@template_parameters@
\end{itemize}
\end{minipage}
\begin{minipage}[t]{0.5\textwidth}
\begin{itemize}
\item \verb@template_arguments@
\item \verb@template_@
\item \verb@exceptions@
\item \verb@instantiation@
\item \verb@position@
\item \verb@value@
\item \verb@elaborated_type_specifier@
\item \verb@access_specifier@
\item \verb@constexpr_specifier@
\item \verb@noexcept_specifier@
\item \verb@const_specifier@
\item \verb@inheritance_specifier@
\item \verb@linkage_specifier@
\item \verb@storage_specifier@
\item \verb@is_pure@
\item \verb@is_pack@
\item \verb@original_type@
\end{itemize}
\end{minipage}

For example:

\begin{minted}{cpp}
template <typename T>
struct metaobject_category
 : metaobject_category<typename evaluate<T>::type>
{ };
\end{minted}

This will allow to compose metaobject expressions into algorithms. For example:

\begin{minted}[tabsize=4]{cpp}
// print the number of members of the scope where mycls is defined
cout << size<members<scope<mirrored(mycls)>>>() << endl;

// print the name of the first base class of mycls
cout <<
	base_name<base_class<at<base_classes<mirrored(mycls)>, 0>>>()
<< endl;

// print the access specifier keyword of the second base of mycls
cout <<
	keyword<access_specifier<at<base_classes<mirrored(mycls)>, 1>>>()
<< endl;

// print the fully qualified name of the scope of
// the source type of the third member of mycls
cout <<
	full_name<scope<type<at<members<mirrored(mycls)>, 2>>>>()
<< endl;
\end{minted}

