\subsection{MetaOperator}
\label{concept-MetaOperator}

\begin{tikzpicture}
\node [concept] (Metaobject) {Metaobject};
\node [concept] (MetaNamed)[above right=of Metaobject] {MetaNamed}
	edge [inheritance, bend right] (Metaobject);
\node [concept] (MetaScoped)[below right=of Metaobject] {MetaScoped}
	edge [inheritance, bend left] (Metaobject);
\node [concept] (MetaNamedScoped)[below right=of MetaNamed, above right=of MetaScoped] {MetaNamedScoped}
	edge [inheritance, bend right] (MetaNamed)
	edge [inheritance, bend left] (MetaScoped);
\node [concept] (MetaFunction)[above right=of MetaNamedScoped] {MetaFunction}
	edge [inheritance, bend right] (MetaNamedScoped);
\node [concept] (MetaOperator)[below right=of MetaNamedScoped] {MetaOperator}
	edge [inheritance] (MetaFunction);
\end{tikzpicture}


\meta{Operator} is a \meta{Function} which reflects an operator.

In addition to the requirements inherited from \meta{Function},
models of \meta{Operator} must conform to the following:

The \verb@metaobject_category@ template class specialization for a \meta{Operator} should
inherit from \verb@operator_tag@:

\begin{minted}{cpp}
template <>
struct metaobject_category<MetaOperator>
 : operator_tag
{ };
\end{minted}

