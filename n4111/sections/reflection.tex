\section{Reflection operator}

The metaobjects reflecting some program feature \verb@X@ as
described above should be made available to the user by
the means of a new operator or expression.
More precisely, the reflection operator returns a type conforming to a particular
metaobject concept, depending on the reflected expression.

Since adding a new keyword has the potential to break existing code,
we do not insist on any particular expression, here follows a list of suggestions
in order of preference (from the most to the least preferrable):

\begin{itemize}
\item{\verb@mirrored(X)@}
\item{\verb@reflected(X)@}
\item{\verb@reflexpr(X)@}
\item{\verb@|X@}
\item{\verb@[[X]]@}
\item{\verb@<<X>>@}
\end{itemize}

The reflected expression \verb@X@ in the items listed above can be any of the following:

\begin{itemize}
\item{\verb@::@} -- The global scope, the returned metaobject is a {\meta{GlobalScope}}.
\item{{\em Namespace name}} -- (\verb@std@) the returned metaobject is a {\meta{Namespace}}.
\item{{\em Type name}} -- (\verb@long double@) the returned metaobject is a {\meta{Type}}.
\item{{\em \verb@typedef@ name}} -- (\verb@std::size_t@ or \verb@std::string@)
     the returned metaobject is a {\meta{Typedef}}.
\item{{\em Template name}} -- (\verb@std::tuple@ or \verb@std::map@)
     the returned metaobject is a {\meta{Template}}.
\item{{\em Class name}} -- (\verb@std::thread@ or \verb@std::map<int, double>@)
     the returned metaobject is a {\meta{Class}}.
\item{{\em Function name}} -- (\verb@std::sin@ or \verb@std::string::size@) the returned metaobject
     is a {\meta{OverloadedFunction}}.
\item{{\em Function signature}} -- (\verb@std::sin(double)@ or \verb@std::string::front(void) const@)
     the returned metaobject is a {\meta{Function}}. The signature may be specified without the
     return value type.
\item{{\em Constructor signature}} -- (\verb@std::pair<char, double>::pair(char, double)@
     or \verb@std::string::string(void)@) the returned metaobject is a {\meta{Constructor}}.
\item{{\em Variable name}} -- (\verb@std::errno@) the returned metaobject is a {\meta{Variable}}.
%\item{TODO}
\end{itemize}

The reflection operator or expression should have access to \verb@private@ and
\verb@protected@ members of classes. The following should be valid:

\begin{minted}{cpp}
struct A
{
	int a;
};

class B
{
protected:
	int b;
};

class C
 : protected A
 , public B
{
private:
	int c;
};

typedef mirrored(A::a) meta_A_a;
typedef mirrored(B::b) meta_B_b;
typedef mirrored(C::a) meta_C_a;
typedef mirrored(C::b) meta_C_b;
typedef mirrored(C::c) meta_C_c;

\end{minted}
