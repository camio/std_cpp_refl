\documentclass[11pt,a4paper,oneside]{scrartcl}

\newcommand{\mchname}{Mat\'{u}\v{s} Chochl\'{i}k}
\newcommand{\mchmail}{chochlik@gmail.com}
\newcommand{\docname}{Static reflection (rev. 2)}
\newcommand{\docnum}{N4111}
\newcommand{\docdate}{2014-07-02}

\usepackage[utf8]{inputenc}
\usepackage{url}
\usepackage[colorlinks=true]{hyperref}
\usepackage{parskip}
\usepackage[titletoc]{appendix}

\usepackage{listings}
\usepackage{minted}
\lstset{basicstyle=\footnotesize\ttfamily,breaklines=true}

\usepackage{fancyhdr}
\setlength{\headheight}{14pt}
\pagestyle{fancyplain}
\lhead{\fancyplain{}{\docnum - \docname}}
\rhead{}
\rfoot{\fancyplain{}{\thepage}}
\cfoot{}

\usepackage[pdftex]{graphicx}
\DeclareGraphicsExtensions{.pdf,.png,.jpg,.mps,.eps}
\graphicspath{{images/}}

\usepackage{tikz}
\usetikzlibrary{arrows,positioning}
\tikzstyle{concept}=[
	rectangle,	
	very thick,
	draw=red!80!black!80,
	top color=white,
	bottom color=red!20,
	node distance=0.5em and 1.5em
]
\tikzstyle{inheritance}=[
	->,
	shorten >=1pt,
	>=open triangle 90,
	very thick
]
   

\setcounter{tocdepth}{3} 

\title{\docname}

\author{\mchname (\mchmail)}

\newcommand{\concept}[1]{\hyperref[concept-#1]{\em{#1}}}
\newcommand{\meta}[1]{\concept{Meta#1}}

\begin{document}

\begin{tabular}{r l}
Document number: & \docnum\\
Date: & \docdate\\
Project: & Programming Language C++, SG7, Reflection\\
Reply-to: & \mchname (\href{mailto:\mchmail}{\mchmail})\\
\end{tabular}

\begin{center}
\vskip 2em
{\Huge \docname}
\vskip 1em
{\emph \mchname}
\vskip 2em
\end{center}

\paragraph{Abstract}

In N3996 \cite{n3996} we presented some ideas on the design and possible
implementation of a compile-time reflection facility for standard C++.
N3996 also contained extensive discussion about usefulness of reflection,
description of several use-cases and motivational examples from the Mirror
reflection utilities \cite{mirror-doc-cpp11}.
The actual proposal was however found to be confusing in some points.
This paper tries to make the proposal more concise and detailed.

\tableofcontents

\section{Introduction}

Reflection and reflective programming can be used
for a wide range of tasks such as implementation of serialization-like operations,
remote procedure calls, scripting, automated GUI-generation,
implementation of several software design patterns, etc.
C++ as one of the most prevalent programming languages 
lacks a standardized reflection facility.

In this paper we propose the addition of native support for
compile-time reflection to C++ and a library built
on top of the metadata provided by the compiler.

The basic static metadata provided by compile-time reflection
should be as complete as possible to be applicable in a wide
range of scenarios and allow to implement custom higher-level
static and dynamic reflection libraries and reflection-based
utilities.

The term \emph{reflection} refers to the ability of a computer program
to observe and possibly alter its own structure and/or its behavior.
This includes building new or altering the existing data structures,
doing changes to algorithms or changing the way the program code
is interpreted. Reflective programming is a particular kind
of \emph{metaprogramming}.

The advantage of using reflection is in the fact that everything
is implemented in a single programming language, and the human-written
code can be closely tied with the customizable reflection-based
code which is automatically generated by compiler metaprograms,
based on the metadata provided by reflection.

The solution proposed in this paper is based on the
\href{http://kifri.fri.uniza.sk/~chochlik/mirror-lib/html/}{\em Mirror}
reflection utilities~\cite{mirror-doc-cpp11} and on several years
of user experience with reflection-based metaprogramming.

\section{Metaobject concepts}

We propose that the basic metadata describing a program written
in C++ should be made available through a set of {\em anonymous} types
and related functions and templates
defined by the compiler. These types should describe various program
constructs like, namespaces, types, typedefs, classes, their member variables
(member data), member functions, inheritance, templates, template parameters,
enumerated values, etc.

The compiler should generate metadata for the program constructs defined
in the currently processed translation unit. Indexed sets (ranges) of metaobjects,
like scope members, parameters of a function, etc. should be listed
in the order of appearance in the processed source code.

Since we want the metadata to be available at compile-time,
different base-level constructs should be reflected by
{\em "statically" different} metaobjects and thus by {\em different} types.
For example a metaobject reflecting the global scope namespace should
be a different {\em type} than a metaobject reflecting the \verb@std@
namespace, a metaobject reflecting the \verb@int@ type should
have a different type then a metaobject reflecting the \verb@double@
type, a metaobject reflecting \verb@::foo(int)@ function should
have a different type than a metaobject reflecting \verb@::foo(double)@,
function, etc.

In a manner of speaking these special types (metaobjects) should become
"instances" of the meta-level concepts (static interfaces which
should not exist as concrete types, but rather only at the
"specification-level" similar for example to the iterator concepts).
This section describes a set of metaobject concepts,
their interfaces, tag types for metaobject classification and
functions (or operators) providing access to the metaobjects.

This section conceptualy describes the requirements that various metaobjects
need to satisfy in order to be considered models of the individual
concepts.

\subsection{Categorization and Traits}

In order to provide means for distinguishing between regular types
and metaobjects generated by the compiler,
the \verb@is_metaobject@ trait should be added
and should inherit from \verb@true_type@ for metaobjects (types generated
by the compiler providing metadata) and from \verb@false_type@
for non-metaobjects (native or user defined types):

\begin{minted}{cpp}
template <typename T>
struct is_metaobject
 : false_type
{ };
\end{minted}

\subsubsection{Metaobject category tags}
\label{metaobject-category-tags}

To distiguish between various metaobject kinds (satisfying different concepts
as described below) a set of tag \verb@struct@s (indicating the kind of the metaobject)
should be added:

\begin{minted}{cpp}
struct specifier_tag
{
	typedef specifier_tag type;
};

struct namespace_tag
{
	typedef namespace_tag type;
};

struct global_scope_tag
{
	typedef global_scope_tag type;
};

struct type_tag
{
	typedef type_tag type;
};

struct typedef_tag
{
	typedef typedef_tag type;
}; 

struct class_tag
{
	typedef class_tag type;
};

struct function_tag
{
	typedef function_tag type;
};

struct constructor_tag
{
	typedef constructor_tag type;
};

struct operator_tag
{
	typedef operator_tag type;
};

struct overloaded_function_tag
{
	typedef overloaded_function_tag type;
};

struct enum_tag
{
	typedef enum_tag type;
};

struct enum_class_tag
{
	typedef enum_class_tag type;
};

struct inheritance_tag
{
	typedef inheritance_tag type;
};

struct constant_tag
{
	typedef constant_tag type;
};

struct variable_tag
{
	typedef variable_tag type;
};

struct parameter_tag
{
	typedef parameter_tag type;
};
\end{minted}

These tags are referred-to as \verb@MetaobjectCategory@ below:

\subsubsection{Specifier category tags}
\label{specifier-category-tags}

Similar to the \hyperref[metaobject-category-tags]{metaobject tag} types,
a set of tag types for individual C++ specifier keywords should be defined:

\begin{minted}{cpp}
// indicates no specifier
struct none_tag
{
	typedef none_tag type;
};

struct extern_tag
{
	typedef extern_tag type;
};

struct static_tag
{
	typedef static_tag type;
};

struct mutable_tag
{
	typedef mutable_tag type;
};

struct register_tag
{
	typedef register_tag type;
};

struct thread_local_tag
{
	typedef thread_local_tag type;
};

struct const_tag
{
	typedef const_tag type;
};

struct virtual_tag
{
	typedef virtual_tag type;
};

struct private_tag
{
	typedef private_tag type;
};

struct protected_tag
{
	typedef protected_tag type;
};

struct public_tag
{
	typedef public_tag type;
};

struct class_tag
{
	typedef class_tag type;
};

struct struct_tag
{
	typedef struct_tag type;
};

struct union_tag
{
	typedef union_tag type;
};

struct enum_tag
{
	typedef enum_tag type;
};

struct enum_class_tag
{
	typedef enum_class_tag type;
};

struct constexpr_tag
{
	typedef constexpr_tag type;
};

\end{minted}

These tags are collectively referred-to as \verb@SpecifierCategory@ below.

\subsection{StringConstant}
\label{concept-StringConstant}

A {\em StringConstant} is a class conforming to the following:

\begin{lstlisting}
struct StringConstant
{
	// null terminated char array
        static constexpr const char value[Length+1] = {..., '\0'};

	// implicit conversion to null terminated c-string
        constexpr operator const char* (void) const
        {
                return value;
        }
};

constexpr const char StringConstant::value[Length+1];
\end{lstlisting}

Concrete models of {\em StringConstant} are used to return compile-time string values.
For example the \verb@_str_void@ type defined below, conforms to the {\em StringConstant}
concept:

\begin{lstlisting}
template <char ... C>
struct string_constant
{
	static constexpr const char value[sizeof...(C)+1] = {C...,'\0'};

	constexpr operator const char* (void) const
	{
		return value;
	}
};

template <char ... C>
constexpr const char string_constant::value[sizeof...(C)+1];

//...

typedef string_constant<'v','o','i','d'> _str_void;

cout << _str_void::value << std::endl;
cout << _str_void() << std::endl;
static_assert(sizeof(_str_void::value) == 4+1, "");
\end{lstlisting}

The strings stored in the \verb@value@ array should be UTF-8 encoded.

\subsection{Metaobject}
\label{concept-Metaobject}

A {\em Metaobject} is a stateless anonymous type generated by the compiler which
provides metadata reflecting a specific program feature. Each metaobject
should satisfy the following:

The \verb@is_metaobject@ template should return \verb@true_type@.

\begin{lstlisting}
template <>
struct is_metaobject<Metaobject>
 : true_type
{ };
\end{lstlisting}

A template struct \verb@metaobject_category@ should be defined and should return one of the
\hyperref[metaobject-category-tags]{metaobject category tags}, depending on
the actual kind of the metaobject.

\begin{lstlisting}
template <typename T>
struct metaobject_category;

template <>
struct metaobject_category<Metaobject>
 : MetaobjectCategory
{ };
\end{lstlisting}


\subsection{MetaSpecifier}
\label{concept-MetaSpecifier}

{\em MetaSpecifier} is a \meta{object} reflecting a C++ specifier. In addition to the requirements
inherited from \meta{object}, types conforming to this concept must satisfy the following:

The \verb@metaobject_category@ template should return \verb@specifier_tag@ for all {\em MetaSpecifiers}.

\begin{lstlisting}
template <>
struct metaobject_category<MetaSpecifier>
 : specifier_tag
{ };
\end{lstlisting}

A template struct \verb@specifier_category@ should be defined and should inherit from one of the
\hyperref[specifier-category-tags]{specifier category tags}, depending on
the actual reflected specifier.

\begin{lstlisting}
template <typename T>
struct specifier_category;

template <>
struct specifier_category<MetaSpecifier>
 : SpecifierCategory
{ };
\end{lstlisting}

A template struct \verb@keyword@ should be defined and should return
the keyword matching the reflected specifier as a
\hyperref[concept-StringConstant]{\em StringConstant}.

\begin{lstlisting}
template <typename T>
struct keyword;

template <>
struct keyword_category<MetaSpecifier>
 : StringConstant
{ };
\end{lstlisting}


\subsection{MetaNamed}
\label{concept-MetaNamed}

{\em MetaNamed} is a \meta{object} reflecting program constructs, which have a name
(are identified by an identifier) like namespaces, types, functions, variables, parameters, etc.

In addition to the requirements inherited from \meta{object}, the following requirements must
be satisfied:

A template struct \verb@base_name@ should be defined an should return the base name
of the reflected construct, without the nested name specifier as a
\hyperref[concept-StringConstant]{\em StringConstant}:

\begin{lstlisting}
template <typename T>
struct base_name;

template <>
struct base_name<MetaNamed>
 : StringConstant
{ };
\end{lstlisting}

For namespace \verb@std@ the value should be \verb@"std"@, for namespace
\verb@foo::bar::baz@ it should be \verb@"baz"@, for the global scope the
value should be an empty string.

For \verb@std::vector<int>::iterator@ it should be \verb@"iterator"@. For derived,
qualified types like \verb@volatile std::vector<const foo::bar::fubar*> * const *@
it should be \verb@"volatile vector<const fubar*> * const *"@, etc.




\section{Reflection operator}

The metaobjects reflecting some program feature \verb@X@ as
described above should be made available to the user by
the means of a new operator or expression.
More precisely, the reflection operator should return a type conforming to a particular
metaobject concept, depending on the reflected expression.

Since adding a new keyword has the potential to break existing code,
we do not insist on any particular expression, here follows a list of suggestions
in order of preference (from the most to the least preferrable):

\begin{itemize}
\item{\verb@mirrored(X)@}
\item{\verb@reflected(X)@}
\item{\verb@reflexpr(X)@}
\item{\verb@|X@}
\item{\verb@[[X]]@}
\item{\verb@<<X>>@}
\end{itemize}

The reflected expression \verb@X@ in the items listed above can be any of the following:

\begin{itemize}
\item{\verb@::@} -- The global scope, the returned metaobject is a {\meta{GlobalScope}}.
\item{{\em Namespace name}} -- (\verb@std@) the returned metaobject is a {\meta{Namespace}}.
\item{{\em Type name}} -- (\verb@long double@) the returned metaobject is a {\meta{Type}}.
\item{{\em \verb@typedef@ name}} -- (\verb@std::size_t@ or \verb@std::string@)
     the returned metaobject is a {\meta{Typedef}}.
\item{{\em Template name}} -- (\verb@std::tuple@ or \verb@std::map@)
     the returned metaobject is a {\meta{Template}}.
\item{{\em Class name}} -- (\verb@std::thread@ or \verb@std::map<int, double>@)
     the returned metaobject is a {\meta{Class}}.
\item{{\em Function name}} -- (\verb@std::sin@ or \verb@std::string::size@) the returned metaobject
     is a {\meta{OverloadedFunction}}.
\item{{\em Function signature}} -- (\verb@std::sin(double)@ or \verb@std::string::front(void) const@)
     the returned metaobject is a {\meta{Function}}. The signature may be specified without the
     return value type.
\item{{\em Constructor signature}} -- (\verb@std::pair<char, double>::pair(char, double)@
     or \verb@std::string::string(void)@) the returned metaobject is a {\meta{Constructor}}.
\item{{\em Variable name}} -- (\verb@std::errno@) the returned metaobject is a {\meta{Variable}}.
%\item{TODO}
\end{itemize}

The reflection operator or expression should have access to \verb@private@ and
\verb@protected@ members of classes. The following should be valid:

\begin{minted}{cpp}
struct A
{
	int a;
};

class B
{
protected:
	int b;
};

class C
 : protected A
 , public B
{
private:
	int c;
};

typedef mirrored(A::a) meta_A_a;
typedef mirrored(B::b) meta_B_b;
typedef mirrored(C::a) meta_C_a;
typedef mirrored(C::b) meta_C_b;
typedef mirrored(C::c) meta_C_c;

\end{minted}

\subsection{Context-dependent reflection}

We also propose to define a set of special expressions that can be used
inside of the reflection operator, to obtain metadata based on the context
where it is invoked, instead of the identifier.

\subsubsection{Namespaces}

If the \verb@this::namespace@ expression is used as the argument of the reflection
operator, then it should return a \meta{Namespace} reflecting the namespace
inside of which the reflection operator was invoked.

For example:

\begin{minted}{cpp}

typedef mirrored(this::namespace) _meta_gs;

\end{minted}

reflects the global scope namespace and is equivalent to

\begin{minted}{cpp}

typedef mirrored(::) _meta_gs;

\end{minted}

For named namespaces:

\begin{minted}{cpp}

namespace foo {

typedef mirrored(this::namespace) _meta_foo;

namespace bar {

typedef mirrored(this::namespace) _meta_foo_bar;

} // namespace bar

} // namespace foo
\end{minted}

\subsubsection{Classes}

If the \verb@this::class@ expression is used as the argument of the reflection
operator, then it should return a \meta{Class} reflecting the class
inside of which the reflection operator was invoked.

For example:

\begin{minted}{cpp}

struct foo
{
	const char* _name;

	// reflects foo
	typedef mirrored(this::class) _meta_foo1;

	foo(void)
	 : _name(base_name<mirrored(this::class)>())
	{ }

	void f(void)
	{
		// reflects foo
		typedef mirrored(this::class) _meta_foo2;
	}

	double g(double, double);

	struct bar
	{
		// reflects foo::bar
		typedef mirrored(this::class) _meta_foo_bar;
	};
};

double foo::g(double a, double b)
{
	// reflects foo
	typedef mirrored(this::class) _meta_foo3;
	return a+b;
}

class baz
{
private:
	typedef mirrored(this::class) _meta_baz;
};

typedef mirrored(this::class); // <- error: not used inside of a class.

\end{minted}

\subsubsection{Functions}

If the \verb@this::function@ expression is used as the argument of the reflection
operator, then it should return a \meta{Function} reflecting the function or operator
inside of which the reflection operator was invoked.

For example:

\begin{minted}{cpp}

void foobar(void)
{
	// reflects this particular overload of the foobar function
	typedef mirrored(this::function) _meta_foobar;
}

int foobar(int i)
{
	// reflects this particular overload of the foobar function
	typedef mirrored(this::function) _meta_foobar;
	return i+1;
}

class foo
{
private:
	void f(void)
	{
		// reflects this particular overload of foo::f
		typedef mirrored(this::function) _meta_foo_f;
	}

	double f(void)
	{
		// reflects this particular overload of foo::f
		typedef mirrored(this::function) _meta_foo_f;
		return 12345.6789;
	}
public:
	foo(void)
	{
		// reflects this constructor of foo
		typedef mirrored(this::function) _meta_foo_foo;
	}

	friend bool operator == (foo, foo)
	{
		// reflects this operator
		typedef mirrored(this::function) _meta_foo_eq;
	}

	typedef mirrored(this::function) _meta_fn; // <- error
};

typedef mirrored(this::function) _meta_fn; // <- error

\end{minted}


\section{Additions to the library}
\label{section-Library}

In order to simplify composition of the metaobjects and metafunctions defined
\hyperref[section-Concepts]{above}, several further additions to the standard
library should be made.

\subsection{Metaobject expressions}

A {\em metaobject expression} is a class which can be {\em evaluated}
into a \meta{object}. By default any class, that has a member typedef
called \verb@type@, which is a model of \meta{object}, is a metaobject expression.

\begin{minted}{cpp}
struct SomeMetaobjectExpression
{
	typedef Metaobject type;
};

\end{minted}

And thus, any \meta{object} is also a {\em metaobject expression}.

Generally, however, any type for which the \verb@evaluate@ metafunction
(described below), "returns" a \meta{object} is a {\em metaobject expression}.

\subsubsection{\texttt{evaluate}}

A class template called \verb@evaluate@ should be defined and should "return" a \meta{object}
resulting from a {\em metaobject expression}:

\begin{minted}{cpp}
template <class MetaobjectExpression>
struct evaluate
 : Metaobject
{ };
\end{minted}

that could be implemented for example as follows:

\begin{minted}{cpp}
template <class X, bool IsMetaobject>
struct do_evaluate;

template <class X>
struct do_evaluate<X, true>
 : X
{ };

template <class X>
struct do_evaluate<X, false>
 : do_evaluate<
	typename X::type,
	is_metaobject<typename X::type>::value
> { };

template <class X>
struct evaluate
 : do_evaluate<X, is_metaobject<X>::value>
{ };

\end{minted}

The users should be allowed to add specializations of \verb@evaluate@
for other types if necessary.

\subsection{Default implementation of metafunctions}

The default implementation of the metafunction template classes defined above,
should follow this pattern:

\begin{minted}{cpp}
template <typename T>
struct Template;

template <typename T>
struct Template
 : Template<typename evaluate<T>::type>
{ };
\end{minted}

Where \verb@Template@ is each of the following:

\begin{minipage}[t]{0.5\textwidth}
\begin{itemize}
\item \verb@metaobject_category@
\item \verb@specifier_category@
\item \verb@keyword@
\item \verb@base_name@
\item \verb@full_name@
\item \verb@named_typedef@
\item \verb@named_mem_var@
\item \verb@scope@
\item \verb@members@
\item \verb@overloads@
\item \verb@type@
\item \verb@base_classes@
\item \verb@base_class@
\item \verb@base_type@
\item \verb@result_type@
\item \verb@parameters@
\item \verb@template_parameters@
\end{itemize}
\end{minipage}
\begin{minipage}[t]{0.5\textwidth}
\begin{itemize}
\item \verb@template_arguments@
\item \verb@template_@
\item \verb@exceptions@
\item \verb@instantiation@
\item \verb@position@
\item \verb@value@
\item \verb@elaborated_type_specifier@
\item \verb@access_specifier@
\item \verb@constexpr_specifier@
\item \verb@noexcept_specifier@
\item \verb@const_specifier@
\item \verb@inheritance_specifier@
\item \verb@linkage_specifier@
\item \verb@storage_specifier@
\item \verb@is_pure@
\item \verb@is_pack@
\item \verb@original_type@
\end{itemize}
\end{minipage}

For example:

\begin{minted}{cpp}
template <typename T>
struct metaobject_category
 : metaobject_category<typename evaluate<T>::type>
{ };
\end{minted}

This will allow to compose metaobject expressions into algorithms. For example:

\begin{minted}[tabsize=4]{cpp}
// print the number of members of the scope where mycls is defined
cout << size<members<scope<mirrored(mycls)>>>() << endl;

// print the name of the first base class of mycls
cout <<
	base_name<base_class<at<base_classes<mirrored(mycls)>, 0>>>()
<< endl;

// print the access specifier keyword of the second base of mycls
cout <<
	keyword<access_specifier<at<base_classes<mirrored(mycls)>, 1>>>()
<< endl;

// print the fully qualified name of the scope of
// the source type of the third member of mycls
cout <<
	full_name<scope<type<at<members<mirrored(mycls)>, 2>>>>()
<< endl;
\end{minted}


\section{Rationale}

This section explains some of the design decisions behind this proposal and
answers several frequently asked questions.

\subsection{Why metaobjects, why not reflect directly?}

{\textbf Q:}{\em Why should we define a set of metaobject concepts, let the compiler generate
models of these concepts and use those to obtain the metadata? Why not just extend the existing
type traits?}

{\textbf A:} The most important reason is the completeness and the scope of reflection.
Type traits (as they are defined now) work just with types. A reflection facility should
however provide much more metadata.
It should be able to reflect namespaces, functions, constructors, inheritance, variables, etc.

For example:

\begin{minted}{cpp}
pair<long, string> my_var;

// OK, we can print the name of the type of a variable:
cout << type_name<decltype(my_var)>() << endl;
// But we really, really want to print the name of the variable
// (without the use of the preprocessor)
cout << type_name<my_var>() << endl; // Error
// similar with namespaces:
cout << type_name<std::chrono>() << endl; // Error
// etc.
\end{minted}


Doing reflection with type traits limits the scope, because of the rules defining what
can be a template parameter. This rules could be updated to allow for example an expression
representing a particular class constructor to  be passed as a template argument.
Also currently there is no expression for specifying (not invoking) a constructor
or a particular function overload, so additional rules would have to be added.

This would (in our opinion) be a much more drastic change to the standard, than
the adoption of this proposal. If expressions denoting a namespace or a particular
constructor or a function overload were added just for the purpose of reflecting
them (with the \verb@mirroed@ keyword), then all the changes could be localized
in the reflection subsystem and remain invalid in the core language:

\begin{minted}{cpp}
mirrored(std::current_thread); // OK - MetaNamespace
std::current_thread; // error - not a primary expression

mirrored(std::sin); // OK - MetaOverloadedFunction
std::sin; // error - cannot resolve overload

mirrored(std::sin(double)); // OK - MetaFunction
std::sin(double); // error - invalid expression
//etc.
\end{minted}


Second reason is access to private and protected members. There are many use-cases where
access to non-public class members through reflection is desired. If reflection was
done through type traits directly on the class members, it would be either impossible
to reflect non-public members or the access rules would have to be changed to somehow
allow access in reflection expressions:

\begin{minted}{cpp}
class C
{
private:
	typedef int T;
public:
};

assert(some_trait<C::T>::value); //OK, we are reflecting so we have access
\end{minted}

but not outside:

\begin{minted}{cpp}
C::T x = 0; // Error, C::T is private
\end{minted}

With the reflection operator like \verb@mirrored(X)@, the access rules would have
to be updated only to allow the reflection operator to have access to everthing.
At the first glance, the following two expressions;

\begin{minted}{cpp}
some_trait<C::T>::value
\end{minted}

and

\begin{minted}{cpp}
mirrored(C::T)
\end{minted}

look similar and so the changes to the access rules could seem similar too, but
that is not the case. The (single) \verb@mirrored@ operator would have special status,
on the other hand type traits are regular templates (with some magic inside) and
all (several dozens of them) would need to be distinguished from all the other templates
in the \verb@std@ namespace, which should not have private access.

Having said that, we do not object to extending the type traits where it does make sense.

One other reason for having a new reflection operator is, that there already is an
existing (very limited) reflection operator, namely \verb@typeid@ which "returns"
a compiler-generated "metaobject" -- \verb@std::type_info@. We are aware that there
are differences between \verb@typeid@ and \verb@mirrored@, but the basic idea is similar.

\subsection{Why are the metaobjects anonymous?}

{\textbf Q:}{\em Why should the metaobjects be anonymous types as opposed to
types with well defined and standardized names or concrete template classes, (possibly with some
special kind of parameter accepting different arguments than types and constants)?}

{\textbf A:} We wanted to avoid defining a specific naming convention, because it would
be difficult to do so and very probably not user friendly (see C++ name mangling). There
already is a precedent for anonymous types -- for example C++ {\em lambdas}.

Another option would be to define a concrete set of template classes like:

\begin{minted}{cpp}
namespace std {

template <typename T>
class meta_type /* Model of MetaType */
{ };

}
\end{minted}

which could work with types, classes, etc., but would not work with namespaces, constructors,
etc. (see also the Q/A above):

\begin{minted}{cpp}
namespace std {

template <something X> //<- Problem
class meta_constructor /* Model of MetaConstructor */
{ };

template <something X> //<- Problem
class meta_namespace /* Model of MetaNamespace */
{ };

}

typedef std::meta_namespace<std> meta_std; //<- Problem
\end{minted}

Instead of this, the metaobjects are anonymous and their (internal) identification
is left to the compiler. From the users POV, the metaobject can be distinguished
by the means of the metaobject traits and tags as \hyperref[section-Concepts]{described above}.

\section{Unresolved Issues}

\begin{itemize}
	\item {\em Normalization of names returned by \verb@Named::base_name()@ and \verb@Named::full_name()@:}
	The strings returned by the \verb@base_name@ and \verb@full_name@ functions should be
	implementation-independent and the same on every platform/compiler.
	

	\item {\em Returning names as compile-time strings:} It could be advantageous if even
	the names of various metaobjects were compile-time constants and could be introspected
	or used as template parameters.
\end{itemize}

\section{Acknowledgements}

Thanks to Chandler Carruth for presenting the N4111 proposal at the Urbana meeting.
Also thanks to Axel Naumann for helping with this proposal.


\renewcommand\refname{\arabic{section}\hspace{1em}References}

\stepcounter{section}
\addcontentsline{toc}{section}{\refname}

\begin{thebibliography}{100}

\bibitem{mirror-doc-cpp11}
Mirror C++ reflection library documentation (C++11 version),
\url{http://kifri.fri.uniza.sk/~chochlik/mirror-lib/html/}.

\end{thebibliography}{100}



\begin{appendices}
\appendix
\section{Examples}

This section contains multiple examples of usage of the additions proposed above.
The examples assume that the \verb@mirrored@ operator (described above) is used
to obtain the metaobjects and the types, templates, etc. are defined in the
\verb@std::meta@ namespace.

For the sake of brevity

\begin{minted}{cpp}
using namespace std;
\end{minted}

is assumed.

\subsection{Basic traits}

Usage of the \verb@is_metaobject@ trait on non-metaobjects:

\begin{minted}[tabsize=4]{cpp}
static_assert(not(is_metaobject<int>()), "");
static_assert(not(is_metaobject<std::string>()), "");
static_assert(not(is_metaobject<my_class>()), "");
static_assert(not(meta::is_class_member<meta_gs>()), "");
\end{minted}


\subsection{Global scope reflection}

\begin{minted}[tabsize=4]{cpp}
// reflected global scope
typedef mirrored(::) meta_gs;

static_assert(is_metaobject<meta_gs>(), "");

// Is a MetaNamed
static_assert(meta::has_name<meta_gs>(), "");
// Is a MetaScoped
static_assert(meta::has_scope<meta_gs>(), "");
// Is a MetaScope
static_assert(meta::is_scope<meta_gs>(), "");
// Is not a MetaClassMember
static_assert(not(meta::is_class_member<meta_gs>()), "");

// Is a MetaGlobalScope
static_assert(
	is_base_of<
		meta::global_scope_tag,
		meta::category<meta_gs>
	>(), ""
);

// Global scope is its own scope
static_assert(
	is_base_of<
		meta_gs,
		meta::scope<meta_gs>
	>(), ""
);

// Empty base and full name
assert(strlen(meta::base_name<meta_gs>()) == 0);
assert(strcmp(meta::base_name<meta_gs>(), "") == 0);

assert(strlen(meta::full_name<meta_gs>()) == 0);
assert(strcmp(meta::full_name<meta_gs>(), "") == 0);

// the sequence of members
typedef meta::members<meta_gs>::type meta_gs_members;

static_assert(
	meta::size<meta_gs_members>() == 20, // YMMV
	""
);

\end{minted}


\subsection{Namespace reflection}

\begin{minted}[tabsize=4]{cpp}
// reflected namespace std
typedef mirrored(std) meta_std;

static_assert(is_metaobject<meta_std>(), "");

// Is a MetaNamed
static_assert(meta::has_name<meta_std>(), "");
// Is a MetaScoped
static_assert(meta::has_scope<meta_std>(), "");
// Is a MetaScope
static_assert(meta::is_scope<meta_std>(), "");
// Is not a MetaClassMember
static_assert(not(meta::is_class_member<meta_std>()), "");

// Is a MetaNamespace
static_assert(
	is_base_of<
		meta::namespace_tag,
		metaobject_category<meta_std>
	>(), ""
);

// The scope of namespace std is the global scope
static_assert(
	is_base_of<
		meta_gs,
		meta::scope<meta_std>
	>(), ""
);

// The base and full name
assert(strlen(meta::base_name<meta_std>()) == 3);
assert(strcmp(meta::base_name<meta_std>(), "std") == 0);
assert(strlen(meta::full_name<meta_std>()) == 3);
assert(strcmp(meta::full_name<meta_std>(), "std") == 0);
\end{minted}

\subsection{Type reflection}

\begin{minted}[tabsize=4]{cpp}
// reflected type unsigned int
typedef mirrored(unsigned int) meta_uint;

static_assert(is_metaobject<meta_uint>(), "");

// Is a MetaNamed
static_assert(meta::has_name<meta_uint>(), "");
// Is a MetaScoped
static_assert(meta::has_scope<meta_uint>(), "");
// Is not a MetaScope
static_assert(not(meta::is_scope<meta_uint>()), "");
// Is not a MetaClassMember
static_assert(not(meta::is_class_member<meta_uint>()), "");

// Is a MetaType
static_assert(
	is_base_of<
		meta::type_tag,
		meta::category<meta_uint>
	>(), ""
);

// The scope of unsigned int is the global scope
static_assert(
	is_base_of<
		meta_gs,
		meta::scope<meta_uint>
	>(), ""
);

// The original type
static_assert(
	is_same<
		unsigned int,
		meta::original_type<meta_uint>::type
	>(), ""
);

assert(strlen(meta::base_name<meta_uint>()) == 12);
assert(strcmp(meta::base_name<meta_uint>(), "unsigned int") == 0);
assert(strlen(meta::full_name<meta_uint>()) == 12);
assert(strcmp(meta::full_name<meta_uint>(), "unsigned int") == 0);
\end{minted}

\subsection{Typedef reflection}

\begin{minted}[tabsize=4]{cpp}
// reflected typedef std::size_t
typedef mirrored(std::size_t) meta_size_t;

static_assert(is_metaobject<meta_size_t>(), "");

static_assert(meta::has_name<meta_size_t>(), "");
static_assert(meta::has_scope<meta_size_t>(), "");
static_assert(not(meta::is_scope<meta_size_t>()), "");
static_assert(not(meta::is_class_member<meta_size_t>()), "");

// Is a MetaTypedef
static_assert(
	meta::is_alias<meta_size_t>(), ""
);

// The scope of std::size_t is the namespace std
static_assert(
	is_base_of<
		meta_std,
		meta::scope<meta_size_t>
	>(), ""
);

// The original type
static_assert(
	is_same<
		std::size_t,
		meta::original_type<meta_size_t>::type
	>(), ""
);

// the "source" type of the typedef
typedef meta::type<meta_size_t>::type meta_size_t_type;
static_assert(
	is_base_of<
		meta::type_tag,
		meta::category<meta_size_t_type>
	>(), ""
);

// The original type
static_assert(
	is_same<
		std::size_t,
		meta::original_type<meta_size_t_type>::type
	>(), ""
);

assert(strlen(meta::base_name<meta_size_t>()) == 6);
assert(strcmp(meta::base_name<meta_size_t>(), "size_t") == 0);
assert(strlen(meta::full_name<meta_size_t>()) == 11);
assert(strcmp(meta::full_name<meta_size_t>(), "std::size_t") == 0);
// YMMV
assert(strlen(meta::base_name<meta_size_t_type>()) == 12);
assert(strcmp(meta::base_name<meta_size_t_type>(), "unsigned int") == 0);
\end{minted}


\subsection{Class reflection}

\begin{minted}[tabsize=4]{cpp}
struct A
{
	int a;
};

class B
{
private:
	bool b;
public:
	typedef int T;
};

class C
 : public A
 , virtual protected B
{
public:
	static constexpr char c = 'C';

	struct D : A
	{
		static double d;
	} d;
};

union U
{
	long u;
	float v;
};

typedef mirrored(A) meta_A;
typedef mirrored(B) meta_B;
typedef mirrored(C) meta_C;
typedef mirrored(C::D) meta_D;
typedef mirrored(B::T) meta_T;
typedef mirrored(U) meta_U;

// classes are scopes
static_assert(meta::is_scope<meta_A>(), "");
static_assert(meta::is_scope<meta_B>(), "");
static_assert(meta::is_scope<meta_C>(), "");
static_assert(meta::is_scope<meta_D>(), "");
static_assert(meta::is_scope<meta_U>(), "");

// A, B, C, C::D and U are all elaborated types
assert(is_base_of<meta::class_tag, metaobject_category<meta_A>>());
assert(is_base_of<meta::class_tag, metaobject_category<meta_B>>());
assert(is_base_of<meta::class_tag, metaobject_category<meta_C>>());
assert(is_base_of<meta::class_tag, metaobject_category<meta_D>>());
assert(is_base_of<meta::class_tag, metaobject_category<meta_U>>());

static_assert(!meta::is_class_member<meta_A>(), "");
static_assert(!meta::is_class_member<meta_B>(), "");
static_assert(!meta::is_class_member<meta_C>(), "");
static_assert( meta::is_class_member<meta_D>(), "");
static_assert( meta::is_class_member<meta_T>(), "");
static_assert(!meta::is_class_member<meta_U>(), "");

// typenames
assert(strcmp(meta::base_name<meta_A>(), "A") == 0);
assert(strcmp(meta::base_name<meta_B>(), "B") == 0);
assert(strcmp(meta::full_name<meta_D>(), "C::D") == 0);

// reflected elaborated type specifiers for A, B and U
typedef meta::elaborated_type_specifier<meta_A>::type meta_A_ets;
typedef meta::elaborated_type_specifier<meta_B>::type meta_B_ets;
typedef meta::elaborated_type_specifier<meta_U>::type meta_U_ets;

// specifier keywords
assert(strcmp(meta::keyword<meta_A_ets>(), "struct") == 0);
assert(strcmp(meta::keyword<meta_B_ets>(), "class") == 0);
assert(strcmp(meta::keyword<meta_U_ets>(), "union") == 0);

// specifier tags
assert(is_base_of<meta::struct_tag, meta::specifier_category<meta_A_ets>>());
assert(is_base_of< meta::class_tag, meta::specifier_category<meta_B_ets>>());
assert(is_base_of<meta::union_tag, meta::specifier_category<meta_U_ets>>());

// reflected sequences of members of the A,B and C classes
typedef meta::members<meta_A>::type meta_A_members;
typedef meta::members<meta_B>::type meta_B_members;
typedef meta::members<meta_C>::type meta_C_members;

static_assert(meta::size<meta_A_members>() == 1, ""); // A::a
static_assert(meta::size<meta_B_members>() == 2, ""); // B::b,B::T
static_assert(meta::size<meta_C_members>() == 3, ""); // C::c,C::D,C::d

// reflected members of B and C
typedef meta::at<meta_B_members, 0>::type meta_B_b;
typedef meta::at<meta_B_members, 1>::type meta_B_T;
typedef meta::at<meta_C_members, 0>::type meta_C_c;
typedef meta::at<meta_C_members, 1>::type meta_C_D;
typedef meta::at<meta_C_members, 2>::type meta_C_d;

assert(is_base_of<meta::variable_tag, metaobject_category<meta_B_b>>());
assert(is_base_of<meta::typedef_tag, metaobject_category<meta_B_T>>());
assert(is_base_of<meta::class_tag, metaobject_category<meta_C_D>>());

// MetaClassMembers
static_assert( meta::is_class_member<meta_B_b>(), "");
static_assert( meta::is_class_member<meta_B_T>(), "");
static_assert( meta::is_class_member<meta_C_D>(), "");
static_assert( meta::is_class_member<meta_C_d>(), "");

// access specifiers
typedef meta::access_specifier<meta_B_B>::type meta_B_b_access;
typedef meta::access_specifier<meta_C_D>::type meta_C_D_access;

// specifier keywords
assert(strcmp(meta::keyword<meta_B_b_access>(), "private") == 0);
assert(strcmp(meta::keyword<meta_C_D_access>(), "public") == 0);

// sequence of base classes of C
typedef meta::base_classes<meta_C>::type meta_C_bases;

static_assert(meta::size<meta_C_bases>() == 2, ""); // A, B

// MetaInheritances of C->A and C->B
typedef meta::at<meta_C_bases, 0>::type meta_C_base_A;
typedef meta::at<meta_C_bases, 1>::type meta_C_base_B;

// inheritance specifiers
typedef meta::inheritance_specifier<meta_C_base_A>::type meta_C_base_A_it;
typedef meta::inheritance_specifier<meta_C_base_B>::type meta_C_base_B_it;

// access specifiers
typedef meta::access_specifier<meta_C_base_A>::type meta_C_base_A_acc;
typedef meta::access_specifier<meta_C_base_B>::type meta_C_base_B_acc;

// specifier keywords
assert(strcmp(meta::keyword<meta_C_base_A_it>(), "") == 0);
assert(strcmp(meta::keyword<meta_C_base_B_it>(), "virtual") == 0);
assert(strcmp(meta::keyword<meta_C_base_A_acc>(), "public") == 0);
assert(strcmp(meta::keyword<meta_C_base_B_acc>(), "protected") == 0);

// specifier tags
static_assert(
	is_base_of<
		meta::none_tag,
		meta::specifier_category<meta_C_base_A_it>
	>(), ""
);
static_assert(
	is_base_of<
		meta::virtual_tag,
		meta::specifier_category<meta_C_base_B_it>
	>(), ""
);
static_assert(
	is_base_of<
		meta::public_tag,
		meta::specifier_category<meta_C_base_A_acc>
	>(), ""
);

// base classes
static_assert(
	is_base_of<
		meta_A,
		meta::base_class<meta_C_base_A>
	>(), ""
);
static_assert(
	is_base_of<
		meta_B,
		meta::base_class<meta_C_base_B>
	>(), ""
);

\end{minted}


\subsection{Enumeration reflection}

\begin{minted}[tabsize=4]{cpp}
enum E
{
	val_a = 1,
	val_b = 2,
	val_c = 3,
	val_d = 4
};

// reflected enumeration
typedef mirrored(E) meta_E;
// reflected enum values
typedef mirrored(val_a) meta_val_a;
typedef mirrored(val_d) meta_val_d;

// enums are not scopes
static_assert(not(meta::is_scope<meta_E>()), "");
// other traits
static_assert(meta::has_scope<meta_E>(), "");
static_assert(meta::has_name<meta_E>(), "");
static_assert(meta::has_scope<meta_val_a>(), "");
static_assert(meta::has_name<meta_val_a>(), "");

// the categories
assert(is_base_of<meta::enum_tag, metaobject_category<meta_E>>());
assert(is_base_of<meta::constant_tag, metaobject_category<meta_val_a>>());

// names
assert(strcmp(meta::base_name<meta_E>(), "E") == 0);
assert(strcmp(meta::base_name<meta_val_a>(), "val_a") == 0);
assert(strcmp(meta::full_name<meta_val_d>(), "val_d") == 0);

%// reflected elaborated type specifiers for E
%typedef meta::elaborated_type_specifier<meta_E>::type meta_E_ets;

%// specifier keyword
%assert(strcmp(meta::keyword<meta_E_ets>(), "enum") == 0);

%// the members
%typedef meta::members<meta_E>::type meta_E_members;

%assert(meta::size<meta_E_members>() == 4);
%assert(is_base_of<meta_val_a, meta::at<meta_E_members, 0>>());
%assert(is_same<meta_val_d, meta::at<meta_E_members, 3>::type>());

// the scope of the enum values is the same
// as the scope of the enum type
assert(is_same<
	meta::scope<meta_E>::type,
	meta::scope<meta_val_a>::type
>());
assert(is_same<
	meta::scope<meta_E>::type,
	meta::scope<meta::at<meta_E_members, 1>>::type
>());

\end{minted}

\subsection{Strongly-typed enumeration reflection}

\begin{minted}[tabsize=4]{cpp}
enum class E : unsigned short
{
	val_a = 1,
	val_b = 2,
	val_c = 3,
	val_d = 4
};

// reflected enumeration
typedef mirrored(E) meta_E;
// reflected enum values
typedef mirrored(E::val_a) meta_E_val_a;
typedef mirrored(E::val_d) meta_E_val_d;

// enum classes are scopes
static_assert(meta::is_scope<meta_E>(), "");
// other traits
static_assert(meta::has_scope<meta_E>(), "");
static_assert(meta::has_name<meta_E>(), "");
static_assert(meta::has_scope<meta_E_val_a>(), "");
static_assert(meta::has_name<meta_E_val_a>(), "");

// the categories
assert(is_base_of<
	meta::enum_class_tag,
	metaobject_category<meta_E>
>());
assert(is_base_of<
	meta::constant_tag,
	metaobject_category<meta_E_val_a>
>());

// names
assert(strcmp(meta::base_name<meta_E>(), "E") == 0);
assert(strcmp(meta::base_name<meta_E_val_a>(), "val_a") == 0);
assert(strcmp(meta::full_name<meta_E_val_d>(), "E::val_d") == 0);

%// reflected elaborated type specifiers for E
%typedef meta::elaborated_type_specifier<meta_E>::type meta_E_ets;

%// specifier keyword
%assert(strcmp(meta::keyword<meta_E_ets>(), "enum class") == 0);

%// the members
%typedef meta::members<meta_E>::type meta_E_members;

%assert(meta::size<meta_E_members>() == 4);
%assert(is_base_of<meta_E_val_a, meta::at<meta_E_members, 0>>());
%assert(is_same<meta_E_val_d, meta::at<meta_E_members, 3>::type>());

// the scope
assert(is_same<meta_E, meta::scope<meta_E_val_a>::type>());
assert(not(is_same<
	meta::scope<meta_E>::type,
	meta::scope<meta::at<meta_E_members, 1>>::type
>()));
assert(is_same<
	meta::scope<meta_E>::type,
	meta::scope<meta::scope<meta::at<meta_E_members, 1>>>::type
>());

\end{minted}


%TODO



\end{appendices}

\end{document}
