\subsection{Source text-based metaprogramming}

In scripting languages metaprogramming often takes the form of dynamically creating
a new script purely through text operations followed by the execution of that script.

While using this approach is more complicated with compiled languages it is not
unheard of. A C++ source can be created by a program in C++, compiled by a compiler
(invoked from that same or from a different program) into a shared library and then
dynamically loaded and executed.

In some cases this approach could be used to generate source code on a local machine,
which is then compiled and executed on a remote machine with a different architecture.

\begin{minted}[tabsize=4]{cpp}

class foo64bit
{
/* ... */
};

class foo32bit
{
/* ... */
};

#if __THIS_IS_64BIT_ARCH
typedef foo64bit default_foo;
#else
typedef foo32bit default_foo;
#endif

struct plugin
{
	virtual void process_foo(default_foo&) = 0;
	/* ... */
};

\end{minted}

We want to programatically create a new logging\footnote{
We're again trying to stick to the basics, here. Much more complicated
examples could be devised.
} implementation of \verb@plugin@
and we don't want to rewrite this program every time
the interface is updated.

Furthermore it is possible that we will be generating the code
on a 32-bit machine and then compiling and executing it on a 64-bit machine
or vice-versa.

\begin{minted}[tabsize=4]{cpp}

template <typename MetaFunction>
void print_func_impl(MetaFunction)
{
	using std::cout;
	using std::endl;

	// the result can have a typedef-ined type
	// and we want to print here the typedef name
	cout << base_name<result_type<MetaFunction>>() << " ";
	cout << base_name<MetaFunction>() << "(";

	for_each<parameters<MetaFunction>>(
		[](auto meta_param)
		{
			typedef decltype(meta_param) MetaParam;
			if(position<MetaParam>() > 0)
			{
				cout << ", ";
			}
			// the parameter can have a typedef-ined type
			// and we want to print here the typedef name
			cout << base_name<type<MetaParam>>() << " ";
			cout << base_name<MetaParam>();
		}
	);

	cout << base_name<MetaFunction>() << ")";
	cout << " override" << endl;
	cout << "{" << endl;
	cout << "  _do_log_call<MetaFunction>(" << endl;

	for_each<parameters<MetaFunction>>(
		/* Print out a parameter list for the call */
	);

	cout << "  );" << endl;

	/* Print out the rest of the implementation */

	cout << "}" << endl;
}

void main(void)
{
	using std::cout;
	using std::endl;

	cout << "#include <foo/plugin.hpp>" << endl;
	/* etc. */

	cout << "class logging_plugin" << endl;
	cout << " : virtual public plugin" << endl;
	cout << "{" << endl;
	cout << "private:" << endl;
	cout << "  template <typename MetaFunction, typename ... P>" << endl;
	cout << "  void _do_log_call(const P&...);" << endl;
	cout << "public:" << endl;

	for_each<members<mirrored(plugin)>>(print_func_impl);

	cout << "};" << endl;
	return 0;
}

\end{minted}

The example above could be semi-automated using the preprocessor
and (demangled) \verb@type_info::name@. The problem is that
\verb@type_info@ is not aware of the fact that the \verb@default_foo@
parameter type is a typedef and it would instead return either
\verb@"foo64bit"@ or \verb@"foo32bit"@ based on the architecture
on which the script was generated.

In N4111 (and subsequent papers) it is proposed that reflection
is aware of \verb@typedef@s and distinguishes between
\verb@typedef@s and their {\em underlying types}.

This example shows the following features:

\begin{itemize}
\item{typedef reflection,}
\item{class member reflection,}
\item{use of the reflection operator,}
\item{use of the \verb@base_name@, \verb@type@ and \verb@original_type@ templates. }
\end{itemize}

