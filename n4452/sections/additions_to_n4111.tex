\section{Additions to N4111}

The examples described in this paper use several features not described in N4111
(these will be added to the next revision of that paper -- N4451):

\subsection{Source file and line}

Template class \verb@source_file@ should be defined for \meta{object}s
and should "return" a compile-time string containing the path to the source file
where the base-level construct reflected by the metaobject was declared.

\begin{minted}[tabsize=4]{cpp}
template <>
struct source_file<MetaObject>
 : String
{ };
\end{minted}

Template class \verb@source_line@ should be defined for \meta{object}s
and should inherit from \verb@integral_constant<unsigned, Line>@ where
\verb@Line@ is the line number in the source file
where the base-level construct reflected by the metaobject was declared.

\begin{minted}[tabsize=4]{cpp}
template <>
struct source_file<MetaObject>
 : String
{ };
\end{minted}

\subsection{For each element in Metaobject sequence}

Template function \verb@for_each@, should be defined for every metaobject sequence
and should call the specified unary functor taking values of types conforming to
the same metaobject concept as the elements of the metaobject sequence as arguments.

\begin{minted}[tabsize=4]{cpp}
template <typename MetaobjectSequence, typename UnaryFunc>
void for_each(UnaryFunc func)
{
	/* call func on each element in the sequence */
}
\end{minted}

The interface of \meta{objectSequence} as defined in N4111 should be enough
to define a single generic implementation of this function without the need
to write specialization for every type modelling this concept.

\subsection{The \meta{Positional} concept}

The \meta{Positional} concept defines the interface for metaobjects reflecting
base-level constructs having a fixed position, like function or template parameters,
class inheritance clauses, etc.

The \verb@has_position@ template class can be used to distinguish metaobjects modelling
this concept. Is should inherit from \verb@true_type@ for \meta{Positional}s and from
\verb@false_type@ otherwise.

\begin{minted}{cpp}
template <typename X>
struct has_position
 : false_type
{ };

template <>
struct has_position<MetaPositional>
 : true_type
{ };
\end{minted}

The \verb@position@ template class inheriting
from \verb@integral_constant<size_t, I>@ type (where \verb@I@ is
a zero-based position of the reflected base-level language construct)
can be used to obtain the value of the index.

\begin{minted}{cpp}
template <typename T>
struct position;

template <>
struct position<MetaPositional>
 : integral_constant<size_t, I>
{ };
\end{minted}

\subsection{Position of a base in the list of base classes}

Every model of \meta{Inheritance} should also conform to the \meta{Positional}
concept described above.

\subsection{Pointers to reflected variables and functions}

For models of \meta{Variable} and \meta{Function} the \verb@pointer@ template class
should be defined as:

\begin{minted}[tabsize=4]{cpp}
template <>
struct pointer<MetaVariable>
{
	typedef _unspecified_ type;

	static type get(void);
};

template <>
struct pointer<MetaFunction>
{
	typedef _unspecified_ type;

	static type get(void);
};
\end{minted}

For \meta{Variable}s or \meta{Function}s reflecting namespace-level variables or functions
the \verb@get@ function should return a pointer to that variable or function respectively.

If the \meta{Variable} or \meta{Function} reflect class members then the \verb@get@ function
should return a pointer to data member or pointer to member function respectively.

\subsection{Context-dependent reflection}
\label{appendix-context-dependent-reflection}

Special expressions in the form of \verb@this::{namespace,class,function}@ should be added
as valid arguments for the reflection operator and should return metaobjects depending
on the context where such invocation of the reflection operator was used.

\begin{itemize}
\item{\verb@mirrored(this::namespace)@} reflects the namespace inside of which the reflection
operator was invoked.
\item{\verb@mirrored(this::class)@} reflects the class inside of which the reflection
operator was invoked. This should also work inside of member functions, constructors and operators
of that class.
\item{\verb@mirrored(this::function)@} reflects the function inside of which the reflection
operator was invoked.
\end{itemize}

\subsection{Turning compile-time strings into identifiers}
\label{appendix-operator-identifier}

Inspired by the idea of {\em name literals} as mentioned on the WG mailing list,
we suggest to consider adding a new functionality to the core language, allowing to specify
identifiers as compile-time constant C-string literal expression, i.e. expressions
evaluating into values of \verb@constexpr const char [N]@.

This could be implemented either by using a new operator (or recycling an old one),
or maybe by using generalized attributes.
In the use-cases described in this paper the \verb@identifier@ operator is used, but we do not have
any strong preference for the name of this operator.

For example:

\begin{minted}[tabsize=4]{cpp}

identifier("int") identifier("main")(
	int idenitifier("argc"),
	const identifier("char")* identifier("argv")
)
{
	using namespace identifier(base_name<mirrored(std)>::value);
	for(int i=0; i<argc; ++i)
	{
		cout << argv[i] << endl;
	}
	return 0;
}

\end{minted}

would be equivalent to

\begin{minted}[tabsize=4]{cpp}
int main(int argc, const char* argv)
{
	using namespace std;
	/* ... */
}
\end{minted}

The content of the constexpr string passed as the argument to \verb@identifier@
should be encoded in the source character set and subject to the same restrictions
which are placed on identifiers.

The idea is to replace preprocessor token pasting with much more flexible constexpr C++ expressions.
Adding this feature would also allow to remove the \verb@named_mem_var@ and
\verb@named_typedef@ metafunctions which were in N4111 defined as part of
the interface of \meta{Named}.

This addition has the potential to complicate the processing of a translation unit
by the compiler and would logically fit somewhere between phases 6 and 8 as described
in the standard. If the use of regular templates for the purpose of creating the
constexpr identifier strings would be too complicated to implement, phase 6 could
be extended to allow simple compile-time text manipulation (comparison, concatenation,
substrings, etc.) by a set of dedicated functions.

