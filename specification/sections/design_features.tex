\subsection{Desired features} 

The proposed reflection facility is designed with the following
goals in mind:

\begin{itemize}
\item {\em Reusability}: The provided metadata should be reusable
in many situations and for many different purposes, not only
the obvious ones. This is closely related to {\em completeness} (below).

\item {\em Flexibility}: The basic reflection and the libraries
built on top of it should be designed
in a way that they are eventually usable during both compile-time
and run-time and under various paradigms (object-oriented, functional, etc.),
depending on the application needs.

\item {\em  Encapsulation}: The metadata should be accessible
through conceptually well-defined interfaces.

\item {\em  Stratification}: Reflection should be non-intrusive,
and the meta-level should be separated from the base-level language
constructs it reflects. Also, reflection should not be implemented
in a all-or-nothing manner. Things that are not needed, should not generally
be compiled-into the final application.

\item {\em  Ontological correspondence}: The meta-level facilities should
correspond to the ontology of the base-level C++ language constructs
which they reflect. This basically means that all existing language
features should be reflected and new ones should not be invented.
This rule may have some important exceptions, like the reflection of
containers. 

\item {\em  Completeness}: The proposed reflection facility should
provide as much useful metadata as possible, including various specifiers,
(like constness, storage-class, access, etc.), namespace members,
enumerated types, iteration of namespace members and much more.

\item {\em  Ease of use}: Although reflection-based metaprogramming
allows to implement very complicated things, simple things
should be kept simple.

\item {\em  Cooperation with other librares}: Reflection should be
usable with the existing introspection facilites (like \verb@type_traits@)
already provided by the standard library and with other libraries.
\end{itemize}
