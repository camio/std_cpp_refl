\subsubsection{Lagoon}

Lagoon defines run-time polymorphic interfaces and classes implementing these
interfaces and wrapping the compile-time metaobjects from Mirror and Puddle.
While Rubber is more suitable for simple decoupling of reflection-based
algorithms from the real types of the metaobjects that the algorithms
operate on, Lagoon is full-blown run-time reflection utility that can be
even decoupled from the application using it and loaded dynamically on-demand.

This example queries the meta-types reflecting types in the global scope,
orders them by the value of \verb@sizeof@ and prints their names:

\begin{lstlisting}
#include <mirror/mirror.hpp>
#include <lagoon/lagoon.hpp>
#include <lagoon/range/extract.hpp>
#include <lagoon/range/sort.hpp>
#include <lagoon/range/for_each.hpp>
#include <iostream>

int main(void)
{
    using namespace lagoon;
    typedef shared<meta_named_scoped_object> shared_mnso;
    typedef shared<meta_type> shared_mt;
    //
    // traverses the range of meta-objects passed as
    // the first argument and on each of them executes
    // the functor passed as the second argument
    for_each(

        // sorts the range passed as the first argument
        // using the functor passed as the second argument
        // for comparison
        sort(

            // extracts only those having the meta_type
            // interface
            extract<meta_type>(

                // gets all members of the global scope
                reflected_global_scope()->members()
            ),

            // compares two meta-types on the value
            // of sizeof(reflected-type)
            [](const shared_mt& a, const shared_mt& b)
            {
                return a->size_of() < b->size_of();
            }
        ),

        // prints the full name of a type
        [](const shared_mt& member)
        {
            std::cout << member->full_name() << std::endl;
        }
    );
    return 0;
}
\end{lstlisting}

This application prints the following on the standard output:

\begin{verbatim}
void
bool
char
unsigned char
short int
unsigned short int
wchar_t
int
long int
unsigned int
unsigned long int
float
double
long double
\end{verbatim}

The following example is more complex and shows the usage of Lagoon's
object factories, in this case a factory using a text-script similar
to C++ uniform initializers to provide input data from which a set
of instances is constructed:

\begin{lstlisting}
#include <mirror/mirror_base.hpp>
#include <mirror/pre_registered/basic.hpp>
#include <mirror/pre_registered/class/std/vector.hpp>
#include <mirror/pre_registered/class/std/map.hpp>
#include <mirror/utils/quick_reg.hpp>
#include <lagoon/lagoon.hpp>
#include <lagoon/utils/script_factory.hpp>
#include <iostream>

namespace morse {

// declares an enumerated class for morse code symbols
enum class signal { dash = '-', dot = '.' };

// declares a type for a sequence of morse code symbols
typedef std::vector<signal> sequence;

// declares a type for storing morse code entries
typedef std::map<char, sequence> code;

} // namespace morse

MIRROR_REG_BEGIN

// registers the morse namespace
MIRROR_QREG_GLOBAL_SCOPE_NAMESPACE(morse)
// registers the signal enumeration
MIRROR_QREG_ENUM(morse, signal, (dash)(dot))

MIRROR_REG_END

int main(void)
{
    try
    {
        using namespace lagoon;

        // a factory builder class provided by Lagoon
        // that can be used together with a meta-type
        // to build a factory
        c_str_script_factory_builder builder;

        // a class storing the input data for the factory
        // built by the builder
        c_str_script_factory_input in;

        // the input data for the factory
        auto data = in.data();

	// polyomrphic meta-type reflecting the morse::code type
        auto meta_morse_code = reflected_class<morse::code>();

        // a polymorphic factory that can be used to construct
        // instances of the morse::code type, that is built by
        // the builder and the meta-type reflecting morse::code.
        auto morse_code_factory = meta_morse_code->make_factory(
            builder,
            raw_ptr(&data)
        );

        // the input string for this factory
        const char input[] = "{ \
            {'A', {dot, dash}}, \
            {'B', {dash, dot, dot, dot}}, \
            {'C', {dash, dot, dash, dot}}, \
            {'D', {dash, dot, dot}}, \
            {'E', {dot}}, \
            {'F', {dot, dot, dash, dot}}, \
            {'G', {dash, dash, dot}}, \
            {'H', {dot, dot, dot, dot}}, \
            {'I', {dot, dot}}, \
            {'J', {dot, dash, dash, dash}}, \
            {'K', {dash, dot, dash}}, \
            {'L', {dot, dash, dot, dot}}, \
            {'M', {dash, dash}}, \
            {'N', {dash, dot}}, \
            {'O', {dash, dash, dash}}, \
            {'P', {dot, dash, dash, dot}}, \
            {'Q', {dash, dash, dot, dash}}, \
            {'R', {dot, dash, dot}}, \
            {'S', {dot, dot, dot}}, \
            {'T', {dash}}, \
            {'U', {dot, dot, dash}}, \
            {'V', {dot, dot, dot, dash}}, \
            {'W', {dot, dash, dash}}, \
            {'X', {dash, dot, dot, dash}}, \
            {'Y', {dash, dot, dash, dash}}, \
            {'Z', {dash, dash, dot, dot}}, \
            {'1', {dot, dash, dash, dash, dash}}, \
            {'2', {dot, dot, dash, dash, dash}}, \
            {'3', {dot, dot, dot, dash, dash}}, \
            {'4', {dot, dot, dot, dot, dash}}, \
            {'5', {dot, dot, dot, dot, dot}}, \
            {'6', {dash, dot, dot, dot, dot}}, \
            {'7', {dash, dash, dot, dot, dot}}, \
            {'8', {dash, dash, dash, dot, dot}}, \
            {'9', {dash, dash, dash, dash, dot}}, \
            {'0', {dash, dash, dash, dash, dash}} \
        }";

        // passes the input data to the factory
        in.set(input, input+sizeof(input));

        // uese the factory built above to create
        // a new instance of the morse::code type
        raw_ptr pmc = morse_code_factory->new_();

        // cast of the raw pointer returned by the factory
        // to the concrete type (morse::code)
        morse::code& mc = *raw_cast<morse::code*>(pmc);

        // the morse::code type is just a map of char to
	// a vector of morse signals, this prints them
        // to cout in a standard way
        for(auto i = mc.begin(), e = mc.end(); i != e; ++i)
        {
            std::cout << "Morse code for '" << i->first << "': ";
            auto j = i->second.begin(), f = i->second.end();
            while(j != f)
            {
                std::cout << char(*j);
                ++j;
            }
            std::cout << std::endl;
        }

        // uses the meta-type reflecting morse::code to delete
        // the instance constructed by the factory
        meta_morse_code->delete_(pmc);
    }
    catch(std::exception const& error)
    {
        std::cerr << "Error: " << error.what() << std::endl;
    }
    //
    return 0;
}
\end{lstlisting}

This application has the following output:

\begin{verbatim}
Morse code for '0': -----
Morse code for '1': .----
Morse code for '2': ..---
Morse code for '3': ...--
Morse code for '4': ....-
Morse code for '5': .....
Morse code for '6': -....
Morse code for '7': --...
Morse code for '8': ---..
Morse code for '9': ----.
Morse code for 'A': .-
Morse code for 'B': -...
Morse code for 'C': -.-.
Morse code for 'D': -..
Morse code for 'E': .
Morse code for 'F': ..-.
Morse code for 'G': --.
Morse code for 'H': ....
Morse code for 'I': ..
Morse code for 'J': .---
Morse code for 'K': -.-
Morse code for 'L': .-..
Morse code for 'M': --
Morse code for 'N': -.
Morse code for 'O': ---
Morse code for 'P': .--.
Morse code for 'Q': --.-
Morse code for 'R': .-.
Morse code for 'S': ...
Morse code for 'T': -
Morse code for 'U': ..-
Morse code for 'V': ...-
Morse code for 'W': .--
Morse code for 'X': -..-
Morse code for 'Y': -.--
Morse code for 'Z': --..
\end{verbatim}

For more examples of usage see ~\cite{mirror-doc-lagoon-examples}.
