\subsubsection{Rubber}

Rubber is a OOP-style run-time type erasure utility built on top
of Mirror and Puddle. It again follows the metaobject concept hierarchy of Mirror and Puddle.
Rubber allows to access and store metaobjects of the same category in a single
type, so in contrast to Mirror and Puddle where a meta-type reflecting the \verb@int@
type and a meta-type reflecting the \verb@double@ type have different types
in Rubber they can both be stored in a variable of the same type.
Rubber does not use virtual functions but rather pointers to existing
functions implemented by Mirror to achieve run-time polymorphism.

The first example shows the usage of type-erased metaobjects with a C++11
lambda function which could not be used with Mirror's or Puddle's meda-objects
(because lambdas are not templated):

\begin{minted}{cpp}
#include <mirror/mirror.hpp>
#include <rubber/rubber.hpp>
#include <iostream>

int main(void)
{
    // use the Mirror's for_each function, but erase
    // the types of the iterated compile-time metaobjects
    // before passing them as arguments to the lambda function.
    mirror::mp::for_each<
        mirror::members<
            MIRRORED_GLOBAL_SCOPE()
        >
    >(
        // the rubber::meta_named_scoped_object type is
        // constructible from a Mirror MetaNamedScopedObject
        [](const rubber::meta_named_scoped_object& member)
        {
            std::cout <<
                member.self().construct_name() <<
                " " <<
                member.base_name() <<
                std::endl;
        }
    );
    return 0;
}
\end{minted}

This simple application prints the following on the standard output:

\begin{verbatim}
namespace std
namespace boost
namespace mirror
type void
type bool
type char
type unsigned char
type wchar_t
type short int
type int
type long int
type unsigned short int
type unsigned int
type unsigned long int
type float
type double
type long double
\end{verbatim}

The next example prints different information for different categories
of metaobjects:

\begin{minted}{cpp}
#include <mirror/mirror.hpp>
#include <rubber/rubber.hpp>
#include <iostream>
#include <vector>

int main(void)
{
    using namespace rubber;
    mirror::mp::for_each<
        mirror::members<
            MIRRORED_GLOBAL_SCOPE()
        >
    >(
        eraser<meta_scope, meta_type, meta_named_object>(
            [](const meta_scope& scope)
            {
                std::cout <<
                    scope.self().construct_name() <<
                    " '" <<
                    scope.base_name() <<
                    "', number of members = " <<
                    scope.members().size() <<
                    std::endl;
            },
            [](const meta_type& type)
            {
                std::cout <<
                    type.self().construct_name() <<
                    " '" <<
                    type.base_name() <<
                    "', size in bytes = " <<
                    type.sizeof_() <<
                    std::endl;
            },
            [](const meta_named_object& named)
            {
                std::cout <<
                    named.self().construct_name() <<
                    " '" <<
                    named.base_name() <<
                    "'" <<
                    std::endl;
            }
        )
    );
    return 0;
}
\end{minted}

It has the following output:

\begin{verbatim}
namespace 'std', number of members = 20
namespace 'boost', number of members = 0
namespace 'mirror', number of members = 0
type 'void', size in bytes = 0
type 'bool', size in bytes = 1
type 'char', size in bytes = 1
type 'unsigned char', size in bytes = 1
type 'wchar_t', size in bytes = 4
type 'short int', size in bytes = 2
type 'int', size in bytes = 4
type 'long int', size in bytes = 8
type 'unsigned short int', size in bytes = 2
type 'unsigned int', size in bytes = 4
type 'unsigned long int', size in bytes = 8
type 'float', size in bytes = 4
type 'double', size in bytes = 8
type 'long double', size in bytes = 16
\end{verbatim}

For more examples of usage see ~\cite{mirror-doc-rubber-examples}.
