\subsubsection{Puddle}

Puddle is a OOP-style (mostly) compile-time interface built on top
of Mirror. It copies the metaobject concept hierarchy of Mirror,
but provides a more "object-ish" interface as shown below:

Instead of Mirror's:

\begin{lstlisting}
static_assert(
  is_public<
    access_type<
      at_c<
        member_variables<
          reflected<person>
        >,
        0
      >
    >
  >::value,
  "Shoot, persons first mem. variable is not public!"
)

\end{lstlisting}

Puddle allows to do the following:

\begin{lstlisting}
assert(
  reflected_type<person>()
    member_variables().
      at_c<0>().
        access_type().
          is_public()
);
\end{lstlisting}

or a more complex example, in which a reflection-based algorithm
traverses the global scope namespace and its nested scopes
and prints information about their members:

\begin{lstlisting}
struct object_printer
{
  std::ostream& out;
  int indent_level;

  std::ostream& indented_output(void)
  {
    for(int i=0;i!=indent_level;++i)
      out << "  ";
    return out;
  }

  template <class MetaObject>
  void print_details(MetaObject obj, mirror::meta_object_tag)
  {
  }

  template <class MetaObject>
  void print_details(MetaObject obj, mirror::meta_scope_tag)
  {
    out << ": ";
    if(obj.members().empty())
    {
      out << "{ }";
    }
    else
    {
      out << "{" << std::endl;
      object_printer print_members = {out, indent_level+1};
      obj.members().for_each(print_members);
      indented_output() << "}";
    }
  }

  template <class MetaObject>
  void print(MetaObject obj, bool last)
  {
    indented_output()
      << obj.self().construct_name()
      << " "
      << obj.base_name();
    print_details(obj, obj.category());
    if(!last) out << ",";
    out << std::endl;
  }
  template <class MetaObject>
  void operator()(MetaObject obj, bool first, bool last)
  {
    print(obj, last);
  }

  template <class MetaObject>
  void operator()(MetaObject obj)
  {
    print(obj, true);
  }


int main(void)
{
  object_printer print = {std::cout, 0};
  print(puddle::adapt<MIRRORED_GLOBAL_SCOPE()>());
  return 0;
}
\end{lstlisting}

which prints the following on the standard output:

\begin{verbatim}
   global scope : {
     namespace std: {
       class string: { },
       class wstring: { },
       template pair,
       template tuple,
       template initializer_list,
       template allocator,
       template equal_to,
       template not_equal_to,
       template less,
       template greater,
       template less_equal,
       template greater_equal,
       template deque,
       class tm: {
         member variable tm_sec,
         member variable tm_min,
         member variable tm_hour,
         member variable tm_mday,
         member variable tm_mon,
         member variable tm_year,
         member variable tm_wday,
         member variable tm_yday,
         member variable tm_isdst
       },
       template vector,
       template list,
       template set,
       template map
     },
     namespace boost: {
       template optional
     },
     namespace mirror: { },
     type void,
     type bool,
     type char,
     type unsigned char,
     type wchar_t,
     type short int,
     type int,
     type long int,
     type unsigned short int,
     type unsigned int,
     type unsigned long int,
     type float,
     type double,
     type long double
   }
\end{verbatim}

For more examples of usage see ~\cite{mirror-doc-puddle-examples}.
