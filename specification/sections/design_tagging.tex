\subsection{Annotations and relations}

Strict adhering to the principle of {\em Ontological correspondence} can pose
a problem and decrease the usefullness of reflection in certain cases.
Probably the most important case in C++ is the reflection of containers
which are not implemented as first-class citizens but rather by libraries.

The principle of ontological correspondence says that the meta-level facilities
should not invent metaobjects that do not correspond to base-level
language features. In C++ where containers are basically regular
classes internally implementing a data structure that is not standardized
and is platform-specific and having only a public interface defined 
by the standard, automated reflection-based implementation of operations
like serialization, de-serialization, and others may become complicated.

The purpose of serialization (for example as a part of RPC) may be to
convert an instance of a container class into an external representation
that can be used to transfer the instance (for example via network) 
to a machine running the receiving application, but having a different
OS, compiler, using a different implementation of the standard library
and thus a different implementation of the container class.

If the serialization algorithm would use the description of the internal
(non-standard vendor-specific) structure of a class, then the receiving
application would not be able to restore the instance, because the 
internal structure of the class would be different.

Because of this a high-level mechanism is required, that would allow
reflection-based metaalgorithms to handle such cases.

One possible option is to break the principle of correspondence
and add new concepts for metaobjects allowing for example
"high-level" traversal or insertion of elements in an arbitrary
container. This approach has its advantages and could be implemented
for such special classes as containers, but is unsystematic.

Another option is to allow the base-level constructs to be annotated
and let the metaobjects to provide these annotations to the metaalgorithms.

These annotations should take the form of {\em tags}: identifiers assigned
to base-level constructs, like types, variables, class members, parameters,
etc. and binary directional {\em relations} between two language constructs,
for example a relation between a class member and a constructor parameter
initializing the class member, a relation between (a pair of) container
element traversal functions (like begin and end in std. containers) and 
functions or constructors doing element insertion into the same container
class.
