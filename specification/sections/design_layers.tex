\subsection{Layered approach and extensibility}

The purpose of this section is to show that a {\em static} $\to$ {\em dynamic}
and {\em basic} $\to$ {\em complex} approach in designing reflection
can accomodate a wide variety of programming styles and is arguably
the "best" one. We do not propose to add all layers described
below into the standard library. They are mentioned here only to
show that a well designed compile-time reflection is a good foundation
for many (if not all) other reflection facilities.

The Mirror reflection utilities \cite{mirror-doc-cpp11} on which this
proposal is based, implements several distinct components which
are stacked on top of each other. From the low-level metadata, through
a functional-style compile-time interface to a completely dynamic
object-oriented run-time layer (all described in greater detail below).

\subsubsection{Basic metaobjects}
The very basic metadata, which are in Mirror
provided (registered) by the user (or an automated command-line tool) via a set
of preprocessor macros. This approach is both inconvenient and error-prone
in many situations, but also has its advantages.

We propose that a standard compiler should make these metadata available
to the programmer through the static basic metaobject interfaces. These should
serve as the basis for other (standard and non-standard) higher-level
reflection libraries and utilities.

In the Mirror utilities the basic metadata is not used directly by
applications.

\subsubsection{Mirror}

Mirror is a compile-time functional-style reflective programming library,
which is based directly on the basic metadata and is suitable for generic programming,
similar to the standard \verb@type_traits@ library.

Mirror is the original library from which the Mirror reflection utilities started.

It provides a more user-friendly and rich interface than the basic-metaobjects.
and a set of metaprogramming utilities which allow
to write compile-time meta-programs, which can generate efficient
and optimized program code using only those metadata that are required.

The following text contains several (rather simple) examples of usage
and the functional style of the algorithms based on metadata provided by Mirror.

The first example prints some information about the members of selected
namespaces to \verb@std::cout@.

\begin{lstlisting}
struct info_printer
{
    template <typename MetaObject>
    void operator()(MetaObject mo) const
    {
        MIRRORED_META_OBJECT(MetaObject) mmo;
        std::cout
            << mmo.construct_name()
            << ": "
            << mo.full_name()
            << std::endl;
    }
};

int main(void)
{
    using namespace mirror;

    // print the info about each of the members
    // of the global scope
    mirror::mp::for_each<
        members<

            // this should be in standard C++
            // be replaced by a specialstandard library
            // function or operator
            MIRRORED_GLOBAL_SCOPE()
        >
    >(info_printer());

    // print the info about each of the members
    // of the std namespace
    mp::for_each<
        members<

            // this should be in standard C++
            // be replaced by a special standard
            // library function or operator
            MIRRORED_NAMESPACE(std)
        >
    >(info_printer());
    //
    return 0;
}
\end{lstlisting}

This program produces the following output:

\begin{verbatim}
   namespace: std
   namespace: boost
   type: void
   type: bool
   type: char
   type: unsigned char
   type: wchar_t
   type: short int
   type: int
   type: long int
   type: unsigned short int
   type: unsigned int
   type: unsigned long int
   type: float
   type: double
   type: long double
   class: std::string
   class: std::wstring
   class: std::tm
   template: std::pair
   template: std::tuple
   template: std::allocator
   template: std::equal_to
   template: std::not_equal_to
   template: std::less
   template: std::greater
   template: std::less_equal
   template: std::greater_equal
   template: std::vector
   template: std::list
   template: std::deque
   template: std::map
   template: std::set
\end{verbatim}

The next example gets all types in the global scope,
applies some \verb@type_traits@ modifiers like \verb@std::add_pointer@
\verb@std::add_const@ and for each of such modified types calls a functor
that prints the names of the individual types to the standard output:

\begin{lstlisting}
struct name_printer
{
    template <typename MetaNamedObject>
    void operator()(MetaNamedObject mo) const
    {
        std::cout << mo.base_name() << std::endl;
    }
};

int main(void)
{
  using namespace mirror;

  // this function calls the name_printer functor passed
  // as the function argument on each element in the 
  // range that is passed as the template argument
  mp::for_each<

    // this template transforms the elements in the range
    // passed as the first argument by the unary template
    // passed as the second argument
    mp::transform<

      // this template filters out only those metaobjects
      // that satisfy the predicate passed as the second
      // argument from the range of metaobjects passed
      // as the first argument
      mp::only_if<

        // this template "returns" a range of metaobjects
        // reflecting the members of the namespace
        // (or other scope) that is passed as argument
        members<

          // this macro expands into a class
          // conforming to the Mirror's MetaNamespace
          // concept and provides metadata describing
          // the global scope namespace.
          // in the proposed solution for standard C++
          // this should be relaced by a special stdlib
          // function or by an operator.
          MIRRORED_GLOBAL_SCOPE()
        >,

        // this is a lambda function testing if its first
        // argument falls to the MetaType category
        mp::is_a<
          mp::arg<1>,
          meta_type_tag
        >
      >,

      // this is a unary lambda function that modifies
      // the type passed as its argument by
      // the add_pointer and add_const type traits
      apply_modifier<
        mp::arg<1>,
        mp::protect<
          std::add_pointer<
            std::add_const<
              mp::arg<1>
            >
          >
        >
      >
    >
  >(name_printer());
  std::cout << std::endl;
  return 0;
}

\end{lstlisting}

This short program produces the following output:

\begin{verbatim}
   void const *
   bool const *
   char const *
   unsigned char const *
   wchar_t const *
   short int const *
   int const *
   long int const *
   unsigned short int const *
   unsigned int const *
   unsigned long int const *
   float const *
   double const *
   long double const *
\end{verbatim}

The printing of names is definitely not the only usage
of reflection. The scope of this proposal does not allow
to include and fully explain the more elaborated applications.
For some other examples of usage see ~\cite{mirror-doc-mirror-examples}.


\subsubsection{Puddle}

Puddle is a OOP-style (mostly) compile-time interface built on top
of Mirror. It copies the metaobject concept hierarchy of Mirror,
but provides a more "object-ish" interface as shown below:

Instead of Mirror's:

\begin{lstlisting}
static_assert(
  is_public<
    access_type<
      at_c<
        member_variables<
          reflected<person>
        >,
        0
      >
    >
  >::value,
  "Shoot, persons first mem. variable is not public!"
)

\end{lstlisting}

Puddle allows to do the following:

\begin{lstlisting}
assert(
  reflected_type<person>()
    member_variables().
      at_c<0>().
        access_type().
          is_public()
);
\end{lstlisting}

or a more complex example, in which a reflection-based algorithm
traverses the global scope namespace and its nested scopes
and prints information about their members:

\begin{lstlisting}
struct object_printer
{
  std::ostream& out;
  int indent_level;

  std::ostream& indented_output(void)
  {
    for(int i=0;i!=indent_level;++i)
      out << "  ";
    return out;
  }

  template <class MetaObject>
  void print_details(MetaObject obj, mirror::meta_object_tag)
  {
  }

  template <class MetaObject>
  void print_details(MetaObject obj, mirror::meta_scope_tag)
  {
    out << ": ";
    if(obj.members().empty())
    {
      out << "{ }";
    }
    else
    {
      out << "{" << std::endl;
      object_printer print_members = {out, indent_level+1};
      obj.members().for_each(print_members);
      indented_output() << "}";
    }
  }

  template <class MetaObject>
  void print(MetaObject obj, bool last)
  {
    indented_output()
      << obj.self().construct_name()
      << " "
      << obj.base_name();
    print_details(obj, obj.category());
    if(!last) out << ",";
    out << std::endl;
  }
  template <class MetaObject>
  void operator()(MetaObject obj, bool first, bool last)
  {
    print(obj, last);
  }

  template <class MetaObject>
  void operator()(MetaObject obj)
  {
    print(obj, true);
  }


int main(void)
{
  object_printer print = {std::cout, 0};
  print(puddle::adapt<MIRRORED_GLOBAL_SCOPE()>());
  return 0;
}
\end{lstlisting}

which prints the following on the standard output:

\begin{verbatim}
   global scope : {
     namespace std: {
       class string: { },
       class wstring: { },
       template pair,
       template tuple,
       template initializer_list,
       template allocator,
       template equal_to,
       template not_equal_to,
       template less,
       template greater,
       template less_equal,
       template greater_equal,
       template deque,
       class tm: {
         member variable tm_sec,
         member variable tm_min,
         member variable tm_hour,
         member variable tm_mday,
         member variable tm_mon,
         member variable tm_year,
         member variable tm_wday,
         member variable tm_yday,
         member variable tm_isdst
       },
       template vector,
       template list,
       template set,
       template map
     },
     namespace boost: {
       template optional
     },
     namespace mirror: { },
     type void,
     type bool,
     type char,
     type unsigned char,
     type wchar_t,
     type short int,
     type int,
     type long int,
     type unsigned short int,
     type unsigned int,
     type unsigned long int,
     type float,
     type double,
     type long double
   }
\end{verbatim}

For more examples of usage see ~\cite{mirror-doc-puddle-examples}.

\subsubsection{Rubber}

Rubber is a OOP-style run-time type erasure utility built on top
of Mirror and Puddle. It again follows the metaobject concept hierarchy of Mirror and Puddle.
Rubber allows to access and store metaobjects of the same category in a single
type, so in contrast to Mirror and Puddle where a meta-type reflecting the \verb@int@
type and a meta-type reflecting the \verb@double@ type have different types
in Rubber they can both be stored in a variable of the same type.
Rubber does not use virtual functions but rather pointers to existing
functions implemented by Mirror to achieve run-time polymorphism.

The first example shows the usage of type-erased metaobjects with a C++11
lambda function which could not be used with Mirror's or Puddle's meda-objects
(because lambdas are not templated):

\begin{lstlisting}
#include <mirror/mirror.hpp>
#include <rubber/rubber.hpp>
#include <iostream>

int main(void)
{
    // use the Mirror's for_each function, but erase
    // the types of the iterated compile-time metaobjects
    // before passing them as arguments to the lambda function.
    mirror::mp::for_each<
        mirror::members<
            MIRRORED_GLOBAL_SCOPE()
        >
    >(
        // the rubber::meta_named_scoped_object type is
        // constructible from a Mirror MetaNamedScopedObject
        [](const rubber::meta_named_scoped_object& member)
        {
            std::cout <<
                member.self().construct_name() <<
                " " <<
                member.base_name() <<
                std::endl;
        }
    );
    return 0;
}
\end{lstlisting}

This simple application prints the following on the standard output:

\begin{verbatim}
namespace std
namespace boost
namespace mirror
type void
type bool
type char
type unsigned char
type wchar_t
type short int
type int
type long int
type unsigned short int
type unsigned int
type unsigned long int
type float
type double
type long double
\end{verbatim}

The next example prints different information for different categories
of metaobjects:

\begin{lstlisting}
#include <mirror/mirror.hpp>
#include <rubber/rubber.hpp>
#include <iostream>
#include <vector>

int main(void)
{
    using namespace rubber;
    mirror::mp::for_each<
        mirror::members<
            MIRRORED_GLOBAL_SCOPE()
        >
    >(
        eraser<meta_scope, meta_type, meta_named_object>(
            [](const meta_scope& scope)
            {
                std::cout <<
                    scope.self().construct_name() <<
                    " '" <<
                    scope.base_name() <<
                    "', number of members = " <<
                    scope.members().size() <<
                    std::endl;
            },
            [](const meta_type& type)
            {
                std::cout <<
                    type.self().construct_name() <<
                    " '" <<
                    type.base_name() <<
                    "', size in bytes = " <<
                    type.sizeof_() <<
                    std::endl;
            },
            [](const meta_named_object& named)
            {
                std::cout <<
                    named.self().construct_name() <<
                    " '" <<
                    named.base_name() <<
                    "'" <<
                    std::endl;
            }
        )
    );
    return 0;
}
\end{lstlisting}

It has the following output:

\begin{verbatim}
namespace 'std', number of members = 20
namespace 'boost', number of members = 0
namespace 'mirror', number of members = 0
type 'void', size in bytes = 0
type 'bool', size in bytes = 1
type 'char', size in bytes = 1
type 'unsigned char', size in bytes = 1
type 'wchar_t', size in bytes = 4
type 'short int', size in bytes = 2
type 'int', size in bytes = 4
type 'long int', size in bytes = 8
type 'unsigned short int', size in bytes = 2
type 'unsigned int', size in bytes = 4
type 'unsigned long int', size in bytes = 8
type 'float', size in bytes = 4
type 'double', size in bytes = 8
type 'long double', size in bytes = 16
\end{verbatim}

For more examples of usage see ~\cite{mirror-doc-rubber-examples}.

\subsubsection{Lagoon}

Lagoon defines run-time polymorphic interfaces and classes implementing these
interfaces and wrapping the compile-time metaobjects from Mirror and Puddle.
While Rubber is more suitable for simple decoupling of reflection-based
algorithms from the real types of the metaobjects that the algorithms
operate on, Lagoon is full-blown run-time reflection utility that can be
even decoupled from the application using it and loaded dynamically on-demand.

This example queries the meta-types reflecting types in the global scope,
orders them by the value of \verb@sizeof@ and prints their names:

\begin{lstlisting}
#include <mirror/mirror.hpp>
#include <lagoon/lagoon.hpp>
#include <lagoon/range/extract.hpp>
#include <lagoon/range/sort.hpp>
#include <lagoon/range/for_each.hpp>
#include <iostream>

int main(void)
{
    using namespace lagoon;
    typedef shared<meta_named_scoped_object> shared_mnso;
    typedef shared<meta_type> shared_mt;
    //
    // traverses the range of meta-objects passed as
    // the first argument and on each of them executes
    // the functor passed as the second argument
    for_each(

        // sorts the range passed as the first argument
        // using the functor passed as the second argument
        // for comparison
        sort(

            // extracts only those having the meta_type
            // interface
            extract<meta_type>(

                // gets all members of the global scope
                reflected_global_scope()->members()
            ),

            // compares two meta-types on the value
            // of sizeof(reflected-type)
            [](const shared_mt& a, const shared_mt& b)
            {
                return a->size_of() < b->size_of();
            }
        ),

        // prints the full name of a type
        [](const shared_mt& member)
        {
            std::cout << member->full_name() << std::endl;
        }
    );
    return 0;
}
\end{lstlisting}

This application prints the following on the standard output:

\begin{verbatim}
void
bool
char
unsigned char
short int
unsigned short int
wchar_t
int
long int
unsigned int
unsigned long int
float
double
long double
\end{verbatim}

The following example is more complex and shows the usage of Lagoon's
object factories, in this case a factory using a text-script similar
to C++ uniform initializers to provide input data from which a set
of instances is constructed:

\begin{lstlisting}
#include <mirror/mirror_base.hpp>
#include <mirror/pre_registered/basic.hpp>
#include <mirror/pre_registered/class/std/vector.hpp>
#include <mirror/pre_registered/class/std/map.hpp>
#include <mirror/utils/quick_reg.hpp>
#include <lagoon/lagoon.hpp>
#include <lagoon/utils/script_factory.hpp>
#include <iostream>

namespace morse {

// declares an enumerated class for morse code symbols
enum class signal { dash = '-', dot = '.' };

// declares a type for a sequence of morse code symbols
typedef std::vector<signal> sequence;

// declares a type for storing morse code entries
typedef std::map<char, sequence> code;

} // namespace morse

MIRROR_REG_BEGIN

// registers the morse namespace
MIRROR_QREG_GLOBAL_SCOPE_NAMESPACE(morse)
// registers the signal enumeration
MIRROR_QREG_ENUM(morse, signal, (dash)(dot))

MIRROR_REG_END

int main(void)
{
    try
    {
        using namespace lagoon;

        // a factory builder class provided by Lagoon
        // that can be used together with a meta-type
        // to build a factory
        c_str_script_factory_builder builder;

        // a class storing the input data for the factory
        // built by the builder
        c_str_script_factory_input in;

        // the input data for the factory
        auto data = in.data();

	// polyomrphic meta-type reflecting the morse::code type
        auto meta_morse_code = reflected_class<morse::code>();

        // a polymorphic factory that can be used to construct
        // instances of the morse::code type, that is built by
        // the builder and the meta-type reflecting morse::code.
        auto morse_code_factory = meta_morse_code->make_factory(
            builder,
            raw_ptr(&data)
        );

        // the input string for this factory
        const char input[] = "{ \
            {'A', {dot, dash}}, \
            {'B', {dash, dot, dot, dot}}, \
            {'C', {dash, dot, dash, dot}}, \
            {'D', {dash, dot, dot}}, \
            {'E', {dot}}, \
            {'F', {dot, dot, dash, dot}}, \
            {'G', {dash, dash, dot}}, \
            {'H', {dot, dot, dot, dot}}, \
            {'I', {dot, dot}}, \
            {'J', {dot, dash, dash, dash}}, \
            {'K', {dash, dot, dash}}, \
            {'L', {dot, dash, dot, dot}}, \
            {'M', {dash, dash}}, \
            {'N', {dash, dot}}, \
            {'O', {dash, dash, dash}}, \
            {'P', {dot, dash, dash, dot}}, \
            {'Q', {dash, dash, dot, dash}}, \
            {'R', {dot, dash, dot}}, \
            {'S', {dot, dot, dot}}, \
            {'T', {dash}}, \
            {'U', {dot, dot, dash}}, \
            {'V', {dot, dot, dot, dash}}, \
            {'W', {dot, dash, dash}}, \
            {'X', {dash, dot, dot, dash}}, \
            {'Y', {dash, dot, dash, dash}}, \
            {'Z', {dash, dash, dot, dot}}, \
            {'1', {dot, dash, dash, dash, dash}}, \
            {'2', {dot, dot, dash, dash, dash}}, \
            {'3', {dot, dot, dot, dash, dash}}, \
            {'4', {dot, dot, dot, dot, dash}}, \
            {'5', {dot, dot, dot, dot, dot}}, \
            {'6', {dash, dot, dot, dot, dot}}, \
            {'7', {dash, dash, dot, dot, dot}}, \
            {'8', {dash, dash, dash, dot, dot}}, \
            {'9', {dash, dash, dash, dash, dot}}, \
            {'0', {dash, dash, dash, dash, dash}} \
        }";

        // passes the input data to the factory
        in.set(input, input+sizeof(input));

        // uese the factory built above to create
        // a new instance of the morse::code type
        raw_ptr pmc = morse_code_factory->new_();

        // cast of the raw pointer returned by the factory
        // to the concrete type (morse::code)
        morse::code& mc = *raw_cast<morse::code*>(pmc);

        // the morse::code type is just a map of char to
	// a vector of morse signals, this prints them
        // to cout in a standard way
        for(auto i = mc.begin(), e = mc.end(); i != e; ++i)
        {
            std::cout << "Morse code for '" << i->first << "': ";
            auto j = i->second.begin(), f = i->second.end();
            while(j != f)
            {
                std::cout << char(*j);
                ++j;
            }
            std::cout << std::endl;
        }

        // uses the meta-type reflecting morse::code to delete
        // the instance constructed by the factory
        meta_morse_code->delete_(pmc);
    }
    catch(std::exception const& error)
    {
        std::cerr << "Error: " << error.what() << std::endl;
    }
    //
    return 0;
}
\end{lstlisting}

This application has the following output:

\begin{verbatim}
Morse code for '0': -----
Morse code for '1': .----
Morse code for '2': ..---
Morse code for '3': ...--
Morse code for '4': ....-
Morse code for '5': .....
Morse code for '6': -....
Morse code for '7': --...
Morse code for '8': ---..
Morse code for '9': ----.
Morse code for 'A': .-
Morse code for 'B': -...
Morse code for 'C': -.-.
Morse code for 'D': -..
Morse code for 'E': .
Morse code for 'F': ..-.
Morse code for 'G': --.
Morse code for 'H': ....
Morse code for 'I': ..
Morse code for 'J': .---
Morse code for 'K': -.-
Morse code for 'L': .-..
Morse code for 'M': --
Morse code for 'N': -.
Morse code for 'O': ---
Morse code for 'P': .--.
Morse code for 'Q': --.-
Morse code for 'R': .-.
Morse code for 'S': ...
Morse code for 'T': -
Morse code for 'U': ..-
Morse code for 'V': ...-
Morse code for 'W': .--
Morse code for 'X': -..-
Morse code for 'Y': -.--
Morse code for 'Z': --..
\end{verbatim}

For more examples of usage see ~\cite{mirror-doc-lagoon-examples}.

