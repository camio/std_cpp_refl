\subsection{Basic overview}

In P0194R0 we propose to add native support for
compile-time reflection to C++ by the means of lightweight, compiler-generated
types -- {\em metaobjects}, providing metadata describing various program
declarations.

The metaobjects exist only at the type-level, they do not have any constructors
or members and cannot be instantiated. Their sole purpose is to give a
first class identity
to the reflected entity (namespace, typedef, function, parameter, specifier, etc.),
so that we can pass it as argument or return value in metaprograms and to
separate the reflection of the declaration from the querying of metadata.

The metadata can be obtained from a metaobject by using one of the class templates
which comprise its interface.
Since there are many different kinds of
base-level reflectable declarations, the metaobjects reflecting them are
modeling various {\em metaobject concepts}. The metaobjects can be inspected
by {\em metaobject traits}, which indicate whether a metaobjects has or has
not a particular property or if is falls into a particular category.

We introduce a new reflection operator -- \verb@reflexpr@ which returns a
metaobject type reflecting its operand.

P0194R0 also defines the initial subset
of metaobject concepts which we assume to be essential
and which will provide a good starting point for future extensions.

When finalized, the metaobjects can be together with some additions
to the standard library, used to implement other third-party libraries
providing both compile-time and run-time, high-level reflection.

