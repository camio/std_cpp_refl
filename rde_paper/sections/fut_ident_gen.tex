\subsection{Generating identifiers programmatically}
\label{fut-ident-gen}

Some of the use cases from the committee's CFP, for example the one described
in section \ref{use-case-struct-transf}, depend on the ability to generate
identifiers programmatically.

\subsubsection{Identifier from a generic compile-time string}
\label{fut-op-identifier1}

We will start by the most powerful, but also the most difficult to implement
option -- an operator, say \verb@identifier(const char (&)[N])@\footnote{We
do not insist on this particular name in case it should be implemented.},
which \say{creates} an identifier from an arbitrary compile-time string, so that:

\begin{minted}[tabsize=4]{cpp}
identifier("std")::identifier("size_t")
identifier("foo")(int identifier("i"), int j);
\end{minted}

would be equivalent to 

\begin{minted}[tabsize=4]{cpp}
std::size_t foo(int i, int j);
\end{minted}

Remember that the string-returning operations defined on the named metaobjects,
for example the \verb@meta::get_name@ operation, return \verb@constexpr@
arrays of \verb@char@s. This means that we can use them to copy the name of
a declaration, from a metaobject reflecting that declaration:

\begin{minted}[tabsize=4]{cpp}
struct bar
{
	int foo;
};

using meta_bar_foo = reflexpr(bar::foo);

std::size_t identifier(meta::get_name_v<meta_bar_foo>)(int i, int j);
\end{minted}

which is again equivalent to:

\begin{minted}[tabsize=4]{cpp}
std::size_t foo(int i, int j);
\end{minted}

It's important to note, that in order to do anything more complex than just
copying the identifiers of other declarations, we would also need some compile-time
string manipulation utility. For example to create identifiers like
\verb@get_%@ or \verb@set_%@ where \verb@%@ is a declaration name, we would
need to have a function for compile-time string concatenation, like
\verb@ct_concat@:

\begin{minted}[tabsize=4]{cpp}
using meta_bar_foo = reflexpr(bar::foo);

int identifier(ct_concat("get_", meta::get_name_v<meta_bar_foo>))(const bar&);
\end{minted}

which would be equivalent to:

\begin{minted}[tabsize=4]{cpp}
int get_foo(const bar&);
\end{minted}

While being very appealing and powerful, implementing this feature may
also prove to be very challenging, especially taking the different phases
of compilation into account.

\subsubsection{Identifier formatting}

Another option is a simplified version of the operator \verb@identifier@,
which would take a format string {\em literal} and optionally a pack of
\meta{Named} metaobjects:

\begin{minted}[tabsize=4]{cpp}
struct bar
{
	int foo;
};

using meta_bar = reflexpr(bar);
using meta_foo = reflexpr(bar::foo);

identifier("int")
identifier("get_%1_%2", meta_bar, meta_foo)(const identifier("%1", meta_bar)&);
\end{minted}

which would result in:

\begin{minted}[tabsize=4]{cpp}
int get_bar_foo(const bar&);
\end{minted}

In this case we would only allow string {\em literals}, not arbitrary
\verb@constexpr@ string-returning expressions and the identifier formatting
would be handled inside of the compiler.

Also note that we could combine this with the reversible reflection described
in section \ref{fut-reverse-reflection} and reuse the \verb@reflexpr@ operator
instead of adding a new operator like \verb@identifier@:

\begin{minted}[tabsize=4]{cpp}
reflexpr("int")
reflexpr("get_%1_%2", meta_bar, meta_foo)(const reflexpr("%1", meta_bar)&);
\end{minted}

\subsubsection{\texttt{named\_data\_member} and \texttt{named\_member\_typedef}}
\label{fut-named-member-X}

If we don't want to implement the operator \verb@identifier@ or expose it
to the users directly, the third option is to add new operations like
\verb@named_member_typedef@  and \verb@named_data_member@ for \meta{Named}
metaobjects {\em equivalent} to the following:

\begin{minted}[tabsize=4]{cpp}
template <MetaNamed MO, typename X>
struct named_member_typedef
{
	struct type
	{
		// same as
		// typedef X identifier(meta::get_name_v<MO>);
		typedef X __builtin_identifier_of(MO);
	};
};

template <MetaNamed MO, typename X>
struct named_data_member
{
	struct type
	{
		// same as
		// X identifier(meta::get_name_v<MO>);
		X __builtin_identifier_of(MO);

		template <typename ... P>
		type(P&& ... p)
		 : __builtin_identifier_of(MO)(forward<P>(p)...) 
		{ }
	};
};

template <MetaNamed MO, typename X>
using named_member_typedef_t = typename named_member_typedef<MO, X>::type;

template <MetaNamed MO, typename X>
using named_data_member_t = typename named_data_member<MO, X>::type;
\end{minted}

So for example the specializations of \verb@named_member_typedef@ and
\verb@named_data_member@ for the metaobject reflecting \verb@bar::foo@ would
look like this:

\begin{minted}[tabsize=4]{cpp}
template <typename X>
struct named_member_typedef<reflexpr(bar::foo), X>
{
	struct type
	{
		typedef X foo;
	};
};

template <typename X>
struct named_data_member<reflexpr(bar::foo), X>
{
	struct type
	{
		X foo;

		template <typename ... P>
		type(P&& ... p)
		 : foo(forward<P>(p)...) 
		{ }
	};
};
\end{minted}

With some clever metaprogramming we can combine specializations
of \verb@named_data_member_t<...>@ or \verb@named_member_typedef_t<...>@
into structures having member types or variables with names, matching other
existing declarations.

Note that this option does not give use the ability to change the identifiers,
just copy them and it's generally less powerful.
On the other hand, implementing it may be much less complicated
than the first two options.

