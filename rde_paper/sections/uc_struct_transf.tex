\subsection{Structure data member transformations}
\label{use-case-struct-transf}

We need to create a new structure which has data members with the same or similar
names as an original structure, but we need to change some of the properties
of the data members, usually their types.

For example we need to transform a structure like:

\begin{minted}[tabsize=4]{cpp}
struct foo
{
	bool b;
	char c;
	double d;
	float f;
	string s;
};
\end{minted}

into

\begin{minted}[tabsize=4]{cpp}
struct rdbs_table_placeholder_foo
{
	column_placeholder<bool>::type b;
	column_placeholder<char>::type c;
	column_placeholder<double>::type d;
	column_placeholder<float>::type f;
	column_placeholder<string>::type s;
};
\end{minted}

or create a structure-of-arrays, which was one of the \say{targeted use cases}
from the committee's CFP~\cite{ISOCPP-N3814}: 

\begin{minted}[tabsize=4]{cpp}
struct soa_foo
{
	vector<bool> bs;
	vector<char> cs;
	vector<double> sd;
	vector<float> fs;
	vector<string> ss;
};
\end{minted}

The primary obstacle in implementing this use case with the help of reflection
is at the moment the fact that we do not have the ability to create C++
identifiers \say{programmatically}, at least not without the help of
the preprocessor.

We either have to add the ability to create identifiers from compile-time strings
which may be fairly complicated , or look for some simpler workarounds.

First, let's say that we have a magic operator \verb@identifier@ which creates an
identifier from a compile time string, so that:

\begin{minted}[tabsize=4]{cpp}
identifier("long") identifier("foo")(int i, int j);
\end{minted}

would be equivalent to 

\begin{minted}[tabsize=4]{cpp}
long foo(int i, int j);
\end{minted}

With the help of multiple inheritance and the
\hyperref[fut-facade]{\texttt{expand\_sequence}} helper template described
in section \ref{fut-facade}, we can create a new structure that is
nearly equivalent to \verb@soa_foo@ via metaprogramming:

\begin{minted}[tabsize=4]{cpp}

template <typename MetaDataMember>
struct soa_single_member
{
	// vector<T> Xs;
	// assuming that we have a ct_concat function which
	// does compile-time concatenation of two strings
	vector<meta::get_original_type_t<meta::get_type_t<MetaDataMember>>>
		identifier(ct_concat(meta::get_name_v<MetaDataMember>, "s"));

	/* constructors, forwarding parameters to the vector, ... */
};

template <typename ... MetaDataMembers>
struct soa_inherit_all
 : soa_single_member<MetaDataMembers>...
{ /* constructors forwarding parameters to the inherited soa_single_members */ };

template <typename T>
struct soa
 : meta::expand_sequence_t<
	meta::get_data_members_t<reflexpr(T)>,
	soa_inherit_all
> { /* constructors indirectly forwarding parameters to soa_inherit_all */ };

using soa_foo = soa<foo>;
\end{minted}

If we don't want to implement the magic operator \verb@identfier@ or expose it
to the users directly, we could add a new operation \verb@named_mem_var@
for \meta{Named} metaobjects equivalent to the following:

\begin{minted}[tabsize=4]{cpp}
template <MetaNamed MO, typename X>
struct named_mem_var
{
	struct type
	{
		// same as
		// X identifier(meta::get_name_v<MO>);
		X __builtin_identifier_of(MO);

		template <typename ... P>
		type(P&& ... p)
		 : __builtin_identifier_of(MO)(forward<P>(p)...) 
		{ }
	};
};

template <MetaNamed MO, typename X>
using named_mem_var = typename named_mem_var_t<MO, X>::type;
\end{minted}

So for example the specialization of \verb@named_mem_var@ for the reflection
of \verb@foo::c@ would look like this:

\begin{minted}[tabsize=4]{cpp}
template <typename X>
struct named_mem_var<reflexpr(foo::c), X>
{
	struct type
	{
		X c;

		template <typename ... P>
		type(P&& ... p)
		 : c(forward<P>(p)...) 
		{ }
	};
};
\end{minted}

We can then combine the specializations of \verb@named_mem_var<...>::type@
via multiple inheritance as before:

\begin{minted}[tabsize=4]{cpp}

template <typename MetaDataMember>
using soa_single_member = meta::named_mem_var_t<
	MetaDataMember,
	vector<meta::get_original_type_t<meta::get_type_t<MetaDataMember>>>
>;

template <typename ... MetaDataMember>
struct soa_inherit_all
 : soa_single_member<MetaDataMembers>...
{ /* constructors forwarding parameters to the inherited soa_single_members */ };

template <typename T>
struct soa
 : meta::expand_sequence_t<
	meta::get_data_members_t<reflexpr(T)>,
	soa_inherit_all
> { /* constructors indirectly forwarding parameters to soa_inherit_all */ };

using soa_foo = soa<foo>;
\end{minted}

Note that with this workaround we don't have the ability to change the name
of the data members from \verb@c@ to \verb@cs@, etc., but the \verb@named_mem_var@
template is potentially much simpler to implement.
