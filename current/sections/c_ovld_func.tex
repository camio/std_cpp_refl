\subsection{MetaOverloadedFunction}
\label{concept-MetaOverloadedFunction}

Models of \meta{OverloadedFunction} reflect overloaded functions.
\meta{Function}s (and \meta{Operator}s, \meta{Initializer}s, \meta{Constructor}s, etc.)
are not direct members of scopes (they are not listed in the \concept{MetaobjectSequence}
"returned" by the \verb@members<MetaScope>@ template class).
Instead, all functions, operators and constructors with the same name, (and even those that are not
overloaded in a specific scope) are grouped into a \meta{OverloadedFunction}. Individual overloaded \meta{Function}s
in the group can be obtained through the interface of \meta{OverloadedFunction} (specifically through the
\verb@overloads@ template described below). The same also applies to \meta{Constructor}s and \meta{Operator}s.

The rationale for this is that direct scope members, i.e. metaobjects accessible through the \meta{Scope}'s
\verb@members@ template class should have unique names, which would not be the case if \meta{Function}s
were direct scope members.

The \verb@scope@ of an \meta{OverloadedFunction} is the same as the \verb@scope@
of all \meta{Function}s grouped by that \meta{OverloadedFunction}.

In addition to the requirements inherited from \meta{NamedScoped},
models of \meta{OverloadedFunction} are subject to the following:

The \verb@category@ template class specialization for a \meta{OverloadedFunction}
should inherit from \verb@overloaded_function_tag@:

\begin{minted}{cpp}
template <>
struct category<MetaOverloadedFunction>
 : overloaded_function_tag
{ };
\end{minted}

\subsubsection{\texttt{overloads}}

A template class called \verb@overloads@ should be defined and should
return a \concept{MetaobjectSequence} of \meta{Function}s, reflecting
the individual overloads:

\begin{minted}{cpp}
template <>
struct overloads<MetaOverloadedFunction>
 : MetaobjectSequence<MetaFunction>
{ };
\end{minted}

