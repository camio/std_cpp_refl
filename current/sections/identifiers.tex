\section{Identifier pasting}

As a part of this proposal we suggest to consider adding a new functionality to the core language,
allowing to specify identifiers as compile-time constant C-string literal expression, i.e. expressions
evaluating into values of \verb@constexpr const char [N]@.

This could be implemented either by using a new operator (or recycling an old one),
or maybe by using generalized attributes.
In the examples below the \verb@identifier@ operator is used, but we do not have
any strong preference for the name of this operator.

For example:

\begin{minted}{cpp}

identifier("int") identifier("main")(
	int idenitifier("argc"),
	const identifier("char")* identifier("argv")
)
{
	using namespace identifier(base_name<mirrored(std)>::value);
	for(int i=0; i<argc; ++i)
	{
		cout << argv[i] << endl;
	}
	return 0;
}

\end{minted}

would be equivalent to

\begin{minted}{cpp}
int main(int argc, const char* argv)
{
	using namespace std;
	return 0;
}
\end{minted}

The content of the string literal passed as the argument to \verb@identifier@
should be encoded in the source character set and subject to the same restrictions
which are placed on identifiers.

For example:

\begin{minted}{cpp}

class foo
{
	identifier("some-type#") x; // Error

	double identifier("static"); // Error

	float identifier("void"); // Error

	long identifier("<bar>"); // Error

	identifier("std::vector<std::unique_ptr<bar>>") v; // OK
};

\end{minted}

The idea is to replace preprocessor token concatenation with much more
flexible constexpr C++ expressions.
Adding identifier pasting would allow to replace the \verb@named_mem_var@
and \verb@named_typedef@ metafunctions which were in N4111 defined as part of
the interface of \meta{Named} with a much more powerful feature.

