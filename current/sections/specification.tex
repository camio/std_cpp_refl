\section{Metaobject concept specification}
\label{section-current-Concepts}

We propose that the basic metadata describing a program written
in C++ should be made available through a set of {\em anonymous} types
defined by the compiler and through related functions and template classes.
At the moment these types should describe only the following program
constructs: namespaces\footnote{in a limited form}, types, typedefs,
classes and their data members.

In the future, the set of metaobjects should be extended to reflect also
class inheritance, free functions, class member functions, templates,
template parameters, enumerated values, specifiers, etc.
See appendix~\ref{section-all-Concepts} for more details.

The compiler should generate metadata for the program constructs defined
in the currently processed translation unit, when requested by invoking
the reflection operator. Members of ordered sets (sequences) of metaobjects,
like scope members, parameters of a function, and so on, should be listed
in the order of appearance in the processed source code.

Since we want the metadata to be available at compile-time,
different base-level constructs should be reflected by
{\em statically different} metaobjects and thus by {\em different} types.
For example a metaobject reflecting the global scope namespace should
be a different {\em type} than a metaobject reflecting the \verb@std@
namespace\footnote{this means that they should be distinguishable for
example by the \texttt{std::is\_same} type trait},
a metaobject reflecting the \verb@int@ type should
have a different type then a metaobject reflecting the \verb@double@
type, etc.

This section describes a set of metaobject concepts,
their interfaces\footnote{the requirements that the various metaobjects
need to satisfy in order to be considered models of the individual
concepts},
tag types for metaobject classification and operators providing access to the metaobjects.

Unless stated otherwise, all named templates proposed and described below should
go into the \verb@std::meta@ nested namespace in order to contain reflection-related
definitions and to help avoiding potential name conflicts.

Also note, that in the sections below, the examples use names for concrete
metaobjects, like \verb@__meta_std_string@, etc. This convention
is {\em NOT} part of this proposal. The actual naming of the metaobjects
should be left to the compiler implementations and for all purposes,
from the user's point of view, the metaobjects should be anonymous types.

\subsection{Metaobject}
\label{concept-Metaobject}

A \meta{object} is a stateless anonymous type generated by the compiler
(when requested by the programmer through the 
\hyperref[section-reflection-operator]{\texttt{reflexpr} operator}),
providing metadata reflecting a specific program feature.

\subsubsection{\texttt{is\_metaobject}}

In order to distinguish between regular types and metaobjects generated
by the compiler, the \texttt{is\_metaobject} trait should be added
to the \texttt{std} namespace as one of the type traits. 

\begin{minted}{cpp}
template <typename T>
struct is_metaobject
{
	typedef bool value_type;
	static constexpr const bool value = /*
		true: if T is a metaobject
		false: otherwise
	*/

	typedef integral_constant<bool, value> type;

	operator value_type (void) const noexcept;
	value_type operator(void) const noexcept;
};

template <typename T>
using is_metaobject_t = typename is_metaobject<T>::type;

template <typename T>
constexpr bool is_metaobject_v = is_metaobject<T>::value;
\end{minted}

The expression \texttt{is\_metaobject<X>::value} should evaluate to \texttt{true}
if \texttt{X} is a metaobject generated by the compiler, otherwise it should
be \texttt{false}.

There are also several other trait templates nested in the namespace
\texttt{std::meta} which provide further information about metaobjects.
These traits are listed below together with their related metaobject concepts.



\subsubsection{Definition}

\begin{minted}[tabsize=8]{cpp}

template <typename T>
concept bool Metaobject =
	is_metaobject_v<T>;

\end{minted}


The following operations are defined for each type satisfying the \meta{object}
concept.


\subsubsection{\texttt{get\_source\_file}}

returns the source file path of the declaration of a base-level program feature reflected by a \meta{object}.

\begin{minted}[tabsize=4]{cpp}

template <typename T>
requires Metaobject<T>
struct get_source_file {
	typedef StringConstant type;
	typedef const char value_type[N+1];
	static constexpr const char value[N+1];

	operator const char* (void) const noexcept;
	const char* operator (void) const noexcept;
};


template <typename T>
using get_source_file_t = typename get_source_file<T>::type;
template <typename T>
constexpr bool get_source_file_v = get_source_file<T>::value;

\end{minted}

\subsubsection{\texttt{get\_source\_line}}

returns the source file line of the declaration of a base-level program feature reflected by a \meta{object}.

\begin{minted}[tabsize=4]{cpp}

template <typename T>
requires Metaobject<T>
struct get_source_line {
	typedef integral_constant<unsigned, value> type;
	typedef unsigned value_type;
	static constexpr const unsigned value;

	operator unsigned(void) const noexcept;
	unsigned operator(void) const noexcept;
};


template <typename T>
using get_source_line_t = typename get_source_line<T>::type;
template <typename T>
constexpr bool get_source_line_v = get_source_line<T>::value;

\end{minted}

\subsubsection{\texttt{get\_source\_column}}

returns the source file column of the declaration of a base-level program feature reflected by a \meta{object}.

\begin{minted}[tabsize=4]{cpp}

template <typename T>
requires Metaobject<T>
struct get_source_column {
	typedef integral_constant<unsigned, value> type;
	typedef unsigned value_type;
	static constexpr const unsigned value;

	operator unsigned(void) const noexcept;
	unsigned operator(void) const noexcept;
};


template <typename T>
using get_source_column_t = typename get_source_column<T>::type;
template <typename T>
constexpr bool get_source_column_v = get_source_column<T>::value;

\end{minted}


The templates returning source information for for built-in types and other
such base-level constructs which are declared internally by the compiler
should return an empty string as a source file path and zero as source file
line and column.

\subsection{MetaobjectSequence}
\label{concept-MetaobjectSequence}

A \meta{object} is representing an ordered sequence of metaobjects.


\subsubsection{\texttt{is\_sequence}}

The \texttt{meta::is\_sequence}
trait indicates that the \meta{object} passed as argument is a \meta{objectSequence}.


\begin{minted}[tabsize=4]{cpp}
namespace meta {

template <typename T> requires Declaration<T>
struct is_sequence : integral_constant<bool, ... > { };

template <typename T>
constexpr bool is_sequence_v = is_sequence<T>::value;

} // namespace meta
\end{minted}



\subsubsection{Definition}

\begin{minted}[tabsize=4]{cpp}

template <typename T>
concept bool MetaobjectSequence =
	Metaobject<T> &&
	meta::is_sequence_v<T>;

\end{minted}



\subsubsection{\texttt{get\_size}}

returns a number of elements in the sequence.

\begin{minted}[tabsize=4]{cpp}

template <typename T>
requires MetaobjectSequence<T>
struct get_size
{
	typedef size_t value_type;
	static constexpr const size_t value;
	typedef integral_constant<value_type, value> type;

	operator value_type (void) const noexcept;
	value_type operator(void) const noexcept;
};


template <typename T>
using get_size_t = typename get_size<T>::type;

\end{minted}

\subsubsection{\texttt{get\_element}}

returns the i-th element in a MetaobjectSequence.

\begin{minted}[tabsize=4]{cpp}

template <typename T1, size_t Index>
requires MetaobjectSequence<T1>
struct get_element
{
	typedef Metaobject type;
};


template <typename T1, size_t Index>
using get_element_t = typename get_element<T1, Index>::type;

\end{minted}


\subsection{MetaSpecifier}
\label{concept-MetaSpecifier}

A \meta{Specifier} reflects a C++ specifier.


\subsubsection{\texttt{is\_specifier}}

The \texttt{meta::is\_specifier}
trait indicates if the \meta{object} passed as argument is a \meta{Specifier}.


\begin{minted}[tabsize=4]{cpp}
namespace meta {

template <typename T>
requires Metaobject<T>
struct is_specifier
{
	typedef bool value_type;
	static constexpr const bool value;
	typedef integral_constant<bool, value> type;

	operator value_type (void) const noexcept;
	value_type operator(void) const noexcept;
};

template <typename T>
using is_specifier_t = typename is_specifier<T>::type;

template <typename T>
constexpr bool is_specifier_v = is_specifier<T>::value;

} // namespace meta
\end{minted}



\subsubsection{Definition}

\begin{minted}[tabsize=4]{cpp}

template <typename T>
concept bool MetaSpecifier =
	Metaobject<T> &&
	meta::is_specifier_v<T>;

\end{minted}


\subsubsection{\texttt{get\_keyword}}

returns the keyword of the specifier reflected by a \meta{Specifier}.

\begin{minted}[tabsize=4]{cpp}

template <typename T>
requires MetaSpecifier<T>
struct get_keyword
{
	typedef const char value_type[N+1];

	static constexpr const char value[N+1];

	typedef StringConstant type;

	operator const char* (void) const noexcept;
	const char* operator (void) const noexcept;
};


template <typename T>
using get_keyword_t = typename get_keyword<T>::type;

\end{minted}



\subsection{MetaNamed}
\label{concept-MetaNamed}

A \meta{Named} is a \meta{object} reflecting a named base-level construct
(namespace, type, variable, function, etc.).


\subsubsection{\texttt{has\_name}}

The \texttt{meta::has\_name}
trait indicates that the \meta{object} passed as argument is a \meta{Named}.


\begin{minted}[tabsize=4]{cpp}
namespace meta {

template <typename T>
requires Metaobject<T>
struct has_name
{
	typedef bool value_type;
	static constexpr const bool value = /*
		true: if T is a MetaNamed
		false: otherwise
	*/

	typedef integral_constant<bool, value> type;

	operator value_type (void) const noexcept;
	value_type operator(void) const noexcept;
};

template <typename T>
using has_name_t = typename has_name<T>::type;

template <typename T>
constexpr bool has_name_v = has_name<T>::value;

} // namespace meta
\end{minted}



\subsubsection{Definition}

\begin{minted}[tabsize=8]{cpp}

template <typename T>
concept bool MetaNamed =
	Metaobject<T> &&
	meta::has_name_v<T>;

\end{minted}


\subsubsection{\texttt{get\_name}}

returns the basic name of the a named declaration reflected by a \meta{Named}.

\begin{minted}[tabsize=4]{cpp}

template <typename T>
requires MetaNamed<T>
struct get_name
{
	typedef const char value_type[N+1];
	static constexpr const char value[N+1];
	typedef StringConstant type;

	operator const char* (void) const noexcept;
	const char* operator (void) const noexcept;
};


template <typename T>
using get_name_t = typename get_name<T>::type;

\end{minted}


\subsection{MetaTyped}
\label{concept-MetaTyped}

A \meta{Typed} is a \meta{object} reflecting base-level construct with a type
(like a variable).


\subsubsection{\texttt{has\_type}}

The \texttt{meta::has\_type}
trait indicates that the \meta{object} passed as argument is a \meta{Typed}.


\begin{minted}[tabsize=4]{cpp}
namespace meta {

template <typename T> requires Metaobject<T>
struct has_type {
	typedef integral_constant<bool, value> type;
	typedef bool value_type;
	static constexpr const bool value;

	operator bool(void) const noexcept;
	bool operator(void) const noexcept;
};

template <typename T>
using has_type_t = typename has_type<T>::type;
template <typename T>
constexpr bool has_type_v = has_type<T>::value;

} // namespace meta
\end{minted}



\subsubsection{Definition}

\begin{minted}[tabsize=8]{cpp}

template <typename T>
concept bool MetaTyped =
	Metaobject<T> &&
	meta::has_type_v<T>;

\end{minted}


\subsubsection{\texttt{get\_type}}

returns the MetaType reflecting the type of base-level construct with a type reflected by a \meta{Typed}.

\begin{minted}[tabsize=4]{cpp}

template <typename T>
requires MetaTyped<T>
struct get_type
{
	typedef MetaType type;
};


template <typename T>
using get_type_t = typename get_type<T>::type;

\end{minted}


\subsection{MetaScoped}
\label{concept-MetaScoped}

A \meta{Scoped} is a \meta{object} reflecting base-level construct declared
inside of a scope.


\subsubsection{\texttt{has\_scope}}

The \texttt{meta::has\_scope}
trait indicates that the \meta{object} passed as argument is a \meta{Scoped}.


\begin{minted}[tabsize=4]{cpp}
namespace meta {

template <typename T> requires Metaobject<T>
struct has_scope {
	typedef integral_constant<bool, value> type;
	typedef bool value_type;
	static constexpr const bool value;

	operator bool(void) const noexcept;
	bool operator(void) const noexcept;
};

template <typename T>
using has_scope_t = typename has_scope<T>::type;
template <typename T>
constexpr bool has_scope_v = has_scope<T>::value;

} // namespace meta
\end{minted}



\subsubsection{Definition}

\begin{minted}[tabsize=8]{cpp}

template <typename T>
concept bool MetaScoped =
	Metaobject<T> &&
	meta::has_scope_v<T>;

\end{minted}


\subsubsection{\texttt{get\_scope}}

returns the MetaScope reflecting the scope of a scoped declaration reflected by a \meta{Scoped}.

\begin{minted}[tabsize=4]{cpp}

template <typename T>
requires MetaScoped<T>
struct get_scope
{
	typedef MetaScope type;
};


template <typename T>
using get_scope_t = typename get_scope<T>::type;

\end{minted}


\subsection{MetaScope}
\label{concept-MetaScope}

A \meta{Scope} is a \meta{Named} and possibly a \meta{Scoped} reflecting a scope.


\subsubsection{\texttt{is\_scope}}

The \texttt{meta::is\_scope}
trait indicates that the \meta{object} passed as argument is a \meta{Scope}.


\begin{minted}[tabsize=4]{cpp}
namespace meta {

template <typename T> requires Metaobject<T>
struct is_scope {
	typedef bool value_type;
	static constexpr const bool value;
	typedef integral_constant<bool, value> type;

	operator bool(void) const noexcept;
	bool operator(void) const noexcept;
};

template <typename T>
using is_scope_t = typename is_scope<T>::type;
template <typename T>
constexpr bool is_scope_v = is_scope<T>::value;

} // namespace meta
\end{minted}



\subsubsection{Definition}

\begin{minted}[tabsize=8]{cpp}

template <typename T>
concept bool MetaScope =
	Metaobject<T> &&
	meta::is_scope_v<T>;

\end{minted}


\subsection{MetaAlias}
\label{concept-MetaAlias}

A \meta{Alias} is a \meta{Named} reflecting a type or namespace alias.  


\subsubsection{\texttt{is\_alias}}

The \texttt{meta::is\_alias}
trait indicates if the \meta{object} passed as argument is a \meta{Alias}.


\begin{minted}[tabsize=4]{cpp}
namespace meta {

template <typename T>
requires Metaobject<T>
struct is_alias
{
	typedef bool value_type;
	static constexpr const bool value;
	typedef integral_constant<bool, value> type;

	operator value_type (void) const noexcept;
	value_type operator(void) const noexcept;
};

template <typename T>
using is_alias_t = typename is_alias<T>::type;

template <typename T>
constexpr bool is_alias_v = is_alias<T>::value;

} // namespace meta
\end{minted}



\subsubsection{Definition}

\begin{minted}[tabsize=8]{cpp}

template <typename T>
concept bool MetaAlias =
	MetaNamed<T> &&
	meta::is_alias_v<T>;

\end{minted}


\subsubsection{\texttt{get\_aliased}}

returns the MetaNamed reflecting the original declaration of a type or namespace alias reflected by a \meta{Alias}.

\begin{minted}[tabsize=4]{cpp}

template <typename T>
requires MetaAlias<T>
struct get_aliased {
	typedef MetaNamed type;
};


template <typename T>
using get_aliased_t = typename get_aliased<T>::type;

\end{minted}



\subsection{MetaGlobalScope}
\label{concept-MetaGlobalScope}

A \meta{GlobalScope} is a \meta{Scope} reflecting the global scope.


\subsubsection{\texttt{is\_global\_scope}}

The \texttt{meta::is\_global\_scope}
trait indicates that the \meta{object} passed as argument is a \meta{GlobalScope}.


\begin{minted}[tabsize=4]{cpp}
namespace meta {

template <typename T> requires Metaobject<T>
struct is_global_scope {
	typedef bool value_type;
	static constexpr const bool value;
	typedef integral_constant<bool, value> type;

	operator bool(void) const noexcept;
	bool operator(void) const noexcept;
};

template <typename T>
using is_global_scope_t = typename is_global_scope<T>::type;
template <typename T>
constexpr bool is_global_scope_v = is_global_scope<T>::value;

} // namespace meta
\end{minted}



\subsubsection{Definition}

\begin{minted}[tabsize=8]{cpp}

template <typename T>
concept bool MetaGlobalScope =
	MetaNamed<T> &&
	MetaScope<T> &&
	meta::is_global_scope_v<T>;

\end{minted}


