\subsection{Categorization and Traits}

In order to provide means for distinguishing between regular types
and metaobjects generated by the compiler,
the \verb@is_metaobject@ trait should be added to the standard \verb@type_traits@ library
and should inherit from \verb@true_type@ for metaobjects\footnote{types generated
by the compiler providing metadata} and from \verb@false_type@
for non-metaobjects\footnote{native or user defined types}:

\begin{minted}{cpp}
template <typename T>
struct is_metaobject
 : false_type
{ };
\end{minted}

\subsubsection{Metaobject category tags}
\label{metaobject-category-tags}

To distiguish between various metaobject kinds\footnote{metaobjects satisfying different concepts
as described below} a set of tag \verb@struct@s indicating the metaobject kind
should be added:

\begin{minted}{cpp}

struct namespace_tag { };

struct global_scope_tag { };

struct type_tag { };

struct class_tag { };

struct enum_tag { };

struct enum_class_tag { };

struct variable_tag { };

\end{minted}

These tags are referred-to as \verb@MetaobjectCategory@:

