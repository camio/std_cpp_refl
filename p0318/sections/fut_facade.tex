\subsection{Additional utilities}
\label{fut-facade}

In this section we describe several useful metaprogramming utilities which
could be added later to the standard library to simplify the use of the basic
reflection primitives in certain cases.
 
\subsubsection{\texttt{identity}}
\label{fac-identity}

The \verb@identity@ template should be defined as follows:

\begin{minted}[tabsize=4]{cpp}
template <typename X>
struct identity
{
	typedef X type;
};
\end{minted}

The primary purpose of \verb@identity@ is to wrap types which {\em cannot} be
instantiated\footnote{like the metaobjects returned by \texttt{reflexpr}}
into a template which still carries information about these types, but
{\em can} be used to create run-time variables.

For example:

\begin{minted}[tabsize=4]{cpp}
using MO = reflexpr(int);

void v; // error
MO mo; // error

identity<void> idv; // OK
identity<MO> idmo; // OK
\end{minted}
 
This can be used for example to pass metaobjects into lambda functions,
templated constructors or operators. 

\subsubsection{\texttt{for\_each}}
\label{fac-for-each}

The \verb@for_each@ function should sequentially call a specified
\verb@UnaryFunction@ on instances of \verb@identity@-wrapped metaobjects from
a \meta{ObjectSequence}. It should {\em be equivalent} to the following:

\begin{minted}[tabsize=4]{cpp}
namespace meta {
\end{minted}
\begin{minted}[xleftmargin=1em,tabsize=4]{cpp}
template <typename MetaObjectSequence, typename UnaryFunction>
void for_each(UnaryFunction func)
{
	func(identity<Metaobject1>());
	func(identity<Metaobject2>());
	/* ... */
	func(identity<MetaobjectN>());
}
\end{minted}
\begin{minted}[tabsize=4]{cpp}
} // namespace meta
\end{minted}

For example:

\begin{minted}[tabsize=4]{cpp}
meta::for_each<meta::get_data_members_t<reflexpr(my_class)>>(
	[](auto idmo)
	{
		using MO = decltype(idmo)::type;
		...
	}
);
\end{minted}

There are some cases where we would like to have the ability to interrupt
the traversal if some condition is met, for example if the user-defined
function returns \verb@true@.

\begin{minted}[xleftmargin=1em,tabsize=4]{cpp}
template <typename MetaObjectSequence, typename UnaryFunction>
void for_each_until(UnaryFunction func)
{
	if(func(identity<Metaobject1>())) return;
	if(func(identity<Metaobject2>())) return;
	/* ... */
	if(func(identity<MetaobjectN>())) return;
}
\end{minted}

\subsubsection{\texttt{unpack\_sequence}}
\label{fac-unpack-sequence}

The \verb@unpack_sequence@ template should instantiate the provided variadic
template with metaobjects from a specified \meta{ObjectSequence} as template
arguments. It should be equivalent to the following:

\begin{minted}[tabsize=4]{cpp}
namespace meta {
\end{minted}
\begin{minted}[xleftmargin=1em,tabsize=4]{cpp}
template <typename MetaObjectSequence, template <typename ...> class Template>
struct unpack_sequence
{
	typedef Template<Metaobject1, Metaobject2, ... MetaobjectN> type;
};

template <typename MetaObjectSequence, template <typename ...> class Template>
using unpack_sequence_t =
	typename unpack_sequence<MetaobjectSequence, Template>::type;
\end{minted}
\begin{minted}[tabsize=4]{cpp}
} // namespace meta
\end{minted}

