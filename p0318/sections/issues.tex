\section{Issues}

\subsection{Anonymous declarations}

\textbf{Q:} {\em How should anonymous namespaces, nested structs, unions or
the global scope for that matter be reflected?}

In our opinion all such declarations should be reflected by
the same metaobjects as their named counterparts, but without conforming
to the \meta{Named} concept.

So for example the \meta{GlobalScope} is a \meta{Namespace}, but
the \verb@has_name@ trait should return \verb@false@ and the \verb@get_name@
operation would be either undefined or return an empty string.

On the other hand, even anonymous namespaces, classes or unions are scopes,
so the metaobjects reflecting them should conform to the \meta{Scope} concept.

Same reasoning applies to anonymous namespaces, nested structs or unions.

\subsection{Reflection of \texttt{struct}s and \texttt{union}s}

\textbf{Q:} {\em Do we need separate \meta{Struct} and \meta{Union}
concepts or is \meta{Class} sufficient to cover all cases?}

We don't have any strong opinion in this regard.

\subsection{Alternatives for breaking access restrictions}
\label{issue-breaking-access}

\textbf{Q:} {\em Some people were objecting to reflection being able to break
access restrictions. Would this capability be more acceptable if it had to be
explicitly allowed or if it could be denied?}

Breaking class member access restrictions is important for many use cases,
like non-intrusive serialization of third-party classes, which may include
the serialization of non-public data-members.
But there may be situations where the programmer would like to prevent
reflection from having access to non-public members.

This may be achieved either by annotating such members with an attribute;

\begin{minted}[tabsize=4]{cpp}
class very_private
{
private:
	[[nonreflectable]] int _really_secret;
public:
	/* ... */
};
\end{minted}

or by explicitly allowing access to reflection, for example by extending
the \verb@friend@ mechanism:

\begin{minted}[tabsize=4]{cpp}
class not_so_private
{
private:
	friend reflexpr;
	int _public_secret;
public:
	/* ... */
};
\end{minted}

If this is deemed necessary, then we favor the first option.

\subsection{Interaction with concepts}

\textbf{Q:} {\em Are there any special interactions between compile-time
reflection and concepts?}

Unresolved.

\subsection{Interaction with attributes}

TODO.

\subsection{Interaction with modules}

\textbf{Q:} {\em Are there any special interactions between compile-time
reflection and modules?}

Unresolved.

