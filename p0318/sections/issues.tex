\section{Issues}

\subsection{Anonymous declarations}

\textbf{Q:} {\em How should anonymous namespaces, nested structs, unions or
the global scope for that matter be reflected?}

In our opinion all such declarations should be reflected by
the same metaobjects as their named counterparts, but without conforming
to the \meta{Named} concept.

So for example the \meta{GlobalScope} is a \meta{Namespace}, but
the \verb@has_name@ trait should return \verb@false@ and the \verb@get_name@
operation would be either undefined or return an empty string.

On the other hand, even anonymous namespaces, classes or unions are scopes,
so the metaobjects reflecting them should conform to the \meta{Scope} concept.

Same reasoning applies to anonymous namespaces, nested structs or unions.

\subsection{Reflection of \texttt{struct}s and \texttt{union}s}

\textbf{Q:} {\em Do we need separate \meta{Struct} and \meta{Union}
concepts or is \meta{Class} sufficient to cover all cases?}

We don't have any strong opinion in this regard.

\subsection{Alternatives for breaking access restrictions}
\label{issue-breaking-access}

\textbf{Q:} {\em Some people were objecting to reflection being able to break
access restrictions. Would this capability be more acceptable if it had to be
explicitly allowed or if it could be denied?}

Breaking class member access restrictions is important for many use cases,
like non-intrusive serialization of third-party classes, which may include
the serialization of non-public data-members.
But there may be situations where the programmer would like to prevent
reflection from having access to non-public members.

This may be achieved either by annotating such members with an attribute;

\begin{minted}[tabsize=4]{cpp}
class very_private
{
private:
	[[nonreflectable]] int _really_secret;
public:
	/* ... */
};
\end{minted}

or by explicitly allowing access to reflection, for example by extending
the \verb@friend@ mechanism:

\begin{minted}[tabsize=4]{cpp}
class not_so_private
{
private:
	friend reflexpr;
	int _public_secret;
public:
	/* ... */
};
\end{minted}

If this is deemed necessary, then we favor the first option.

\subsection{Interaction with attributes}

\textbf{Q:} {\em Should we even reflect attributes? If so, how will the reflection
of generalized attributes look like?}

Currently generalized attributes are used {\em mostly} as hints to the compiler.
They can help the compiler to do some optimizations\footnote{for example the
\texttt{probably(true)} or \texttt{carries\_dependency} attributes} or to
indicate that the user really means to do something what in other circumstances
could be a bug\footnote{like the \texttt{fallthrough} or \texttt{maybe\_unused}
attributes} that the compiler warns about, etc.

Some people argue that generalized attributes are not supposed to have any
semantic effects -- a compiler should be able to completely strip 
a program of all attributes and it should not have any effects on the behavior
of the program.

Attributes could however be viewed more broadly as a generic mechanism for
annotating declarations\footnote{Associating one or more constant values with
the declarations.} in the code, without explicitly saying that the recipient
can only be the compiler.
There are valid use cases for user annotations and since the annotations
usually provide additional conceptual (meta-)information about a declaration,
reflection is a natural mechanism for accessing the annotations.

We favor the view that on the base-level the compiler should be able to
strip attributes\footnote{or to ignore attributes which it does not understand},
but it should reflect them on the meta-level on demand.

Having covered that, the reflection of the \say{simple} attributes like
\verb@[[attr1]]@ or \verb@[[namespace::attr2]]@ could be pretty straightforward:

We add a \meta{Attribute} concept and its trait, say \verb@meta::is_attribute@,
make it conform to \meta{Named} and \meta{Scoped} concepts and then add an
operation to the interface of \meta{Object} like \verb@get_attributes@, returning
a sequence of metaobjects reflecting the attributes on a base-level declaration.

The complication is that attributes can have nearly arbitrary parameters;
\verb@[[probably(true)]]@, \verb@[[deprecated("reason")]]@,
\verb@[[visibility(hidden)]]@, \verb@[[gnu::aligned(64)]]@. The reflection of
such attributes and their arguments is currently unresolved.

\subsection{Interaction with concepts}

\textbf{Q:} {\em Are there any special interactions between compile-time
reflection and concepts?}

Unresolved.

\subsection{Interaction with modules}

\textbf{Q:} {\em Are there any special interactions between compile-time
reflection and modules?}

Unresolved.

