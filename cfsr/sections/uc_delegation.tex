\subsection{Implementing delegation or decorators}

We need to create a decorator class, which wraps an instance of another
class, implements similar interface as the original class, writes info about
each member function call into a log and then delegates the call to the private member object:

\begin{minted}[tabsize=4]{cpp}

class foo
{
public:
	void f1(void);

	int f2(int a, int b);

	double f3(float a, long b, double c, const std::string& d);
};

class logging_foo
{
private:
	foo _obj;
	loglib::log_sink _log;

	template <typename MetaFunction, typename ... P>
	void _do_log_call(const P&...);
public:
	void f1(void)
	{
		_do_log_call<mirrored(this::function)>();
		_obj.f1();
	}

	int f2(int a, int b);
	{
		_do_log_call<mirrored(this::function)>();
		return _obj.f2(a, b);
	}

	double f3(float a, long b, double c, const std::string& d);
	{
		_do_log_call<mirrored(this::function)>();
		return _obj.f3(a, b, c, d);
	}
};

\end{minted}

Obviously the definition of \verb@logging_foo@ is very repetitive and if this
pattern is recurring in the code it may lead to subtle, hard to track bugs,
so we may wish to automate the implementation.

Reflection to the rescue!

\begin{minted}[tabsize=4]{cpp}

template <typename Wrapped>
class logging_base
{
protected:
	Wrapped _obj;
	loglib::log_sink _log;

	template <typename MetaFunction, typename ... P>
	void _do_log_call(const P&...);
};

\end{minted}

\verb@logging_base@ is a common virtual base class holding the wrapped object
and the log sink.

\begin{minted}[tabsize=4]{cpp}

template <typename Wrapped, typename MetaFunction>
class logging_helper
 : virtual public logging_base<Wrapped>
{
public:
	template <typename ... P>
	auto identifier(base_name<MetaFunction>::value)(P&& ... p)
	{
		this->_do_log_call<MetaFunction>(std::forward<P>(p)...);

		auto mfp = pointer<MetaFunction>::get();

		return (this->_obj.*mfp)(std::forward<P>(p)...);
	}
};

\end{minted}

\verb@logging_helper@ is a unit implementing the delegation of a single function
call from the interface of the \verb@Wrapped@ class. 

The \verb@identifier@ operator is used here to define the name of the member function
to be the same as the name of the wrapped function.

If the idea of the \verb@identifier@ operator is scrapped, it would still be doable
in terms of the \verb@named_mem_var@ template as defined in N4111, or some variation
on that theme.

\begin{minted}[tabsize=4]{cpp}

template <typename Wrapped, typename ... MetaFunctions>
class logging_helpers
 : public logging_helper<Wrapped, MetaFunctions>...
{ };

\end{minted}

\verb@logging_helpers@ inherits from multiple \verb@logging_helper@ units
each having a single \meta{Function} reflecting respective member functions
of the \verb@Wrapper@ class.

\begin{minted}[tabsize=4]{cpp}

template <typename Wrapped, typename MetaFunctionSeq, typename IdxSeq>
class logging_impl;

template <typename Wrapped, typename MetaFunctionSeq, std::size_t ... I>
class logging_impl<Wrapped, MetaFunctionSeq, std::index_sequence<I...>>
 : public logging_helpers<Wrapped, at<MetaFunctionSeq, I>...>
{ };

\end{minted}

\verb@logging_impl@ uses a standard \verb@index_sequence@ to extract the
individual \meta{Functions} from the metafunction sequence and passes them
to \verb@logging_helpers@ as a parameter pack.


\begin{minted}[tabsize=4]{cpp}

template <typename Wrapped>
class logging
 : public logging_impl<
	Wrapped,
	members<mirrored(Wrapped)>,
	std::make_index_sequence<size<members<mirrored(Wrapped)>>::value>
>
{ };

typedef logging<foo> logging_foo;

\end{minted}

The \verb@logging@ template makes the use of \verb@logging_impl@ convenient.

Note that the metaobject sequence 'returned' by \verb@members<...>@ should
be filtered to contain only \meta{Function}s.

This programming pattern of creating a new class with the same or similar interface
than another class is quite frequent and includes not just typical decorators or delegation
but also adapters, type-erasures, mock classes used for unit testing,
etc.\footnote{See also the following use-case.}

The following features are shown in this use-case:

\begin{itemize}
\item{class member reflection,}
\item{class member function reflection,}
\item{use of the reflection operator,}
\item{use of the \verb@identifier@ operator or the \verb@named_mem_var@ templates.}
\end{itemize}

